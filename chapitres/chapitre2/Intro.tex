%\section*{Introduction}
%\label{sec:intro}
%-------------------------------------------------------------------%
%-------------------------------------------------------------------%
\section{État de l'art}
\label{chap:eda}
%-------------------------------------------------------------------%
%-------------------------------------------------------------------%
La possibilité pour un robot d'apprendre à reproduire des mouvements à 
partir de démonstrations est intéressante car intuitive. 
Les démonstrations peuvent par exemple provenir d'un robot identique, ou sur ce même
robot par kinestésie (un opérateur va manipuler à la main les membres du robot).
Dans ce cas, on parle de reproduction de mouvements.
On parle d'imitation de mouvements si la démonstration provient d'un humain ou d'un autre type de robot. 
Le cas échéant, le mouvement doit être transformé pour pouvoir être exécuté au mieux
du point de vue de la ressemblance ou des objectifs à atteindre.
L'imitation ou la reproduction de mouvements peut porter sur des mouvements spécifiques comme
la marche~\cite{benallegue10}, sur des mouvements du corps complet en 
respectant les contraintes d'équilibre du robot~\cite{nakaoka03, yamane09a, miura09} ou 
sur les contraintes physiques par optimisation~\cite{suleiman08}.
Les méthodes d'apprentissage
sont largement utilisées pour généraliser les mouvements appris et sont très populaires.
Cette généralisation porte par exemple sur un changement dans l'environnement.

Le problème de l'imitation
peut être formulé en sous-problèmes
s'attaquant à la reconnaissance et à la génération de mouvements.
La méthode de reconnaissance de t\^aches développée dans ces travaux
s'appuie sur le modèle de génération de mouvements en se
plaçant directement dans les espaces des t\^aches.
Nous comparons les approches étudiées dans le domaine vis-à-vis des espaces utilisés pour
représenter ou caractériser les informations véhiculées dans un mouvement.
