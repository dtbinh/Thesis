\section{Apprentissage et représentation du mouvement}
Un aspect critique dans l'apprentissage de mouvement est le
choix de l'espace dans lequel l'apprentissage s'effectue, ou
autrement dit, des variables qui caractérisent la réalisation
des objectifs.
Ce choix détermine la représentation du mouvement reflétant l'expressivité des 
mouvements observés.
Plusieurs solutions dans le choix de l'espace d'apprentissage ont été étudiées.

Dans~\cite{shon05}, il y a deux espaces d'observations: la configuration articulaire d'un humain
acquise par capture de mouvements, et l'espace articulaire du robot.
L'espace articulaire de l'humain est nécessaire car l'objectif
de ces travaux est de réaliser une imitation.
Un mouvement est généré sur un modèle du robot, et un humain imite ce mouvement.
La trajectoire articulaire de l'humain est enregistrée gr\^ace à un système de capture
de mouvements. De cette façon, l'apprentissage porte sur un corpus parallèle
de trajectoires de l'humain et du robot. L'association entre les trajectoires articulaires
humaines et robotiques est donc manuelle. 

Dans~\cite{chalodhorn09}, l'espace d'observation est l'espace articulaire du robot.
L'objectif de ces travaux étant d'effectuer des reconnaissances et des reproductions 
de mouvements. Pour pouvoir appliquer un algorithme d'apprentissage, 
une technique de réduction de dimension est appliquée:
les auteurs proposent une analyse en composantes principales
non linéaire avec contrainte circulaire. Il s'agit d'une méthode qui généralise
l'analyse en composantes principales car elle permet d'obtenir les relations dans les deux
sens entre l'espace de haute dimension et l'espace réduit. La contrainte circulaire permet
d'adapter la technique aux courbes fermées.
En ne considérant que l'espace articulaire,
la sémantique du mouvement n'est pas explicite: l'exécution d'une t\^ache
n'est que la conséquence de la reproduction de la trajectoire articulaire. 

Le cas contraire est illustré dans~\cite{herzog08}.
Les travaux se focalisent sur les mouvements d'un bras droit pour 
des gestes d'atteintes d'objets.
L'objectif étant d'effectuer la reconnaissance et l'imitation de mouvements d'atteintes
effectués par un humain sur un robot humanoïde.
L'espace d'observation est ici composé des trajectoires dans l'espace cartésiens
de points associés à chaque articulation (épaule, coude, pouce, doigt, poing).
Le choix de cet espace d'observation donne une interprétation aux mouvements considérés.
Les déplacements de ces points sont mis à l'échelle du robot, puis sont utilisés
pour le calcul des trajectoires articulaires
que le robot doit exécuter pour imiter un mouvement. Ainsi
l'algorithme d'apprentissage peut associer le mouvement du bras humain 
au mouvement du bras du robot.
D'une manière similaire, les positions, vitesses et accélérations d'un effecteur sont
observées dans~\cite{hersch08} pour généraliser et reproduire des mouvements.
Les mouvements de démonstrations sont appliqués directement sur le robot par un opérateur.
Des informations de visions sont utilisées pour généraliser les mouvements appris à
d'autres conditions initiales et gérer les perturbations.

Dans \cite{montecillo10}, les positions et orientations
de la tête, des mains, des pieds et du torse d'un humain sont observées. 
Ces positions sont transposées à un robot humanoïde 
pour reproduire le mouvement de l'humain en temps réel.
Toutes les trajectoires de t\^aches sont imitées sans faire
la distinction des mouvements pertinents ou non.

L'espace d'observation considéré dans~\cite{billard03}, est l'union de l'espace articulaire
d'un bras humain, et des positions dans l'espace cartésien d'objets à manipuler.
Dans~\cite{billard06, calinon07a, kwon08, eppner09}, l'espace articulaire et les positions
absolues et relatives entre les effecteurs et les objets à manipuler dans l'environnement sont utilisés
comme observations.
Ces espaces permettent de représenter d'une manière plus explicite,
les t\^aches effectuées puisque l'ajout des positions cartésiennes des éléments
permet de spécifier qu'il n'y a pas que les trajectoires articulaires
qui sont importantes dans le mouvement:  les déplacements des objets et des effecteurs
sont caractéristiques du mouvement. L'espace d'observation est donc spécialisé pour un certain
type d'activité.
Les variations de l'environnement seront prises en compte dans l'apprentissage
pour extraire des descriptions génériques des mouvements démontrés.
Dans~\cite{gribovskaya11}, l'apprentissage se base sur les vitesses et les forces
afin de se spécialiser au contexte de l'interaction physique avec un humain.

Dans tous les travaux présentés, le choix des espaces d'observations est directement lié
à la nature des mouvements. La spécialisation des espaces permet 
de décrire (et donc de reconnaître) plus efficacement les mouvements.
Notre approche consiste à généraliser la spécialisation 
en considérant tous les espaces dans lesquels une t\^ache
peut être exécutée.

\section{Reconnaissance du mouvement}
La reconnaissance du mouvement est un problème pouvant s'appliquer à de nombreux domaines
comme la vision ou l'interaction homme-machine et s'intègre naturellement aux applications d'imitations.
Dans le cas des mouvements humains, la reconnaissance peut porter sur plusieurs niveaux. 
Il peut s'agir de suivre un humain, d'estimer sa configuration articulaire,
reconnaître l'identité d'un humain en se basant sur ses mouvements ou encore interpréter 
ses actions et ses comportements.
Une revue complète et générale de l'analyse de mouvements
humains est présentée dans~\cite{moeslund06}.
Une autre synthèse des travaux sur la reconnaissance de l'action
exécutée dans un mouvement est présentée dans~\cite{krueger07}.

L'étude des mouvements et des t\^aches en biomécanique a amené au développement 
de la théorie des \emph{variétés non contrôlées}~\cite{scholz99}.
Pendant la réalisation d'une t\^ache, les degrés de liberté contribuant à la t\^ache d'un humain
seraient classés dans deux catégories: d'une part les degrés de liberté 
qui sont contrôlés par le système nerveux et d'autre part ceux qui ne le sont pas.
Les variables sont considérées comme étant contrôlées si elles sont stabilisées.
%La mesure de l'accomplissement d'une t\^ache est alors définie par un ratio
%de variance dans deux sous-espaces: le sous-espace non contrôlé et son sous-espace orthogonal.
%Une grande variabilité dans le sous-espace des degrés de liberté
%non contrôlé n'affecte pas les performances tandis
%qu'une grande variabilité dans son sous-espace orthogonal dégrade les performances.
Dans~\cite{jacquierbret09}, la théorie des variétés non contrôlées est appliquée
pour l'étude de mouvements d'atteintes d'un humain en présence d'un obstacle qui va
matérialiser des contraintes spatiales.
L'objectif est de comprendre comment le système nerveux central gère la redondance du bras pour une t\^ache 
d'atteinte en fonction de la présence ou non d'un obstacle. Les différentes 
expérimentations menées renforcent l'hypothèse selon laquelle le système nerveux central
adapte les synergies entre les différents degrés de liberté du bras en fonction
des contraintes spatiales (ici des obstacles) et des objectifs.
On peut noter la similitude entre les variétés non contrôlées et leurs sous-espaces
orthogonals avec le formalisme de la pile de t\^aches présenté dans le chapitre~\ref{chap:sot}. 

\subsection{Classification de mouvements}
Le problème de reconnaissance d'action peut se formuler comme un problème de classification.
Dans les travaux en vision présentés dans \cite{chan07}, 
des mouvements humains, issus de séquences d'images, sont classés pour dégager des comportements distincts
en utilisant des informations contextuelles aux mouvements du corps.
Ces informations sont inspirées de la biomécanique et sont
modélisées par un système de connaissance, à base de règles floues qui sont
appliquées aux trajectoires articulaires d'un humain.  Par
exemple, pour un mouvement de marche, les deux bras ne doivent pas être
en même temps devant le torse. Les classes de mouvements considérées sont 
la course et la marche. Elles correspondent à des comportements humains.
L'inconvénient majeur de la méthode est qu'il est difficile de trouver et de formuler
les règles qui définissent une classe de mouvements. Il est aussi 
difficile de gérer le grand nombre de règles si on accumule
tous les comportements.

Peu de techniques sont développées pour la reconnaissance d'actions simultanées.
La méthode présentée dans~\cite{mori02} est dédiée aux actions humaines simultanées.
Encore une fois, le défaut majeur de cette méthode est qu'une action est représentée 
par un ensemble d'heuristiques sélectionnées manuellement.
Par exemple, pour qu'une action appartienne à la classe \emph{levé de main},
le mouvement doit respecter deux règles: la position de la main doit être au
dessus d'un certain seuil, et la main doit se déplacer de bas en haut.
Toutes les heuristiques sont appliquées au mouvement observé pour dégager l'ensemble
des actions exécutées.

D'autres types de classes peuvent être considérés comme dans~\cite{drumwright03, jenkins04}. Ces classes
sont focalisées sur des classes sémantiques (un direct, un crochet, un uppercut, une garde).
Chaque classe est représentée par des exemples de trajectoires articulaires qui 
vont créer un espace par balayage.
Un classificateur bayesien est utilisé pour calculer la probabilité qu'un mouvement appartienne à une
classe particulière sur des trajectoires articulaires observées, en se basant sur les
espaces de chacune des classes sémantiques.
Des sous-espaces articulaires sont utilisés comme critères de classification
de pas de danse dans~\cite{campbell95}.

La classification présentée dans~\cite{kulic08} est une classification adaptative en ligne. Les classes évoluent
au fur et à mesure des observations. Chaque mouvement observé est encodé sous la forme d'une chaîne
de Markov cachée pour être classé.
Dans~\cite{takano05}, les classes de mouvements se basent sur les paramètres des modèles
de Markov cachés encodant les mouvements.

\subsection{Segmentation temporelle}
La segmentation temporelle de mouvements est utilisée pour définir la succession
de mouvements unitaires ou primitifs dans un mouvement observé. L'outil le
plus utilisé dans ce domaine est le modèle de Markov caché.
Cet outil a surtout été appliqué dans le domaine de la reconnaissance 
de parole~\cite{rabiner89}. Mais ses propriétés mathématiques
permettent de l'utiliser pour modéliser des données séquentielles génériques, dont le mouvement.

La segmentation de données de capture de mouvements est étudiée dans~\cite{barbic04}.
La segmentation est de haut niveau: chaque segment unitaire représente 
un comportement (marcher, courir, frapper\ldots). Trois approches y sont évaluées
en se basant sur les dimensions obtenues par analyse en composantes principales,
puis en se basant sur une analyse en composantes principales associée à un modèle probabiliste~\cite{tipping99} 
et enfin en utilisant des mélanges de modèles gaussiens.

Des techniques très simples sont considérées dans~\cite{field08} pour la segmentation
de mouvements. Celles-ci se basent sur les accélérations des corps en mouvement, les changements de directions
(les points où la vitesse devient nulle)
et les analyses en composantes principales probabilistes.
En animation, dans le cadre des bases de données de mouvements,
la segmentation automatique de mouvements peut être utilisée
pour annoter automatiquement des mouvements et retrouver certains types de motifs de mouvements~\cite{so06, beaudoin08}.
Dans le même domaine, une segmentation de mouvements est réalisée en recherchant les 
postures clefs menant à des changements de direction significatifs~\cite{so05}.

Dans~\cite{kulic08a}, la segmentation d'un flux
de données représentant des mouvements humains, se fait en ligne contrairement à la plupart
des méthodes de segmentations. En effet, il est important que l'apprentissage puisse se faire en ligne
afin de pouvoir tenir compte de la présence d'humains dans l'environnement du robot.
De plus l'approche est incrémentale et permet de
prendre en compte les mouvements déjà appris ou appris manuellement dans 
la segmentation. Cette segmentation se base sur des propriétés des vitesses angulaires des articulations.
Dans~\cite{chalodhorn09}, la segmentation est également effectuée en ligne, 
et se base sur la projection des données observées
dans un espace réduit. Un nouveau segment est créé lorsque les nouveaux points projetés ne correspondent
plus au modèle de prédiction de l'espace courant. Les différents segments représentent
des phases de marches (marche avant, pas sur le côté, virage).
Cette approche ressemble à notre méthode de reconnaissance de t\^ache présentée
dans le chapitre~\ref{chap:reco}, dans le sens où 
les données observées sont comparées dans un espace approprié à des comportements attendus.
La différence est que les espaces que l'on considère sont traités de manière parallèle
et sont directement liés à la sémantique du mouvement.

\section{Génération de mouvements}
L'étape finale en imitation est la génération du mouvement à partir
de la représentation des mouvements reconnus. Les primitives moteurs
permettent d'exprimer le contrôle dans un espace de dimension
inférieure à l'espace articulaire. En combinant
les primitives, il est alors possible d'obtenir une 
grande expressivité pour les mouvements générés. Les primitives
peuvent être tirées d'un système dynamique, et peuvent être
modifiées et commutées en ligne~\cite{degallier08}.
Dans~\cite{sugihara05}, les mouvements sont générés en connectant
par une fonction lisse et continue, une posture clef et la posture courante.
Le mouvement généré tient compte des contraintes dynamiques du robot
et la trajectoire du centre de gravité du robot est recalculée.

Dans~\cite{drumwright03} à la manière de~\cite{rose98},
le mouvement est généré en interpolant des exemples appartenant à une classe de mouvements.
Dans~\cite{calinon07a}, des mouvements sont généralisés par 
mélanges de gaussiennes, puis un mouvement est généré par optimisation
d'une fonction de mesure d'imitation. Cette phase d'optimisation
permet d'adapter le mouvement au contexte courant.

Bien que la technique de réduction de dimension utilisée dans~\cite{chalodhorn09} permette
d'obtenir directement des trajectoires articulaires gr\^ace à la connaissance des relations dans les deux
sens entre l'espace de haute dimension et l'espace réduit, une perte de précision
est possible à cause du processus d'encodage et décodage.
Ainsi l'application du mouvement sur le robot ne garantit aucune stabilité. C'est pour 
cette raison que le mouvement généré est ajusté par optimisation d'une fonction de coût
représentant des valeurs capteurs attendues en fonction de l'état du robot 
dans l'espace réduit. 

