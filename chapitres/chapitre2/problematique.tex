\section{Approche proposée}
Notre approche pour la reconnaissance de 
mouvements s'inscrit dans la continuité des travaux réalisés au JRL et au LAAS-CNRS au sein
de l'équipe Gepetto.
Elle se focalise dans l'espace des t\^aches et par conséquent,
à la reconnaissance de t\^aches.
%Ceci permet
%d'utiliser directement la relation entre l'espace des t\^aches 
%et l'espace articulaire pour générer un mouvement.
Plutôt que de construire un modèle discriminant ou génératif en utilisant 
les outils statistiques, nous exploitons le modèle
de génération de mouvements. En effet, ce modèle permet
d'exprimer directement les corrélations non linéaires entre
les variables de mouvements qui sont inhérentes aux mouvements
humanoïdes. Les approches statistiques et les techniques de
réduction de dimensions ne considèrent généralement que les corrélations linéaires.
Nous utilisons le formalisme de la fonction de t\^ache pour projeter 
le mouvement observé dans les espaces de t\^aches, caractérisant un mouvement, 
ainsi que dans les espaces
nuls. Ce formalisme sera présenté dans le chapitre suivant.
\textbf{Contrairement à la majorité des travaux antérieurs qui s'intéressent
aux mouvements composés de séquences de primitives, nous explorons les 
mouvements complexes issus d'un empilement de contôleurs.}

Nous ne considérons pas les séquencements d'actions ni la généralisation de mouvements:
aucun apprentissage n'est réalisé. En effet, nous supposons que les modèles 
de comportements sont connus puisque ce sont directement les modèles 
de générations de mouvements utilisés pour contrôler le robot.
Notre méthode de reconnaissance de t\^ache est formulée comme un problème d'identification
de t\^aches actives parmis un ensemble connus de t\^aches possible. Notre méthode 
est capable de détecter des t\^aches exécutées en parallèle en projetant
les trajectoires observées dans des espaces caractéristiques.
La méthode est aussi capable de gérer les éventuels couplages de t\^aches gr\^ace
à l'utilisation d'un opérateur de projection dans les espaces nuls des t\^aches.

\section{Plan du manuscrit}
Le chapitre~\ref{chap:sot} introduit les généralités sur la 
cinématique inverse et le concept de la fonction de t\^aches.
Ces fonctions de t\^aches sont utilisées pour la génération de mouvements
gr\^ace à une hiérarchisation de contrôleurs. La méthode
de génération de mouvements utilisée tout au long des travaux présentés
est une implémentation de cette approche: la \emph{pile de t\^aches}.
La méthode de reconnaissance présentée dans ce manuscrit s'appuie sur les propriétés associées à 
la fonction de t\^aches. Ces propriétés sont utilisées pour manipuler
les observations de mouvements.

Il est important de se placer dans un contexte proches des applications
réelles. Ainsi, l'analyse des mouvements est précédée par une phase d'acquisition.
Dans nos travaux, l'acquisition des mouvements s'effectue 
par le biais d'un système de capture de mouvements optique.
Le chapitre~\ref{chap:imitation} présente les différentes 
technologies permettant d'enregistrer un mouvement, le traitement
des données collectées ainsi que des exemples d'applications
orientées vers la fonction de t\^aches.

Les techniques présentées dans ces deux chapitres sont exploitées
dans le chapitre~\ref{chap:reco} qui représente
le c\oe ur de la contribution apportée dans ces travaux:
une méthode de reconnaissance de t\^aches pour un robot humanoïde,
capable de reconnaître des t\^aches exécutées en parallèles et de manière précise.
Les différentes expériences en simulation et sur
un vrai robot validant la méthode y sont détaillées.
