 
\markboth{}{}
\pagestyle{empty}

{\small
\noindent\begin{normalsize}\textbf{Reconnaissance de t\^aches par commande inverse}\end{normalsize}\\

Des méthodes efficaces s'appuyant sur des outils statistiques pour réaliser de la reconnaissance de mouvement ont été développé. Ces méthodes reposent sur l'apprentissage de primitives situé dans des espaces approprié, par exemple l'espace latent de l'espace articulaire et/ou d'espace de tâches
adéquat. Les primitives apprises sont souvent séquentielle: un mouvement est segmenté selon l'axe des temps. Dans le cas d'un robot humanoïde, le mouvement peut être décomposé en plusieurs sous-tâches
simultanées. Par exemple dans un scénario de serveur, le robot doit placer une assiette sur la table avec une main tout en maintenant son plateau horizontal avec son autre main. La reconnaissance ne peut donc pas se limiter à une seule et unique tâche par segment de temps consécutif. La méthode présenté dans ces travaux utilise la connaissance des tâches que le robot est capable d'accomplir, ainsi que des contrôleurs qui génèreront les mouvements  pour réaliser une rétro ingénierie sur un mouvement observé. Cette analyse est destiné à reconnaître des tâches qui ont été exécuté de manière simultanées. La méthode repose sur la fonction de tâche et les projection dans l'espace nul des tâches afin de découpler les contrôleurs. L'approche a été appliqué avec succès sur un vrai robot pour distinguer des mouvements visuellement très proches, mais sémantiquement différents. 

\medskip
\noindent 

\textbf{Mots-clefs~:} Analyse de mouvements, pile de t\^aches, robotique

\vspace{0.5cm}

\noindent\begin{normalsize}\textbf{Task recognition by reverse control}\end{normalsize}\\

Efficient methods to perform motion recognition have been developed using statistical tools. Those methods rely on primitives learning in a suitable space, for example the latent space of the joint angle and/or adequate task spaces. The learned primitives are often sequential : a motion is segmented
according to the time axis. When working with a humanoid robot, a motion can be decomposed into simultaneous sub-tasks. For example in a waiter scenario, the robot has to keep some plates horizontal with one of his arms, while placing a plate on the table with its free hand. Recognition can thus not be
limited to one task per consecutive segment of time. The method presented in this work takes advantage of the knowledge of what tasks the robot is able to do and how the motion is generated from this set of known controllers to perform a reverse engineering of an observed motion. This analysis is intended to recognize simultaneous tasks that have been used to generate a motion. The method relies on the task-function formalism and the projection operation into the null space of a task to decouple the controllers. The approach is successfully applied on a real robot to disambiguate motion in different
scenarios where two motions look similar but have different purposes.

\medskip
\noindent 
\textbf{Keywords ~:} Motion analysis, stack of tasks, robotics

}
