%\chapter{Introduction}
%Le d\'eveloppement de la robotique de service stimule la recherche
%dans l'int\'eraction homme-robot. Dans ce contexte, la compr\'ehension
%des actions du robot et de l'être humain \`a partir d'une observation est un d\'efi.
%Bien que les actions volontaires apparaissent au niveau de la planification,
%leurs r\'ealisations s'effectuent dans le monde r\'eel par le biais 
%des mouvements. Comment reconna\^itre une action \`a partir
%d'un mouvement observ\'e? D\'efinir une m\'ethode pour reconna\^itre automatiquement
%l'objectif vis\'e par un robot ex\'ecutant un mouvement donn\'e est un probl\`eme 
%crucial. Les robots manipulateurs mobiles,
%sont construits de façon \`a \'etablir une s\'eparation claire entre les fonctions de navigation et 
%les fonctions de manipulation. Par exemple, le robot PR2 (voir Fig.~\ref{fig:pr2})
%possède des bras uniquement dédié à la manipulation
%et une base mobile uniquement dédié à la navigation. Par cons\'equent, la question de la reconnaissance
%d'actions peut \^etre simple. De ce point de vue, un robot humano\"ide
%peut aussi \^etre divis\'e en deux parties distinctives: les jambes
%et le haut du corps, qui correspondraient aux fonctions de navigation
%et aux fonctions de manipulation.

%\begin{figure}[t]
%  \begin{center}
%    \includegraphics[width=0.4\linewidth]{figures/chapitre5/pr2.ps}
%  \end{center}
%  \caption{Le robot PR2 est construit de façon à séparer les fonctions de navigation
%  et les fonctions de manipulation.}
%  \label{fig:pr2}
%\end{figure}

%\subsection{Problem statement: disambiguating motions in embodied actions.}
%\begin{figure*}[t]
%  \centering
%  \makeatletter
%  \renewcommand{\@thesubfigure}{Scenario \thesubfigure:\hskip\subfiglabelskip}
%  \makeatother

%  \subfigure[La t\^ache globale \emph{Donne moi la balle} est d\'ecompos\'e en
%  s\'equences de sous-t\^aches \lbrack localiser la balle\rbrack, \lbrack marcher vers la balle\rbrack, 
%  \lbrack attraper la balle\rbrack, \lbrack localiser l'op\'erateur\rbrack, \lbrack marcher vers l'op\'erateur\rbrack, 
%  et \lbrack donner la balle\rbrack. Les mouvements \lbrack marcher vers\rbrack, \lbrack attraper\rbrack, 
%  \lbrack donner\rbrack~apparaissent comme une s\'equence structurant l'action (extrait de~\cite{yoshida07}).]{
%  \makebox[\linewidth]{
%  \begin{tabular}{c@{}c@{}c@{}c@{}c@{}}
%    \includegraphics[height=2.4cm]{figures/chapitre5/purpleBall1.ps}&
%    \includegraphics[height=2.4cm]{figures/chapitre5/purpleBall2.ps}&
%    \includegraphics[height=2.4cm]{figures/chapitre5/purpleBall3.ps}&
%    \includegraphics[height=2.4cm]{figures/chapitre5/purpleBall4.ps}&
%    \includegraphics[height=2.4cm]{figures/chapitre5/purpleBall5.ps}\\
%  \end{tabular}
%  \label{fig:introExample:purpleBall}
%  }
%  }
%  \subfigure[Pour attraper la balle situ\'ee entre ses pieds, le robot doit s'en \'eloigner en faisant quelques pas
%  en arri\`ere. Dans cette exp\'erience, la partie \emph{reculer} n'est pas g\'er\'e par un module de logiciel.
%  Elle fait partie int\'egrale de l'action complexe \emph{attraper} (extrait de~\cite{kanoun10}).]{
%  \makebox[\linewidth]{
%  \begin{tabular}{c@{}c@{}}
%    \includegraphics[height=1.7cm]{figures/chapitre5/graspFeet1.ps}&
%    \includegraphics[height=1.7cm]{figures/chapitre5/graspFeet2.ps}\\
%  \end{tabular}
%  \label{fig:introExample:graspFeet}
%  }
%  }
%  \subfigure[Pour attraper la balle en face de lui (\`a gauche), le robot
%  atteint une posture dans laquelle le bras gauche est utilis\'e pour garder son \'equilibre. 
%  Dans la figure de droite, le robot effectue deux actions en parall\`ele:
%  attraper la balle devant lui tout en attrapant une autre balle se trouvant derri\`ere lui 
%  (bien sûr, la balle qui se trouve derri\`ere le robot a \'et\'e volontairement plac\'e \`a la position 
%  finale de la main gauche illustré dans la figure de gauche). Il est 
%  impossible de diff\'erencier visuellement les deux postures. Cependant,
%  est-il possible de discriminer ces deux mouvements?]{
%  \makebox[\linewidth]{
%  \includegraphics[height=2.4cm]{figures/chapitre5/spotDiff1H.ps}
%  }
%  \label{fig:introExample:spotDiff}
%  }
%  \caption{Exemple d'introduction d'actions complexes.}
%  \label{fig:introExample}
%\end{figure*}
%Cette séparation est illustrée par exemple dans le sc\'enario \emph{Donne moi la balle rose}~\cite{yoshida07},
%effectu\'e par le robot humano\"ide HRP-2 au LAAS-CNRS (Fig.~\ref{fig:introExample:purpleBall}). 
%Pour atteindre son objectif, le robot HRP-2 d\'ecompose sa mission en sous-t\^aches \'el\'ementaires,
%chaque mission est prise en charge par un module logiciel d\'edi\'e. 
%Pour atteindre la balle, le robot doit marcher vers celle-ci.
%\emph{Marcher} s'apparente \`a une action \'el\'ementaire qui est n\'ecessaire dans la 
%r\'esolution du probl\`eme initial (\emph{Donne moi la balle rose}).
%elle est prise en charge par un module de locomotion d\'edi\'e.
%
%Cependant, dans le second sc\'enario (Fig.~\ref{fig:introExample:graspFeet}),
%le HRP-2 doit attraper la balle situ\'e entre ses pieds~\cite{kanoun10}. Pour atteindre
%cet objectif, le robot doit faire quelques pas en arri\`ere avant d'attraper la balle. Dans
%cette exp\'erience, il n'y a pas de module d\'edi\'e \`a la marche. L'action \emph{reculer}, est une
%cons\'equence directe de l'action \emph{attraper}. Cette action est int\'egr\'ee au mouvement 
%du corps, par exemple les jambes contribuent naturellement \`a la r\'ealisation de l'action
%via une g\'en\'eration d'un mouvement complexe. Enfin, la Fig.~\ref{fig:introExample:spotDiff}
%introduit le but de ce chapitre. Dans le cas illustr\'e dans la figure de gauche, le robot
%effectue un mouvement de simple prise (une seule main). Dans le cas illustr\'e dans la figure de 
%droite, le robot effectue deux prises en parall\`ele. L'ambig\"uit\'e de ces mouvements provient
%du r\^ole jou\'e par la main gauche. Dans le premier cas, la main gauche contribue
%\`a la t\^ache de saisie en compensant le d\'es\'equilibre introduit par le d\'eplacement
%de la main droite. Dans le second cas, le bras gauche bouge pour saisir une autre balle. 
%Dans les deux cas, les mouvements sont visuellement proches.

%%%%%%%%%%%%%%%%%%%%%%%%%%%%%%%%%%%%%%%%%%%%%%%%%%%
Le développement de la robotique de service est étroitement liée
à une grande variété de problèmes scientifiques. L'objectif d'une grande partie de ce domaine
de recherche est que l'intégration des robots dans l'environnement 
humain soit suffisamment fiable et efficace pour être utile.
Les exemples de problèmes à 
résoudre concernent la navigation et la perception de l'environnement.
C'est cette classe de problème qui a été tout d'abord résolue et des solutions ont été appliquées avec succès 
sous l'incarnation de robots aspirateurs, nettoyeurs de piscines ou 
robots explorateurs.

À ces fonctionnalités de navigations, des fonctionnalités
de manipulations peuvent être ajoutées.
Les robots mobiles manipulateurs tels 
que le PR2~\cite{PR2} sont équipés de bras 
destinés aux actions de manipulations d'objets situés dans l'environnement
(voir figure~\ref{fig:pr2}).
\begin{figure}[t]
  \begin{center}
    \includegraphics[width=0.3\linewidth]{figures/chapitre5/pr2.ps}
  \end{center}
  \caption{Le robot PR2 de Willow garage.}
  \label{fig:pr2}
\end{figure}
Ces robots sont construits de manière à découpler les fonctionnalités de navigation
et de manipulation, il est ainsi
possible d'associer directement des espaces aux t\^aches à effectuer.
Par exemple, pour la navigation, les roues du robot vont évoluer dans l'espace défini par le plan
du sol. Tandis que pour la manipulation, les actions du bras vont être exprimées
dans l'espace cartésien pour asservir la position de l'effecteur à une position désirée.
Les deux domaines des actions sont bien découplés, et des opérations de 
programmation, debuggage, diagnostique et reconnaissance peuvent alors être traitées facilement en effectuant
des observations dans ces deux espaces.

La même séparation de fonctionnalités peut aussi s'appliquer
aux robots humanoïdes en considérant que les jambes 
remplissent la fonction de navigation et que le haut du corps
s'occupe de la manipulation. La navigation se traduit
alors par une planification d'empreintes de pas que le robot doit suivre dans l'espace à deux
dimensions représentant le sol pour atteindre un objectif. À la manière des robots manipulateurs,
les actions de manipulation seront exprimées dans l'espace cartésien.
Cette approche est illustrée dans le scénario \emph{Donne moi la balle rose}~\cite{yoshida07},
effectué par le robot humanoïde HRP-2 au LAAS-CNRS.
La mission du robot est décomposée en sous-missions qui sont gérées par des modules de logiciels
indépendants: localiser la balle, marcher vers la balle, attraper la balle,
localiser l'opérateur, marcher vers l'opérateur et donner la balle (voir la figure~\ref{fig:purpleBall}).
\begin{figure}[t]
  \centering
  %\begin{tabular}{c@{}c@{}c@{}c@{}c@{}}
  \subfigure{
  \includegraphics[width=0.3\linewidth]{figures/chapitre5/purpleBall1.ps}
  }
  \subfigure{
  \includegraphics[width=0.3\linewidth]{figures/chapitre5/purpleBall2.ps}
  }
  \subfigure{
  \includegraphics[width=0.3\linewidth]{figures/chapitre5/purpleBall3.ps}
  }
  \subfigure{
  \includegraphics[width=0.3\linewidth]{figures/chapitre5/purpleBall4.ps}
  }
  \subfigure{
  \includegraphics[width=0.3\linewidth]{figures/chapitre5/purpleBall5.ps}
  }
  %\end{tabular}
  \caption[\emph{Donne moi la balle}.]{La mission globale \lbrack Donne moi la balle\rbrack est décomposée en
  séquences de sous-missions \lbrack localiser la balle\rbrack, \lbrack marcher vers la balle\rbrack, 
  \lbrack attraper la balle\rbrack, \lbrack localiser l'opérateur\rbrack, \lbrack marcher vers l'opérateur\rbrack, 
  et \lbrack donner la balle\rbrack. Les mouvements \lbrack marcher vers\rbrack, \lbrack attraper\rbrack, 
  \lbrack donner\rbrack~apparaissent comme une séquence structurant l'action (extrait de~\cite{yoshida07}).}
  \label{fig:purpleBall}
\end{figure}

D'une manière générale, le découpage en sous parties fonctionnelles n'est pas toujours
possible à cause de la complexité du robot. C'est le cas par exemple
d'un robot humanoïde où des couplages entre les différents
membre, qui varient en fonction de l'action à réaliser, sont trop nombreux.
Le scénario illustré par la figure~\ref{fig:graspFeet} ressemble beaucoup
au scénario précédent cependant, la méthodologie appliquée pour accomplir la mission est très différente.
Ici, la mission du robot est d'attraper une balle se situant entre ses pieds. Pour
accomplir la mission, le robot doit reculer puis tendre la main vers la balle.
La mission est accomplie en considérant que les pas à effectuer
représentent une extension virtuelle du robot, ramenant la mission à un
problème de cinématique inverse avec contraintes~\cite{kanoun10}.
La marche apparaît alors de façon implicite dans 
l'action \emph{asservir la position de la main vers la position de la balle} qui décrit la mission. 
\begin{figure}
  \centering
  \subfigure{
  \includegraphics[width=0.90\linewidth]{figures/chapitre5/graspFeet1.ps}
    }
  \subfigure{
    \includegraphics[width=0.90\linewidth]{figures/chapitre5/graspFeet2.ps}
    }
  \caption[Planification de pas formulé en problème de cinématique inverse.]{Pour attraper la balle située entre ses pieds, le robot doit s'en éloigner en faisant quelques pas
  en arrière. Dans cette expérience, la partie \emph{reculer} n'est pas gérée par un module de logiciel.
  Elle fait partie intégrale de l'action complexe \emph{attraper la balle} (extrait de~\cite{kanoun10}).}
  \label{fig:graspFeet}
\end{figure}

\section{Problématique}
Dans ce manuscrit, nous cherchons à reconnaître les composantes
du mouvement en cours lorsque le découpage fonctionnel n'est pas possible.
Nous nous intéressons plus particulièrement aux mouvements faisant intervenir
plusieurs objectifs à atteindre.
Dans ce cadre, nous cherchons à montrer que la fonction de t\^aches,
classiquement utilisée pour décrire un mouvement à générer, peut
être utilisée comme une généralisation du découpage fonctionnel, 
pour reconnaître de manière explicite des parties couplées du mouvement.

La figure~\ref{fig:spotDiff} illustre la problématique des travaux présentés dans ce manuscrit.
Dans l'image de gauche, le robot effectue un mouvement d'atteinte avec sa main droite.
Dans l'image de droite, le robot effectue deux mouvements d'atteintes en parallèle.
Les deux mouvements sont visuellement très proches. Pourtant le rôle
de la main gauche est discriminant. Dans le premier cas, la main gauche 
est couplée au mouvement de la main droite pour compenser
le déséquilibre introduit par le déplacement de la main droite.
Dans le second cas, la main gauche agit de manière indépendante pour accomplir l'action
d'atteinte. Le problème soulevé est donc de trouver le moyen
de différencier les deux mouvements en les observant. Nous montrons
qu'en effectuant les observations dans les \emph{bons} espaces,
il est possible d'effectuer une reconnaissance 
de mouvements capable de désambiguïser ces mouvements visuellement proches.
\begin{figure}
  \centering
  \subfigure{
  \includegraphics[trim=220px 10px 220px 0px, width=0.48\linewidth, clip=true]{figures/chapitre1/Lonly.ps}
  }
  \subfigure{
  \includegraphics[trim=220px 10px 220px 0px, width=0.48\linewidth, clip=true]{figures/chapitre1/RL.ps}
  }
  \caption[Comment désambiguïser des mouvements proches?]{Pour attraper la balle en face de lui (\`a gauche), le robot
  atteint une posture dans laquelle le bras gauche est utilisé pour garder son équilibre. 
  Dans la figure de droite, le robot effectue deux actions en parallèle:
  attraper la balle devant lui tout en attrapant une autre balle se trouvant derrière lui 
  (bien sûr, la balle qui se trouve derrière le robot a été volontairement placée à la position 
  finale de la main gauche illustrée dans la figure de gauche). Il est 
  impossible de différencier visuellement les deux postures. Cependant,
  est-il possible de discriminer ces deux mouvements?}
  \label{fig:spotDiff}
\end{figure}

Dans la suite de cette section, nous présentons différentes
approches considérées dans le domaine de l'imitation et la reproduction
de mouvements. Les travaux en imitation de mouvements sont 
reliés aux nôtres dans la mesure où le problème de l'imitation 
consiste à détecter les caractéristiques importantes ou communes d'un ou
plusieurs mouvements observés afin de pouvoir les reproduire ou les généraliser.

