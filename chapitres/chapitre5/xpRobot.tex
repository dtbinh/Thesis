%-------------------------------------------------------------------%
%-------------------------------------------------------------------%
%\section{Exp\'erimentations: Reconnaissance de mouvements effectués par le robot HRP-2}
\label{chap:xpRobot}
%-------------------------------------------------------------------%
%-------------------------------------------------------------------%

Pour valider la m\'ethode présentée plus haut, plusieurs exp\'erimentations
ont \'et\'e effectu\'ees sur le robot HRP-2 en simulation
et sur le vrai robot. Ces exp\'erimentations consistent 
\`a discriminer des paires de mouvements visuellement proches, mais 
faisant intervenir des t\^aches diff\'erentes: les s\'emantiques
de ces mouvements sont diff\'erentes.

\section{Résultats en simulation: reconnaissance de t\^aches effectuées par le robot HRP-2}
\label{sec:simu}
Ce paragraphe détaille une série d'expérimentations effectuées en simulation pour 
valider l'algorithme de reconnaissance.
La simulation nous permet d'observer le comportement nominal de l'algorithme
en absence de bruits issus de capteurs.
La première expérimentation valide simplement l'étape de projection 
du mouvement~(paragraphe \ref{sec:prelimValid}). 
%Fig.~\ref{fig:snapshotXpqdot} and Fig.~\ref{fig:xp3Pqdot} illustrate the effect of the projections.
La seconde partie réunit un ensemble d'expérimentations qui valide l'algorithme de reconnaissance 
de t\^aches~(paragraphe \ref{sec:distinc}).
%Fig.~\ref{fig:XP2RFit} and Fig.~\ref{fig:XP2RLFit} shows how the fitting behaves. 
%Fig.~\ref{fig:RbeforeAfterProj} shows how a projected trajectory evolves after other projections.
Pour chaque expérimentation de cet ensemble, l'algorithme de reconnaissance de t\^aches
est appliqué à deux mouvement visuellement proches:
ces deux mouvements ont été artificiellement construits pour présenter une
ambigu\"ité visuelle lorsqu'on les compare entre eux. La présence de cette
ambiguïté permet d'illustrer l'efficacité de notre algorithme
vis à vis de la précision de la reconnaissance.
Tous les mouvements qui ont servis aux expérimentations sont résumés dans le tableau~\ref{tab:motion}.
La description des t\^aches utilisées pour construire ces mouvements est détaillée ci-dessous.

\subsection{Protocole d'expérimentations}
Les mouvements de référence ont été générés en utilisant le modèle du robot humanoïde
HRP-2. Celui-ci présente 30 degrés de liberté actionnés, plus six degrés de liberté 
pour la base flottante.
Tous les mouvements démarrent de la configuration de repos \emph{half-sitting} (comme illustrée
dans la figure~\ref{fig:halfSit}),
et le centre de masse est situé à zéro dans le repère du bassin lorsque 
celui-ci est la racine de l'arbre cinématique. Cette configuration a pour particularité de ne pas présenter
de singularité.
En cinématique inverse, la position de la base flottante sous-actuée est résolue en contraignant le pied gauche
à rester sur le sol.
Les figure~\ref{fig:spotDiff1},~\ref{fig:spotDiff2} et~\ref{fig:spotDiff3} montrent les postures finales
des mouvements utilisées dans les expérimentations.
\begin{figure}[t]
\begin{center}
\includegraphics[width=0.3\linewidth]{figures/chapitre5/halfSit.ps}
\end{center}
\caption[Pose \emph{half-sitting}.]{Tout les mouvements de référence démarrent à partir de la posture de repos \emph{half-sitting}.}
\label{fig:halfSit}
\end{figure}
\begin{figure}[p]
  \begin{center}
    \includegraphics[width=0.7\linewidth]{figures/chapitre5/spotDiff.ps}
  \end{center}
  \caption[Mouvements d'atteintes.]{Gauche: La posture finale du \emph{mouvement~2.a}; Droite: La posture finale du \emph{mouvement~2.b}.}
  \label{fig:spotDiff1}
\end{figure}
\begin{figure}[p]
\begin{center}
\includegraphics[width=0.50\linewidth]{figures/chapitre5/simu/3/final3.ps}
\end{center}
\caption[Mouvements prise et prise orientée.]{Gauche: La posture finale du \emph{mouvement~3.a}; Droite: La posture finale du \emph{mouvement~3.b}.
La différence entre ces deux mouvements est difficile à percevoir: l'orientation de la main droite
est contrainte à être parallèle au sol dans le \emph{mouvement~3.b}.}
\label{fig:spotDiff2}
%\vspace{-3pt}
\end{figure}
\begin{figure}[p]
\begin{center}
\includegraphics[width=0.50\linewidth]{figures/chapitre5/spotDiff3.ps}
\end{center}
\caption[Mouvements de prise orientée et regard.]{Gauche: La posture finale du \emph{mouvement~4.a}; Droite: La posture finale du \emph{mouvement~4.b}.}
\label{fig:spotDiff3}
\end{figure}

L'ensemble des t\^aches considérées dans les expérimentations sur HRP-2 est : 

\begin{itemize}
  \item \emph{Com} : t\^ache~\eqref{eq:taskCom}, le centre de masse du robot est contraint pour maintenir un équilibre statique (3 DDL)
  \item \emph{Regard} : t\^ache~\eqref{eq:visError}, le robot regarde un point dans l'espace Cartésien, le point de contrôle se situe
à 25cm au dessus du centre de la dernière articulation du cou (2 DDL)
  \item \emph{Double support} : t\^ache~\eqref{eq:taskRelative}, la transformation entre le repère attaché au pied droit et le repère 
  attaché au pied gauche du robot doit être constante (6 DDL)
  \item \emph{Prise gauche/droite} : t\^ache~\eqref{eq:taskPosition}, un repère attaché à la main gauche ou droite du robot atteint un point défini dans l'espace Cartésien, les points de contrôle se situent aux centres des articulations terminales des bras (3 DDL)
  \item \emph{Prise orientée} : t\^ache~\eqref{eq:feature6dE}, similaire à la t\^ache de \emph{prise}, mais la position
    désirée doit \^etre atteinte avec une orientation de la main définie (6 DDL)
  \item \emph{Tête} : t\^ache~\eqref{eq:feature6dE}, un repère attaché à la t\^ete du robot est contrainte en position et en orientation, le point de contrôle se situe au centre de la dernière articulation de la tête (6 DDL) 
  \item \emph{Torse} : t\^ache~\eqref{eq:taskRotation}, un repère attaché au torse du robot est contraint en orientation, le point de contrôle se situe au centre de la dernière articulation associée au torse (3 DDL)
\end{itemize}
\begin{table}[h]
  \centering
    \begin{tabular}{|c|c|c|}
      \hline
      & (a) & (b) \\
      \hline
      & &
      \\
      Mouvement 1 & 
	\begin{tabular}{|c|}
          \hline
          \emph{Prise droite}\\
          \hline
          \emph{Prise gauche}\\
          \hline
          \emph{Regard}\\
	  \hline
	  \emph{Com}\\
	  \hline
	  \emph{Double support}\\
	  \hline
	\end{tabular} & \\
      & &
      \\
      \hline
      & &
      \\
      Mouvement 2 & 
        \begin{tabular}{|c|}
          \hline
          \emph{Prise droite}\\
	  \hline
	  \emph{Com}\\
	  \hline
	  \emph{Double support}\\
	  \hline
	\end{tabular}  
    &
	\begin{tabular}{|c|}
	  \hline
	  \emph{Prise gauche}\\
	  \hline
	  \emph{Prise droite}\\
	  \hline
	  \emph{Com}\\
	  \hline
	  \emph{Double support}\\
	  \hline
	\end{tabular}  
    \\
    & &
    \\
    \hline
    & &
    \\
    Mouvement 3 & 
        \begin{tabular}{|c|}
	  \hline
	  \emph{Regard}\\
	  \hline
	  \emph{Prise droite}\\
	  \hline
	  \emph{Com}\\
	  \hline
	  \emph{Double support}\\
          \hline
        \end{tabular}  
    &
	\begin{tabular}{|c|}
	  \hline
	  \emph{Regard}\\
	  \hline
	  \emph{Prise orientée droite}\\
	  \hline
	  \emph{Com}\\
	  \hline
	  \emph{Double support}\\
	  \hline
	\end{tabular}  
    \\
    & &
    \\
    \hline
    & &
    \\
    Mouvement 4 & 
	\begin{tabular}{|c|}
	  \hline
	  \emph{Regard}\\
	  \hline
	  \emph{Com}\\
	  \hline
	  \emph{Double support}\\
	  \hline
	\end{tabular}  
    &
	\begin{tabular}{|c|}
	  \hline
	  \emph{Prise orienté gauche}\\
	  \hline
	  \emph{Com}\\
	  \hline
	  \emph{Double support}\\
	  \hline
	\end{tabular} 
      \\
    & &
    \\
    \hline
    & &
    \\
    Mouvement 5 &
    \begin{tabular}{|c|}
      \hline
      \emph{Prise droite}\\
      \hline
      \emph{Regard}\\
      \hline
      \emph{Com}\\
      \hline
      \emph{Double support}\\
      \hline
    \end{tabular}
    &
    \begin{tabular}{|c|}
      \hline
      \emph{Prise droite}\\
      \hline
      \emph{Torse}\\
      \hline
      \emph{Com}\\
      \hline
      \emph{Double support}\\
      \hline
    \end{tabular}
    \\
    & &
    \\
    \hline
  \end{tabular}
  \caption{Tableau des mouvements et des t\^aches considérés.}
  \label{tab:motion}
\end{table}

\FloatBarrier
\subsection{Expérimentation 1 : Validation préliminaire}
\label{sec:prelimValid}
Dans cette expérimentation, un mouvement de base est généré.
La reconnaissance de t\^ache par ajustement et le critère d'arrêt de l'algorithme sont validés ici.
Le mouvement de référence est le \emph{mouvement~1.a} (voir le tableau~\ref{tab:motion}).
Le mouvement est donné au programme de détection qui va sélectionner
les t\^aches qui correspondent le mieux par optimisation.

La figure~\ref{fig:snapshotXpqdot} montre des vignettes du mouvement original
et du mouvement résultant des projections successives dans les espaces
nuls des t\^aches détectées:
\emph{prise droite}, \emph{com}, \emph{regard}, \emph{double support} et \emph{prise gauche}.
Le mouvement initial est visualisé en intégrant un champs de vecteurs. Les
projections dans les espaces nuls vont modifier ces vecteurs que l'on intègre pour visualiser
les mouvements projetés. Ces mouvements projetés n'ont donc pas 
de sens physique et seule la quantité de mouvements reste valide.
Ces quantités de mouvements permettent d'avoir une bonne
intuition sur les effets des projections dans les espaces nuls.
Chaque projection annule une partie du mouvement, et le mouvement du robot devient nul quand 
toutes les projections sont appliquées, ce qui signifie que toutes les t\^aches
intervenant dans le mouvement ont été détectées.
\begin{figure*}[p]
\centering
\begin{tabular}{c@{}c@{}c@{}c@{}c@{}c@{}c}
(a)&
\parbox[c]{2.2cm}{\includegraphics[width=\linewidth]{figures/chapitre5/Pqdot0_0.png.ps}} &
\parbox[c]{2.2cm}{\includegraphics[width=\linewidth]{figures/chapitre5/Pqdot0_99.png.ps}} &
\parbox[c]{2.2cm}{\includegraphics[width=\linewidth]{figures/chapitre5/Pqdot0_199.png.ps}} &
\parbox[c]{2.2cm}{\includegraphics[width=\linewidth]{figures/chapitre5/Pqdot0_299.png.ps}} &
\parbox[c]{2.2cm}{\includegraphics[width=\linewidth]{figures/chapitre5/Pqdot0_399.png.ps}} &
\parbox[c]{2.2cm}{\includegraphics[width=\linewidth]{figures/chapitre5/Pqdot0_499.png.ps}}\\

(b)&
\parbox[c]{2.2cm}{\includegraphics[width=\linewidth]{figures/chapitre5/Pqdot1_0.png.ps}} &
\parbox[c]{2.2cm}{\includegraphics[width=\linewidth]{figures/chapitre5/Pqdot1_99.png.ps}} &
\parbox[c]{2.2cm}{\includegraphics[width=\linewidth]{figures/chapitre5/Pqdot1_199.png.ps}} &
\parbox[c]{2.2cm}{\includegraphics[width=\linewidth]{figures/chapitre5/Pqdot1_299.png.ps}} &
\parbox[c]{2.2cm}{\includegraphics[width=\linewidth]{figures/chapitre5/Pqdot1_399.png.ps}} &
\parbox[c]{2.2cm}{\includegraphics[width=\linewidth]{figures/chapitre5/Pqdot1_499.png.ps}}\\

(c)&
\parbox[c]{2.2cm}{\includegraphics[width=\linewidth]{figures/chapitre5/Pqdot2_0.png.ps}} &
\parbox[c]{2.2cm}{\includegraphics[width=\linewidth]{figures/chapitre5/Pqdot2_99.png.ps}} &
\parbox[c]{2.2cm}{\includegraphics[width=\linewidth]{figures/chapitre5/Pqdot2_199.png.ps}} &
\parbox[c]{2.2cm}{\includegraphics[width=\linewidth]{figures/chapitre5/Pqdot2_299.png.ps}} &
\parbox[c]{2.2cm}{\includegraphics[width=\linewidth]{figures/chapitre5/Pqdot2_399.png.ps}} &
\parbox[c]{2.2cm}{\includegraphics[width=\linewidth]{figures/chapitre5/Pqdot2_499.png.ps}}\\

(d)&
\parbox[c]{2.2cm}{\includegraphics[width=\linewidth]{figures/chapitre5/Pqdot3_0.png.ps}} &
\parbox[c]{2.2cm}{\includegraphics[width=\linewidth]{figures/chapitre5/Pqdot3_99.png.ps}} &
\parbox[c]{2.2cm}{\includegraphics[width=\linewidth]{figures/chapitre5/Pqdot3_199.png.ps}} &
\parbox[c]{2.2cm}{\includegraphics[width=\linewidth]{figures/chapitre5/Pqdot3_299.png.ps}} &
\parbox[c]{2.2cm}{\includegraphics[width=\linewidth]{figures/chapitre5/Pqdot3_399.png.ps}} &
\parbox[c]{2.2cm}{\includegraphics[width=\linewidth]{figures/chapitre5/Pqdot3_499.png.ps}}\\

(e)&
\parbox[c]{2.2cm}{\includegraphics[width=\linewidth]{figures/chapitre5/Pqdot4_0.png.ps}} &
\parbox[c]{2.2cm}{\includegraphics[width=\linewidth]{figures/chapitre5/Pqdot4_99.png.ps}} &
\parbox[c]{2.2cm}{\includegraphics[width=\linewidth]{figures/chapitre5/Pqdot4_199.png.ps}} &
\parbox[c]{2.2cm}{\includegraphics[width=\linewidth]{figures/chapitre5/Pqdot4_299.png.ps}} &
\parbox[c]{2.2cm}{\includegraphics[width=\linewidth]{figures/chapitre5/Pqdot4_399.png.ps}} &
\parbox[c]{2.2cm}{\includegraphics[width=\linewidth]{figures/chapitre5/Pqdot4_499.png.ps}}\\

(f)&
\parbox[c]{2.2cm}{\includegraphics[width=\linewidth]{figures/chapitre5/Pqdot5_0.png.ps}} &
\parbox[c]{2.2cm}{\includegraphics[width=\linewidth]{figures/chapitre5/Pqdot5_99.png.ps}} &
\parbox[c]{2.2cm}{\includegraphics[width=\linewidth]{figures/chapitre5/Pqdot5_199.png.ps}} &
\parbox[c]{2.2cm}{\includegraphics[width=\linewidth]{figures/chapitre5/Pqdot5_299.png.ps}} &
\parbox[c]{2.2cm}{\includegraphics[width=\linewidth]{figures/chapitre5/Pqdot5_399.png.ps}} &
\parbox[c]{2.2cm}{\includegraphics[width=\linewidth]{figures/chapitre5/Pqdot5_499.png.ps}}\\
\\
Itération & 0 & 99 & 199 & 299 & 399 & 499\\
\end{tabular}
\caption[Illustrations des projections successives.]{Le mouvement généré par une pile de t\^aches contenant :
\emph{Prise droite}, \emph{Prise gauche}, \emph{Com}, \emph{Regard}, \emph{Double support} 
est représenté dans la ligne (a).
Les autres lignes représentent les projections successives du mouvement dans les espaces
nuls des t\^aches: (b) Prise droite, (c) Com, (d) Regard, (e) Double support, (f) Prise gauche.}
\label{fig:snapshotXpqdot}
\end{figure*}
Comme nous pouvons le constater dans la seconde ligne, le mouvement de la main droite est annulé.
L'annulation de \emph{com} à partir de la troisième ligne est plus difficile à percevoir.
On peut noter la modification du mouvement des jambes. Cette modification correspond au fait que pour
exécuter la t\^ache \emph{Com}, le robot utilise ses jambes.
À la quatrième ligne, l'annulation du mouvement de la t\^ete est très claire.
On note que le mouvement du torse est altéré. Le mouvement du torse est donc justifié par
la t\^ache de regard.
À la cinquième
ligne, l'annulation de la t\^ache \emph{double support} supprime la compensation
faite avec le pied droit. Enfin, comme on peut le voir dans la dernière ligne,
l'annulation de toutes les t\^aches mène à un mouvement nul.
Tous les mouvements du robot ont donc été justifiés.
La figure~\ref{fig:xp3Pqdot} montre l'évolution de la norme du mouvement,
définie par la somme des normes au carré sur le temps de la trajectoire articulaire du robot.
\begin{figure}[t]
\begin{center}
%\includegraphics[height=0.9\linewidth, angle = -90]{figures/chapitre5/PqdotNorms.ps}
\resizebox{.48\textwidth}{!} {
      \input{figures/chapitre5/PqdotNormsProjection.tex}
    }
\end{center}
\caption[Évolution de la norme du mouvement.]{L'évolution de la norme du mouvement après avoir projeté successivement le mouvement dans les
	espaces nuls des t\^aches.}
\label{fig:xp3Pqdot}
\end{figure}
Chaque projection fait strictement décroître la norme de la vitesse
(\ie~la quantité de mouvement). Après cinq projections, le mouvement est complètement
annulé, ce qui confirme que toutes les t\^aches actives ont été découvertes. L'algorithme
s'arrête donc.
\FloatBarrier

\subsection{Expérimentation 2 : Distinction entre deux mouvements proches}
\label{sec:distinc}
En ce qui concerne la détection de mouvement, \^etre capable
de différencier deux mouvements visuellement proches faisant tout de même
intervenir des t\^aches différentes est un problème intéressant.
Les algorithmes pour les systèmes anthropomorphes utilisent plut\^ot le contexte
pour la desambigüisation.
Le travail présenté ici montre que le critère d'ajustement mêlé à la représentation
du mouvement sous forme de pile de t\^aches est suffisant pour distinguer des mouvements visuellement
proches. Trois paires de mouvements ambigus vont \^etre présentées pour illustrer 
la capacité de distinction de mouvement de notre méthode.
Le modèle de comportement de t\^ache choisi est une décroissance exponentielle dont 
les paramètres seront les paramètres de l'optimisation du problème~\ref{optimProblem}.


\subsubsection{Mouvements d'atteintes}
\label{sec:distinc1}
Dans ce paragraphe, deux mouvements sont considérés: 
\emph{mouvement~2.a} et \emph{mouvement~2.b}.
Le premier est un mouvement d'atteinte lointain avec la main droite.
Ce mouvement d'atteinte de la main droite a une influence sur la main gauche par
le biais de la t\^ache \emph{com}: pour retrouver son équilibre,
le robot met sa main gauche en arrière.
Le second mouvement est la même t\^ache d'atteinte de la main droite, mais
une seconde t\^ache est ajoutée. Il s'agit d'une t\^ache d'atteinte de main gauche.
La position désirée pour cette dernière t\^ache a été définie artificiellement comme 
étant la position finale de la main gauche obtenue lors du premier mouvement.

Les état finaux du robot pour les deux mouvements sont illustrés dans la figure~\ref{fig:spotDiffbis}.
\begin{figure}[t]
  \centering
  \subfigure{
  \includegraphics[trim=220px 10px 220px 0px, width=0.43\linewidth, clip=true]{figures/chapitre1/Lonly.ps}
  }
  \subfigure{
  \includegraphics[trim=220px 10px 220px 0px, width=0.43\linewidth, clip=true]{figures/chapitre1/RL.ps}
  }
  \caption[Mouvements d'atteintes.]{États finaux des le \emph{mouvement~2.a} et le \emph{mouvement~2.b}.}
  \label{fig:spotDiffbis}
\end{figure}
Une vidéo montrant les deux mouvements appliqués sur le robot 
HRP-2 est disponible\footnote{\url{http://homepages.laas.fr/shak/videos/}}.
Les deux mouvements se ressemblent, et il est très difficile à l'\oe{}il nu
de dire quel mouvement implique les deux t\^aches d'atteinte sans le contexte.
Dans le premier cas, le mouvement de la main gauche est dû à aux t\^aches \emph{com}
et \emph{prise droite}, étant donné que c'est un effet secondaire pour compenser
l'équilibre de la main droite.
Dans le second cas, le mouvement de la main gauche est découplé,
puisque la main gauche a été dirigée par son propre objectif.
Cependant, dans l'espace de t\^aches approprié, ces mouvements apparaissent
clairement différents.
La figure~\ref{fig:XP2RFit} montre un exemple de résultats de l'ajustement de 
modèles de t\^aches pour la main droite et la main gauche appliqué au \emph{mouvement~2.a}.
\begin{figure*}[p]
\centering
\subfigure{
%\includegraphics[width=0.9\linewidth]{figures/chapitre5/projRHNull.eps}
  \resizebox{.47\textwidth}{!} {
      \input{figures/chapitre5/simu/2a/taskRhandNormInvJerr_0}
      }      }
      \subfigure{
  \resizebox{.47\textwidth}{!} {
      \input{figures/chapitre5/simu/2a/taskLhandNormInvJerr_0}
      }            }
\caption[Ajustement de modèle de t\^ache sur le \emph{mouvement~2.a}.]{L'ajustement de modèle de t\^ache sur le \emph{mouvement~2.a} sur les t\^aches de la 
main droite et gauche. La variable $r$ est le résidu,                                                         
\ie~la distance entre deux courbes. Le mouvement de la main droite est correctement ajusté par
le modèle de la t\^ache, montrant que la t\^ache est active. Le mouvement
de la main gauche n'est pas ajusté correctement, puisque la t\^ache est inactive.
Les deux cas sont facilement distinguables gr\^ace à la valeur du résidu.} 
\label{fig:XP2RFit}
\end{figure*}
Le résidu de l'optimisation est élevé puisque l'ajustement n'est pas possible sur une 
trajectoire qui ne respecte pas le modèle (en particulier, le résidu de la t\^ache \emph{prise gauche} est 
bien plus élevé que celui associé à la t\^ache \emph{prise droite}).
Les résultats de l'algorithme de détection sont résumé dans le 
Tableau~\ref{tab:spotDiff1}.
\begin{table}[t]
\centering
\begin{tabular}{|c|c|c|c|}
\hline
Référence & Détectées & $\int \Vert \dot{q}(t) \Vert ^2 dt$ & $\int \Vert P\dot{q}(t) \Vert ^2 dt$ \\
\hline
\begin{tabular}{c}
Com\\
Prise droite\\
Double support\\
\end{tabular}

&

\begin{tabular}{c}
Com\\
Prise droite\\
Double support\\
\end{tabular}

& 0.364398 & 0.00159355 \\
\hline
\begin{tabular}{c}
Com\\
Prise gauche\\
Prise droite\\
Double support\\
\end{tabular}

&
\begin{tabular}{c}
Com\\
Prise gauche\\
Prise droite\\
Double support\\
\end{tabular}

& 0.538329  & 0.0035343 \\
\hline
\end{tabular}
\caption[Résumé des résultats pour \emph{mouvement~2.a} et du \emph{mouvement~2.b}.]{Résultats de l'algorithme de sélection de t\^aches lors de l'analyse du \emph{mouvement~2.a} et du \emph{mouvement~2.b}.}
\label{tab:spotDiff1}
%\vspace{-20pt}
\end{table}
La première colonne liste les t\^aches 
utilisées dans le mouvement de référence,
la seconde colonne liste les t\^aches sélectionnées par l'algorithme,
la troisième indique la norme du mouvement de référence (quantité de mouvement initialement observée),
et la dernière colonne montre la norme du mouvement de référence projeté dans l'espace nul
des t\^aches sélectionnées.
La quantité de mouvement final est très faible pour les deux mouvements comparé au seuil
défini comme critère d'arrêt 
$\int \Vert P\dot{q}(t) \Vert ^2 dt > \epsilon$
qui est fixé à $\epsilon = 0.07$.

Enfin la figure~\ref{fig:RbeforeAfterProj} illustre comment les normes de la vitesse articulaire
correspondant aux t\^aches \emph{prise droite} et \emph{prise gauche} évoluent
après les projections dans les espaces nuls. 
\begin{figure*}[p]
\centering
  \subfigure[Mouvement 2.a]{
  \resizebox{.47\textwidth}{!} {
      \input{figures/chapitre5/simu/2a/RbeforeAfterProj.tex}
    }
  \label{fig:RbeforeAfterProj:2a}
  }
  \subfigure[Mouvement 2.b]{
  \resizebox{.47\textwidth}{!} {
      \input{figures/chapitre5/simu/2b/RbeforeAfterProj.tex}
  }
  \label{fig:RbeforeAfterProj:2b}
  }
  \caption[Evolution de la trajectoire $\dot{\mbf{q}}$.]{Evolution de la trajectoire $\dot{\mbf{q}}$ projetée dans les espaces des t\^aches
  \emph{prise droite} (en ligne pleine) et 
  \emph{prise gauche} (en pointillés) après les projections successives du mouvement dans l'espace
  nul des t\^aches \emph{prise droite} et l'espace nul de la t\^ache \emph{com}.
  Dans le \emph{mouvement~2.a}, une grande partie du mouvement du bras gauche
  est dû à la t\^ache \emph{com}: la suppression de la t\^ache \emph{com} va annuler presque tout
  le mouvement du bras gauche.
  Dans le \emph{mouvement~2.b}, le mouvement du bras gauche n'est pas seulement issu de
  la t\^ache \emph{com}, mais principalement de la t\^ache \emph{prise gauche}.
  La suppression de la t\^ache \emph{com} va seulement annuler une petite partie
  du mouvement du bras gauche.}
\label{fig:RbeforeAfterProj}
\end{figure*}
La projection du \emph{mouvement~2.a} dans l'espace nul de la t\^ache
\emph{prise droite} va faire décroître la norme de la vitesse articulaire théorique
associée tout en laissant la norme de la vitesse articulaire théorique de la t\^ache
\emph{prise gauche} inchangée. La figure~\ref{fig:RbeforeAfterProj:2a} montre
que la vitesse articulaire théorique associée à la t\^ache \emph{prise gauche}
est diminuée après la projection du mouvement dans l'espace nul de la t\^ache \emph{com}.
Ceci explique que le bras gauche du robot a été déplacé par la t\^ache \emph{com}.
De plus, cette projection empêche la t\^ache \emph{prise gauche} d'être détectée par l'algorithme
dans des itérations futures à cause de mouvements parasites.
Cependant, après la projection du \emph{mouvement~2.b} dans l'espace nul de la t\^ache \emph{prise droite}
et la t\^ache \emph{com}, la norme de la trajectoire de la vitesse articulaire associée à la t\^ache
\emph{prise gauche} reste significative (figure~\ref{fig:RbeforeAfterProj:2b}). 
Ceci signifie que la t\^ache \emph{com} n'a que peu d'influence sur le mouvement du
bras gauche, et que le mouvement de ce bras est dû à une autre t\^ache.
Par conséquent, l'algorithme de sélection de t\^ache va continuer à chercher la t\^ache
qui a contrôlée le bras gauche.
Après la détection des deux t\^aches principales (pour le \emph{mouvement~2.a}) et des 
trois t\^aches principales (pour le \emph{mouvement~2.b}), 
la norme de la dernière trajectoire de la vitesse articulaire n'est pas nulle parce que
l'algorithme de sélection de t\^ache n'est pas terminé et d'autres t\^aches n'ont pas encore
été sélectionnées.
La t\^ache de \emph{double support} est alors détectée mais les courbes associées 
ne sont pas tracées par souci de clarté, car elles sont quasi-nulles.
D'autre part, la figure~\ref{fig:XP2RLFit} montre l'ajustement de modèle de t\^ache pour le \emph{mouvement~2.b}.
\begin{figure*}[p]
\centering
%\includegraphics[width=0.9\linewidth]{figures/chapitre5/projRHNull.eps}
\subfigure{
  \resizebox{.47\textwidth}{!} {
      \input{figures/chapitre5/simu/2b/taskRhandNormInvJerr_0}
    }      
    }
\subfigure{
  \resizebox{.47\textwidth}{!} {
      \input{figures/chapitre5/simu/2b/taskLhandNormInvJerr_0}
    }
    }
\caption[Ajustement de modèle de t\^ache pour le \emph{mouvement~2.b}.]{L'ajustement de modèle de t\^ache pour le \emph{mouvement~2.b} pour les t\^aches de 
\emph{prise droite} et \emph{prise gauche}. Les deux mouvements de la main gauche et de
la main droite sont correctement ajustés par le modèle, avec un petit résidu: 
les t\^aches sont bien détectées.}
\label{fig:XP2RLFit}
\end{figure*}
\FloatBarrier
\subsubsection{Prise VS Prise orientée}
\label{sec:distinc2}
\begin{table}[t]
  \centering
  \begin{tabular}{|c|c|c|c|}
    \hline
    Référence & Détectées & $\int \Vert \dot{q}(t) \Vert ^2 dt$ & $\int \Vert P\dot{q}(t) \Vert ^2 dt$ \\
    \hline
    \begin{tabular}{c}
      Com\\
      Regard\\
      Prise\\
      Double support\\
    \end{tabular}
    &
    \begin{tabular}{c}
      Com\\
      Regard\\
      Prise\\
      Double support\\
    \end{tabular}
    & 0.619266 & 0.00245631 \\
    \hline
    \begin{tabular}{c}
      Com\\
      Regard\\
      Prise orientée\\
      Double support\\
    \end{tabular}
    &
    \begin{tabular}{c}
      Com\\
      Regard\\
      Prise\\
      Prise orientée\\
      Double support\\
    \end{tabular}
    & 0.717041 & 0.00344557\\
    \hline
  \end{tabular}
  \caption[Résultats de l'algorithme pour l'analyse du \emph{mouvement~3.a} et du \emph{mouvement~3.b}.]{Résultats de l'algorithme de sélection de t\^ache 
  pour l'analyse du \emph{mouvement~3.a} et du \emph{mouvement~3.b}.}
  \label{tab:spotDiff2}
  %\vspace{-3pt}
\end{table}
\begin{figure*}[t]
\centering
\subfigure{
%\includegraphics[width=0.9\linewidth]{figures/chapitre5/projRHNull.eps}
  \resizebox{.48\textwidth}{!} {
      \input{figures/chapitre5/simu/3a/taskRhandScrewNormInvJerr_0}
      }      }
      \subfigure{
  \resizebox{.48\textwidth}{!} {
      \input{figures/chapitre5/simu/3b/taskRhandScrewNormInvJerr_0}
    }
    }
\caption[L'ajustement de modèle de t\^ache \emph{prise orientée} échoue.]{L'ajustement de modèle de t\^ache \emph{prise orientée}
échoue, car le résidu est plus élevé, pour le \emph{mouvement~3.a} et réussi pour le \emph{mouvement~3.b}.}
\label{fig:spotDiff3:screw}
\end{figure*}
Dans ce paragraphe, le \emph{mouvement~3.a} et le \emph{mouvement~3.b} sont analysés.
Les deux mouvements partagent la même position de but pour la main droite.
La seule différence entre les deux démonstrations est la présence d'une contrainte 
d'orientation sur la main droite dans le \emph{mouvement~3.b}.
Les postures finales de ces mouvements sont illustrées dans la figure~\ref{fig:spotDiff2}.
Le tableau~\ref{tab:spotDiff2} résume les résultats de l'algorithme de sélection de t\^ache
qui s'est déroulé avec succès.
Lors de l'analyse du \emph{mouvement~3.b}, la t\^ache de \emph{prise} a également
été sélectionnée. Ce qui est parfaitement logique car dans ce cas précis,
la t\^ache \emph{prise} est une sous-t\^ache de \emph{prise orientée}: la t\^ache \emph{prise}
est identique à la t\^ache \emph{prise orientée} sans la contrainte d'orientation.
Par contre, lors de l'analyse du \emph{mouvement~3.a}, la t\^ache
\emph{prise orientée} n'est pas sélectionnée.
La figure~\ref{fig:spotDiff3:screw} montre que l'ajustement de la t\^ache
\emph{prise orientée} échoue pour le \emph{mouvement~3.a} mais 
réussi pour le \emph{mouvement~3.b}.
Comme précédemment,
les projections annulent les mouvements résiduels qui pourraient mener à de futures détections erronées.
Le mouvement résiduel après les projections successives sont finalement proches de zéro, ce qui 
prouve que toutes les t\^aches ont été détectées.

\subsubsection{Prise orientée VS Regard}
\label{sec:distinc3}
Les deux mouvements considérés sont le \emph{mouvement~4.a} et le \emph{mouvement~4.b}.
Le \emph{mouvement~4.a} peut \^etre décrit par le scénario suivant: 
un objet se trouve devant le robot, et crée une occlusion dans le champ de vision du robot.
Pour se débarrasser de cette occlusion, le robot se penche sur sa droite.
Lorsque le robot se penche, la main gauche est entrainée par le torse par un effet
secondaire involontaire.
La position et l'orientation de cette main sont enregistrées comme étant l'état
désiré pour la t\^ache de prise orientée dans le \emph{mouvement~4.b}.
Dans le \emph{mouvement~4.b}, le regard n'est pas contr\^olé. Le mouvement de la t\^ete
est cette fois-ci, un effet secondaire du mouvement de la main gauche.
Les postures finales du robot pour ces deux mouvements sont illustrées dans la figure~\ref{fig:spotDiff3},
et les résultats des analyses sont résumés dans le Tableau~\ref{tab:spotDiff3}.
\begin{table}[t]
\centering
\begin{tabular}{|c|c|c|c|}
\hline
Référence & Détectée & $\int \Vert \dot{q}(t) \Vert ^2 dt$ & $\int \Vert P\dot{q}(t) \Vert ^2 dt$ \\
\hline
\begin{tabular}{c}
Com\\
Regard\\
Double support\\
\end{tabular}

&

\begin{tabular}{c}
Com\\
Regard\\
Double support\\
\end{tabular}

& 0.534478 & 0.0545944 \\
\hline
\begin{tabular}{c}
Com\\
Prise orientée\\
Double support\\
\end{tabular}

&
\begin{tabular}{c}
Com\\
Prise orientée\\
Double support\\
\end{tabular}

& 0.558084 & 0.0023297\\
\hline
\end{tabular}
\caption[Résultats de l'algorithme pour l'analyse du \emph{mouvement~4.a} et du \emph{mouvement~4.b}.]{Résultats de l'algorithme de sélection de t\^aches à partir de l'analyse du \emph{mouvement~4.a} 
et du \emph{mouvement~4.b}.}
\label{tab:spotDiff3}
%\vspace{-12pt}
\end{table}

Comme précédemment, les bonnes t\^aches ont été détectées pour chaque mouvement.
Le résidu après les projections de toutes les t\^aches détectées étant très bas, 
toutes les t\^aches ont été correctement détectées et soustraites au mouvement.\\

Pour conclure, nous venons de montrer expérimentalement que, sans bruits, l'algorithme de détection
se comporte parfaitement, \ie~que toute les t\^aches actives dans le mouvement analysé
sont détectées, sans faux positifs, et la suppression des effets de ces t\^aches, via les projections
du mouvement original dans les espaces nuls, conduit à un mouvement quasi nul. En effet, le mouvement
n'est pas totalement annulé à cause du bruit numérique. Dans le paragraphe suivant
nous réalisons des expérimentations dans un cadre réaliste en utilisant 
un vrai robot et de véritables capteurs qui fourniront des signaux bruités,
correspondant aux trajectoires articulaires du robot.

\section{Expérimentations sur le robot}
\label{sec:real}
Dans ce paragraphe, nous démontrons la validité de l'algorithme de
reconnaissance de t\^ache dans un cadre réaliste en analysant des signaux
bruités. Cette fois, le mouvement de référence est joué sur un véritable
robot HRP-2 et est observé en utilisant un système de capture de mouvement (figure~\ref{fig:hrp2Markers}).
Comme dans les expérimentations en simulations, l'algorithme de sélection de t\^aches est appliqué
sur des paires de mouvements visuellements proches.
Dans la suite de ce paragraphe, nous détaillons le processus d'acquisition 
des trajectoires articulaires par le biais du système de capture de mouvements.
Puis, deux paires de mouvements seront analysées par l'algorithme de reconnaissance
de t\^aches.

\subsection{Protocol expérimental}\label{sec:xpset}
Un mouvement est généré par la pile de t\^aches, puis est exécuté sur le robot HRP-2 équipé de marqueurs sur 
chacun de ses corps (voir figure~\ref{fig:hrp2Markers}).
\begin{figure}[t]
  \centering
  \begin{tabular}{cc}
    \includegraphics[height=0.4\linewidth]{figures/chapitre5/hrp2Markers.ps} &
    \includegraphics[height=0.4\linewidth]{figures/chapitre5/skel.ps} \\
  \end{tabular}
  \caption{Ensemble des marqueurs, et squelette virtuel pour le robot HRP-2.}
  \label{fig:hrp2Markers}
\end{figure}
Le système de capture de mouvements utilisé est composé de 10 caméras 
infrarouges, et enregistre les données à une fréquence de 200Hz.
Les données collectées à partir de ces marqueurs sont utilisés pour
construire un squelette virtuel qui est ajusté à la hiérarchie cinématique du
robot (figure~\ref{fig:hrp2Markers}).
Le système de capture de mouvements fournit une trajectoire pour chacun des
corps du robot.
Les données sont obtenues sous forme de trajectoires 6D de chaque corps
du robot dans l'espace, sans aucune contrainte articulaire.

L'analyse du mouvement est effectuée sur la trajectoire articulaire.
Par conséquent, les trajectoires articulaires doivent être calculées à partir des
données issues de la capture de mouvements.
Les trajectoires articulaires sont calculées de manière classique, par minimisation
de la distance entre les matrices de transformations $\mTransfMatrix{W}{R}{r_j}(\mbf{q})$
associées à l'origine des articulations du robot et des matrices de transformations mesurées
$\tensor[^{W}]{\mathbf{\hat{R}}}{_{m_j}}(t)$ par le système de capture de mouvement,
en tenant compte des limites articulaires du robot. 
\begin{eqnarray}
  \mbf{\hat{q}}(t) =  & \underset{\mbf{q}}\argmin & \sum_j^m \Vert \tensor[^{W}]{\mathbf{\hat{R}}}{_{r_j}}(t) \ominus \tensor[^{W}]{\mathbf{R}}{_{r_j}}(\mbf{q}) \Vert ^2\\
    & \text{s.t.} & q_{i\mathrm{min}} \leq q_i \leq q_{i\mathrm{max}}, \; i = 1..n
  \label{mocapOpti}
\end{eqnarray}
où $m$ est le nombre de corps, $n$ le nombre
d'articulations, $\mbf{q}$ est le vecteur de configuration du robot,
$\ominus$ est l'opérateur de distance dans SO(3), 
$\tensor[^{W}]{\mathbf{R}}{_{r_j}}(\mbf{q})$ est calculée en utilisant le modèle cinématique du robot, 
et $\tensor[^{W}]{\mathbf{\hat{R}}}{_{m_i}}$ est obtenue par:
\begin{equation}
  \tensor[^{W}]{\mathbf{\hat{R}}}{_{r_j}}(t) = \tensor[^{W}]{\mathbf{R}}{_{W_{C}}} \times \tensor[^{W_{C}}]{\mathbf{\hat{R}}}{_{m_j}}(t) \times \tensor[^{m_j}]{\mathbf{R}}{_{r_j}}    
  \label{mocapMatrix}
\end{equation}
où $\tensor[^{W}]{\mathbf{R}}{_{W_{C}}}$ est 
la matrice de transformation entre le repère du monde
et le repère de la zone d'expérimentation de la capture de mouvements.
$\mTransfMatrix{m_j}{R}{r_j}$ sont des 
transformations constantes dû à la différence entre un modèle de squelette quelconque de la capture de
mouvement et du modèle cinématique du robot utilisé lors de la phase de calibration.
Cette phase de calibration consiste à calculer les transformations $\tensor[^{W}]{\mathbf{R}}{_{W_{C}}}$
et $\tensor[^{m_j}]{\mathbf{R}}{_{r_j}}$ à partir d'une posture connue.
$\tensor[^{W_{C}}]{\mathbf{\hat{R}}}{_{m_j}}$
est la matrice de transformation mesurée associée à l'origine 
du corps virtuel $i$ dans le repère de référence du système de capture de mouvements.

Les trajectoires articulaires obtenues sont utilisées pour effectuer la reconnaissance de t\^aches
par retro-ingénierie.

Sur le véritable robot, une flexibilité est introduite après l'articulation de la cheville.
Elle interfère avec le mouvement au tout début car l'accélération angulaire
au niveau des articulations est importante. La flexibilité n'est pas modélisée,
ni dans l'algorithme de contr\^ole, ni dans la méthode de détection.
Pour ne pas \^etre influencé par la flexibilité, les 100 premières millisecondes
des mouvements analysés sont ignorées. C'est dans cette phase que les effets de la
flexibilité sont les plus importants dûs à l'accélération initiale typique d'une t\^ache
de régulation proportionnelle en cinématique inverse.
Dans les deux prochains paragraphes, nous considérons que les mesures sont directement
les trajectoires $\mbf{\hat{q}}$ données par~\eqref{mocapOpti}.

\subsection{Prise VS Maintien de l'équilibre}
Cette expérience correspond à celle effectuée en simulation dans le paragraphe~\ref{sec:distinc1}:
le premier mouvement est un mouvement de la main gauche produit par le couplage entre
la prise lointaine de la main droite, et la t\^ache \emph{com} (\emph{mouvement~2.a}). 
Le second mouvement correspondant à une double prise, a été construit pour 
présenter une ambiguïté avec le premier mouvement~(\emph{mouvement~2.b} dans le tableau~\ref{tab:motion}).
Le tableau~\ref{tab:spotDiffReal1} présente les résultats de l'algorithme de reconnaissance
pour le \emph{mouvement~2.a} et le \emph{mouvement~2.b} avec le critère d'arrêt égal à $ \epsilon = 0.07$.
\begin{table}[t]
  \centering
  \begin{tabular}{|c|c|c|c|}
    \hline
    Référence & Détectée & $\int \Vert \dot{q}(t) \Vert ^2 dt$ & $\int \Vert P\dot{q}(t) \Vert ^2 dt$ \\
    \hline
    \begin{tabular}{c}
      Com\\
      Prise droite\\
      Double support\\
    \end{tabular}

    &

    \begin{tabular}{c}
      Com\\
      Prise droite\\
      Double support\\
    \end{tabular}

    & 0.104835 & 0.0482885 \\
    \hline
    \begin{tabular}{c}
      Com\\
      Prise gauche\\
      Prise droite\\
      Double support\\
    \end{tabular}

    &
    \begin{tabular}{c}
      Com\\
      Prise gauche\\
      Prise droite\\
      Double support\\
    \end{tabular}

    & 0.142293  & 0.0541836 \\
    \hline
  \end{tabular}
  \caption[Résultats de l'algorithme sur le vrai robot.]{Résultats de l'algorithme de sélection de t\^aches pour l'analyse du \emph{mouvement~2.a} et du \emph{mouvement~2.b} joués sur le vrai robot.}
  \label{tab:spotDiffReal1}
\end{table}

En ce qui concerne le \emph{mouvement~2.a}, l'ordre d'extraction des t\^aches est:
\emph{prise droite}, \emph{double support} et \emph{com}.
Tandis que pour le \emph{mouvement~2.b},
les t\^aches extraites sont: \emph{prise droite},  \emph{com}, \emph{double support} et \emph{prise gauche}.

Les figure~\ref{fig:exp1:headFit} et \ref{fig:exp1:taskLhand} montrent l'ajustement
du modèle de t\^ache à la première itération de l'algorithme pour les t\^aches \emph{tête} et \emph{prise gauche}. 
Comme prévu, l'ajustement est correct seulement pour la t\^ache 
\emph{prise gauche} dans le \emph{mouvement~2.b} tandis que cette t\^ache est rejetée dans
le \emph{mouvement~2.a} à cause de son ajustement incorrect.
\begin{figure*}[t]
  \centering
  \subfigure[Mouvement 2.a]{
  \resizebox{.48\textwidth}{!} {
  \input{figures/chapitre5/realRobot/2a/taskHeadNormInvJerr_0.tex}
  }
  \label{fig:exp1:headFit:R}
  }
  \subfigure[Mouvement 2.b]{
  \resizebox{.48\textwidth}{!} {
  \input{figures/chapitre5/realRobot/2b/taskHeadNormInvJerr_0.tex}
  }
  \label{fig:exp1:headFit:RL}
  }
  \caption[T\^ache non sélectionnée.]{Ajustement des modèles de t\^ache à la première itération de l'algorithme de sélection de
  t\^aches pour les t\^aches \emph{tête} dans le \emph{mouvement~2.a} et \emph{mouvement~2.b} 
  Pour les deux mouvements, $r$ est élevé: la t\^ache \emph{tête}, qui n'est pas active, n'est
  pas sélectionnée.}
  \label{fig:exp1:headFit}
\end{figure*}
\begin{figure*}[t]
  \centering
  \subfigure[Mouvement 2.a]{
  \resizebox{.48\textwidth}{!} {
  \input{figures/chapitre5/realRobot/2a/taskLhandNormInvJerr_0.tex}
  }                           
  \label{fig:exp1:taskLhand:R}
  }
  \subfigure[Mouvement 2.b]{
  \resizebox{.48\textwidth}{!} {
  \input{figures/chapitre5/realRobot/2b/taskLhandNormInvJerr_0.tex}
  }
  \label{fig:exp1:taskLhand:RL}
  }
  \caption[La t\^ache \emph{prise gauche} n'est pas pertinente dans le \emph{mouvement~2.a}.]{Ajustement des modèles de t\^ache à la première itération de l'algorithme de sélection de t\^aches
  pour la t\^ache \emph{prise gauche} dans le \emph{mouvement~2.a} et \emph{mouvement~2.b} 
  La t\^ache \emph{prise gauche} n'est pas pertinente dans le \emph{mouvement~2.a}, 
  mais est sélectionnée dans le \emph{mouvement~2.b}.}
  \label{fig:exp1:taskLhand}
\end{figure*}

Bien que la t\^ache \emph{tête} ne soit impliquée dans aucun des mouvements,
la tête n'est pas fixe dans l'espace à cause du mouvement 
du torse et des jambes causés par les t\^aches actives.
Dans le \emph{mouvement~2.b}, la t\^ache candidate \emph{t\^ete} impliquerait moins de mouvement 
que dans le \emph{mouvement~2.a}. La raison est que lorsque les deux 
mains opposées du robot sont contraintes, le torse n'a pas beaucoup de liberté. 
Cependant, la t\^ache \emph{tête} n'est pas gardée pour aucun des deux mouvements (le résidu $r$ est élevé).

L'évolution du mouvement projeté dans les espaces de la t\^ache 
\emph{prise gauche} et \emph{prise droite}
après les projections successives dans les espaces nuls
automatiquement sélectionnées est illustrée dans la 
figure~\ref{fig:exp1:Evolution2L}. Pour le \emph{mouvement~2.a},
le mouvement du bras gauche est annulé après avoir annulé la t\^ache
\emph{com}.
\begin{figure*}[p]
  \centering
  \subfigure[Mouvement 2.a]{
  \resizebox{.47\textwidth}{!} {
  \input{figures/chapitre5/realRobot/2a/RevolutionLH.tex}
  }                           
  \label{fig:exp1:Evolution2L:a}
  }
  \subfigure[Mouvement 2.b]{
  \resizebox{.47\textwidth}{!} {
  \input{figures/chapitre5/realRobot/2b/RevolutionLH.tex}
  }
  \label{fig:exp1:Evolution2L:b}
  }
  \caption[Evolution du mouvement après annulation de t\^aches.]{Norme du mouvement associés aux t\^aches \emph{prise gauche} (\ie mouvement reconstruit de
  la main gauche).
  Le mouvement est nul après avoir retiré la t\^ache \emph{com} dans le 
  \emph{mouvement~2.a}. Dans ce cas, cela signifie que le mouvement de la main 
  gauche provenait de la t\^ache \emph{com}.
  Tandis que dans le \emph{mouvement~2.b}, le mouvement de la main gauche
  n'est pas annulé après avoir retiré la t\^ache 
  \emph{com}. Le mouvement devient nul uniquement après avoir retiré la t\^ache
  \emph{prise gauche}: le mouvement de la main gauche provenait de la t\^ache
  \emph{prise gauche}.}
  \label{fig:exp1:Evolution2L}
\end{figure*}
Dans le \emph{mouvement~2.b}, le mouvement de la main gauche n'est pas annulé par 
la projection dans l'espace nul de la t\^ache 
\emph{com}, mais devient nul après projection dans l'espace nul de la t\^ache
\emph{prise gauche}.
La figure~\ref{fig:exp1:Evolution2R} montre comment évolue le mouvement reconstruit de la main
droite lorsque le mouvement dû à une t\^ache est enlevé. On peut voir
que le bras droit bouge à cause des t\^aches \emph{prise droite} et
\emph{com}.
\begin{figure*}[p]
  \centering
  \subfigure[Mouvement 2.a]{
  \resizebox{.47\textwidth}{!} {
  \input{figures/chapitre5/realRobot/2a/RevolutionRH.tex}
  }                           
  \label{fig:exp1:Evolution2R:a}
  }
  \subfigure[Mouvement 2.b]{
  \resizebox{.47\textwidth}{!} {
  \input{figures/chapitre5/realRobot/2b/RevolutionRH.tex}
  }
  \label{fig:exp1:Evolution2R:b}
  }
  \caption[Annulations succéssives de t\^aches.]{Mouvement reconstruit de la main droite après les projections successives.
  Le mouvement de la main droite est une conséquence des t\^aches
  \emph{prise droite} et \emph{com}, comme c'est expliqué par le fait 
  que le mouvement de la main droite est annulé après la suppression
  des t\^aches \emph{prise droite} et \emph{com}.}
  \label{fig:exp1:Evolution2R}
\end{figure*}
La figure~\ref{fig:exp1:PqdotNorms} montre l'évolution de la quantité de mouvement
après chaque sélection de t\^ache dans les deux mouvements.
La quantité décroit au fur et à mesure des projections, jusqu'à ce que le mouvement
restant soit principalement du bruit issu des capteurs.
\begin{figure*}[p]
  \centering
  \subfigure[Mouvement 2.a]{
  \resizebox{.47\textwidth}{!} {
    \input{figures/chapitre5/realRobot/2a/PqdotNormsR}
  }
  \label{fig:exp1:PqdotNormsR}
  }
  \subfigure[Mouvement 2.b]{
  \resizebox{.47\textwidth}{!} {
    \input{figures/chapitre5/realRobot/2b/PqdotNormsRL}
  }
\label{fig:exp1:PqdotNormsRL}
}
\caption[Évolution de la norme du mouvement après annulations successives.]{Evolution de la norme du mouvement après projections successives
du mouvement dans les espaces nuls des t\^aches pour le 
\emph{mouvement~2.a} et \emph{mouvement~2.b}.}
\label{fig:exp1:PqdotNorms}
\end{figure*}

\subsection{Regard VS torse}
Les mouvements considérés sont les 
\emph{mouvement~5.a} et \emph{mouvement~5.b}.
Dans le premier cas, le robot regarde sa main droite tout en effectuant une prise avec
celle-ci.
Dans le second cas, le robot effectue une prise identique, mais doit maintenir
son torse dans la posture initiale. La figure~\ref{fig:motion5}
montre la posture finale du robot pour ces deux mouvements.
\begin{figure}[t]
  \centering
  \begin{tabular}{cc}
    \includegraphics[width=0.40\linewidth]{figures/chapitre5/realRobot/5a/5aFinal1.ps} &
    \includegraphics[width=0.40\linewidth]{figures/chapitre5/realRobot/5b/5bFinal1.ps} \\
  \end{tabular}
  \caption{Posture finale pour le \emph{mouvement~5.a} et le \emph{mouvement~5.b}.}
  \label{fig:motion5}
\end{figure}
%%%%%%%%%%%%%%%%%%%%%%%%%%%%%%%%%%%%%%%5
Le tableau~\ref{tab:result5} résume les résultats de l'algorithme de sélection de t\^aches.
Malgré le bruit, les t\^aches sont correctement détectées. Les deux mouvements sont
différenciés correctement.
Enfin, le résidu très faible après l'annulation de toutes les t\^aches
indique qu'il n'y a pas d'autres t\^aches à reconnaitre. 
\begin{table}[t]
  \centering
  \begin{tabular}{|c|c|c|c|}
    \hline
    Référence & Détectée & $\int \Vert \dot{q}(t) \Vert ^2 dt$ & $\int \Vert P\dot{q}(t) \Vert ^2 dt$ \\
    \hline
    \begin{tabular}{c}
      Com\\
      Regard\\
      Prise droite\\
      Double support\\
    \end{tabular}

    &

    \begin{tabular}{c}
      Com\\
      Regard\\
      Prise droite\\
      Double support\\
    \end{tabular}

    & 0.320001 & 0.0785168 \\
    \hline
    \begin{tabular}{c}
      Torse\\
      Com\\
      Prise droite\\
      Double support\\
    \end{tabular}

    &
    \begin{tabular}{c}
      Torse\\
      Com\\
      Prise droite\\
      Double support\\
    \end{tabular}

    & 0.216742 & 0.0516134 \\
    \hline
  \end{tabular}
  \caption{Résultats de l'algorithme de sélection de t\^aches à partir de l'analyse des \emph{mouvement~5.a} 
  et \emph{mouvement~5.b} joués sur le vrai robot.}
  \label{tab:result5}
\end{table}
La figure~\ref{fig:exp6:PqdotNorms5} montre l'évolution de la norme
du mouvement après les projections dans les espaces nuls des t\^aches: 
\emph{Prise droite},  \emph{regard}, \emph{double support} and \emph{com} pour
le \emph{mouvement~5.a}, et les t\^aches \emph{prise droite}, \emph{double support}, \emph{com}
et \emph{torse} pour le \emph{mouvement~5.b}.
\begin{figure*}[t]
  \centering
  \subfigure[Mouvement 5.a]{
  \resizebox{.46\textwidth}{!} {
    \input{figures/chapitre5/realRobot/5a/PqdotNorms5a}
  }
  }
  \subfigure[Mouvement 5.b]{
  \resizebox{.46\textwidth}{!} {
    \input{figures/chapitre5/realRobot/5b/PqdotNorms5b}
  }
}
\caption[Évolution de la norme du mouvement.]{Évolution de la norme du mouvement après avoir projeté avec succès
le mouvement dans les espaces nuls des t\^aches pour les mouvements
\emph{mouvement~5.a} et \emph{mouvement~5.b}.}
\label{fig:exp6:PqdotNorms5}
\end{figure*}
Comme dans les précédentes expérimentations, la projection fait strictement décroitre
la quantité de mouvement à chaque itération de l'algorithme. La quantité de mouvement restant 
à la fin de l'algorithme est clairement dû au bruit.
\FloatBarrier
