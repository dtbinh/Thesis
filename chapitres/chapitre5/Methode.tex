%-------------------------------------------------------------------%
%-------------------------------------------------------------------%
\section{M\'ethode}
\label{chap:taskReco}
%-------------------------------------------------------------------%
%-------------------------------------------------------------------%
\subsection{Hypothèses}
On suppose que le modèle de comportement de toute les t\^aches potentiellement utilisées sont connus
($\mbf{e}$, $\mbf{\dot{e}^*}$, $\mbf{J}$). Le modèle du robot
ainsi que toutes les t\^aches susceptibles d'intervenir
dans le mouvement sont connus. L'ensemble de ces t\^aches est appelé le \emph{lot de t\^aches}
et représente les capacités du robot.
Le mouvement observé est supposé avoir été généré en utilisant un sous-ensemble
inconnu du lot de t\^aches.
On suppose aussi que les t\^aches intervenant dans le mouvement observé sont compatibles
au sens des projections~$\mbf{P}$ définies dans~\eqref{eq:projector},
c'est-à-dire qu'il n'y a pas de singularité algorithmique~\cite{chiaverini97}. Enfin
l'ensemble des t\^aches actives ne changent pas d'état pendant le mouvement
(une pile de t\^aches fixe est valide pendant toute la durée du mouvement).
Enfin, le mouvement observé est donné sous la forme de trajectoire articulaire: $\mbf{\hat{\dot{q}}}(t)$.
Dans notre cas, cette trajectoire articulaire sera reconstruite à partir de données issues
d'un système de capture de mouvements comme décrit dans le paragraphe~\ref{sec:xpset}.

\subsection{Présentation générale}
\label{sec:alg1:selec}
L'idée générale est de reconstruire de manière itérative
la pile de t\^aches qui aurait généré le mouvement observé.
La trajectoire articulaire observée est projetée dans 
chacun des espaces de t\^aches issu d'un lot de t\^aches connus.
Les trajectoires projetées sont ensuite comparé par optimisation à une trajectoire
théorique calculée gr\^ace au modèle de génération.
Cette comparaison correspond à une fonction de coût dont le score
va servir de critère de selection. La t\^ache qui donne la trajectoire
projetée de plus faible score est sélectionnée comme étant une
t\^ache intervenant dans le mouvement.

Les effets de la t\^ache sélectionnée sont annulés en projetant la trajectoire
articulaire dans l'espace nul de la t\^ache. Le mouvement résultant est alors utilisé
pour l'itération suivante.

Si à l'itération $n$ le mouvement résultant de l'annulation d'une t\^ache est nul, alors
tous les mouvements présent dans le mouvement initial ont été justifiés 
par l'exécution des t\^aches sélectionnées. Par conséquent,
la pile de t\^aches qui aurait généré le mouvement observé
est constituée des t\^aches sélectionnées.
\newcommand{\shOUTPUT}{\textbf{Output: }}
\newcommand{\shINPUT}{\textbf{Input: }}

\begin{algorithm}[t]
  \caption{Algorithme de sélection de t\^aches}
  \label{alg:taskSelection}
\begin{algorithmic}[1]
  \STATE \shINPUT $\mathbf{\hat{\dot{q}}}(t)$
\STATE \shOUTPUT $activePool$
\STATE $\mathbf{P}_A\mathbf{\dot{q}}(t)\gets \mathbf{\hat{\dot{q}}}(t)$
\WHILE{$\int \Vert \mathbf{P}_A\mathbf{\dot{q}}(t) \Vert ^2 dt > \epsilon$ }
  \FOR{task $i = 1..n$}
    \STATE $r_i \gets \mathrm{taskFitting}(i, activePool)$
  \ENDFOR
  \STATE $i_{select} \gets \mathrm{argmin}(r_i)$
  \STATE $activePool.\mathrm{push}(i_{select})$
  \STATE $\mathbf{P}_A\mathbf{\dot{q}}(t) \gets \mathrm{projection}(i_{select}, \mathbf{P}_A\mathbf{\dot{q}}(t))$
\ENDWHILE
\end{algorithmic}
\end{algorithm}
L'algorithme est présenté dans l'algorithme~\ref{alg:taskSelection}.
La trajectoire de vitesse articulaire du mouvement observé est notée
$\mathbf{\hat{\dot{q}}}(t)$.
$\mbf{P}_A\mbf{\dot{q}}(t)$ représente les résultats des mouvements projetés dans les espaces nuls des t\^aches
sélectionnées. $A$ est l'ensemble des t\^aches qui ont été selectionnées.
Avant la première itération, $\mbf{P\dot{q}}(t)$ est initialisée avec la trajectoire  
à analyser. Ensuite, à chaque itération, cette trajectoire est projetée
dans l'espace nul de la t\^ache sélectionnée.
$r_i$ représente le score de la fonction de coût de l'optimisation. $activePool$ représente
l'ensemble des t\^aches sélectionnées lors du déroulement de l'algorithme. 

Dans le cas où le mouvement observé a été exactement
généré par la pile de t\^aches, la trajectoire articulaire $\mathbf{P\dot{q}}$ après les
projections dans l'espace nul de toute les t\^aches est nulle.
Cependant, en présence de bruits (comme par exemple un bruit issu 
de vrais capteurs lors de l'acquisition du mouvement),
un résidu est systématiquement obtenu. Il faut donc utilisé une valeur
seuil comme critère d'arrêt: la boucle s'arrête lorsque le résidu
des projections passe en dessous du niveau du bruit de la chaine d'acquisition.
$\epsilon$ représente ce seuil: lorsque la norme du mouvement
est inférieure à ce seuil, l'algorithme s'arrête.

La section suivante décrit les deux fonctions principales de l'algorithme:
\begin{itemize}
  \item $\textsf{projection}(i, \mathbf{\dot{q}}(t))$ calcule la projection
de la vitesse $\dot{\mathbf{q}}(t)$ dans l'espace nul de la t\^ache $i$.
  \item $\textsf{taskFitting}(i, activePool)$ s'occupe de l'ajustement de 
courbe du mouvement observé et du mouvement théorique et calcule donc un score. 
Ce processus est détaillé dans la section~\ref{sec:alg2:proj}.
\end{itemize}

\subsection{Projection du mouvement}
\label{sec:alg2:nullification}
%\subsubsection{General case}
Une reconstruction des trajectoires dans l'espace des t\^aches à partir des trajectoires 
angulaires peut être obtenue en multipliant cette trajectoire articulaire par la 
jacobienne de la t\^ache calculée en chaque point.

Afin d'annuler les effets d'une t\^ache détectée $i$, 
la trajectoire articulaire est projetée dans l'espace nul
de cette t\^ache en la multipliant par le projecteur dans l'espace nul
de toute les t\^aches à annuler.
\`A chaque instant $t$ du mouvement :
\begin{equation}
  \mbf{P}_A\mbf{\dot{q}}(t) \leftarrow \mbf{P}_{A+i}(t) \mbf{P}_A\mbf{\dot{q}}(t) 
  \label{eq:projection}
\end{equation}
où $A$ est l'ensemble des t\^aches déjà sélectionnées et $\mbf{P}_A(t)$ est mis à jour:
\begin{equation*}
  \mbf{P_{A+i}}(t) = \mbf{P}_A(t) - (\mbf{J_{i}}(t)\mbf{P}_A(t))^+(\mbf{J_{i}}(t)\mbf{P}_A(t))
  \label{eq:projectionupdate}
\end{equation*}
Le projecteur $\mbf{P}_A(t)$ est initialisé à $\mbf{P}_A^0(t) = \mbf I$. Le mouvement
restant après projection $\mbf{P}_A\mbf{\dot{q}}(t)$ est analysé pour détecter
les t\^aches potentiellement restantes.

L'opération de projection va annuler les effets de l'espace
de la t\^ache sélectionnée dans le mouvement. Dans l'espace articulaire,
ces effets sont les suivants: d'une part, la composante de mouvement
qui est indépendante des autres t\^aches est complètement annulée, et d'autre part, la composante 
de mouvement qui est couplée avec d'autres t\^aches est modifiée.
Le premier effet est bénéfique car il permet d'éviter
de fausses détections causées par un mouvement présent dans 
l'espace de la t\^ache sélectionnée.
Cependant, la partie couplée doit \^etre traitée avec attention
à cause de sa modification comme expliqué dans l'exemple suivant:

%\medskip
%\subsubsection{Example with two tasks}
En considérant un mouvement composé de deux t\^aches arbitraires $\mbf{e}_a$ et $\mbf{e}_b$.
La loi de commande est donnée par:
\begin{equation}
  \mbf{\dot{q}} = \mbf{J}_a^+\mbf{\dot{e}}_a^* - (\mbf{J}_b\mbf{P}_a)^+ (\mbf{\dot{e}}_b^* - \mbf{J}_b\mbf{J}_a^+\mbf{\dot{e}}_a^*)
  \label{eq:controlLaw2Tasks}
\end{equation}

Si $\mbf{e}_a$ est détectée en premier, \eqref{eq:projection} est appliquée avec
$i_{select} = a$. Le produit de~(\ref{eq:controlLaw2Tasks}) par $\mbf{P}_a$
annule le mouvement dans l'espace de la t\^ache $a$:
\begin{equation}
  \mbf{P}_a \mbf{\dot{q}} = \underbrace{\mbf{P}_a \mbf{J}_a^+\mbf{\dot{e}}_a^*}_{\mbf{0}} + \mbf{P}_a (\mbf{J}_b\mbf{P}_a)^+ (\mbf{\dot{e}}_b^* - \mbf{J}_b\mbf{J}_a^+\mbf{\dot{e}}_a^*)  
  \label{eq:cancel1_2:1}
\end{equation}
Le premier terme est nul par définition de $\mbf{P}_a$.
Le mouvement provoqué par $\mbf{P}_a \mbf{\dot{q}}$ dans l'espace de la t\^ache $b$ est 
obtenue par son produit avec $\mbf{J}_b$:
\begin{equation}
  \mbf{J}_b \mbf{P}_a \mbf{\dot{q}} = \mbf{\dot{e}}_b^* -
  \mbf{J}_b  \mbf{J}_a^+ \mbf{\dot{e}}_a^*  
  \label{eq:J2P1q}
\end{equation}
puisque $\mbf{J}_b\mbf{P}_a(\mbf{J}_b\mbf{P}_a)^+ = \mbf{I}$ par hypothèse.
Le premier terme est la composante indépendante de la seconde t\^ache, et
le second terme est la composante couplée avec la t\^ache $a$.
Par conséquent, la projection dans l'espace nul de la t\^ache découverte à la première itération
va faire apparaitre le terme de couplage dont il faudra tenir compte pour la reconnaissance d'autres t\^aches.
Ce couplage sera traité dans la prochaine section.

\subsection{Ajustement de t\^ache par optimisation} \label{sec:alg2:proj}
$\medoth{}(t)$ représente la trajectoire causée par le mouvement à analyser courant $\mbf{P}_A\mbf{\dot{q}}$ 
dans l'espace de la t\^ache observée $\mbf{e}$. La quantification de la pertinence
d'une t\^ache donnée $\mbf{e}$, par rapport au mouvement,
est obtenue en appliquant une optimisation par les moindre carrés
entre le véritable mouvement observé projeté dans l'espace de la t\^ache et le comportement de 
référence d'une t\^ache dont les paramètres sont inconnus:
\begin{equation}
  \mbf{\hat x} = \underset{\mbf{x}}\argmin \frac{\int \Vert \medoth{}(t) - \medot{\mbf x}(t) \Vert^2 \mathrm{dt}}{\int \Vert \medoth{}(t) \Vert^2 \mathrm{dt}}
\label{optimProblem}
\end{equation}

\noindent où $\medoth{}(t)$ est la trajectoire observée et $\medot{\mbf{x}}(t)$ est la trajectoire
générée par le modèle de t\^ache en utilisant les paramètres $\mbf{x}$.
Le score est donc le résidu de l'optimisation après avoir essayé d'obtenir la meilleure correspondance 
entre la trajectoire observée et le modèle théorique dans l'espace des t\^aches.

Dans quelques cas, la trajectoire $\medoth{}$ peut \^etre observée directement.
Par exemple, si l'espace de t\^ache est la position 3D d'une main, une observation
directe est possible. Cependant, en général, une observation directe n'est pas possible.
Typiquement, le centre de masse du robot est difficile à observer avec précision dans le cadre d'une t\^ache
de maintien du centre de masse dans le polygone de sustentation.
De plus, il n'est pas possible d'observer directement l'effet d'une succession de projections.
$\medoth{}$ est alors obtenue gr\^ace au modèle géométrique du robot et
de sa trajectoire articulaires lors du mouvement: $\mbf{J}_i\mbf{\hat{\dot{q}}}$.
Cependant, procéder de cette manière amène à~(\ref{eq:J2P1q}), où le couplage
des t\^aches précédemment sélectionnées apparait.
Le mouvement projeté est alors augmenté par un terme qui va compenser ce couplage.
\begin{equation}
  \medoth{}(t) = \mbf{J}_i\mbf{P}_A \mbf{\dot{q}}(t) + \mbf{J}_i\mbf{J}_A^+\medoth{A}(t)
  \label{eq:compensation}
\end{equation}
où $A$ est l'ensemble des t\^aches qui a déjà été détecté et $i$ est la t\^ache candidate.
$\mbf{P}_A$ est le projecteur dans l'espace nul des t\^aches $A$, $\mbf{J}_A$ est 
la jacobienne des t\^aches $A$. Le terme $\mbf{J}_i\mbf{J}_A^+\medoth{A}(t)$ représente la 
composante du mouvement de la t\^ache $i$ couplé aux t\^aches de l'ensemble $A$.
Tous les termes correspondant à $A$ sont connus puisqu'ils ont été identifiés
dans la précédente itération de l'algorithme.

\medskip
En pratique, l'observation est échantillonnée et l'intégrale dans~\eqref{optimProblem} est une somme.
Le problème d'optimisation~(\ref{optimProblem}) est en général un problème non
linéaire. Pour le résoudre numériquement, le solveur CFSQP
a été utilisé~\cite{lawrence97}. Le résultat de l'optimisation produit
en même temps le résidu utilisé comme critère de sélection de la t\^ache la plus probable,
et les valeurs numériques associées à la t\^ache (par exemple le gain et la position désirée, quand il
s'agit de t\^aches proportionnelles).

%The task reference behavior $\mbf{e_x}$ can, for example, be defined by an exponential regulation.
%In that case, the characteristic trajectory in the task
%space can be obtained by analytical integration of the differential equation $\mbf{\dot{e}} = -\lambda\mbf{e}$ :
%it is an exponential decrease with three input parameters :
%
%\begin{equation}
%p_{\mbf x}(t) = x_1 \mathrm{e}^{(-x_2 t)} + x_3
%\label{eqTask}
%\end{equation}
%
%The vector of parameters $(x_1, x_2, x_3)$ has then a material interpretation:
%$x_3$ is the value at convergence ($x_3=s^*$). $x_1+x_3$ is the initial value ($x_1=s(0)-s^*$).
%And $x_2$ is the gain ($x_2=\lambda$).

\subsection{Invariance à l'ordre d'annulation}
La précédente formulation de détection-projection permet d'annuler les effets de
bords des t\^aches précédemment détectées sans introduire de couplage venant des t\^aches
non détectées. Nous montrons maintenant que l'ordre dans lequel les t\^aches sont détectées
et les projections suivantes n'affectent pas la détection des t\^aches restantes.

Considérons un mouvement observé ayant été généré par une pile de t\^aches à
deux étages où la t\^ache $a$ est prioritaire
par rapport à la t\^ache $b$~(\ref{eq:controlLaw2Tasks}).
\`A partir du mouvement observé initiallement, noté $\mbf{\dot{q}}$, on observe une trajectoire dans l'espace
de la t\^ache $i$. La t\^ache est ensuite supprimé du mouvement. Le mouvement
résultant est $\mbf{P}_i \mbf{\dot{q}}$. On observe alors une trajectoire dans l'espace de la tâche $j$.
Nous montrons dans la suite que les observations dans les espaces des t\^aches sont invariablement les mêmes
pour $i=a$ et $j=b$ ou pour $i=b$ et $j=a$. C'est-à-dire que 
les deux t\^aches ayant servies à générer le mouvement
peuvent \^etre détectées (et supprimées) dans n'importe quel
ordre.

Le mouvement de référence de l'espace de t\^ache $i$
est noté $\medotc{i}$. Le mouvement observé par la t\^ache $i$ dans le mouvement
initial $\mqdot$ est noté $\medoth{i}$, tandis que le mouvement observé par la t\^ache
$j$ après avoir supprimé le mouvement issu de la t\^ache $i$ $\mP{j} \mqdot$ est noté~$\medothp{j}{i}$.

\begin{proposition}
  Soit un mouvement généré par une pile de t\^aches à deux étages. Alors
$\medoth{i} = \medothp{i}{j} = \medotc{i}$, pour $i=a$ et $j=b$, et réciproquement, pour $i=b$ et $j=a$.
\end{proposition}
%\subsubsection{Direct implication: removing Task $a$, then Task $b$}
\noindent\textbf{Preuve:}
La preuve se fait en deux parties, en considérant que $a$ est annulée en premier ($i=a$ et $j=b$),
puis que $b$ est annulée en premier ($i=b$ et $j=a$).\\

\noindent\textbf{Détection de la t\^ache $a$, puis de la t\^ache $b$:}

  Cette implication est immédiate.
\`A la première itération, l'observation dans l'espace de t\^ache $a$ est directement
$\medoth{a} = \mbf{J}_a\mbf{\hat{\dot{q}}} = \medotc{a}$.  
Après la suppression de la t\^ache $a$, l'observation dans l'espace de la t\^ache $b$
est obtenue à partir de \eqref{eq:compensation}:
\begin{align*}
 \medothp{b}{a} &=  \mbf{J}_b\mbf{\hat{\dot{q}}} +  \mbf{J}_b \mbf{J}_a^+ \medoth{a}\\
 &= \mbf{J}_b(\mbf{J}_b \mbf{P}_a)^+ (\medotc{b} - \mbf{J}_b \mbf{J}_a^+\medotc{a}) + \mbf{J}_b \mbf{J}_a^+ \medotc{a}
\end{align*}
impliquant directement $\medothp{b}{a}=\medotc{b}$, puisque $\mbf{J}_b(\mbf{J}_b \mbf{P}_a)^+ = \mbf{J}_b \mbf{P}_a(\mbf{J}_b \mbf{P}_a)^+ = \mbf{I}$ et $\medoth{a} = \medotc{a}$.

L'observation obtenu dans l'espace de t\^ache $b$ est indépendante de la projection
dans l'espace nul de la t\^ache $a$.\\

\noindent\textbf{Détection de la t\^ache $b$, puis de la t\^ache $a$:}

\`A la première itération, l'observation dans l'espace de la t\^ache $b$ est directement
\begin{equation*}
\medoth{b}  = \mJ{b} \mJp{a} \medotc{a} + \mJ{b} ( \mJ{b} \mP{a} )^+ ( \medotc{b} - \mJ{b} \mJp{a} \medotc{a} )
\end{equation*}
ce qui implique $\medoth{b} = \medotc{b}$ puisque $\mJ{b} ( \mJ{b} \mP{a} )^+ = \mbf{I}$.

La projection du mouvement initial~(\ref{eq:controlLaw2Tasks}) dans l'espace nul de
la t\^ache $b$ à la fin de la première itération conduit au calcul de la trajectoire 
dans l'espace de la t\^ache $a$ suivant:
%\begin{equation}
%  \mbf{P}_b \mbf{\hat{\dot{q}}} = \mbf{P}_b \mbf{J}_a^+\mbf{\hat{\dot{e}}}_a + 
%  \mbf{P}_b (\mbf{J}_b \mbf{P}_a)^+ (\mbf{\hat{\dot{e}}}_b - \mbf{J}_b \mbf{J}_a^+ \mbf{\hat{e}}_a)
%  \label{eq:proof:P2q}
%\end{equation}
\begin{align*}
\medothp{a}{b} =& \mJ{a} \mP{b} \mJp{a} \medotc{a} +
 \mJ{a} \mP{b} (\mJ{b} \mP{a} )^+ ( \medotc{b} - \mJ{b} \mJp{a} \medotc{a})
+ \mJ{a} \mJp{b} \medoth{b} \\
=& (\mJ{a} \mP{b} \mJp{a} - \mJ{a} \mP{b} (\mJ{b} \mP{a} )^+ \mJ{b} \mJp{a}) \medotc{a} \\
&+ (\mJ{a} \mP{b} (\mJ{b} \mP{a} )^+ + \mJ{a} \mJp{b}) \medotc{b}
\end{align*}
Nous montrons que cette somme est en réalité simplement égal à $\medotc{a}$
en montrant d'abord que le coefficient gauche du premier terme 
$(\mJ{a} \mP{b} \mJp{a} - \mJ{a} \mP{b} (\mJ{b} \mP{a} )^+ \mJ{b} \mJp{a})$, 
qu'on note $\mbf{C}_1$, est égal à l'identité.
Puis en montrant que le second coefficient gauche $(\mJ{a} \mP{b} (\mJ{b} \mP{a} )^+ + \mJ{a} \mJp{b})$,
qu'on note $\mbf{C}_2$, est nul.
$\mbf{C}_1$ peut s'écrire en développant l'operateur 
de projection $\mP{b}$ en utilisant $\mbf{P} = \mbf{I} - \mbf{J}^+ \mbf{J}$:
\begin{align*}
\mbf{C}_1 =& \mJ{a} \mJp{a} -   \mJ{a} \mJp{b} \mJ{b} \mJp{a} \\
&- \mJ{a} (\mJ{b} \mP{a} )^+ \mJ{b} \mJp{a}  + \mJ{a} \mJp{b} \mJ{b} (\mJ{b} \mP{a} )^+ \mJ{b} \mJp{a}
\end{align*}
Le premier terme est l'identité par hypothèse sur la t\^ache $a$, car
$(\mJ{b} \mP{a} )^+ = \mP{b}(\mJ{b} \mP{a} )^+$, le troisième terme est nul par définition. 
Par hypothèse sur les t\^aches $a$ et $b$, $(\mJ{b} \mP{a} )^+$ 
est l'inverse généralisé de $\mJ{b}$, et donc $\mJ{b} (\mJ{b} \mP{a} )^+ \mJ{b} = \mJ{b}$,
le quatrième terme est égal au second et se simplifie. Donc $\mbf{C}_1 = \mbf{I}$.

De la même façon pour $\mbf{C}_2$, le projecteur $\mP{b}$ est developpé:
\[
\mbf{C}_2 = \mJ{a} (\mJ{b} \mP{a} )^+ - \mJ{a} \mJp{b}\mJ{b} (\mJ{b} \mP{a} )^+ + \mJ{a} \mJp{b}
\]
Comme précédemment, le premier terme est nul, $(\mJ{b} \mP{a} )^+$ 
est l'inverse généralisé de $\mJ{b}$, le second terme est égal au troisième et se simplifie.

Ceci prouve que le même ajustement est obtenu indépendamment de la t\^ache qui a été
détectée la première. ~\QED 

%\end{proof}

\medskip
La preuve peut \^etre étendue en utilisant les m\^emes arguments 
pour un ensemble de $n$ t\^aches.
D'une façon similaire, il est trivial de prouver que quelque soit l'ordre
de détection, le mouvement calculé après toutes les projections est nul.

