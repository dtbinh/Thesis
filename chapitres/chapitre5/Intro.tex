%\section*{Introduction}
%\label{sec:intro}
%-------------------------------------------------------------------%
%-------------------------------------------------------------------%
\chapter{Reconnaissance de t\^ache en robotique par commande inverse}
\label{chap:reco}
%-------------------------------------------------------------------%
%-------------------------------------------------------------------%
Nous présentons dans ce chapitre une méthode de reconnaissance de t\^aches
particulièrement adaptée à la reconnaissance de t\^aches exécutées en parallèle.
%Ce chapitre pr\'esente une m\'ethode exploitant les 
%outils du formalisme de la fonction de t\^ache pour
%effectuer de la reconnaissance de t\^ache. La m\'ethode
%repose sur une analyse du mouvement semblable \`a de
%la r\'etro-ing\'enierie.
%
L'originalité de la méthode présentée ici est d'utiliser les propriétés de la fonction de t\^aches,
classiquement utilisée pour la génération de mouvements,
pour effectuer la reconnaissance de t\^aches. L'idée principale est d'effectuer
une rétro-ingénierie sur un mouvement observé, connaissant l'ensemble 
des t\^aches qui peuvent appara\^itre dans ce mouvement et en utilisant les 
trajectoires engendrées par les lois de commandes dans 
les espaces de t\^aches comme étant des trajectoires caractéristiques.
Sous l'hypothèse que le mouvement a été généré en empilant un ensemble
de contrôleurs, le mouvement est traité afin de rechercher les comportements
connus dans chaque espace des t\^aches.
Nous appelons cette approche de rétro-ingénierie la \emph{commande inverse}.
Alors que toutes les approches présentées dans le paragraphe suivant ne peuvent reconnaitre que
des t\^aches non parallèles, nous nous appuyons sur les principes de la redondance
en contrôle pour reconnaitre des ensembles de t\^aches parallèles.
Lors de la phase de reconnaissance, le mouvement est projeté dans 
des espaces orthogonaux aux espaces des t\^aches déjà détectées.
Cela assure un découplage efficace des t\^aches effectuées par le robot.

Nous avons introduit dans le chapitre~\ref{chap:sot} les fondements sur lesquels 
les travaux présentés s'appuient : le formalisme de la fonction de t\^aches.
L'algorithme proposé pour effectuer l'analyse de mouvement est présenté dans les
paragraphes suivants.
Enfin, des expérimentations validant la méthode sont présentées en simulation
et sur le robot HRP-2.

\section{Etat de l'art}
La reconnaissance de t\^aches est un probl\`eme qui est apparu assez
t\^ot en robotique. Dans la communaut\'e de la vision, ce probl\`eme
est g\'en\'eralement consid\'er\'e pour les mouvements non-structur\'es:
aucune hypoth\`eses ne sont faites, que ce soit sur la forme ou la rigidit\'e
des corps qui bougent.

Cependant, les informations sont issues de l'environnement et la reconnaissance
est principalement faite \`a partir du contexte, par exemple, les points
saillants en temps et en position (le flux de vid\'eo 2D est considéré comme 
étant une fonction 3D)~\cite{laptev05}. Ces points sont appris d'une base de donn\'ees,
puis associ\'es pendant une d\'emonstration. L'environnement est aussi
utilis\'e pour effectuer une extraction d'arri\`ere plan pour 
d\'egager une silhouette et effectuer, par exemple, une reconnaissance 
de d\'emarche~\cite{liu05}.

La structure du corps poly-articul\'e est souvent suppos\'ee connue et est
utilis\'ee pour estimer sa pose. Par exemple l'estimation d'une
trajectoire de posture d'un humano\"ide dans~\cite{zordan03} est ex\'ecut\'ee
en utilisant des donn\'ees issues d'un syst\`eme de capture de mouvements pour 
guider un mod\`ele physique connu.
Dans le reste de ce chapitre, nous consid\'erons que le modèle cinématique du corps poly-articul\'e
est connu.

%In the quest of robot autonomy, research and development in Robotics is
%dominated by the stimulating competition between abstract symbol manipulation and physical signal
%processing, between discrete data structures and continuous variables.
%Indeed finding a proper way to relate the discrete space of symbols and the continuous space
%of controllers is a challenge.\\
%%%
Les outils statistiques ont \'et\'e appliqu\'es avec succ\`es \`a la 
reconnaissance d'actions et \`a l'analyse de mouvements~\cite{schaal03}.
Ces outils sont utilis\'es pour cr\'eer des symboles, et par extension,
les d\'etecter dans un mouvement.
Par exemple, une m\'ethode pour le contr\^ole bas\'e comportement est 
pr\'esentée dans~\cite{drumwright03, drumwright04}. 
Les comportements sont d\'efinis comme \'etant des symboles de mouvements
(par exemple un direct, un crochet, un uppercut, une garde).
Les comportements sont mod\'elis\'es par apprentissage \`a partir d'une 
s\'erie d'exemples.
Une r\'eduction de dimension est alors appliqu\'ee pour avoir une classification
significative. La reconnaissance est g\'er\'ee par
un classifieur bayésien qui reconnait une trajectoire
dans l'espace articulaire ou l'espace cart\'esien. 
L'imitation est une extension de la reconnaissance.
Elle est effectuée en interpolant des exemples connus pour obtenir
des trajectoires réalisables.
L'introduction des processus de décision markoviens partiellement observables 
(partially observable Markov decision process, POMDP) ou des inférences bayésiennes~\cite{pearl88} 
ont relancé le domaine de la modelisation d'action~\cite{kaelbling98} ces dernières années. 
De telles techniques sont appliquées à l'apprentissage de primitives motrices~\cite{peters08} 
et à la segmentation de mouvements~\cite{calinon10, inamura04}. 
Les modèles de Markov cachés (HMMs) ont été largement utilisés dans des travaux antérieurs,
par exemple, pour exécuter une reconnaissance d'action ou de démarche~\cite{gu10}
ou pour générer des mouvements proches du mouvement humain~\cite{kwon08}. 
Dans ces travaux, des données issues de captures de mouvements humain sont utilisées
pour construire une base dans l'espace articulaire pour chaque classe de mouvement
en utilisant une technique de réduction de dimension.
En parallèle, des HMMs sont entrainés pour capturer les caractéristiques relatives
à la classe de mouvement dans l'espace des t\^aches.
La génération est obtenue en trouvant la combinaison linéaire optimale des éléments 
de la base qui va maximiser la probabilité du HMM entrainé.
Bien que l'espace des t\^aches soit considéré comme étant un espace où les caractéristiques
doivent être extraites, la méthode est limitée à un espace de t\^ache spécifique par mouvement
généré.
Dans~\cite{liang09} des modèles de Markov à tailles variable sont utilisés
pour apprendre des actions atomiques d'humain. Une séquence d'actions atomiques représente
un comportement complet.
D'une manière générale, l'efficacité des techniques de reconnaissance basé sur les outils 
statistiques est dépendante de la qualité de la base de données construite lors de la phase 
d'apprentissage.
Plusieurs démonstrations pour chaque cas particulier sont nécessaire afin
d'extraire les invariants qui vont discriminer les t\^aches.
La couverture des données d'apprentissage peut aussi limiter l'efficacité de la reconnaissance.
Enfin, l'ensemble des démonstrations doivent être associées
aux bons symboles, ce qui est généralement donné à l'algorithme
d'apprentissage.

D'une autre façon, la reconnaissance peut se baser sur des critères spécifiques qui
sont des a priori donnés au système.
Dans~\cite{nakaoka07}, seules les trajectoires du robot sont utilisées pour
distinguer les différentes phases de mouvement. Dans ces travaux là, une t\^ache
est un mouvement corps complet du robot durant un segment temporel.
Le mouvement global est une séquence de t\^aches. Chaque t\^ache possède 
ses propres paramètres appelés \emph{skill parameters}.
La méthode de reconnaissance de t\^ache est décomposée en deux étapes:
d'abord, étiqueter tous les segments temporels du mouvement 
observé avec les noms des t\^aches.
La seconde est l'estimation des \emph{skill parameters} pour chaque segment.
Chaque t\^ache est détectée par l'analyse de trajectoires projetées
dans un espace spécifique. Par exemple, la t\^ache \emph{faire un pas} est détectée
en analysant la trajectoire d'un pied;
une t\^ache de \emph{squat} est détectée en analysant la trajectoire verticale
du bassin. Le critère utilisé pour la détection et le choix des espaces de projections
sont ad-hoc, construits manuellement pour un mouvement particulier qui doit être imité par
le robot.
D'une manière similaire, ~\cite{muhlig09}
utilise un ensemble d'espaces spécifiques dans lequel le mouvement 
observé est projeté. Des critères ad-hoc
%Observations of a movement from sensors (camera, or motion capture system)
sont utilisés pour choisir automatiquement l'ensemble des espaces de t\^aches qui va 
décrire le mieux un mouvement donné, dans le but de focaliser la technique 
d'apprentissage dans ce nouvel espace. Le critère utilisé pour la
sélection de l'espace de t\^ache est exprimé par
des fonctions de coût inspirées de neurosciences:
la saillance de l'objet manipulé, la variance de la dimension d'un espace à travers
plusieurs démonstrations, et quelques heuristiques qui indiquent que les mouvements
fatiguants ou inconfortables sont provoqués par l'exécution de t\^aches. Ces heuristiques
règlent le problème de détection de t\^aches d'immobilité.
Cette méthode présentée ajoute des informations de plus haut niveau que les analyses
purement statistiques et s'appuie sur les espaces de t\^aches comme 
étant des espaces appropriés pour représenter le mouvement. Cependant, l'efficacité
de la sélection de t\^aches dépend de la puissance des heuristiques.

L'approche commune de ces travaux est de projeter le mouvement observé dans un espace réduit
particulier, dans lequel la reconnaissance est plus facile. 
Ces espaces sont choisis arbitrairement~\cite{nakaoka07}, 
séléctionnés automatiquement~\cite{muhlig09}, ou obtenus par apprentissage~\cite{peters08}. 
L'analyse en composantes principales (PCA) est une technique de réduction de dimension qui 
permet de trouver des espaces réduits. 
Malheureusement cette technique ne tient pas compte des corrélations non linéaires
entre les différents espaces réduits, et donc la technique est inadapté aux mouvements humanoïdes
qui comporte de telles corrélations non linéaires.
Il existe des techniques d'analyse en composantes principales non linéaires. Ces 
techniques sont plus adaptées aux mouvements humanoïdes~\cite{chalodhorn09}.
Cependant les espaces réduits sont dégagés automatiquement, or, en contrôle,
des espaces de tailles réduites, présentant des corrélations non linéaires, sont construit pour définir des objectifs et
moduler le comportement du robot.
Par exemple l'approche de la fonction de t\^aches~\cite{samson91} 
exprime des commandes génériques dans un espace de t\^ache donné de dimension $n$.
L'approche a été étendue pour gérer un ensemble hiérarchique de t\^aches~\cite{siciliano91, nakamura87}
en utilisant la redondance du système.
Ces espaces de t\^ache représentent ainsi des espaces parfaitement appropriés pour
effectuer des analyses et reconnaître des motifs particuliers.
Nous proposons ainsi d'utiliser directement ces contrôleurs connus pour la génération
de mouvement plutôt que d'utiliser des primitives déterminées automatiquement a posteriori.
\medskip
%In this work the control law is obtained by a \emph{stack of tasks}
%(SoT)~\cite{mansard07} ie hierarchical inverse kinematics
%\cite{siciliano91}.\\                                       

%In this paper another point of view is taken: the control theory based
%one. Control theory constitutes the second corpus that accompanies
%robotics development. Originating from mechanics and applied
%mathematics, it focuses on robot motion control~\cite{murray94,
%siciliano10}. Among all the
%contributions on linear and non-linear systems, robot control theory has
%provided efficient concepts for motion generation. The research
%initiated by A. Li\'egeois~\cite{liegeois77} on redundant robots (i.e. robots that have
%more degrees of freedom than necessary to perform a given task), and
%then developed by Y. Nakamura~\cite{nakamura91}, B. Siciliano, J.J. Slotine~\cite{siciliano91} and O.
%Khatib~\cite{khatib87} introduce mathematical machinery based on linear algebra and
%numerical optimization that allows for clever ways to model the symbolic
%notion of \emph{task}~\cite{samson91}.\\

%Those \emph{tasks} are defined by their tasks spaces (eg. position of the hand
%in an arbitrary frame\cite{nakamura86a,khatib87}, or position of a visual feature in the image
%plane \cite{espiau92,hutchinson96a}), a reference behavior in those tasks spaces
%(eg. exponential decrease to zero) and by the differential link between
%the task space and the actuator space, typically the task Jacobian.
%Given a set of active task, the corresponding control law can be
%obtained by inverting the equation of motion of the robot. 

%In this work, the analyzed motion is not seen as a sequence of motion primitives but as a
%superimposition of controllers. The recognition of sequences
%is out of the scope of the paper.
%The task-function is generally used to generate a motion\cite{siciliano91, mansard07}, and

Dans le chapitre~\ref{chap:sot}, le formalisme de la fonction de t\^aches 
a été présenté. Chaque t\^ache peut \^etre utilisée pour générer un
modèle de mouvement dans diverses situations.
Dans ce sens, une t\^ache est à la fois un contr\^oleur qui peut 
générer un mouvement, mais également un descripteur du mouvement
exécuté.
Ce descripteur est quasi symbolique, mais contient aussi des paramètres
additionnels qui caractérisent la façon dont il est exécuté : par exemple,
une t\^ache régie par une décroissance exponentielle est paramétrée par
$\lambda$ qui caractérise la vitesse de décroissance. Dans ce chapitre, 
nous proposons d'utiliser les t\^aches comme un ensemble de descripteurs
pour reconnaitre un mouvement observé, en identifiant l'ensemble
des t\^aches qui ont été utilisées pour générer ce mouvement (voir la figure~\ref{fig:taskReco}).

\begin{figure}[t]
  \begin{center}
    \includegraphics[width=0.6\linewidth]{figures/chapitre5/taskReco.ps}
  \end{center}
  \caption[La reconnaissance de t\^aches.]{Nous proposons de déterminer automatiquement quelles sont les t\^aches,
  parmi un ensemble connu, qui ont servi à générer le mouvement observé.}
  \label{fig:taskReco}
\end{figure}

L'ensemble des t\^aches identifiées est utilisé pour caractériser le mouvement observé,
par exemple pour distinguer deux mouvements visuellement proches.

Nous présentons d'abord les hypothèses considérées pour l'application
de la méthode de reconnaissance de t\^aches, ensuite, les différentes
étapes constituant la reconnaissance de t\^aches sont présentées.
Ensuite, des expérimentations sont effectuées en simulation pour illustrer le comportement
de la méthode dans des conditions idéales, puis en utilisant
un système capture de mouvement pour acquérir les mouvements à analyser joués par un vrai robot.
Ces expérimentations illustrent la précision et la robustesse aux bruits de la reconnaissance de t\^aches en 
distinguant des paires de mouvements visuellement proches. Enfin,
des expérimentations prospectives sur la validité de la méthode
pour l'analyse du mouvement humain sont présentées.
