%-------------------------------------------------------------------%
%-------------------------------------------------------------------%
\section{Exp\'erimentation sur l'humain}
\label{chap:xpHumain}
%-------------------------------------------------------------------%
%-------------------------------------------------------------------%
La méthode de reconnaissance de t\^aches présentée dans ce chapitre donne
de très bon résultats lorsque l'on connaît à la fois le modèle
des t\^aches mis en \oe uvre, et le modèle géométrique 
du robot pour calculer les différentes opération de projections. 
La reconnaissance de t\^aches pour le robot humanoïde se basait sur la connaissance
parfaite du comportement du robot lors de l'exécution d'une t\^ache.
Nous avions appliqué comme exemple de comportement, une décroissance exponentielle.
On souhaite maintenant étudier la possibilité
d'appliquer la méthode de reconnaisssance sur des mouvements humains en
adaptant les modèles de comportement de t\^ache et en utilisant 
des modèles de génération de mouvement humains.
En effet, la méthode n'est pas 
dépendante du comportement des t\^aches à retrouver.
Deux points sont donc à vérifier, correspondant
à la sélection de t\^aches d'une part (paragraphe~\ref{sec:alg2:proj}),
et d'autre part l'annulation du mouvement provoqué par l'exécution
d'une t\^ache (paragraphe~\ref{sec:alg2:nullification}).
Dans le cadre de cette thèse, seul le premier point 
a été validé en se plaçant dans le cas particulier du 
mouvement d'atteinte, dont le modèle de comportement chez l'humain 
est bien connu. 
Il s'agit du modèle du minimum jerk qui sera utilisé pour caractériser 
la trajectoire de l'effecteur atteignant sa cible.
La généralisation à d'autres types de t\^aches
ainsi que le second point
sont laissés en perspectives.

\subsection{Comportement d'une t\^ache d'atteinte par un humain}
Nous nous focalisons sur  un type de t\^ache particulier:
la t\^ache d'atteinte.
En effet, des études ont mis en évidence le profil  
de trajectoires humaines pour les mouvement d'atteinte.
Celles-ci respectent un profil de type \emph{minimum jerk}~\cite{flash85}.
\label{subsection:modelJerk}
Cette connaissance de profil de trajectoire correspond
bien à l'hypothèse requise pour l'application de notre méthode 
de reconnaissance, qui est la connaissance d'un modèle de comportement
lors de l'exécution d'une t\^ache.
La trajectoire de la partie du corps d'un humain 
effectuant ce mouvement d'atteinte suit un profil du type minimum jerk
(de la même manière que dans la partie précédente, la trajectoire du robot
correspondait à une décroissance exponentielle).
La trajectoire en position reliant deux points $A$ et $B$ est dite
de jerk minimum si elle correspond à l'optimum du problème:
\begin{eqnarray}
  \min \iiint_0^{\infty} \Vert \dddot{f}(t) \Vert dt \\
  f(0) = A\\
  f(\infty) = B
  \label{eq:minJerkTraj}
\end{eqnarray}

\subsection{Modélisation du minimum jerk}
Plutôt que de raisonner sur la forme implicite du problème~\eqref{eq:minJerkTraj},
il est souvent choisi dans la littérature de représenter les trajectoires 
en minimum jerk comme une classe spécique de polynômes du troisième ordre, 
dont on sait que le jerk est faible.
Nous choisissons de modéliser le jerk non pas comme un polynôme de degré 3,
mais comme l'intégrale triple d'un signal à quatre créneaux correspondant respectivement
aux phases d'accélération linéaires, de décélération, d'accélération et
enfin d'accélération nulle. Ainsi notre modèle comportera 6 paramètres:
les trois valeurs crêtes, et les trois instants relatifs de commutations.
Ce choix est motivé par le fait que la méthode de reconnaissance utilise
une méthode d'optimisation numérique pour la phase d'ajustement
entre les trajectoires reconstruites (par observation) et le modèle de comportement.
Utiliser les coefficients d'un polynôme comme paramètre 
d'optimisations ne permet pas de restreindre l'espace de recherche
aux courbes de type minimum jerk, et par conséquent,
modéliser une t\^ache par un polynôme n'est pas efficace.
Au contraire, la modélisation en signal créneaux
permet d'assurer la minimalité de la fonction de jerk
sans passer par des contraintes artificielles sur les 
coefficients du ploynôme représentant la trajectoire.
La trajectoire en position est
alors obtenue par triple intégration du jerk.

Le jerk est alors défini ainsi:
\begin{equation}
  \mathrm{jerk}(t) =
  \left\{
      \begin{array}{rll}
        K_1& if& 0<t< \Delta t_1 \\
        K_2& if& t_1<t< \Delta t_1 + \Delta t_2\\
        K_3& if& t_2<t< \Delta t_1 + \Delta t_2 + \Delta t_3\\
	0& if& t> \Delta t_1 + \Delta t_2 + \Delta t_3
      \end{array}
    \right.
\label{jerk}
\end{equation}
La figure~\ref{fig:jerk} illustre notre modélisation par un exemple de minimum jerk, et 
sa trajectoire en position correspondante.
\begin{figure}[t]
\begin{center}
%\includegraphics[height=0.9\linewidth, angle = -90]{img/jerkTrajectory.ps}
    \resizebox{0.4\textwidth}{!} {
      \input{figures/chapitre5/jerkTrajectory.pstex_t}
    }
\end{center}
\caption{Un exemple de minimum jerk et sa trajectoire en position correspondante.}
\label{fig:jerk}
\end{figure}

Les mouvements considérés étant des trajectoires
d'atteinte, le mouvement de la main se termine
avec une position constante de celle-ci.
Donc à $t = \Delta t_1 + \Delta t_2 + \Delta t_3$ la vitesse et l'accélération
sont nulles. Ce qui amène à un système de deux équations
avec comme inconnues $\Delta t_3$ et $K_3$, ce qui contraint la durée et le
comportement de la phase finale pour atteindre une vitesse et accélération nulle. 
%The problem is to make this trajectory continuous, so
%that the optimization performance is not poor.
%This is done by constraining the final relative time,
%and the final jerk value~: the velocity and acceleration
%at $t=t_1 + t_ 2+ t_3$ is null. It leads to a system of 2 equations
%which will be solved over $K_3$ and $t_3$.\\
Le problème d'optimisation pour l'ajustement de courbe devient:
\begin{equation}
\begin{array}{ll}
  & \underset{\mbf{x}}\min \int \Vert \mbf{\hat{p}}(t) - \mbf{p}_\mbf{x}(t) \Vert ^2 + K_1^2 + K_2^2 \mathrm{dt}\\
  \text{s.t.} & \mbf{p}_\mbf{x}(t) = \iiint \mathrm{jerk}(t) \mathrm{dt}
\end{array}
\label{minJerkOpti}
\end{equation}
\noindent où $\mbf{x}=[dt_1\ dt_2\ K_1\ K_2]$, $\mbf{\hat{p}}$ est la trajectoire mesurée.
La seconde partie de la fonction de coût $K_1^2 + K_2^2$ est introduite
pour limiter les paramètres $K_1$ et $K_2$,
afin que les trajectoires obtenues par intégration du jerk soient 
d'un ordre de grandeur raisonnable. Comme dans la section~\ref{sec:alg2:proj}, on utilise le solveur 
FSQP avec une forme discrétisée de~\eqref{minJerkOpti}, la distance entre $\mbf{p}_\mbf{x}$ et $\mbf{\hat{p}}$.

\subsection{Validation du modèle choisi}
Dans cette expérimentation, nous validons 
l'algorithme présenté dans le paragraphe~\ref{sec:alg2:proj}
pour la détection de trajectoire d'atteinte chez l'humain.
Un humain est équipé d'un ensemble de 23 marqueurs qui 
vont permettre au système de capture de mouvement
d'enregistrer des démonstrations.
Le sujet doit atteindre le haut d'une bouteille en y posant son doigt, 
tout en gardant un pied au sol et en gardant la position de la main secondaire
constante. La position finale de la main
est donc clairement définie et constante, et son orientation
n'est pas contrainte. La bouteille
est suffisamment éloignée du sujet pour l'obliger à lever un
pied pour atteindre la bouteille. Lors de cette expérimentation, la trajectoire de la main
qui atteint la bouteille est étudiée.
La figure~\ref{fig:humanXP} illustre la position finale du mouvement.

\begin{figure}[t]
\begin{center}
\includegraphics[width=0.6\linewidth]{figures/chapitre5/imgHuman.ps}
\end{center}
\caption{Position finale du mouvement d'atteinte humain.}
\label{fig:humanXP}
\end{figure}
La trajectoire de la main dominante obtenue par capture de mouvement est donnée
à un programme d'optimisation pour réaliser l'ajustement~\eqref{minJerkOpti} de la trajectoire observée
sur notre modèle théorique~\eqref{jerk}.
La figure~\ref{fig:minJerkFitting1}, montre que l'optimisation arrive 
à faire correspondre la trajectoire observée au modèle proposé.
\begin{figure}[t]
\begin{center}
%\includegraphics[height=0.9\linewidth, angle=-90]{img/FittingMinJerk.ps}
\resizebox{.4\textwidth}{!} {
      \input{figures/chapitre5/FittingMinJerk.pstex_t}
    }
\end{center}
\caption{Ajustement du modèle de minimum jerk sur une trajectoire d'atteinte réelle.}
\label{fig:minJerkFitting1}
\end{figure}

%\subsection{Conclusion et futures expérimentations}

\subsection{Analogie avec la pile de t\^aches}
L'hypothèse des \emph{variétés non contrôlées}
définit une séparation entre les espaces contrôlés et les espaces non-contrôlés~\cite{scholz99}.
Ainsi une grande variabilité dans l'espace non-contrôlé (\emph{UCM}) ne 
va pas dégrader les performances d'une t\^ache,
tandis qu'une grande variabilité dans son sous-espace orthogonal va
dégrader ces performances.

Dans le cas particulier des t\^aches d'atteintes considérées
dans les différentes expérimentations menées dans~\cite{jacquierbret09},
la présence de contraintes géométriques matérialisées 
par un obstacle entraîne une grande variabilité au niveau
de l'espace non-contrôlé du coude: pour éviter l'obstacle,
un humain va modifier la trajectoire du coude.
Une approximation de cet espace non-contrôlé est réalisée en utilisant le noyau 
de la jacobienne. Ainsi, cet espace 
représente l'espace nul de la t\^ache d'atteinte.
Les expérimentations menées dans ces travaux
montrent qu'il
est cohérent d'utiliser les mécanismes liés à la redondance,
bien connus dans le cas robotique, pour expliquer
la gestion de la redondance dans les mouvements humains.
