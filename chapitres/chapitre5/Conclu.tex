\section*{Conclusion}
Ce chapitre a présenté une méthode pour identifier quelles t\^aches sont
impliquées dans un mouvement observé, sans utiliser d'informations contextuelles:
seule la trajectoire observée est analysée.
L'analyse est conduite par la connaissance du comportement
d'un robot lors de l'exécution d'une t\^ache (par exemple, une décroissance
exponentielle).
Le mouvement analysé est supposé \^etre généré par un ensemble
de contr\^oleurs appartenant à un lot de t\^aches connu.
Le problème de la reconnaissance de t\^aches est traité en procédant par une rétro-ingénierie
du mouvement.
La trajectoire observée est analysée dans chaque espace de t\^aches connu pour décider
quelles t\^aches sont actives en comparant ces trajectoires aux comportements théoriques attendu.
La méthode est généralisable sur les comportements d'une t\^ache, et par conséquent,
n'importe quelle loi de commande utilisée pour générer un mouvement 
peut \^etre utilisée dans cette méthode pour caractériser une t\^ache.

La méthode a été appliquée avec succès dans différent scénarios pour discriminer des mouvements
visuellement proches, en simulation (paragraphe~\ref{sec:simu}) et sur un véritable 
robot HRP-2 (paragraphe~\ref{sec:real}).
Ces mouvements ont été construits spécialement pour présenter des ambiguïtés, dans le but
d'illustrer l'efficacité de la méthode proposée.\\

Dans toutes ces expériences, une hypothèse forte considérée était que le mouvement
analysé ne doit pas comporter de t\^aches qui changent d'état (active, non-active).
Une solution intéressante à explorer serait d'ajouter une dimension temporelle
à la reconnaissance de t\^ache: il s'agirait de déterminer la durée de validité
d'une pile de t\^aches, ainsi un mouvement serait segmenté temporellement
en succession d'instances de piles de t\^aches comme montré dans la Fig.~\ref{fig:trajSegmentation}.
La segmentation temporelle s'effectuera en
introduisant les temps de début et de fin de t\^aches dans les paramètres d'optimisation.

\begin{figure}[t]
\begin{center}
\includegraphics[width=0.6\linewidth]{figures/chapitre5/trajSegmentation2.ps}
\end{center}
\caption{Segmentation temporelle d'une trajectoire en séquence de piles de t\^aches.}
\label{fig:trajSegmentation}
\end{figure}

Dans l'expérience de le paragraphe~\ref{chap:xpHumain},
un modèle de t\^ache adapté aux mouvements humain pour des t\^aches d'atteintes est vérifié.
Par extension, il serait ainsi possible d'appliquer
notre méthode de reconnaissance de t\^aches pour ce type de mouvement.
Le problème étant que les mouvements humains sont beaucoup plus complexes à décrire
que les mouvements des robots. 
Les étapes futures s'attacheront à trouver dans la littérature du domaine
des modèles pour d'autres t\^aches que le mouvement d'atteinte. D'autre part,
pour les mouvements d'atteinte, nous chercherons à valider le modèle du minimum jerk
comme étant un critère discriminant (c'est-à-dire dans le cas du scénario 2.a,
si la main gauche ne suit pas une telle loi).
Une solution pour généraliser à des modèles de t\^aches divers serait 
de s'appuyer sur des méthodes d'apprentissage. On pourrait ainsi extraire
un modèle à partir d'une série de mouvement, par exemple en s'appuyant sur
des processus gaussien. L'utilisation de ces techniques ont montrées
des résultats prometteurs dans le domaine des mouvements humains~\cite{wang08a}.
Les travaux présentés dans~\cite{alvarez09} illustrent une approche hybride
de modélisation de mouvement par apprentissage et modèle physique:
les données de haute dimension représentant des mouvements humains
sont résumé par des variables latentes ayant une interprétation physique.
Le modèle serait ensuite re-utilisé en projetant le mouvement
observé dans l'espace de la variable apprise. La distance 
au modèle appris serait alors utilisée comme une mesure
d'activation de la t\^ache.

Dans notre méthode de reconnaissance,
l'étape des annulations des t\^aches par projections dans les espaces nuls
est primordiale. Outre le fait d'étudier la
validité du modèle de minimum jerk pour caractériser l'exécution
ou la non-exécution d'une t\^ache, une des voies à explorer est d'étudier si 
un modèle cinématique approché d'un humain, pour calculer les projections, peut être suffisant
pour pouvoir déterminer si toute les t\^aches ont été détectées.
L'utilisation d'un modèle plus réaliste et complet,
tel un modèle de génération de mouvement dynamique~\cite{saab11},
pourra être étudié. En effet, en considérant un modèle dynamique,
les inerties des corps seront prises en compte dans le mécanisme de projection.
Enfin, la validité de la pile de t\^aches
sera étudier pour des mouvements humain, c'est-à-dire caractériser si
les mécanismes de fusion d'objectifs multiples (ou comment est gérée la redondance) 
chez l'humain peut \^etre représenter sous une forme de pile de t\^aches.

