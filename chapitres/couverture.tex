%\addtocounter{page}{-1}%ça c'est pour revenir à 0
%\fontfamilly{phv}

%%  1ere de Couverture:



\thispagestyle{empty}
\begin{center}
  {\LARGE 
\textbf{THÈSE}\\[\baselineskip]
  }
  présentée\\[\baselineskip]
  {\Large
\textbf{devant l'Institut National des Sciences Appliquées de Toulouse}\\[\baselineskip]
  }
  pour obtenir\\[\baselineskip]
  {\large 
le grade de~: \emph{\textsc{Docteur de l'Institut National des Sciences Appliquées de Toulouse}}\\
Sp\'ecialit\'e \textsc{Systèmes}\\[\baselineskip]
  }
  par\\[\baselineskip]
  {\large 
Sovannara HAK\\[\baselineskip]
  }
Équipe d'accueil~: LAAS-CNRS - Équipe {\sc GEPETTO}\\
École Doctorale~: Edsys\\
  Titre de la thèse~:\\[\baselineskip]
  {\LARGE     \textbf{   Reconnaissance de t\^aches par commande inverse\\}}
  \vfill
  Projet R-Blink Contrat ANR-08JCJC-0075-01\\
  \vfill
  Soutenance prévue le 02/11/2011 devant la commission d'examen\\[\baselineskip]
\end{center}

\begin{center}
\begin{tabular}{r@{\protect\hspace{0.5cm}}ll@{\protect\hspace{1.0cm}}l}
%M.~:&Président&DU JURY &Président\\
%MM.~:
&Aude&BILLARD\\
&Bernard&ESPIAU\\
&Nicolas&MANSARD\\
&Olivier&STASSE\\
&Rachid&ALAMI\\%&Président\\
&Jean-Paul&LAUMOND\\

\end{tabular}
\end{center}


%  \whitepage










%\titre{~\\
%\huge Enchainement de tâches}

%\soutenue
%%   Laisser cette ligne en commentaire sauf pour la version finale.
%%   (la premiere page contiendra "a soutenir le ..." 
%%   au lieu de "soutenue le ...")


%% Les différents champs de la couverture...
%\datesout{Apres-Demain (deja?)}
%\Auteur{Nicolas}{Mansard}
%\Equipe{Projet \sc{lagadic} - IRISA}{Université de Rennes 1}
%% La composition du jury : prénom, nom, titre 
%\President{Président}{du jury}        %% le président du jury
%\Advisor{Fran\c cois}{CHAUMETTE}
%\Rapporteur{Jean-Paul}{LAUMOND}
%\Rapporteur{Oussama}{KHATIB}
%% Si vous avez N rapporteurs ça marche toujours...
%\Examinateur{Bruno}{SICILIANO}
%\Examinateur{Etienne}{DOMBRE}
%\Examinateur{Albert}{BENVENISTE}
%% idem...

%\ordre{299792458.6}  %% le numéro d'ordre donné par la Sco

%\makethese    %% crée la couverture.


%% une page blanche (deuxième de couverture)

