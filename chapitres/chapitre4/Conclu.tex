\section*{Conclusion}

Dans ce chapitre, nous avons présenté différents outils existant pour
acquérir des mouvements réels en détaillant les avantages et inconvénients des différentes 
techniques. Les techniques de capture de mouvements permettent
d'analyser les données récoltées afin d'étudier un mouvement, mais aussi
d'animer des objets ou des personnages virtuels en s'inspirant 
de mouvements réels. La possibilité d'éditer 
les données représentant les mouvements permet d'adapter un mouvement
à différente morphologie ou de créer des variations de mouvements.
Nous avons aussi souligné l'importance
du choix des t\^aches qui une fois exécutées, vont permettre de conserver les
caractéristiques du mouvement.
Nous avons présenté comment un système de capture de mouvements a permit
de fournir des trajectoires de références pour des t\^aches de suivi de trajectoires,
fournissant ainsi un moyen d'élargir l'expressivité des mouvements générés pour le robot.
Ces t\^aches peuvent ensuite être utilisées, via une pile de t\^aches, pour reproduire sur un robot HRP-2
un mouvement difficilement définissable par un modèle de génération: le mouvement
d'attente. Dans le chapitre suivant le problème de reconnaissance de t\^aches
est introduit, et les données issues de la capture de mouvements seront utilisées 
pour analyser des trajectoires associées à un mouvement.
