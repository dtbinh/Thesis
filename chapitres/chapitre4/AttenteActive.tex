
%-------------------------------------------------------------------%
%-------------------------------------------------------------------%
\subsection{Application au mouvement d'attente}
\label{chap:activeWait}
%-------------------------------------------------------------------%
%-------------------------------------------------------------------%
Lorsqu'un humain attend, l'intégralité de son corps est en mouvement.
Cependant, il n'est pas possible d'exprimer facilement une t\^ache d'attente
(au sens fonction de t\^ache).
C'est pour cette raison que pour reproduire ce mouvement, nous séléctionnons
un ensemble de t\^aches qui vont, par combinaison, reproduire
les caractéristiques du mouvement.
\subsubsection{Réplication du mouvement}
Les t\^aches de suivi de trajectoire des mains
et de l'orientation de la tête, présentées dans le paragraphe précédent,
sont choisies comme étant les t\^aches caractérisant un mouvement 
d'attente. Les trajectoires correspondantes sont donc mesurées sur un humain
à l'aide du système de capture de mouvement, et serviront de trajectoires
de références.
Pour garantir la stabilité du robot, deux t\^aches
sont ajoutées en haute priorité (\emph{double support} et \emph{centre de masse}): 
les deux pieds sont contraints à rester au sol, et le centre de masse
est fixe en $x$ et en $y$ (la hauteur n'est pas contrainte).
Pour éviter les postures trop proches des singularités,
une t\^ache de posture est utilisée en faible priorité: lors du mouvement, le robot
gardera une posture proche de la configuration de repos (configuration \emph{half-sitting}).
La figure~\ref{fig:sotActiveWait} illustre la pile de t\^aches utilisée pour réaliser
un mouvement d'attente.
\begin{figure}[t]
  \begin{center}
    \begin{tabular}{|c|}
      \hline
      Posture \emph{Half-sitting} (30D)\\
      \hline
      Main droite (3D)\\
      \hline
      Main gauche (3D)\\
      \hline
      Tête (6D)\\
      \hline
      Centre de masse (2D)\\
      \hline
      Double support (6D)\\
      \hline
    \end{tabular}
  \end{center}
  \caption{La pile de t\^aches qui va permettre de reproduire le mouvement d'attente.}
  \label{fig:sotActiveWait}
\end{figure}
Cette pile de t\^aches sera exécutée par un robot HRP-2.

%L'outil de capture de mouvement est donc utilisé pour obtenir
%des trajectoires de références correspondant
%aux trajectoires 3D des mains, et de la trajectoire 6D (position
%et orientation) de la tête d'un humain.

\subsubsection{Edition du mouvement}
Pour donner un aspect aléatoire au mouvement exécuté par le robot,
plusieurs démonstrations de mouvements d'attente sont enregistrées.
En ce qui concerne le mouvement d'attente, les positions atteintes
par les différents membres ne sont pas importantes. Ce sont les
déplacements de ces membres qui caractérisent ce mouvement.
Il n'est donc pas capital que le suivi de trajectoire soit parfaitement accompli.
Toutes les trajectoires sont éditées afin que les positions de début et de
fin soient identiques pour chaque ensemble de trajectoires. Ainsi n'importe quel séquencement
des trajectoires reste continu.
Pour chaque t\^ache, une des trajectoires enregistrées 
correspondant à cette t\^ache est choisie aléatoirement
puis est définie comme trajectoire de référence. Lorsque la trajectoire 
a été exécutée en totalité, une nouvelle trajectoire est choisie comme référence et ainsi de suite (voir
figure~\ref{fig:activeLoop}).
\begin{figure}[t]
  \begin{center}
    \resizebox{0.6\textwidth}{!}{
    \input{figures/chapitre4/activeWaitOpPoint.tex}
    }
  \end{center}
  \caption[Reproduction du mouvement d'attente.]{Pour chaque t\^ache de suivi, des trajectoires sont choisies comme
  référence parmi un ensemble de trajectoires enregistrées.}
  \label{fig:activeLoop}
\end{figure}
La figure~\ref{fig:activeWaitStrips} illustre le mouvement d'attente réalisé par le robot
HRP-2 en utilisant la pile de t\^aches, et les vidéos associées sont 
disponibles\footnote{\url{http://homepages.laas.fr/shak/videos/}}.
\begin{figure}[p]
  \begin{center}
    \subfigure{
    \includegraphics[width=0.3\linewidth]{figures/chapitre4/ActiveWait/aw1.ps}
    }
    \subfigure{
    \includegraphics[width=0.3\linewidth]{figures/chapitre4/ActiveWait/aw2.ps}
    }
    \subfigure{
    \includegraphics[width=0.3\linewidth]{figures/chapitre4/ActiveWait/aw3.ps}
    }
    \subfigure{
    \includegraphics[width=0.3\linewidth]{figures/chapitre4/ActiveWait/aw4.ps}
    }
    \subfigure{
    \includegraphics[width=0.3\linewidth]{figures/chapitre4/ActiveWait/aw5.ps}
    }\\

    \subfigure{
    \includegraphics[width=0.3\linewidth]{figures/chapitre4/ActiveWait/awz2.ps}
    }
    \subfigure{
    \includegraphics[width=0.3\linewidth]{figures/chapitre4/ActiveWait/awz3.ps}
    }
    \subfigure{
    \includegraphics[width=0.3\linewidth]{figures/chapitre4/ActiveWait/awz4.ps}
    }
  \end{center}
  \caption{Le mouvement d'attente réalisé par le robot HRP-2.}
  \label{fig:activeWaitStrips}
\end{figure}

\subsection{Adaptation du mouvement au modèle}
Les travaux présentés dans~\cite{shin01} introduisent la notion
d'\emph{importance} d'un effecteur. Il s'agit d'un critère qui permet de
déterminer si la position d'un effecteur est plus importante que la configuration articulaire associée
à sa chaîne cinématique.
L'importance des effecteurs est déterminée en analysant la posture d'un personnage
en relation avec son environnement: lorsqu'un effecteur touche
un objet (distance nulle), la position de l'effecteur est considérée comme étant plus importante
que la configuration articulaire. Par exemple, lorsqu'une main touche un objet,
il est important de conserver ce contact et donc la position de la main doit
être recopiée en priorité. Par contre si la main est en l'air alors,
la configuration articulaire est plus importante car il est possible que la position
de la main ne soit pas à portée du personnage, à cause de la structure différente,
qui reproduit le mouvement. La figure~\ref{fig:ballet} illustre un exemple
de mouvement où en fonction de l'effecteur, la configuration articulaire
ou la position doit être respectée.
\begin{figure}[p]
  \begin{center}
    \resizebox{0.4\textwidth}{!}{
    \input{figures/chapitre4/ballet.tex}
    }
  \end{center}
  \caption[Que faut-il reproduire?]{Pour reproduire le mouvement de la danseuse, il est plus important
  de respecter la position de la main droite touchant la barre que la configuration articulaire associée.
  En ce qui concerne le pied gauche, il est plus important de respecter la configuration articulaire pour
  conserver l'allure du mouvement.}
  \label{fig:ballet}
\end{figure}

Ce paragraphe s'appuie sur le travail réalisé par~\cite{ramos11} dans le 
cadre de l'édition de mouvement dynamique issu d'un mouvement humain pour l'adapter
à la dynamique d'un robot. L'exemple présenté illustre également l'importance
du choix des t\^aches à reproduire. L'utilisation de la dynamique dans la génération de 
mouvements a pour but de rendre le mouvement exécuté le plus réaliste possible. 

Considérons un mouvement de yoga réalisé par
un humain est enregistré par capture de mouvements (voir figure~\ref{fig:yogaL}).
\begin{figure}[t]
  \begin{center}
    \subfigure{
    \includegraphics[width=0.25\linewidth]{figures/chapitre4/yogaL1.ps}
    }
    \subfigure{
    \includegraphics[width=0.25\linewidth]{figures/chapitre4/yogaL2.ps}
    }
    \subfigure{
    \includegraphics[width=0.25\linewidth]{figures/chapitre4/yogaL3.ps}
    }
    \subfigure{
    \includegraphics[width=0.25\linewidth]{figures/chapitre4/yogaL4.ps}
    }
    \subfigure{
    \includegraphics[width=0.25\linewidth]{figures/chapitre4/yogaL.ps}
    }
  \end{center}
  \caption{Un mouvement de yoga enregistré gr\^ace au système de capture de mouvements.}
  \label{fig:yogaL}
\end{figure}
Si le mouvement
est converti en trajectoire articulaire et est appliqué directement 
sur le robot en cinématique, alors plusieurs problème liés 
à la différence de structures apparaissent. Le pied d'appui décolle du sol
(problème connu en animation sous le nom de \emph{foot skating} ou \emph{foot sliding}), 
les mains rentrent en collision puis transpercent le torse (voir figure~\ref{fig:yogaKine}). 
\begin{figure}[t]
  \begin{center}
    \subfigure{
    \includegraphics[width=0.3\linewidth]{figures/chapitre4/yogaHRP2Kine1.ps}
    }
    \subfigure{
    \includegraphics[width=0.3\linewidth]{figures/chapitre4/yogaHRP2Kine2.ps}
    }
    \subfigure{
    \includegraphics[width=0.3\linewidth]{figures/chapitre4/yogaHRP2Kine3.ps}
    }
    \subfigure{
    \includegraphics[width=0.3\linewidth]{figures/chapitre4/yogaHRP2Kine4.ps}
    }
    \subfigure{
    \includegraphics[width=0.3\linewidth]{figures/chapitre4/yogaHRP2Kine5.ps}
    }
  \end{center}
  \caption[Mouvement de yoga en cinématique.]{Le mouvement de yoga reproduit en cinématique en suivant les trajectoires articulaires
  calculées à partir des données issues de la capture de mouvements. Plusieurs problèmes apparaissent:
  le pied d'appui perd son contact avec le sol, les mains rentrent en collision et transpercent le torse.}
  \label{fig:yogaKine}
\end{figure}

En rejouant la trajectoire mais cette fois ci en respectant la dynamique du robot 
(en utilisant une extension de la pile de t\^aches en dynamique~\cite{saab11}), le 
pied d'appui est respecté mais le résultat n'est plus le mouvement de yoga puisque le robot perd
l'équilibre à cause de ses propres contraintes dynamiques différente de l'humain (voir figure~\ref{fig:yogaDynFall}).

\begin{figure}[t]
  \begin{center}
    \subfigure{
    \includegraphics[width=0.3\linewidth]{figures/chapitre4/yogaDynFall1.ps}
    }
    \subfigure{
    \includegraphics[width=0.3\linewidth]{figures/chapitre4/yogaDynFall2.ps}
    }
    \subfigure{
    \includegraphics[width=0.3\linewidth]{figures/chapitre4/yogaDynFall3.ps}
    }
    \subfigure{
    \includegraphics[width=0.3\linewidth]{figures/chapitre4/yogaDynFall4.ps}
    }
    \subfigure{
    \includegraphics[width=0.3\linewidth]{figures/chapitre4/yogaDynFall5.ps}
    }
  \end{center}
  \caption[Mouvement de yoga en dynamique.]{Le mouvement de yoga reproduit en dynamique. Le robot perd l'équilibre à cause de ses contraintes
  dynamiques.}
  \label{fig:yogaDynFall}
\end{figure}

En analysant le mouvement initial, on peut déterminer les caractéristiques importantes 
du mouvement. Dans l'exemple du mouvement de yoga, la position relative d'une main par rapport à l'autre
est importante, de même la position du centre de masse (et donc l'équilibre) doivent être contrôlées correctement.
En choisissant ces caractéristiques comme étant les t\^aches à exécuter pour reproduire le mouvement,
le mouvement généré devient correct (voir figure~\ref{fig:yogaDyn}).
\begin{figure}[t]
  \begin{center}
    \subfigure{
    \includegraphics[width=0.3\linewidth]{figures/chapitre4/yogaDyn1.ps}
    }
    \subfigure{
    \includegraphics[width=0.3\linewidth]{figures/chapitre4/yogaDyn2.ps}
    }
    \subfigure{
    \includegraphics[width=0.3\linewidth]{figures/chapitre4/yogaDyn3.ps}
    }
    \subfigure{
    \includegraphics[width=0.3\linewidth]{figures/chapitre4/yogaDyn4.ps}
    }
    \subfigure{
    \includegraphics[width=0.3\linewidth]{figures/chapitre4/yogaDyn5.ps}
    }
  \end{center}
  \caption[Mouvement de yoga valide en dynamique.]{Le mouvement de yoga reproduit en dynamique après édition du mouvement
  original pour définir correctement les t\^aches caractéristiques: la posture, la distance relative entre les
  deux mains, et le contrôle du centre de masse. Le mouvement généré est fidèle à la démonstration (sans collision ni
  perte d'équilibre).}
  \label{fig:yogaDyn}
\end{figure}
Enfin la figure~\ref{fig:yogaR} illustre le mouvement appliqué sur le vrai robot.
\begin{figure}[t]
  \begin{center}
    \subfigure{
    \includegraphics[trim=0px 300px 0px 300px, clip=true, width=0.3\linewidth]{figures/chapitre4/treepose1.ps}
    }
    \subfigure{
    \includegraphics[trim=0px 300px 0px 300px, clip=true, width=0.3\linewidth]{figures/chapitre4/treepose2.ps}
    }
    \subfigure{
    \includegraphics[trim=0px 300px 0px 300px, clip=true, width=0.3\linewidth]{figures/chapitre4/treepose3.ps}
    }
    \subfigure{
    \includegraphics[trim=0px 300px 0px 300px, clip=true, width=0.3\linewidth]{figures/chapitre4/treepose4.ps}
    }                                                     
    \subfigure{
    \includegraphics[trim=0px 300px 0px 300px, clip=true, width=0.3\linewidth]{figures/chapitre4/treepose5.ps}
    }
  \end{center}
  \caption[Mouvement de yoga sur le robot.]{Le mouvement de yoga reproduit sur le robot HRP-2.}
  \label{fig:yogaR}
\end{figure}

\FloatBarrier
