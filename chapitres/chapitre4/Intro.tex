%\section*{Introduction}
%\label{sec:intro}
%-------------------------------------------------------------------%
%-------------------------------------------------------------------%
\chapter{Imitation sans reconnaissance}
\label{chap:imitation}
%-------------------------------------------------------------------%
%-------------------------------------------------------------------%

L'imitation sans reconnaissance consiste à exécuter un mouvement
en utilisant directement des données issues d'une démonstration. Aucune
reconnaissance de ce qui doit être reproduit n'est effectuée.
Pour obtenir les données à partir d'une démonstration, les
techniques de capture de mouvements sont utilisées pour encoder 
un mouvement dans un format adapté aux besoins.
Dans ce chapitre, nous présentons les 
principales technologies utilisées pour effectuer de la capture
de mouvements, puis nous présenterons un peu plus en détail le système utilisé
dans le cadre de nos travaux ainsi que les différentes étapes qui constituent
l'utilisation de la capture de mouvements:
calibration du système, définition de la structure à capturer, interprétation
et phase de post-traitements des données acquises.
Ensuite nous appliquons le principe de la pile de t\^aches, présenté
dans le chapitre précédent, pour reproduire un mouvement difficilement 
reconnaissable car n'obéissant pas à une loi particulière.
Nous prenons l'exemple du mouvement d'attente d'un humain, pour le reproduire sur
un robot HRP-2.
Ce type de mouvement peut se traduire par la réalisation de mouvements inconscients de l'humain
lorsqu'il exécute l'action \emph{ne rien faire}.

