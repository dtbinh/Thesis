\FloatBarrier
\newpage
\section*{Notations mathématiques et conventions}
Dans ce manuscrit, les scalaires sont notés en minuscules comme par exemple $s$.
Les vecteurs sont notés en minuscule grasses: $\mbf{v}$ et les matrices en majuscules
grasses: $\mbf{M}$. Sauf mention contraire, $n$ représente le nombre de degrés de liberté d'un 
robot. $\mbf{I}$ est une matrice identité. Une étoile en exposant dénote une
valeur désirée: $e^*$ et un accent circonflexe dénote une valeur observée: $\hat{p}$.
Le point et le double point sont utilisés pour noter les vitesses et les 
accélérations: $\dot{\mbf{v}}$, $\ddot{\mbf{a}}$.
Des notations supplémentaires seront présentées quand elles seront nécessaires.
