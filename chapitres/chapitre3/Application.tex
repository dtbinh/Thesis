\section{Applications de la pile de t\^aches}
Dans ce paragraphe les différents types de t\^aches robotiques utilisées dans la suite des travaux
sont décrites. Puis un exemple d'exécution de t\^aches en parallèle est donné pour
illustrer les effets des couplages des t\^aches.
Dans le chapitre suivant nous présentons un exemple applicatif de génération de 
mouvement dans le cadre de l'imitation sans reconnaissance d'un mouvement
humanoïde. 
\subsection{Exemples de t\^aches robotique}
\subsubsection{T\^ache position 6D}
Un repère qui est attaché à un corps du robot doit atteindre une position 6D désirée.
L'erreur correspond à la transformation homogène $\mbf{M}_e = \mbf{M}(\mbf{M}^*)^{-1}$ permettant de passer de la position
actuelle à la position désirée avec 
$\mbf{M} = \left( \begin{array}{cc} \mbf{R}& \mbf{t}\\ \mbf{0} & \mbf{1} \end{array} \right)$.
La figure~\ref{fig:feature6d} illustre ce type de t\^ache.
\begin{figure}[t]
  \begin{center}
    \resizebox{0.95\textwidth}{!}{
    \input{figures/chapitre3/feature6d.tex}
    }
  \end{center}
  \caption{Exemple d'une t\^ache de position 6D.}
  \label{fig:feature6d}
\end{figure}
Cette représentation matricielle est redondante car il suffit de trois variables
indépendante pour identifier une matrice de rotation dans le groupe des rotations
SO(3). Une représentation vectorielle dans le groupe SE(3) (\emph{Special Euclidian Group} qui 
est isomorphe à $\mathbb{R}^3 \times$SO(3) ) est donc préférée.
Le vecteur erreur $\mbf{e}$ est défini par:
\begin{equation}
  \mbf{e} = \left( \begin{array}{c} \mbf{t}_e \\ \mbf{u}_{\theta, e} \end{array} \right)
  \label{eq:feature6dE}
\end{equation}
\noindent où $\mbf{u}_\theta$ est le vecteur représentant une rotation équivalente
à une matrice de rotation $\mbf{R}$: sa direction correspond à l'axe de rotation
et sa norme $\Vert \mbf{u}_\theta \Vert$ correspond à la valeurs de l'angle de rotation par rapport à son axe (voir figure~\ref{fig:utheta}).
On note que pour un suivi de trajectoire, l'erreur $\mbf{e}(t)  = \left( \begin{array}{c} \mbf{t}_e(t) \\ \mbf{u}_{\theta, e}(t) \end{array} \right) $ doit être petit.
\begin{figure}[t]
  \begin{center}
    \resizebox{0.1\textwidth}{!}{
    \input{figures/chapitre3/utheta.tex}
    }
  \end{center}
  \caption{Représentation d'une rotation avec un vecteur.}
  \label{fig:utheta}
\end{figure}

Dans la suite du document, on notera $\mbf{e}_{rh}$ la t\^ache de prise droite,
asservissant le point de contrôle de la dernière articulation du poignet
(telle que donnée dans le modèle classique du robot), voir figure~\ref{fig:feature6d}.
Cette t\^ache met en \oe uvre les chaînes articulaires du \emph{free flyer}, du torse
et du bras droit. Symétriquement, la t\^ache de prise main gauche sera noté $mbf{e}_{lh}$.
On nommera les t\^aches de position 6D \emph{prise orientée}.

\subsubsection{T\^ache de position 3D et de rotation 3D}
Il est possible de ne sélectionner qu'une partie des dimensions 
d'une t\^ache de position 6D en supprimant les lignes de la jacobienne
correspondant aux dimensions que l'on ne veut pas considérer.
On peut ainsi définir des t\^aches de positions dans n'importe quel
sous-espace de dimensions inférieures.
Par convention dans la suite du manuscrit, sauf mention contraire, lorsque l'on parle de t\^ache de position 3D,
nous ne considérons que les composantes de translations: 
\begin{equation}
\mbf{e} = \mbf{t}_e
\label{eq:taskPosition}
\end{equation}
De même, lorsque l'on parle de t\^ache de rotation 3D,
   nous ne considérons que les composantes de rotations:
\begin{equation}
\mbf{e} = \mbf{u}_{\theta, e} 
\label{eq:taskRotation}
\end{equation}

Dans la suite du document, les t\^aches nommées \emph{prise} sont des t\^aches
de position en 3D. 

\subsubsection{T\^ache de position du centre de masse}
L'équilibre statique d'un robot humanoïde est vérifiée lorsque la 
projection au sol du centre de masse (com) est à l'intérieur du polygone de support du robot
(voir figure~\ref{fig:supportPoly}).
\begin{figure}[t]
  \begin{center}
    \resizebox{0.3\textwidth}{!}{
    \input{figures/chapitre3/supportPoly.tex}
    }
  \end{center}
  \caption{Le polygone de support est défini par les frontières des empreintes de pieds.}
  \label{fig:supportPoly}
\end{figure}
Il est donc intéressant de contrôler la position du centre de masse. Sa position est
calculée en fonction de la position du centre de masse de chacun des corps composant le robot:
\begin{equation}
  \mbf{p}_{\textsf{com}} = \frac{1}{\sum_i m_i} \sum_i m_i\mbf{p}_{\textsf{com}_i}
  \label{eq:com}
\end{equation}
La t\^ache de position du centre de masse correspond à un cas particulier 
de la t\^ache de position 3D: les colonnes de la jacobienne associée 
à cette t\^ache sont pondérées par les masses des corps du robot~\cite{boulic96, sugihara02}.
L'erreur est donc exprimé par:
\begin{equation}
e = \mbf{p}_{com} - \mbf{p}_{com}^*
\label{eq:taskCom}
\end{equation}

\subsubsection{T\^ache regard}
Cette t\^ache consiste à centrer un point dans le plan image d'une caméra
situé au niveau de la t\^ete du robot (voir figure~\ref{fig:featureVis}).
Nous décrivons brièvement le principe général dans ce paragraphe. Une description
complète, avec notamment le calcul de la jacobienne est présentée dans~\cite{espiau92, chaumette06, chaumette07}.
\begin{figure}[t]
  \begin{center}
    \resizebox{0.95\textwidth}{!}{
    \input{figures/chapitre3/featureVis.tex}
    }
  \end{center}
  \caption[T\^ache de regard.]{La t\^ache de regard centre un point $\mbf{p}^*$ de l'espace
  dans le plan image de la caméra situé au niveau de la tête du robot.}
  \label{fig:featureVis}
\end{figure}
Un point $\mbf{M} = \left( \begin{array}{c} X\\ Y\\ Z\end{array} \right)$ dans l'espace
est projeté dans l'image d'une caméra, ayant un centre optique $\mbf{C}$ et une 
distance focale $f$, en un point $m$ (voir figure~\ref{fig:cameraProj}).
La position du point $m = \left( \begin{array}{c} x\\ y \end{array} \right)$ est alors obtenue par la relation:
\begin{equation}
  \left( \begin{array}{c} x\\ y \end{array} \right) = 
    \left( \begin{array}{c} Xf/Z\\ Yf/Z \end{array} \right) 
  \label{eq:camProj3D}
\end{equation}
\begin{figure}[t]
  \begin{center}
    \resizebox{0.4\textwidth}{!}{
    \input{figures/chapitre3/cameraProj.tex}
    }
  \end{center}
  \caption{Un point $\mbf{M}$ dans l'espace se projète en un point $m$ dans l'image de la caméra.}
  \label{fig:cameraProj}
\end{figure}
Ainsi le vecteur erreur est défini par :
\begin{equation}
  \mbf{e} = \left( \begin{array}{c} x_e \\ y_e \end{array} \right)
    = \left( \begin{array}{c} x-x^* \\ y-y^* \end{array} \right)
  \label{eq:visError}
\end{equation}
Un exemple d'utilisation de cette t\^ache est présenté dans~\cite{mansard07a}.

\subsubsection{T\^ache double support}
En supposant que les deux pieds du robot sont posés au sol,
la t\^ache de double support consiste à conserver une transformation géométrique
constante entre le repère attaché au pied gauche, et le repère attaché au pied droit.
Ainsi les deux pieds resteront en contact avec le sol (voir figure~\ref{fig:doubleSupport}).
\begin{figure}[t]
  \begin{center}
    \resizebox{0.4\textwidth}{!}{
    \input{figures/chapitre3/doubleSupport.tex}
    }
  \end{center}
  \caption[T\^ache de double support]{La t\^ache double support permet de garder une transformation constante
  égale à $\mbf{M}^*$ entre les repères attachés aux deux pieds du robot $\mathcal{R}_d$ et $\mathcal{R}_g$.}
  \label{fig:doubleSupport}
\end{figure}
Il s'agit en fait d'un autre cas particulier de la t\^ache de position 6D. 
La transformation entre deux repères attachés
à deux corps du robot doit rester constante:
\begin{equation}
\mbf{e} = \left( \begin{array}{c} \tensor[^{r}]{\mbf{t}}{_{l}} \\ \tensor[^{r}]{\mbf{u}}{_{\theta, l}} \end{array} \right)
\label{eq:taskRelative}
\end{equation}
\noindent avec $\tensor[^{r}]{\mbf{t}}{_{l}}$ la translation entre un repère attaché
au pied droit et un repère attaché au pied gauche et $\tensor[^{r}]{\mbf{u}}{_{\theta, l}}$
la rotation entre un repère attaché au pied droit et un repère attaché au pied gauche.

\subsubsection{T\^ache de posture}
Une t\^ache de posture permet de définir un état référence dans l'espace de configuration:
\begin{equation}
  \mbf{e} = \mbf{q} - \mbf{q}^*
  \label{eq:postureTask}
\end{equation}
La figure~\ref{fig:posture} illustre une t\^ache de posture.
\begin{figure}[t]
  \begin{center}
    \resizebox{0.3\textwidth}{!}{
    \input{figures/chapitre3/posture.tex}
    }
  \end{center}
  \caption[T\^ache de posture]{Un exemple de t\^ache de posture, pour l'articulation d'une jambe. $\mbf{q}^*$ étant la 
  configuration désirée.}
  \label{fig:posture}
\end{figure}

La correction de l'erreur agit directement sur $\mbf{q}$ et la jacobienne est une matrice identité.
Le contrôle de la posture pour un robot humanoïde est 
étudié en détail dans le chapitre 4 de~\cite{sentis07}. Les t\^aches de posture
sont utilisées pour optimiser des critères de performances (comme les efforts des actionneurs)
et de reproduire des comportements humain.

\subsection{Couplage des t\^aches}
Gr\^ace au formalisme de la fonction de t\^ache
et de la représentation du mouvement sous forme
de pile de t\^aches il est possible de combiner 
de manière efficace des t\^aches indépendantes à réaliser
pour générer un mouvement pour un robot.
Un couplage entre des t\^aches apparait si pour les réaliser il est nécessaire
d'utiliser des sous-chaînes articulaires communes.
En considérant la pile de t\^aches illustrée dans la figure~\ref{fig:couplingSot},
les t\^aches de regard, et de position de mains gauche et droite  ont des
influences mutuelles.
\begin{figure}[t]
  \begin{center}
    \begin{tabular}{|c|}
      \hline
      Main gauche\\
      \hline
      Main droite\\
      \hline
      Regard\\
      \hline
      Centre de masse\\
      \hline
      Double support\\
      \hline
    \end{tabular}
  \end{center}
  \caption{La pile de t\^aches utilisé pour illustrer le couplage.}
  \label{fig:couplingSot}
\end{figure}

La figure~\ref{fig:fullSot} illustre le mouvement
complet issu de cette pile de t\^aches. Pour illustrer les 
couplages, les différentes t\^aches de la piles sont
isolées pour générer plusieurs mouvements. 
\begin{figure}[t]
  \begin{center}
    \subfigure{
    \resizebox{0.36\linewidth}{!}{
    \includegraphics[trim=250px 0px 250px 0px, clip=true]{figures/chapitre3/GRL1.ps}
    }
    }
    \subfigure{
    \resizebox{0.36\linewidth}{!}{
    \includegraphics[trim=250px 0px 250px 0px, clip=true]{figures/chapitre3/GRL2.ps}
    }
    }
    \subfigure{
    \resizebox{0.36\linewidth}{!}{
    \includegraphics[trim=250px 0px 250px 0px, clip=true]{figures/chapitre3/GRL3.ps}
    }
    }
    \subfigure{
    \resizebox{0.36\linewidth}{!}{
    \includegraphics[trim=250px 0px 250px 0px, clip=true]{figures/chapitre3/GRL4.ps}
    }
    }
  \end{center}
  \caption[Exemple de t\^aches couplées dans un mouvement.]{Un exemple de mouvement comportant des t\^aches couplées. Les t\^aches
  exécutées sont la t\^ache de double support, le positionnement du centre de masse,
  l'orientation du regard, le positionnement de la main droite et gauche.}
  \label{fig:fullSot}
\end{figure}

La pile de t\^aches de base comporte les t\^aches de double support
et de positionnement du centre de masse.
La t\^ache de regard est ajoutée. Pour exécuter 
cette dernière t\^aches, le robot utilise son torse pour se pencher en avant.
Les bras du robot sont donc entraînés par le torse.
Un mouvement au niveau notamment de la main droite
peut être observé alors qu'aucune t\^ache ne contrôle explicitement
le bras droit (voir figure~\ref{fig:gazeCoupling}).
\begin{figure}[t]
  \begin{center}
    \subfigure{
    \resizebox{0.36\linewidth}{!}{
    \includegraphics[trim=250px 0px 250px 0px, clip=true]{figures/chapitre3/G1.ps}
    }
    }
    \subfigure{
    \resizebox{0.36\linewidth}{!}{
    \includegraphics[trim=250px 0px 250px 0px, clip=true]{figures/chapitre3/G2.ps}
    }
    }
    \subfigure{
    \resizebox{0.36\linewidth}{!}{
    \includegraphics[trim=250px 0px 250px 0px, clip=true]{figures/chapitre3/G3.ps}
    }
    }
    \subfigure{
    \resizebox{0.36\linewidth}{!}{
    \includegraphics[trim=250px 0px 250px 0px, clip=true]{figures/chapitre3/G4.ps}
    }
    }
  \end{center}
  \caption[T\^aches de double support, de centre de masse et de regard.]{Le mouvement est généré à partir des t\^aches de double support,
  de positionnement du centre de masse et de la t\^ache de regard.}
  \label{fig:gazeCoupling}
\end{figure}
En ajoutant la t\^ache de position de la main droite avec une position
de référence à l'avant du robot, le mouvement de la main
droite est naturellement modifié. Cependant, le mouvement de la main gauche est aussi
modifié (figure~\ref{fig:gazeRightArm}). Le déplacement de la main gauche 
est causé par le couplage de la t\^ache main droite
et la t\^ache de positionnement du centre de masse en effet, lorsque la main droite
se déplace vers l'avant, le centre de masse se déplace aussi dans cette
direction. Pour compenser le déplacement du centre de masse causé par
le déplacement de la main droite, il est nécessaire de déplacer la main gauche.
Le mouvement du torse est également modifié par rapport au mouvement ne faisant intervenir que
le regard.
\begin{figure}[t]
  \begin{center}
    \subfigure{
    \resizebox{0.36\linewidth}{!}{
    \includegraphics[trim=250px 0px 250px 0px, clip=true]{figures/chapitre3/GR1.ps}
    }
    }
    \subfigure{
    \resizebox{0.36\linewidth}{!}{
    \includegraphics[trim=250px 0px 250px 0px, clip=true]{figures/chapitre3/GR2.ps}
    }
    }
    \subfigure{
    \resizebox{0.36\linewidth}{!}{
    \includegraphics[trim=250px 0px 250px 0px, clip=true]{figures/chapitre3/GR3.ps}
    }
    }
    \subfigure{
    \resizebox{0.36\linewidth}{!}{
    \includegraphics[trim=250px 0px 250px 0px, clip=true]{figures/chapitre3/GR4.ps}
    }
    }
  \end{center}
  \caption[Couplage de la main et du torse.]{La t\^ache de positionnement de la main droite perturbe
  la position du torse et de la main gauche.}
  \label{fig:gazeRightArm}
\end{figure}

%-------------------------------------------------------------------%
%-------------------------------------------------------------------%
%\subsection{Application de la fonction de t\^ache pour la reconnaissance et l'imitation}
%\label{chap:basics:sec:application}
%-------------------------------------------------------------------%
%-------------------------------------------------------------------%

%Nous avons vu précédemment que le formalisme de la fonction
%de t\^aches permettait de générer des mouvements en créant une loi
%de commande satisfaisant l'exécution d'un ensemble de t\^aches 
%rangées par ordre de priorité. 
%Dans ce paragraphe nous présentons une autre application
%du formalisme de la fonction de t\^aches, qui sera développé dans le chapitre 5,
%à savoir utiliser ce formalisme pour effectuer
%de la reconnaissance de t\^aches.
%
%Ici, la reconnaissance
%de t\^aches consiste à déterminer l'ensemble des t\^aches (ou contr\^oleurs)
%qui ont été utilisées pour générer le mouvement observé (voir la figure~\ref{fig:sotReco}).
%\begin{figure}[t]
%  \begin{center}
%    \includegraphics[width=\linewidth]{figures/chapitre3/sotReco.ps}
%  \end{center}
%  \caption{Un mouvement est observé, puis après analyse, les t\^aches qui sont les plus probablement actives sont sélectionnées
%  parmi un ensemble de t\^aches candidates. Les valeurs numériques associées à ces t\^aches sont calculées. Ainsi une
%  pile de t\^aches est complètement reconstruite, permettant de générer des lois de commande pour n'importe quel
%  robot.}
%  \label{fig:sotReco}
%\end{figure}
%Le mouvement observé est analysé en se basant sur la connaissance 
%du comportement des t\^aches ainsi que la connaissance de toutes les t\^aches qui
%sont susceptibles d'être utilisées.
%Gr\^ace à ce formalisme, l'imitation est une conséquence directe de la reconnaissance,
%puisqu'imiter un mouvement revient à exécuter une pile de t\^aches identique.
