\section*{Conclusion}

Dans ce chapitre nous avons présenté le formalisme sur lequel
se base la suite des travaux présentés dans ce manuscrit: le formalisme de la fonction de 
t\^aches. Les méthodes utilisées pour la résolution de la cinématique inverse
sont utilisés d'une manière analogue pour manipuler les fonctions de t\^aches.
Nous avons présenté une méthode pour construire efficacement des lois de commandes,
qui permettent de réaliser plusieurs t\^aches parallèles en
respectant un ordre de priorité, en les combinant sous
forme de pile de t\^aches. Les t\^aches de priorités inférieures
sont réalisées aux mieux en exploitant l'espace nul des t\^aches
plus prioritaires afin de ne pas perturber leurs réalisations.
Chaque t\^ache est exprimée dans l'espace le plus adaptée à celle-ci.
Cette caractéristique permet d'avoir une grande expressivité
pour le contr\^ole d'un robot.
Une t\^ache peut \^etre définie dans un espace identique
aux données issues d'un capteur. Ce qui permet par exemple de suivre
facilement l'évolution de l'exécution de la t\^ache.
Dans le chapitre suivant, nous présentons une application directe
de la structure de la pile de t\^aches pour effectuer
une imitation de mouvements non structurés sans reconnaissance.


