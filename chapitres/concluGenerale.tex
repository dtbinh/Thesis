Les travaux présentés dans cette thèse ont portés
sur la reconnaissance de tâches qu'effectuent un système anthropomorphe.
Nous avons d'abord présenté les fondements sur lesquels reposent nos
travaux: le formalisme de la fonction de t\^aches.
Ce formalisme permet d'exprimer intuitivement des commandes pour un 
robot en décrivant l'objectif dans un espace approprié. Nous avons également donné
des exemples typiques de t\^aches robotiques exprimées
dans ce formalisme. La génération de mouvements est effectuée en construisant une pile
de t\^aches rangées par ordre de priorités.
Les t\^aches 
de haute priorité ne sont pas perturbées par
des t\^aches de priorités inférieures.

Les techniques utilisées pour acquérir des mouvements ont été présentées.
Ces techniques sont applicables aussi bien sur un robot que sur un humain.
Aussi nous avons donné un exemple d'une méthode générique pour reproduire 
un comportement humain sur un robot en utilisant les mécanismes de contrôle
de robot présentés précedemment.

Enfin, nous avons montré qu'en se plaçant directement dans l'espace dans lequel 
une tâche est exprimée, il est possible d'utiliser le modèle
de génération de mouvement comme un modèle de prédiction en résolvant
un problème d'optimisation. Le résidu du problème d'optimisation
produit une mesure d'adéquation entre un mouvement observé et
une loi de génération de mouvements. Ainsi
en supposant que les modèles de générations de mouvements et que les tâches
pouvant intervenir dans un mouvement sont connus et 
qu'aucune tâche ne change d'état (active, inactive) durant le mouvement observé, il est possible 
de reconstruire la pile de t\^aches ayant généré ce mouvement.

%une sémantique aux mouvements consid\'er\'es.
%Cependant, on ne s'est limit\'e qu'\`a faire de la reconnaissance
%dans des segments temporels dans lesquels les
%t\^aches mises en jeu restent dans le m\^eme \'etat.
%Par la suite, la gestion de s\'equence de t\^aches pourrait \^etre
%\'etudi\'e.
%Les travaux pr\'eliminaires sur la reconnaissance de t\^aches humaines
%pr\'esentent des r\'esultats prometteurs.

\section*{Contributions}
\subsection*{Capture de mouvements}
Dans le cadre de l'acquisition de mouvements, il était nécessaire d'observer, de manière externe au robot,
les trajectoires articulaires du robot à partir d'un système de capture de mouvements.
Ainsi, un programme de recallage de modèle par optimisation
a été développé. Le mouvement du squelette défini par la configuration géométrique
des points suivis par la capture de mouvements est transféré au modèle
du robot HRP-2.
Ce programme a aussi été utilisé 
dans les travaux d'éditions de mouvements dynamiques de l'équipe dans lesquels 
les mouvements sont montrés par un humain. Les mouvements obtenus
sont transposés au robot pour donner une première trajectoire articulaire.
Cette trajectoire articulaire est ensuite modifiée en utilisant
des t\^aches afin de corriger les erreurs introduites par le changement de modèle et 
d'adapter la dynamique du mouvement aux contraintes physiques du robot.

\subsection*{Génération de mouvements d'attente}
Pour la génération de mouvements d'attente, un module logiciel a été
créé pour assigner les lots de trajectoires mesurées par capture de mouvements
aux trajectoires de références des t\^aches. Les mouvements
générés ont été utilisés pour \emph{donner une impression de vie} au robot
HRP-2 durant une démonstration publique du robot.

\subsection*{Reconnaissance de t\^aches}
Une méthode de reconnaissance de mouvements capable de détecter 
des t\^aches effectuées en parallèle et possédant éventuellement
des couplages a été développée. Cette méthode s'appuie
sur des opérateurs de projection dans les espaces des t\^aches et leurs 
espaces orthogonaux pour d'une part projeter les mouvements à analyser
dans des espaces caractéristiques d'une t\^ache et d'autre part
pour annuler et découpler les t\^aches détectées.
La méthode de la reconnaissance de t\^aches a été implémentée
en utilisant les différentes librairies développées
en collaboration par l'équipe Gepetto au LAAS-CNRS Toulouse et le JRL Tsukuba.

\section*{Perspectives}
En ce qui concerne la reconnaissance de t\^aches,
les perspectives envisagées sont réliées aux limites
de la méthode. La reconnaissance ne s'applique que sur des segments
temporels dans lesquels les t\^aches mises en jeu ne changent pas d'état.
Ainsi il est envisageable d'ajouter des paramètres de temporels représentant le
début et la fin d'une t\^ache dans le problème 
d'optimisation des paramètres du modèle de génération de mouvements pour détecter
ces changements d'états.
L'étape qui suit logiquement est la reconnaissance de séquence de pile de t\^aches.
Un apprentissage des transitions entre les différentes instances 
de pile de t\^aches pourra être réalisé afin de construire des graphes 
de mouvements offrant une grande réactivité pour la reconnaissance et la 
reproduction.

D'autre part, le formalisme de la fonction de t\^aches
s'étend très bien en dynamique: notre méthode de reconnaissance pourra
être appliquée à des t\^aches dynamiques.
Toutes les propriétés de la fonction de t\^aches sont 
conservées. Seuls les modèles de génération et les espaces d'observation diffèrent.
Il semble donc immédiat d'adapter l'approche à la dynamique.\\

Nous avons donné une définition d'une t\^ache robotique.
Cette définition est accompagnée d'outils et de propriétés
puissantes qui nous ont permis de projeter des trajectoires
dans des espaces spécialisés où la reconnaissance est plus
facile. Mais est-il possible d'utiliser les mêmes concepts
sur un humain ou de définir d'une manière similaire des t\^aches
humaines? 

L'étude du mouvement d'attente nous a amené à réfléchir
sur ce qui caractérise un mouvement ou un comportement.
La méthode utilisée pour reproduire le mouvement
d'attente est très pragmatique: les trajectoires associées 
à certains membres sont reproduites.
L'évaluation du succès d'une imitation d'action 
est immédiate: si l'action à reproduire consiste à déplacer un objet
à un endroit donné, alors la position finale de l'objet
après l'imitation permet de valider le mouvement effectué.
En revanche il n'y a aucun critère objectif permettant de valider
le succès de l'imitation du mouvement d'attente.
Lorsqu'un humain est debout et attend,
des mouvements inconscients apparaissent: le regard
est mobile, la respiration entraîne des mouvements au niveau
du torse. Il n'est pas naturel de rester immobile.
L'exemple des statues vivantes le montre bien, puisque
de gros efforts physiques et de concentration sont nécessaires
pour contrôler sa respiration, retirer toutes les tensions inutiles, maintenir une 
posture constante et se décaler subtilement
lorsque l'attention du spectateur baisse
pour se dégourdir. Dans les deux cas, on peut considérer que l'humain \emph{ne fait rien}.
Cette étude de ce qu'est un modèle générique d'action trouverait une simplification immédiate
en introduisant des méthodes d'apprentissage automatique pour synthétiser plusieurs
démonstrations succéssives en un seul modèle numérique.
