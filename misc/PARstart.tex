% IEEEtran class scratch pad registers
% dimen
\newdimen\trantmpdimenA
\newdimen\trantmpdimenB
% count
\newcount\trantmpcountA
\newcount\trantmpcountB
% token list
\newtoks\trantmptoksA

% \PARstart
% Definition for the big two line drop cap letter at the beginning of the
% first paragraph of journal papers. The first argument is the first letter
% of the first word, the second argument is the remaining letters of the
% first word which will be rendered in upper case.
% In V1.6 this has been completely rewritten to:
% 
% 1. no longer have problems when the user begins an environment
%    within the paragraph that uses \PARstart.
% 2. auto-detect and use the current font family
% 3. revise handling of the space at the end of the first word so that
%    interword glue will now work as normal.
% 4. produce correctly aligned edges for the (two) indented lines.
% 
% We generalize things via control macros - playing with these is fun too.
% 
% the number of lines that are indented to clear it
\def\PARstartDROPLINES{2}
% minimum number of lines left on a page to allow a \@PARstart
% Does not take into consideration rubber shrink, so it tends to
% be overly cautious
\def\PARstartMINPAGELINES{2}
% the depth the letter is lowered below the baseline
% the height (and size) of the letter is determined by the sum
% of this value and the height of a capital "T" in the current
% font. It is a good idea to set this value in terms of the baselineskip
% so that it can respond to changes therein.
\def\PARstartDROPDEPTH{1.1\baselineskip}
% This is the separation distance from the drop letter to the main text.
% Lengths that depend on the font (i.e., ex, em, etc.) will be referenced
% to the font that is active when PARstart is called. 
\def\PARstartSEP{0.15em}


% definition of \PARstart
% THIS IS A CONTROLLED SPACING AREA, DO NOT ALLOW SPACES WITHIN THESE LINES
% 
% The token \PARstartfont will be globally defined after the first use
% of \PARstart and will be a font command which creates the big letter
% The first argument is the first letter of the first word and the second
% argument is the rest of the first word(s).
\def\PARstart#1#2{\par{%
% if this page does not have enough space, break it and lets start
% on a new one
%\tranneedspace{\PARstartMINPAGELINES\baselineskip}{\relax}%
% calculate the desired height of the big letter
% it extends from the top of a capital "T" in the current font
% down to \PARstartDROPDEPTH below the current baseline
\settoheight{\trantmpdimenA}{T}%
\addtolength{\trantmpdimenA}{\PARstartDROPDEPTH}%
% extract the name of the current font in bold
% and place it in \PARstartFONTNAME
\def\PARstartGETFIRSTWORD##1 ##2\relax{##1}%
{\bfseries%
\edef\PARstartFONTNAMESPACE{\fontname\font\space}%
\xdef\PARstartFONTNAME{\expandafter\PARstartGETFIRSTWORD\PARstartFONTNAMESPACE\relax}}%
% define a font based on this name with a point size equal to the desired
% height of the drop letter
\font\PARstartsubfont\PARstartFONTNAME\space at \trantmpdimenA\relax%
% save this value as a counter (integer) value (sp points)
\trantmpcountA=\trantmpdimenA%
% now get the height of the actual letter produced by this font size
\settoheight{\trantmpdimenB}{\PARstartsubfont\MakeUppercase{#1}}%
% If something bogus happens like the first argument is empty or the
% current font is strange, do not allow a zero height.
\ifdim\trantmpdimenB=0pt\relax%
\typeout{** WARNING: PARstart drop letter has zero height! (line \the\inputlineno)}%
\typeout{ Forcing the drop letter font size to 10pt.}%
\trantmpdimenB=10pt%
\fi%
% and store it as a counter
\trantmpcountB=\trantmpdimenB%
% Since a font size doesn't exactly correspond to the height of the capital
% letters in that font, the actual height of the letter, \trantmpcountB,
% will be less than that desired, \trantmpcountA
% we need to raise the font size, \trantmpdimenA 
% by \trantmpcountA / \trantmpcountB
% But, TeX doesn't have floating point division, so we have to use integer
% division. Hence the use of the counters.
% We need to reduce the denominator so that the loss of the remainder will
% have minimal affect on the accuracy of the result
\divide\trantmpcountB by 200%
\divide\trantmpcountA by \trantmpcountB%
% Then reequalize things when we use TeX's ability to multiply by
% floating point values
\trantmpdimenB=0.005\trantmpdimenA%
\multiply\trantmpdimenB by \trantmpcountA%
% \PARstartfont is globaly set to the calculated font of the big letter
% We need to carry this out of the local calculation area to to create the
% big letter.
\global\font\PARstartfont\PARstartFONTNAME\space at \trantmpdimenB%
% Now set \trantmpdimenA to the width of the big letter
% We need to carry this out of the local calculation area to set the
% hanging indent
\settowidth{\global\trantmpdimenA}{\PARstartfont\MakeUppercase{#1}}}%
% end of the isolated calculation environment
% add in the extra clearance we want
\advance\trantmpdimenA by \PARstartSEP%
% \trantmpdimenA has the width of the big letter plus the
% separation space and \PARstartfont is the font we need to use
% Now, we make the letter and issue the hanging indent command
% The letter is placed in a box of zero width and height so that other
% text won't be displaced by it.
\noindent\hangindent\trantmpdimenA\hangafter=-\PARstartDROPLINES%
\makebox[0pt][l]{\hspace{-\trantmpdimenA}\raisebox{-\PARstartDROPDEPTH}[0pt][0pt]{\PARstartfont\MakeUppercase{#1}}}\MakeUppercase{#2}}


% V1.6 \CMPARstart is no longer needed as \PARstart now uses whatever
% the current font family is.
% \CMPARstart is provided here for backward compatability.
\let\CMPARstart=\PARstart
