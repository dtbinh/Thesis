%\documentclass[12pt,draft]{article}
\documentclass[10pt]{article}
%% En 12pt c'est possible aussi


%-----------------------------------------------------------------------------%
% New command 								      %
% ----------------------------------------------------------------------------%
%\newcommand{\TODO}[1]{\textcolor{red}{\begin{Large}{#1}\end{Large}}} 
\newcommand{\TODO}[1]{ \textcolor{red}{ \textbf{ #1 } } } 


% ---------------------------------------------------------------------------- %
% --- PACKAGE ---------------------------------------------------------------- %
% ---------------------------------------------------------------------------- %
\usepackage{color}
\RequirePackage{latexsym} % anciens symboles latex
\RequirePackage{makeidx}
\RequirePackage{calc}
\RequirePackage{ifthen}
\RequirePackage{float}
\RequirePackage[frenchb]{babel}
\usepackage{multicol}
\usepackage{amsmath}

\begin{document}

\tableofcontents
% ---------------------------------------------------------------------------- %
% --- CHAPITRES -------------------------------------------------------------- %
% ---------------------------------------------------------------------------- %

 \section*{Introduction G\'en\'erale}
Introduction g\'n\'rale sur la robotique. Pourquoi \'etudier le mouvement,
les intentions du robot, interaction homme robots.

\section{Etat de l'art}
\label{chap:basics:sec:introEda}
%-------------------------------------------------------------------%
%-------------------------------------------------------------------%

Ce chapitre pr\'esente un \'etat de l'art sur les principes et technologies
employ\'ees durant les travaux qui seront pr\'esent\'es.
Tout d'abord, comment observer un mouvement.
Ensuite quelles m\'ethodes peuvent \^etre utilis\'ee pour la reconnaissance d'actions.
Comment faire de l'imitation sans reconnaissance.
Enfin, imitation de d\'emonstrations.

\subsection{Syst\`eme de capture de mouvements}
Diff\'erents types de syst\`emes de capture de mouvements:
optique, magn\'etique, marqueurs actifs, passifs.

%-------------------------------------------------------------------%
%-------------------------------------------------------------------%
\subsection{Reconnaissance de mouvement}
\label{chap:basics:sec:reco}
%-------------------------------------------------------------------%
%-------------------------------------------------------------------%

Techniques utilis\'ees en vision pour faire de la reconnaissance:
plusieurs niveau: reconnaissance de trajectoire articulaire
jusqu'\`a reconnaissance d'action comme par exemple reconna\^itre un \emph{crochet}.

%-------------------------------------------------------------------%
%-------------------------------------------------------------------%
\subsection{Imitation de mouvement sans reconnaissance}
\label{chap:basics:sec:mocap}
%-------------------------------------------------------------------%
%-------------------------------------------------------------------%

Comment faire \'ex\'ecuter un mouvement au robot directement \`a partir
d'une d\'emonstration. Aucun aspect du mouvement ne sera pas
r\'epliqu\'e.

%-------------------------------------------------------------------%
%-------------------------------------------------------------------%
\subsection{Programmation par d\'emonstration}
\label{chap:basics:sec:mocap}
%-------------------------------------------------------------------%
%-------------------------------------------------------------------%

Comment programmer des mouvements \`a appliquer au robot
\`a partir de d\'emonstrations. Typiquement les approches statistiques,
avec apprentissage.

%-------------------------------------------------------------------%
%-------------------------------------------------------------------%
\subsection*{Conclusion}
\label{chap:basics:sec:conclusion}
%-------------------------------------------------------------------%
%-------------------------------------------------------------------%

Conclusion sur l'\'etat de l'art. Faiblesse des approches statistiques,
manque de parall\'elisme.


\section{Probl\'ematique}
\label{chap:2:sec:intro}
%-------------------------------------------------------------------%
%-------------------------------------------------------------------%

Compr\'ehension du mouvement du robot. Comment reproduire un mouvement 
d\'estructur\'e. Comment utiliser l'outil de g\'en\'eration
de mouvements pour effectuer de la reconnaissance avec un certain
niveau s\'emantique. Traiter le cas des t\^aches \'ex\'ecut\'ees
en parall\`ele.
 
\section{Pile de t\^aches}
\label{chap:3:sec:intro}
%-------------------------------------------------------------------%
%-------------------------------------------------------------------%

Etat de l'art et pr\'esentation de la pile de t\^aches.
\subsection*{Conclusion}

Nous avons d\'ecrit le formalisme de la fonction de t\^aches. Nous allons 
utiliser ces outils pour faire de la reconnaissance.

\section{Imitation sans reconnaissance}
\label{chap:4:sec:intro}
%-------------------------------------------------------------------%
%-------------------------------------------------------------------%

Pour imiter un mouvement non structur\'e,
la reconnaissance n'a pas de sens. Pourtant ce type de mouvement
peut avoir un int\'eret. Ce chapitre pr\'esente une m\'ethode
pour appliquer au robot un mouvement d'attente.

\subsection{M\'ethode}
\label{chap:basics:sec:activeWaitMethode}
%-------------------------------------------------------------------%
%-------------------------------------------------------------------%

Description de la m\'ethode pour effectuer un mouvement non structur\'e.
Enregistrement de donn\'ees de capture de mouvements. Puis
utilisation de la pile de t\^aches pour l'\'ex\'ecution sur le robot.

\subsection{Mouvement d'attente active}
\label{chap:basics:sec:activeWait}
%-------------------------------------------------------------------%
%-------------------------------------------------------------------%

Un exemple de mouvement non structur\'e, l'attente.
Il s'agit d'appuyer la pr\'esence du robot. Par exemple
entre deux d\'emonstration scientifique.
\subsection*{Conclusion}

Nous avons montr\'e une mani\`ere pour reproduire 
des mouvements non structur\'e.

\section{Reconnaissance de t\^ache en robotique}
\label{chap:5:sec:intro}
%-------------------------------------------------------------------%
%-------------------------------------------------------------------%

Ce chapitre pr\'esente une m\'ethode exploitant les 
outils du formalisme de la fonction de t\^ache pour
effectuer de la reconnaissance de t\^ache. La m\'ethode
repose sur une analyse du mouvement semblable \`a de
la r\'etro-ing\'enierie.

\subsection{M\'ethode}
\label{chap:basics:sec:taskReco}
%-------------------------------------------------------------------%
%-------------------------------------------------------------------%

Cette section pr\'esente la m\'ethode utilis\'e
pour analyser une trajectoire articulaire afin
d'en d\'eduire les t\^aches qui ont \'et\'e utilis\'ees
pour g\'en\'erer ce mouvement. Cette m\'ethode s'appuie sur
les m\'ecanismes de la fonction de t\^aches pour d\'ecoupler
les diff\'erentes t\^aches impliqu\'ees dans le mouvement.
\subsection{Exp\'erimentation sur le robot}
\label{chap:basics:sec:xpRobot}
%-------------------------------------------------------------------%
%-------------------------------------------------------------------%

Pour valider la m\'ethode, plusieurs exp\'erimentations
on \'et\'e effectu\'ees sur le robot HRP-2 en simulation
et sur le vrai robot. Ces exp\'erimentations consistent 
\`a discriminer des paires mouvements visuellement proche, mais 
faisant intervenir des t\^aches diff\'erentes: les s\'emantiques
de ces mouvements sont diff\'erentes.

\subsection{Exp\'erimentation sur l'humain}
\label{chap:basics:sec:xpHumain}
%-------------------------------------------------------------------%
%-------------------------------------------------------------------%

Cette section pr\'esente les premiers travaux effectu\' pour appliquer
la m\'ethode de reconnaissance de t\^aches \`a l'humain.
Le choix du comportement de r\'ef\'erence se porte
sur des trajectoires de type \emph{minimum jerk}.
\subsection*{Conclusion}

Nous avons valid\'e la m\'ethode pr\'esent\'ee
gr\^ace \`a plusieurs exp\'erimentations, et avons
explor\'e la possibilit\'e d'appliquer
cette m\'ethode \`a l'humain.

\section{Reconnaissance de t\^ache dynamiques}
\label{chap:6:sec:intro}
%-------------------------------------------------------------------%
%-------------------------------------------------------------------%

Le mecanisme utilis\'e en cin\'ematique est appliqu\'e en
dynamique.
\subsection{Pile de t\^aches dynamique}
\label{chap:basics:sec:sotDyn}
%-------------------------------------------------------------------%
%-------------------------------------------------------------------%

Cette section introduit la pile de t\^aches dynamique.
\subsection{Imitation sans reconnaissance en dynamique}
\label{chap:basics:sec:imitDyn}
%-------------------------------------------------------------------%
%-------------------------------------------------------------------%

Cette section pr\'esente un exemple d'application
de la pile de t\^aches dynamique dans le cadre
d'une imitation de mouvement sans reconnaissance.
\subsection{Reconnaissance de t\^aches dynamique}
\label{chap:basics:sec:recoDyn}
%-------------------------------------------------------------------%
%-------------------------------------------------------------------%

Cette section pr\'esente les r\'esultats obtenus
pour la reconnaissance de t\^aches dynamique.
\subsection*{Conclusion}

Nous avons \'etendu la m\'ethode de reconnaissance \`a la dynamique.
 

% %%%%%%%%%%%%%%%%%%%%%%%%%%%%%%%%%%%%%%%%%%%%%%%%%%%%%%%%%%%%%%%%%%%%%%%%%%%%
 \section*{Conclusion et perspectives}
Les travaux pr\'esent\'es permettent d'introduire
une s\'emantique aux mouvements consid\'er\'es.
Cependant, on ne s'est limit\'e qu'\`a faire de la reconnaissance
dans des segments temporels dans lesquels les
t\^aches mises en jeu restent dans le m\^eme \'etat.
Par la suite, la gestion de s\'equence de t\^aches pourrait \^etre
\'etudi\'e.
Les travaux pr\'eliminaires sur la reconnaissance de t\^aches humaines
pr\'esentent des r\'esultats prometteurs.
% %%%%%%%%%%%%%%%%%%%%%%%%%%%%%%%%%%%%%%%%%%%%%%%%%%%%%%%%%%%%%%%%%%%%%%%%%%%%
 


\end{document}


%<!-- Local IspellDict: francais -->
