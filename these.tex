%\documentclass[12pt,draft]{article}
\documentclass[10pt]{book}
%% En 12pt c'est possible aussi

%-----------------------------------------------------------------------------%
% New command 								      %
% ----------------------------------------------------------------------------%
%\newcommand{\TODO}[1]{\textcolor{red}{\begin{Large}{#1}\end{Large}}} 
\newcommand{\TODO}[1]{ \textcolor{red}{ \textbf{ #1 } } } 


% ---------------------------------------------------------------------------- %
% --- PACKAGE ---------------------------------------------------------------- %
% ---------------------------------------------------------------------------- %
\usepackage[utf8]{inputenc}
\usepackage[T1]{fontenc}
\usepackage{ae,aecompl}
\usepackage[frenchb]{babel}
%\usepackage{these}

% ITRO 
\usepackage[dvips]{graphicx}
%\usepackage[pdftex]{graphicx}
%\usepackage{graphicx}  
\usepackage{subfigure} 
\usepackage{amssymb}
\usepackage{amsmath}
\usepackage{url}
\usepackage{multirow}
\usepackage{array}
\usepackage{algorithmic}
\usepackage[french]{algorithm2e}
\usepackage{tensor}
\usepackage{leftidx}
\usepackage{subfigure}
\usepackage{supertabular}
\usepackage{placeins}
\usepackage{amsthm} % assumes amsmath package installed
%\usepackage{tabularx}
%\usepackage{epsfig, floatflt,amssymb} 
%\usepackage{moreverb} %% pour le verbatim en boite
%\usepackage{multirow} %% pour regrouper un texte sur plusieurs lignes dans une table
%\usepackage{url} %% pour citer les url par \url
%\usepackage[all]{xy} %% pour la barre au dessus des symboles
%\usepackage{shorttoc} %% pour plusieurs tables des matières par la commande \shorttableofcontents{Titre}{profondeur}.
%\usepackage{textcomp} %% pour le symbol pour mille par \textperthousand.

\usepackage{hyperref} %% pour la transformation en PDF, ça permet d'obtenir des liens sur les sections ...
\hypersetup{
    colorlinks,%
    citecolor=black,%
    filecolor=black,%
    linkcolor=black,%
    urlcolor=black
}
%\usepackage{slashbox} %% pour couper les colonnes des tableaux en diagonale 
%\usepackage{showkeys} %% pour voir les labels

%\usepackage{textcomp}
%\usepackage[right]{eurosym}
%\usepackage{eurosans} %%pour le symbole \euro
%\usepackage{epic,eepic}
%\usepackage{times}%times,palatino}
%\usepackage{palatino}
%\usepackage{helvetica} : tjrs le pb
%\usepackage{utopia}:pb de cesure

\usepackage{misc/multibib}
%\usepackage{misc/splitbib}



%% macro/racourcis por les symboles et commandes usuelles
% IEEEtran class scratch pad registers
% dimen
\newdimen\trantmpdimenA
\newdimen\trantmpdimenB
% count
\newcount\trantmpcountA
\newcount\trantmpcountB
% token list
\newtoks\trantmptoksA

% \PARstart
% Definition for the big two line drop cap letter at the beginning of the
% first paragraph of journal papers. The first argument is the first letter
% of the first word, the second argument is the remaining letters of the
% first word which will be rendered in upper case.
% In V1.6 this has been completely rewritten to:
% 
% 1. no longer have problems when the user begins an environment
%    within the paragraph that uses \PARstart.
% 2. auto-detect and use the current font family
% 3. revise handling of the space at the end of the first word so that
%    interword glue will now work as normal.
% 4. produce correctly aligned edges for the (two) indented lines.
% 
% We generalize things via control macros - playing with these is fun too.
% 
% the number of lines that are indented to clear it
\def\PARstartDROPLINES{2}
% minimum number of lines left on a page to allow a \@PARstart
% Does not take into consideration rubber shrink, so it tends to
% be overly cautious
\def\PARstartMINPAGELINES{2}
% the depth the letter is lowered below the baseline
% the height (and size) of the letter is determined by the sum
% of this value and the height of a capital "T" in the current
% font. It is a good idea to set this value in terms of the baselineskip
% so that it can respond to changes therein.
\def\PARstartDROPDEPTH{1.1\baselineskip}
% This is the separation distance from the drop letter to the main text.
% Lengths that depend on the font (i.e., ex, em, etc.) will be referenced
% to the font that is active when PARstart is called. 
\def\PARstartSEP{0.15em}


% definition of \PARstart
% THIS IS A CONTROLLED SPACING AREA, DO NOT ALLOW SPACES WITHIN THESE LINES
% 
% The token \PARstartfont will be globally defined after the first use
% of \PARstart and will be a font command which creates the big letter
% The first argument is the first letter of the first word and the second
% argument is the rest of the first word(s).
\def\PARstart#1#2{\par{%
% if this page does not have enough space, break it and lets start
% on a new one
%\tranneedspace{\PARstartMINPAGELINES\baselineskip}{\relax}%
% calculate the desired height of the big letter
% it extends from the top of a capital "T" in the current font
% down to \PARstartDROPDEPTH below the current baseline
\settoheight{\trantmpdimenA}{T}%
\addtolength{\trantmpdimenA}{\PARstartDROPDEPTH}%
% extract the name of the current font in bold
% and place it in \PARstartFONTNAME
\def\PARstartGETFIRSTWORD##1 ##2\relax{##1}%
{\bfseries%
\edef\PARstartFONTNAMESPACE{\fontname\font\space}%
\xdef\PARstartFONTNAME{\expandafter\PARstartGETFIRSTWORD\PARstartFONTNAMESPACE\relax}}%
% define a font based on this name with a point size equal to the desired
% height of the drop letter
\font\PARstartsubfont\PARstartFONTNAME\space at \trantmpdimenA\relax%
% save this value as a counter (integer) value (sp points)
\trantmpcountA=\trantmpdimenA%
% now get the height of the actual letter produced by this font size
\settoheight{\trantmpdimenB}{\PARstartsubfont\MakeUppercase{#1}}%
% If something bogus happens like the first argument is empty or the
% current font is strange, do not allow a zero height.
\ifdim\trantmpdimenB=0pt\relax%
\typeout{** WARNING: PARstart drop letter has zero height! (line \the\inputlineno)}%
\typeout{ Forcing the drop letter font size to 10pt.}%
\trantmpdimenB=10pt%
\fi%
% and store it as a counter
\trantmpcountB=\trantmpdimenB%
% Since a font size doesn't exactly correspond to the height of the capital
% letters in that font, the actual height of the letter, \trantmpcountB,
% will be less than that desired, \trantmpcountA
% we need to raise the font size, \trantmpdimenA 
% by \trantmpcountA / \trantmpcountB
% But, TeX doesn't have floating point division, so we have to use integer
% division. Hence the use of the counters.
% We need to reduce the denominator so that the loss of the remainder will
% have minimal affect on the accuracy of the result
\divide\trantmpcountB by 200%
\divide\trantmpcountA by \trantmpcountB%
% Then reequalize things when we use TeX's ability to multiply by
% floating point values
\trantmpdimenB=0.005\trantmpdimenA%
\multiply\trantmpdimenB by \trantmpcountA%
% \PARstartfont is globaly set to the calculated font of the big letter
% We need to carry this out of the local calculation area to to create the
% big letter.
\global\font\PARstartfont\PARstartFONTNAME\space at \trantmpdimenB%
% Now set \trantmpdimenA to the width of the big letter
% We need to carry this out of the local calculation area to set the
% hanging indent
\settowidth{\global\trantmpdimenA}{\PARstartfont\MakeUppercase{#1}}}%
% end of the isolated calculation environment
% add in the extra clearance we want
\advance\trantmpdimenA by \PARstartSEP%
% \trantmpdimenA has the width of the big letter plus the
% separation space and \PARstartfont is the font we need to use
% Now, we make the letter and issue the hanging indent command
% The letter is placed in a box of zero width and height so that other
% text won't be displaced by it.
\noindent\hangindent\trantmpdimenA\hangafter=-\PARstartDROPLINES%
\makebox[0pt][l]{\hspace{-\trantmpdimenA}\raisebox{-\PARstartDROPDEPTH}[0pt][0pt]{\PARstartfont\MakeUppercase{#1}}}\MakeUppercase{#2}}


% V1.6 \CMPARstart is no longer needed as \PARstart now uses whatever
% the current font family is.
% \CMPARstart is provided here for backward compatability.
\let\CMPARstart=\PARstart


\newcommand{\mPage}{\newpage}

\newcommand{\mMatB}[1]{\left[ \begin{array}{#1}}
\newcommand{\mMatE}{\end{array} \right]}

% --- QUALITATIF --- %
\newcommand{\meb}[1]{\bar{\me{#1}}}
\newcommand{\mH}{\mbf{H}}
\newcommand{\mHa}{\widehat{\mbf{H}}}
\newcommand{\mAq}{\mbf{A_q}}
\newcommand{\mAnq}{\overline{\mAq}}
\newcommand{\msupp}{\textrm{ supp }}

\newcommand{\mindinv}[1]{\oplus #1}
\newcommand{\minv}[1]{^{\mindinv{#1}}}


\newcommand{\ssi}{\emph{ssi }}

\newcommand{\eq}[1]{(\ref{#1})}
\newcommand{\mbf}[1]{{\mathbf{#1}}}
\newcommand{\deriv}[1]{d#1}

\newcommand{\mCri}[1]{\mbf{\mathcal{C}_{#1}}}

\newcommand{\md}{\Delta}

% --- Notation --- %
\newcommand{\mProjec}[1]{\mbf{Proj}(#1)}
\newcommand{\mP}[1]{\mbf{P}_{#1}}
\newcommand{\mPaug}[1]{\mbf{P}_{#1}^{A}}
\newcommand{\mProj}[2]{\mbf{P}_{#1 \dots #2}}
\newcommand{\mTransfMatrix}[3]{\tensor[^{#1}]{\mathbf{#2}}{_{#3}}}

\newcommand{\mC}[1]{\mbf{C_{#1}}}

\newcommand{\mL}[1]{\mbf{L_{#1}}}
\newcommand{\mpL}[1]{\mbf{L_{#1}^{+}}}
\newcommand{\maL}[1]{\mbf{\widehat{L}_{#1}}}
\newcommand{\mapL}[1]{\mbf{\widehat{L_{#1}^{+}} }}
\newcommand{\mpaL}[1]{\mbf{\widehat{L}_{#1}^{+}}}
\newcommand{\mLaug}[1]{\mbf{L_{#1}^{A}}}
\newcommand{\mpLaug}[1]{\mbf{L_{#1}^{A+}}}
\newcommand{\mtL}[1]{\mbf{\widetilde{L_{#1}}}}
\newcommand{\mtpL}[1]{\mbf{\widetilde{L_{#1}}^{+}}}

\newcommand{\mJ}[1]{\mbf{J}_{#1}}
\newcommand{\mpJ}[1]{\mbf{J}_{#1}^{+}}
\newcommand{\maJ}[1]{\mbf{\widehat{J}_{#1}}}
\newcommand{\mapJ}[1]{\mbf{\widehat{J_{#1}^{+}} }}
\newcommand{\mpaJ}[1]{\mbf{\widehat{J}_{#1}^{+}}}
\newcommand{\mJaug}[1]{\mbf{J}_{#1}^{A}}
\newcommand{\mpJaug}[1]{\mbf{J}_{#1}^{A+}}
\newcommand{\mtJ}[1]{\widetilde{\mbf{J}_{#1}}}
\newcommand{\mtpJ}[1]{\widetilde{\mbf{J}_{#1}}^{+}}

\newcommand{\me}[1]{\mbf{e}_{#1}}
\newcommand{\medot}[1]{\mbf{\dot{e}}_{#1}}
\newcommand{\meddot}[1]{\mbf{\ddot{e}_{#1}}}
\newcommand{\mde}[1]{\mbf{\md e}_{#1}}
\newcommand{\meaug}[1]{\mbf{e}_{#1}^{A}}
\newcommand{\medotaug}[1]{\mbf{\dot{e}}_{#1}^{A}}
\newcommand{\medotd}[1]{\mbf{\dot{e}}_{#1}^*}

\newcommand{\mte}[1]{\widetilde{\mbf{e}_{#1}}}
\newcommand{\mtedot}[1]{\widetilde{\dot{\mbf{e}}_{#1}}}
\newcommand{\mteddot}[1]{\widetilde{\ddot{\mbf{e}}_{#1}}}
\newcommand{\mtedotd}[1]{\widetilde{\dot{\mbf{e}}_{#1}}^*}

\newcommand{\ms}[1]{\mbf{s}_{#1}}
\newcommand{\msdot}[1]{\mbf{\dot{s_{#1}}}}
\newcommand{\msd}[1]{\mbf{s^{*}_{#1}}}

\newcommand{\mq}{\mbf{q}}
\newcommand{\mqdot}{\mbf{\dot{q}}}
\newcommand{\mdq}[1]{\mbf{\md q_{#1}}}
\newcommand{\mqr}{\mbf{\rho}}
\newcommand{\mqrlim}{\mbf{\rho}_{lim}}
\newcommand{\mqrdot}{\mbf{\dot{\rho}}}
\newcommand{\mQmax}[1]{\mbf{Q_{#1}}^{MAX}}
\newcommand{\mQmin}[1]{\mbf{Q_{#1}}^{MIN}}

\newcommand{\mr}{\mbf{r}}
\newcommand{\mrdot}{\mbf{\dot{r}}}
\newcommand{\mscrew}{\mbf{v}}

\newcommand{\mres}{\mbf{\rho}}

\newcommand{\mevit}{\mbf{g}^{JointLim}}

\newcommand{\mpot}[1]{V_{#1}}
\newcommand{\mFpot}[1]{\mbf{g}_{#1}}
\newcommand{\mg}{\mbf{g}}
\newcommand{\mgrad}[1]{{\nabla}_{#1}}
\newcommand{\mgradT}[1]{{\nabla}_{#1}^{\top}}

\newcommand{\qlim}[1]{\bar{q}_{#1}}
\newcommand{\qmin}[1]{\bar{q}^{\min}_{#1}}
\newcommand{\qmax}[1]{\bar{q}^{\max}_{#1}}
\newcommand{\qlmin}[1]{\bar{q}^{\min}_{\ell#1}}
\newcommand{\qlmax}[1]{\bar{q}^{\max}_{\ell#1}}

\newcommand{\mqlim}[1]{\qlim{#1}}
\newcommand{\mqmin}[1]{\qmin{#1}}
\newcommand{\mqmax}[1]{\qmax{#1}}
\newcommand{\mqlmin}[1]{\qlmin{#1}}
\newcommand{\mqlmax}[1]{\qlmax{#1}}

\newcommand{\mvv}{\hbox{\boldmath $v$}}
\newcommand{\mvomega}{\hbox{\boldmath $\omega$}}



\newcommand{\mI}{\mbf{I}}

\newcommand{\derive}[2]{\frac{\deriv{#1}}{\deriv{#2}}}
\newcommand{\dpartial}[2]{\frac{\partial{#1}}{\partial{#2}}}

\newcommand{\mdeuxD}{2-D }
\newcommand{\mtroisD}{3-D }
\newcommand{\mdeuxDd}{2-1/2-D }



\newcommand{\att}[1]{\underline{#1}}

% \newcommand{\argmax}[1]{\arg\!\max_{\!\!\!\!\!\!\!\!\!\!\!\!\!\!\!\!#1}}
\newcommand{\argmax}[1]{\mathop{\arg\!\max}_{#1}}
%\newcommand{\argmax}[1]{\mathop{\mbox{argmax}}_{#1}}
\newcommand{\mdint}{\int\!\!\!\int}


\newcommand{\mJp}[1]{\mbf{J}_{#1}^+}
\newcommand{\medotc}[1]{\medot{#1}^*}
\newcommand{\medoth}[1]{\mbf{\widehat{\dot{e}}}_{#1}}
\newcommand{\medothp}[2]{\medoth{#1|#2}}
\DeclareMathOperator*{\argmin}{arg\,min\,} 


%------------------------------------------
% Taille figure
%------------------------------------------

\newcommand{\figw}{0.9\textwidth}
\newcommand{\figh}{4cm}
\newcommand{\minipw}{0.5\textwidth}
%------------------------------------------
% Txt
%------------------------------------------

\newcommand{\clic}{\emph{clic}}
\newcommand{\via}{\emph{via}}
\newcommand{\ie}{c'est-à-dire}
\newcommand{\eg}{par exemple }
\newcommand{\apriori}{\emph{a priori}}
\newcommand{\ihand}{\emph{eye-in-hand}}
\newcommand{\thand}{\emph{eye-to-hand}}
\newcommand{\etal}{\emph{et al.}}
\newcommand{\euclidien}{euclidien}
\newcommand{\euclidienne}{\euclidien ne}
\newcommand{\gaussien}{gaussien}
\newcommand{\gaussienne}{\gaussien ne}
\newcommand{\jacobien}{jacobien}
\newcommand{\jacobienne}{\jacobien ne}

%------------------------------------------
% Acronymes
%------------------------------------------
\newcommand {\afm}{\textsc{AFM}}
\newcommand {\anso}{\textsc{Anso}}
\newcommand {\approche}{\textsc{Approche}}
\newcommand {\arph}{\textsc{Arph}}
\newcommand {\aviso}{\textsc{Aviso}}
\newcommand {\friend}{\textsc{Friend}}
\newcommand {\halo}{\textsc{Halo}}
\newcommand {\kares}{\textsc{Kares}}
\newcommand {\lares}{\textsc{Lares}}
\newcommand {\manus}{\textsc{Manus}}
\newcommand {\sam}{\textsc{Sam}}
\newcommand {\victoria}{\textsc{Victoria}}
\newcommand {\musiic}{\textsc{Musiic}}
%------------------------------------------
% Cameras
%------------------------------------------

%Camera mobile
% dans les equations
\newcommand{\cm}{e}
\newcommand{\Cm}{\mathbf{e}}
% dans le texte
\newcommand{\txtcm}{$\cm${}}
\newcommand{\txtCm}{$\Cm${}}

%Camera fixe
% dans les equations
\newcommand{\cf}{d}
\newcommand{\Cf}{\mathbf{d}}
% dans le texte
\newcommand{\txtcf}{$\cf${}}
\newcommand{\txtCf}{$\Cf${}}

%------------------------------------------
% Reperes
%------------------------------------------
\newcommand{\txtRc}{$\mathcal{R}_c$}
% camera deportee
\newcommand{\txtRcf}{$\mathcal{R}_{\cf}${}}
% camera embarquee
\newcommand{\txtRcm}{$\mathcal{R}_{\cm}$}
% repere monde
\newcommand{\txtRw}{$\mathcal{R}_{w}$}
% repere base
\newcommand{\txtRb}{$\mathcal{R}_{b}$}
% repere lie a l'objet
\newcommand{\txtRo}{$\mathcal{R}_{o}$}
% repere lie a la pince
\newcommand{\txtRg}{$\mathcal{R}_{g}$}
% repere lie a un solide
\newcommand{\txtRs}{$\mathcal{R}_{s}$}


\newcommand{\txtRef}[1]{$\equaRef{#1}$}
\newcommand{\equaRef}[1]{\mathcal{R}_{#1}}

%------------------------------------------
% Matrices de passage
%------------------------------------------
\newcommand{\txtM}[2]{$\equaM{#1}{#2}$}
\newcommand{\equaM}[2]{^{#1}\mathbf{M}_{#2}}

%----------------------------------------------
% composantes de la matrice homogene
%----------------------------------------------

\newcommand{\equaT}[2]{{^{#1}\mathbf{t}}_{#2}}
\newcommand{\equaR}[2]{^{#1}\mathbf{R}_{#2}}
\newcommand{\txtT}[2]{$\equaT{#1}{#2}$}
\newcommand{\txtR}[2]{$\equaR{#1}{#2}$}
%-------------------------------------------
%  Geometrie epipolaire
%-------------------------------------------

% Matrice essentielle
% dans les equations\emph{i.e.}
\newcommand{\cfEcm}{^{\cf}\mathbf{E}_{\cm}}
\newcommand{\cmEcf}{^{\cm}\mathbf{E}_{\cf}}
% dans le texte
\newcommand{\txtcfEcm}{$\cfEcm${}}
\newcommand{\txtcmEcf}{$\cmEcf${}}

% Matrice fondamental
% dans les equations\emph{i.e.}
\newcommand{\cfFcm}{^{\cf}\mathbf{F}_{\cm}}
\newcommand{\cmFcf}{^{\cm}\mathbf{F}_{\cf}}
% dans le texte
\newcommand{\txtcfFcm}{$\cfFcm${}}
\newcommand{\txtcmFcf}{$\cmFcf${}}


%-------------------------------------------
%          points 2D
%-------------------------------------------

% points dans le repere de la camera fixe
\newcommand{\pcf}{^{\cf}\mathbf{p}}
\newcommand{\txtpcf}{$\pcf${}}

% points dans le repere de la camera mobile
\newcommand{\pcm}{^{\cm}\mathbf{p}}
\newcommand{\txtpcm}{$\pcm${}}

% meme chose avec epipole
\newcommand{\ecf}{^{\cf}\mathbf{e}}
\newcommand{\txtecf}{$\ecf${}}
\newcommand{\ecm}{^{\cm}\mathbf{e}}
\newcommand{\txtecm}{$\ecm${}}

% clic
\newcommand{\pclic}{{^d\mathbf{x}_{clic}}}

% point central
\newcommand{\pcentral}{{\mathbf{x}_c}}

% point unit reel frame
\newcommand{\purf}[2]{^{#2}\mathbf{\tilde{x}}_{#1}}
% point unit modele frame
\newcommand{\pumf}[2]{^{#2}\mathbf{x}_{#1}}
% point unit reel frame
\newcommand{\pur}[1]{\mathbf{\tilde{x}}_{#1}}
% point unit modele frame
\newcommand{\pum}[1]{\mathbf{x}_{#1}}


%--------------------------------------------
%       Points 3d
%--------------------------------------------

% points dans le repere de la camera fixe
\newcommand{\Pcf}{^{\cf}\mathbf{X}}
\newcommand{\txtPcf}{$\Pcf${}}

% points dans le repere de la camera mobile
\newcommand{\Pcm}{^{\cm}\mathbf{X}}
\newcommand{\txtPcm}{$\Pcm${}}

% points dans le repere de objet 
\newcommand{\Po}{^{o}\mathbf{X}}
\newcommand{\txtPo}{$\Po${}}

% points dans un repere qq
% un argument qui est la lettre a mettre en exposant en avant de P
\newcommand{\PProjRef}[1]{^{#1}\mathbf{\Pseul}}
\newcommand{\PEucliRef}[1]{^{#1}\bar{\mathbf{\Pseul}}}
\newcommand{\PCalRef}[1]{^{#1}\mathcal{\Pseul}}
\newcommand{\Pseul}{X}
% points qq dans un repere qq
% deux argument un argument qui est la lettre et le second est son exposant
\newcommand{\PNameRef}[2]{^{#2}\mathcal{#1}}


%----------------------------------------------
% Vecteur directeur
%----------------------------------------------
\newcommand{\vdirclic}{\mathbf{w}}


%----------------------------------------------
% Proba
%----------------------------------------------
\newcommand{\vetat}{\mathbf{\eta}}
\newcommand{\Mcov}{\mathbf{P}}
\newcommand{\distance}{D}
%----------------------------------------------
% Homographies
%----------------------------------------------
\newcommand{\equaA}{\mathbf{A}}
\newcommand{\equaK}{\mathbf{K}}

%----------------------------------------------
% Frame
%----------------------------------------------
\newcommand{\iRef}[1]{{\mathbf{i}}}
\newcommand{\jRef}[1]{{\mathbf{j}}}
\newcommand{\kRef}[1]{{\mathbf{k}}}
\newcommand{\uRef}[1]{{\mathbf{u}}}
\newcommand{\vRef}[1]{{\mathbf{v}}}

%----------------------------------------------
% les fonctions implicites
%----------------------------------------------

\newcommand{\paramTri}{\mathcal{P}}
\newcommand{\paramBi}{\mathbf{\rho}}

%----------------------------------------------
%               Les coniques
%----------------------------------------------
\newcommand{\conic}{\mathcal{C}}
% coefficients de la conic
\newcommand{\cParam}{k}
\newcommand{\txtcParam}{$\cParam${}}

% coefficient estimes de la conic
\newcommand{\cParamEst}{\widehat{\cParam}}
\newcommand{\txtcParamEst}{$\cParamEst${}}
% parametre de la conique
\newcommand{\cParamInd}[1]{\cParam_{#1}}
\newcommand{\cParamEstInd}[1]{\widehat{\cParam}_{#1}}
% matrice correspondant a la quadrique
\newcommand{\MatConic}{\mathbf{C}}
% matrice de la forme quadratique
\newcommand{\matConic}{\mathbf{c}}
%-----------------------------------------------
%          Les quadriques
%-----------------------------------------------

\newcommand{\quadric}{\mathcal{Q}}

% matrice correspondant a la quadrique
\newcommand{\MatQuad}{\mathbf{\Gamma}}
% matrice de la forme quadratique
\newcommand{\matQuad}{\mathbf{\Gamma_{33}}}
% les determinant assosice
\newcommand{\DetQuad}{\Delta_{\MatQuad}}
\newcommand{\detQuad}{\Delta_{\matQuad}}
%les rangs
\newcommand{\RangQuad}{R_{\MatQuad}}
\newcommand{\rangQuad}{R_{\matQuad}}

% parametre de la quadrique
\newcommand{\qParam}{q}
\newcommand{\txtqParam}{$\qParam${}}

% parametre de la quadrique dans un repere specifie
\newcommand{\qParamRef}[1]{{^{#1}}\qParam}
\newcommand{\qParamRefInd}[2]{{^{#1}}\qParam_{#2}}
\newcommand{\qParamInd}[1]{\qParam_{#1}}
\newcommand{\qParamEstInd}[1]{\widehat{\qParam}_{#1}}


%---------------------------------------------
\newcommand{\normK}{\sqrt{\sum_i {\cParamInd{ij}}^2 }}
\newcommand{\normKprime}{\sqrt{
({{\qParamInd{7}}^2\!-\!\qParamInd{1}\qParamInd{10}})^2\!+\!
({\qParamInd{8}}^2\!-\!\qParamInd{2}\qParamInd{10})^2\!+\!
(\qParamInd{7}\qParamInd{8}\!-\!\qParamInd{4}\qParamInd{10})^2\!+\!
(\qParamInd{7}\qParamInd{9}\!-\!\qParamInd{5}\qParamInd{10})^2\!+\!
(\qParamInd{7}\qParamInd{7}\!-\!\qParamInd{6}\qParamInd{10})^2\!+\!
({\qParamInd{9}}^2\!-\! \qParamInd{3}\qParamInd{10})^2}}
%---------------------------------------------
\newcommand{\Jconic}{\mathbf J_\conic}
%---------------------------------------------
\newcommand{\fcout}{\chi^2}

%--------------------------------------------
\newcommand{\entropy}{I}
\newcommand{\critere}{\tau}



\usepackage{these}
\usepackage{color}
\usepackage{multicol}
\usepackage{amsmath}

\newcites{sec}{Publications}

%% 2= subsection, 3=subsubsection
\setcounter{secnumdepth}{3}  %% Avec un numero.
\setcounter{tocdepth}{2}     %% Visibles dans la table des matieres

\definecolor{orange}{rgb}{0.99,0.69,0.07}
\definecolor{lightgray}{gray}{0.85}

% ---------------------------------------------------------------------------- %
% --- ENTETE ----------------------------------------------------------------- %
% ---------------------------------------------------------------------------- %
\begin{document}
%\addtocounter{page}{-1}%ça c'est pour revenir à 0
%\fontfamilly{phv}

%%  1ere de Couverture:



\thispagestyle{empty}
\begin{center}
  {\LARGE 
\textbf{THÈSE}\\[\baselineskip]
  }
  présentée\\[\baselineskip]
  {\Large
\textbf{devant l'Institut National des Sciences Appliquées de Toulouse}\\[\baselineskip]
  }
  pour obtenir\\[\baselineskip]
  {\large 
le grade de~: \emph{\textsc{Docteur de l'Institut National des Sciences Appliquées de Toulouse}}\\
Sp\'ecialit\'e \textsc{Systèmes}\\[\baselineskip]
  }
  par\\[\baselineskip]
  {\large 
Sovannara HAK\\[\baselineskip]
  }
Équipe d'accueil~: LAAS-CNRS - Équipe {\sc GEPETTO}\\
École Doctorale~: Edsys\\
  Titre de la thèse~:\\[\baselineskip]
  {\LARGE     \textbf{   Reconnaissance de t\^aches par commande inverse\\}}
  \vfill
  Projet R-Blink Contrat ANR-08JCJC-0075-01\\
  \vfill
  Soutenance prévue le 02/11/2011 devant la commission d'examen\\[\baselineskip]
\end{center}

\begin{center}
\begin{tabular}{r@{\protect\hspace{0.5cm}}ll@{\protect\hspace{1.0cm}}l}
%M.~:&Président&DU JURY &Président\\
%MM.~:
&Aude&BILLARD\\
&Bernard&ESPIAU\\
&Nicolas&MANSARD\\
&Olivier&STASSE\\
&Rachid&ALAMI\\%&Président\\
&Jean-Paul&LAUMOND\\

\end{tabular}
\end{center}


%  \whitepage










%\titre{~\\
%\huge Enchainement de tâches}

%\soutenue
%%   Laisser cette ligne en commentaire sauf pour la version finale.
%%   (la premiere page contiendra "a soutenir le ..." 
%%   au lieu de "soutenue le ...")


%% Les différents champs de la couverture...
%\datesout{Apres-Demain (deja?)}
%\Auteur{Nicolas}{Mansard}
%\Equipe{Projet \sc{lagadic} - IRISA}{Université de Rennes 1}
%% La composition du jury : prénom, nom, titre 
%\President{Président}{du jury}        %% le président du jury
%\Advisor{Fran\c cois}{CHAUMETTE}
%\Rapporteur{Jean-Paul}{LAUMOND}
%\Rapporteur{Oussama}{KHATIB}
%% Si vous avez N rapporteurs ça marche toujours...
%\Examinateur{Bruno}{SICILIANO}
%\Examinateur{Etienne}{DOMBRE}
%\Examinateur{Albert}{BENVENISTE}
%% idem...

%\ordre{299792458.6}  %% le numéro d'ordre donné par la Sco

%\makethese    %% crée la couverture.


%% une page blanche (deuxième de couverture)


~\thispagestyle{empty}%\newpage
\section*{Remerciements}
~\thispagestyle{empty}%\newpage
Je remercie tout d'abord Olivier Stasse, Nicolas Mansard et Jean-Paul Laumond
mes directeurs de thèse pour m'avoir donné l'opportunité d'effectuer cette thèse
dans le domaine de la robotique humanoïde au sein
d'une excellente équipe, pour leurs conseils, disponibilité.
Merci également à Abderrahmane Kheddar pour m'avoir donner la chance de 
faire mon stage de fin d'étude au Japon et ainsi de rencontrer mes futurs
directeurs de thèse.\\

Je remercie également les membres de mon jury de thèse:
Aude Billard et Bernard Espiau, pour avoir accepté d'être
mes rapporteurs, d'avoir pris le temps d'étudier mon manuscrit
et d'en tirer des commentaires intéressant. Olivier Stasse,
et Rachid Alami, membres du jury, pour leurs intêrets pour mes travaux
et la discussion qui en a découlée lors de ma soutenance.\\

Je remercie également les permanents de l'équipe Gepetto:
Florent Lamiraux, Michel Taix, Philippe Souères.
Ensuite je remercie (et souhaite un bon courage pour
ceux qui n'ont pas encore soutenu) les doctorants du groupe RIA que j'ai pu cotoyer
lors de mon séjour au LAAS: Naveed, Mathieu, Ali, Assia, Mokhtar, Diego, Redouane (my frienemy).
Un autre merci revient aux personnes que j'ai pu voir et revoir au JRL ou au LAAS au cours de ma thèse:
Claire, Diane, Mitsu, Adrien, Jean-Rémy, Paul et Amélie.

Les différentes personnes qui m'ont beaucoup aidé en effectuant
les relectures des différents chapitres composant mon manuscrit
de thèse afin de déceler fautes d'orthographes, formulations
douteuses et erreurs: Antonio, Wassim, François et 
Roza\footnote{Bonus dédication inconcevable}.\\

Les personnes qui m'ont fait répéter ma soutenance,
Antonio, Wassim et surtout Nicolas. Sans tes conseils et tes remarques,
la présentation ne se serait pas aussi bien passée.
Je remercie toutes les personnes qui ont assisté à ma soutenance, merci Marc et François pour avoir fait
le déplacement, ma famille pour le soutien et la préparation du pot.\\

Je remercie chaleureusement tout les membres du gang Gepetto, dont les capacités
de manipulations et d'intimidations ne sont plus à prouver,
pour tout les moments partagés au labo et ailleurs:
Manish, Oussama, David, Mathieu, Pancho, Minh, Anh, Alireza, Maxime, 
\emph{los hombres valientes} Sébastien, Thomas, Nicolas P., Duong y se\~nor Oscar, 
mes professeurs de libanais Antonio\footnote{Bonus moustache} et 
Layale\footnote{Bonus saucisse} (gr\^ace à vous et au vocabulaire que j'ai acquis, 
j'ai peut être une chance d'épouser Shakira)
ainsi que bien evidemment, Wassim et Wassima les gepettistes par adoption.\\

Enfin, j'envoie mille merci à toutes les personnes extérieures au laboratoire dont le soutien, sous la
forme de gestes ou de paroles, a été grandement aprécié:
Roza, Mehdi, Marc, Benoit, François.\\

\noindent Les travaux présentés dans ce manuscrit font partie
du projet R-Blink contrat ANR-08JCJC-0075-01.\\


\thispagestyle{empty}\newpage
\thispagestyle{empty}\newpage

\nocitesec{hak11}
\nocitesec{hak10a}
\nocitesec{hak10}
\nocitesec{ramos11}

\bibliographystylesec{plain}
\bibliographysec{misc/bibfr}
\thispagestyle{empty}\newpage
\thispagestyle{empty}
%---------------------------------------------------------------------------- %
%--- INDEX ------------------------------------------------------------------ %
%---------------------------------------------------------------------------- %
~\newpage
\setcounter{page}{1}
%~\newpage
\addcontentsline{toc}{chapter}{Table des mati\`eres}
\markboth{Table des mati\`eres}{Table des mati\`eres}
 
\tableofcontents 

% ---------------------------------------------------------------------------- %
% --- CHAPITRES -------------------------------------------------------------- %
% ---------------------------------------------------------------------------- %
\pagestyle{empty}
 \FloatBarrier
\newpage
\section*{Notations mathématiques et conventions}
Dans ce manuscrit, les scalaires sont notés en minuscules comme par exemple $s$.
Les vecteurs sont notés en minuscule grasses: $\mbf{v}$ et les matrices en majuscules
grasses: $\mbf{M}$. Sauf mention contraire, $n$ représente le nombre de degrés de liberté d'un 
robot. $\mbf{I}$ est une matrice identité. Une étoile en exposant dénote une
valeur désirée: $e^*$ et un accent circonflexe dénote une valeur observée: $\hat{p}$.
Le point et le double point sont utilisés pour noter les vitesses et les 
accélérations: $\dot{\mbf{v}}$, $\ddot{\mbf{a}}$.
Des notations supplémentaires seront présentées quand elles seront nécessaires.

~\newpage
 \chapter{Introduction}
  %Introduction générale sur la robotique. Pourquoi étudier le mouvement,
les intentions du robot, interaction homme robots.

\pagestyle{plain} 
 %\chapter{Introduction}
%Le d\'eveloppement de la robotique de service stimule la recherche
%dans l'int\'eraction homme-robot. Dans ce contexte, la compr\'ehension
%des actions du robot et de l'être humain \`a partir d'une observation est un d\'efi.
%Bien que les actions volontaires apparaissent au niveau de la planification,
%leurs r\'ealisations s'effectuent dans le monde r\'eel par le biais 
%des mouvements. Comment reconna\^itre une action \`a partir
%d'un mouvement observ\'e? D\'efinir une m\'ethode pour reconna\^itre automatiquement
%l'objectif vis\'e par un robot ex\'ecutant un mouvement donn\'e est un probl\`eme 
%crucial. Les robots manipulateurs mobiles,
%sont construits de façon \`a \'etablir une s\'eparation claire entre les fonctions de navigation et 
%les fonctions de manipulation. Par exemple, le robot PR2 (voir Fig.~\ref{fig:pr2})
%possède des bras uniquement dédié à la manipulation
%et une base mobile uniquement dédié à la navigation. Par cons\'equent, la question de la reconnaissance
%d'actions peut \^etre simple. De ce point de vue, un robot humano\"ide
%peut aussi \^etre divis\'e en deux parties distinctives: les jambes
%et le haut du corps, qui correspondraient aux fonctions de navigation
%et aux fonctions de manipulation.

%\begin{figure}[t]
%  \begin{center}
%    \includegraphics[width=0.4\linewidth]{figures/chapitre5/pr2.ps}
%  \end{center}
%  \caption{Le robot PR2 est construit de façon à séparer les fonctions de navigation
%  et les fonctions de manipulation.}
%  \label{fig:pr2}
%\end{figure}

%\subsection{Problem statement: disambiguating motions in embodied actions.}
%\begin{figure*}[t]
%  \centering
%  \makeatletter
%  \renewcommand{\@thesubfigure}{Scenario \thesubfigure:\hskip\subfiglabelskip}
%  \makeatother

%  \subfigure[La t\^ache globale \emph{Donne moi la balle} est d\'ecompos\'e en
%  s\'equences de sous-t\^aches \lbrack localiser la balle\rbrack, \lbrack marcher vers la balle\rbrack, 
%  \lbrack attraper la balle\rbrack, \lbrack localiser l'op\'erateur\rbrack, \lbrack marcher vers l'op\'erateur\rbrack, 
%  et \lbrack donner la balle\rbrack. Les mouvements \lbrack marcher vers\rbrack, \lbrack attraper\rbrack, 
%  \lbrack donner\rbrack~apparaissent comme une s\'equence structurant l'action (extrait de~\cite{yoshida07}).]{
%  \makebox[\linewidth]{
%  \begin{tabular}{c@{}c@{}c@{}c@{}c@{}}
%    \includegraphics[height=2.4cm]{figures/chapitre5/purpleBall1.ps}&
%    \includegraphics[height=2.4cm]{figures/chapitre5/purpleBall2.ps}&
%    \includegraphics[height=2.4cm]{figures/chapitre5/purpleBall3.ps}&
%    \includegraphics[height=2.4cm]{figures/chapitre5/purpleBall4.ps}&
%    \includegraphics[height=2.4cm]{figures/chapitre5/purpleBall5.ps}\\
%  \end{tabular}
%  \label{fig:introExample:purpleBall}
%  }
%  }
%  \subfigure[Pour attraper la balle situ\'ee entre ses pieds, le robot doit s'en \'eloigner en faisant quelques pas
%  en arri\`ere. Dans cette exp\'erience, la partie \emph{reculer} n'est pas g\'er\'e par un module de logiciel.
%  Elle fait partie int\'egrale de l'action complexe \emph{attraper} (extrait de~\cite{kanoun10}).]{
%  \makebox[\linewidth]{
%  \begin{tabular}{c@{}c@{}}
%    \includegraphics[height=1.7cm]{figures/chapitre5/graspFeet1.ps}&
%    \includegraphics[height=1.7cm]{figures/chapitre5/graspFeet2.ps}\\
%  \end{tabular}
%  \label{fig:introExample:graspFeet}
%  }
%  }
%  \subfigure[Pour attraper la balle en face de lui (\`a gauche), le robot
%  atteint une posture dans laquelle le bras gauche est utilis\'e pour garder son \'equilibre. 
%  Dans la figure de droite, le robot effectue deux actions en parall\`ele:
%  attraper la balle devant lui tout en attrapant une autre balle se trouvant derri\`ere lui 
%  (bien sûr, la balle qui se trouve derri\`ere le robot a \'et\'e volontairement plac\'e \`a la position 
%  finale de la main gauche illustré dans la figure de gauche). Il est 
%  impossible de diff\'erencier visuellement les deux postures. Cependant,
%  est-il possible de discriminer ces deux mouvements?]{
%  \makebox[\linewidth]{
%  \includegraphics[height=2.4cm]{figures/chapitre5/spotDiff1H.ps}
%  }
%  \label{fig:introExample:spotDiff}
%  }
%  \caption{Exemple d'introduction d'actions complexes.}
%  \label{fig:introExample}
%\end{figure*}
%Cette séparation est illustrée par exemple dans le sc\'enario \emph{Donne moi la balle rose}~\cite{yoshida07},
%effectu\'e par le robot humano\"ide HRP-2 au LAAS-CNRS (Fig.~\ref{fig:introExample:purpleBall}). 
%Pour atteindre son objectif, le robot HRP-2 d\'ecompose sa mission en sous-t\^aches \'el\'ementaires,
%chaque mission est prise en charge par un module logiciel d\'edi\'e. 
%Pour atteindre la balle, le robot doit marcher vers celle-ci.
%\emph{Marcher} s'apparente \`a une action \'el\'ementaire qui est n\'ecessaire dans la 
%r\'esolution du probl\`eme initial (\emph{Donne moi la balle rose}).
%elle est prise en charge par un module de locomotion d\'edi\'e.
%
%Cependant, dans le second sc\'enario (Fig.~\ref{fig:introExample:graspFeet}),
%le HRP-2 doit attraper la balle situ\'e entre ses pieds~\cite{kanoun10}. Pour atteindre
%cet objectif, le robot doit faire quelques pas en arri\`ere avant d'attraper la balle. Dans
%cette exp\'erience, il n'y a pas de module d\'edi\'e \`a la marche. L'action \emph{reculer}, est une
%cons\'equence directe de l'action \emph{attraper}. Cette action est int\'egr\'ee au mouvement 
%du corps, par exemple les jambes contribuent naturellement \`a la r\'ealisation de l'action
%via une g\'en\'eration d'un mouvement complexe. Enfin, la Fig.~\ref{fig:introExample:spotDiff}
%introduit le but de ce chapitre. Dans le cas illustr\'e dans la figure de gauche, le robot
%effectue un mouvement de simple prise (une seule main). Dans le cas illustr\'e dans la figure de 
%droite, le robot effectue deux prises en parall\`ele. L'ambig\"uit\'e de ces mouvements provient
%du r\^ole jou\'e par la main gauche. Dans le premier cas, la main gauche contribue
%\`a la t\^ache de saisie en compensant le d\'es\'equilibre introduit par le d\'eplacement
%de la main droite. Dans le second cas, le bras gauche bouge pour saisir une autre balle. 
%Dans les deux cas, les mouvements sont visuellement proches.

%%%%%%%%%%%%%%%%%%%%%%%%%%%%%%%%%%%%%%%%%%%%%%%%%%%
Le développement de la robotique de service est étroitement liée
à une grande variété de problèmes scientifiques. L'objectif d'une grande partie de ce domaine
de recherche est que l'intégration des robots dans l'environnement 
humain soit suffisamment fiable et efficace pour être utile.
Les exemples de problèmes à 
résoudre concernent la navigation et la perception de l'environnement.
C'est cette classe de problème qui a été tout d'abord résolue et des solutions ont été appliquées avec succès 
sous l'incarnation de robots aspirateurs, nettoyeurs de piscines ou 
robots explorateurs.

À ces fonctionnalités de navigations, des fonctionnalités
de manipulations peuvent être ajoutées.
Les robots mobiles manipulateurs tels 
que le PR2~\cite{PR2} sont équipés de bras 
destinés aux actions de manipulations d'objets situés dans l'environnement
(voir figure~\ref{fig:pr2}).
\begin{figure}[t]
  \begin{center}
    \includegraphics[width=0.3\linewidth]{figures/chapitre5/pr2.ps}
  \end{center}
  \caption{Le robot PR2 de Willow garage.}
  \label{fig:pr2}
\end{figure}
Ces robots sont construits de manière à découpler les fonctionnalités de navigation
et de manipulation, il est ainsi
possible d'associer directement des espaces aux t\^aches à effectuer.
Par exemple, pour la navigation, les roues du robot vont évoluer dans l'espace défini par le plan
du sol. Tandis que pour la manipulation, les actions du bras vont être exprimées
dans l'espace cartésien pour asservir la position de l'effecteur à une position désirée.
Les deux domaines des actions sont bien découplés, et des opérations de 
programmation, debuggage, diagnostique et reconnaissance peuvent alors être traitées facilement en effectuant
des observations dans ces deux espaces.

La même séparation de fonctionnalités peut aussi s'appliquer
aux robots humanoïdes en considérant que les jambes 
remplissent la fonction de navigation et que le haut du corps
s'occupe de la manipulation. La navigation se traduit
alors par une planification d'empreintes de pas que le robot doit suivre dans l'espace à deux
dimensions représentant le sol pour atteindre un objectif. À la manière des robots manipulateurs,
les actions de manipulation seront exprimées dans l'espace cartésien.
Cette approche est illustrée dans le scénario \emph{Donne moi la balle rose}~\cite{yoshida07},
effectué par le robot humanoïde HRP-2 au LAAS-CNRS.
La mission du robot est décomposée en sous-missions qui sont gérées par des modules de logiciels
indépendants: localiser la balle, marcher vers la balle, attraper la balle,
localiser l'opérateur, marcher vers l'opérateur et donner la balle (voir la figure~\ref{fig:purpleBall}).
\begin{figure}[t]
  \centering
  %\begin{tabular}{c@{}c@{}c@{}c@{}c@{}}
  \subfigure{
  \includegraphics[width=0.3\linewidth]{figures/chapitre5/purpleBall1.ps}
  }
  \subfigure{
  \includegraphics[width=0.3\linewidth]{figures/chapitre5/purpleBall2.ps}
  }
  \subfigure{
  \includegraphics[width=0.3\linewidth]{figures/chapitre5/purpleBall3.ps}
  }
  \subfigure{
  \includegraphics[width=0.3\linewidth]{figures/chapitre5/purpleBall4.ps}
  }
  \subfigure{
  \includegraphics[width=0.3\linewidth]{figures/chapitre5/purpleBall5.ps}
  }
  %\end{tabular}
  \caption[\emph{Donne moi la balle}.]{La mission globale \lbrack Donne moi la balle\rbrack est décomposée en
  séquences de sous-missions \lbrack localiser la balle\rbrack, \lbrack marcher vers la balle\rbrack, 
  \lbrack attraper la balle\rbrack, \lbrack localiser l'opérateur\rbrack, \lbrack marcher vers l'opérateur\rbrack, 
  et \lbrack donner la balle\rbrack. Les mouvements \lbrack marcher vers\rbrack, \lbrack attraper\rbrack, 
  \lbrack donner\rbrack~apparaissent comme une séquence structurant l'action (extrait de~\cite{yoshida07}).}
  \label{fig:purpleBall}
\end{figure}

D'une manière générale, le découpage en sous parties fonctionnelles n'est pas toujours
possible à cause de la complexité du robot. C'est le cas par exemple
d'un robot humanoïde où des couplages entre les différents
membre, qui varient en fonction de l'action à réaliser, sont trop nombreux.
Le scénario illustré par la figure~\ref{fig:graspFeet} ressemble beaucoup
au scénario précédent cependant, la méthodologie appliquée pour accomplir la mission est très différente.
Ici, la mission du robot est d'attraper une balle se situant entre ses pieds. Pour
accomplir la mission, le robot doit reculer puis tendre la main vers la balle.
La mission est accomplie en considérant que les pas à effectuer
représentent une extension virtuelle du robot, ramenant la mission à un
problème de cinématique inverse avec contraintes~\cite{kanoun10}.
La marche apparaît alors de façon implicite dans 
l'action \emph{asservir la position de la main vers la position de la balle} qui décrit la mission. 
\begin{figure}
  \centering
  \subfigure{
  \includegraphics[width=0.90\linewidth]{figures/chapitre5/graspFeet1.ps}
    }
  \subfigure{
    \includegraphics[width=0.90\linewidth]{figures/chapitre5/graspFeet2.ps}
    }
  \caption[Planification de pas formulé en problème de cinématique inverse.]{Pour attraper la balle située entre ses pieds, le robot doit s'en éloigner en faisant quelques pas
  en arrière. Dans cette expérience, la partie \emph{reculer} n'est pas gérée par un module de logiciel.
  Elle fait partie intégrale de l'action complexe \emph{attraper la balle} (extrait de~\cite{kanoun10}).}
  \label{fig:graspFeet}
\end{figure}

\section{Problématique}
Dans ce manuscrit, nous cherchons à reconnaître les composantes
du mouvement en cours lorsque le découpage fonctionnel n'est pas possible.
Nous nous intéressons plus particulièrement aux mouvements faisant intervenir
plusieurs objectifs à atteindre.
Dans ce cadre, nous cherchons à montrer que la fonction de t\^aches,
classiquement utilisée pour décrire un mouvement à générer, peut
être utilisée comme une généralisation du découpage fonctionnel, 
pour reconnaître de manière explicite des parties couplées du mouvement.

La figure~\ref{fig:spotDiff} illustre la problématique des travaux présentés dans ce manuscrit.
Dans l'image de gauche, le robot effectue un mouvement d'atteinte avec sa main droite.
Dans l'image de droite, le robot effectue deux mouvements d'atteintes en parallèle.
Les deux mouvements sont visuellement très proches. Pourtant le rôle
de la main gauche est discriminant. Dans le premier cas, la main gauche 
est couplée au mouvement de la main droite pour compenser
le déséquilibre introduit par le déplacement de la main droite.
Dans le second cas, la main gauche agit de manière indépendante pour accomplir l'action
d'atteinte. Le problème soulevé est donc de trouver le moyen
de différencier les deux mouvements en les observant. Nous montrons
qu'en effectuant les observations dans les \emph{bons} espaces,
il est possible d'effectuer une reconnaissance 
de mouvements capable de désambiguïser ces mouvements visuellement proches.
\begin{figure}
  \centering
  \subfigure{
  \includegraphics[trim=220px 10px 220px 0px, width=0.48\linewidth, clip=true]{figures/chapitre1/Lonly.ps}
  }
  \subfigure{
  \includegraphics[trim=220px 10px 220px 0px, width=0.48\linewidth, clip=true]{figures/chapitre1/RL.ps}
  }
  \caption[Comment désambiguïser des mouvements proches?]{Pour attraper la balle en face de lui (\`a gauche), le robot
  atteint une posture dans laquelle le bras gauche est utilisé pour garder son équilibre. 
  Dans la figure de droite, le robot effectue deux actions en parallèle:
  attraper la balle devant lui tout en attrapant une autre balle se trouvant derrière lui 
  (bien sûr, la balle qui se trouve derrière le robot a été volontairement placée à la position 
  finale de la main gauche illustrée dans la figure de gauche). Il est 
  impossible de différencier visuellement les deux postures. Cependant,
  est-il possible de discriminer ces deux mouvements?}
  \label{fig:spotDiff}
\end{figure}

Dans la suite de cette section, nous présentons différentes
approches considérées dans le domaine de l'imitation et la reproduction
de mouvements. Les travaux en imitation de mouvements sont 
reliés aux nôtres dans la mesure où le problème de l'imitation 
consiste à détecter les caractéristiques importantes ou communes d'un ou
plusieurs mouvements observés afin de pouvoir les reproduire ou les généraliser.



 %\section*{Introduction}
%\label{sec:intro}
%-------------------------------------------------------------------%
%-------------------------------------------------------------------%
\section{État de l'art}
\label{chap:eda}
%-------------------------------------------------------------------%
%-------------------------------------------------------------------%
La possibilité pour un robot d'apprendre à reproduire des mouvements à 
partir de démonstrations est intéressante car intuitive. 
Les démonstrations peuvent par exemple provenir d'un robot identique, ou sur ce même
robot par kinestésie (un opérateur va manipuler à la main les membres du robot).
Dans ce cas, on parle de reproduction de mouvements.
On parle d'imitation de mouvements si la démonstration provient d'un humain ou d'un autre type de robot. 
Le cas échéant, le mouvement doit être transformé pour pouvoir être exécuté au mieux
du point de vue de la ressemblance ou des objectifs à atteindre.
L'imitation ou la reproduction de mouvements peut porter sur des mouvements spécifiques comme
la marche~\cite{benallegue10}, sur des mouvements du corps complet en 
respectant les contraintes d'équilibre du robot~\cite{nakaoka03, yamane09a, miura09} ou 
sur les contraintes physiques par optimisation~\cite{suleiman08}.
Les méthodes d'apprentissage
sont largement utilisées pour généraliser les mouvements appris et sont très populaires.
Cette généralisation porte par exemple sur un changement dans l'environnement.

Le problème de l'imitation
peut être formulé en sous-problèmes
s'attaquant à la reconnaissance et à la génération de mouvements.
La méthode de reconnaissance de t\^aches développée dans ces travaux
s'appuie sur le modèle de génération de mouvements en se
plaçant directement dans les espaces des t\^aches.
Nous comparons les approches étudiées dans le domaine vis-à-vis des espaces utilisés pour
représenter ou caractériser les informations véhiculées dans un mouvement.

 \section{Apprentissage et représentation du mouvement}
Un aspect critique dans l'apprentissage de mouvement est le
choix de l'espace dans lequel l'apprentissage s'effectue, ou
autrement dit, des variables qui caractérisent la réalisation
des objectifs.
Ce choix détermine la représentation du mouvement reflétant l'expressivité des 
mouvements observés.
Plusieurs solutions dans le choix de l'espace d'apprentissage ont été étudiées.

Dans~\cite{shon05}, il y a deux espaces d'observations: la configuration articulaire d'un humain
acquise par capture de mouvements, et l'espace articulaire du robot.
L'espace articulaire de l'humain est nécessaire car l'objectif
de ces travaux est de réaliser une imitation.
Un mouvement est généré sur un modèle du robot, et un humain imite ce mouvement.
La trajectoire articulaire de l'humain est enregistrée gr\^ace à un système de capture
de mouvements. De cette façon, l'apprentissage porte sur un corpus parallèle
de trajectoires de l'humain et du robot. L'association entre les trajectoires articulaires
humaines et robotiques est donc manuelle. 

Dans~\cite{chalodhorn09}, l'espace d'observation est l'espace articulaire du robot.
L'objectif de ces travaux étant d'effectuer des reconnaissances et des reproductions 
de mouvements. Pour pouvoir appliquer un algorithme d'apprentissage, 
une technique de réduction de dimension est appliquée:
les auteurs proposent une analyse en composantes principales
non linéaire avec contrainte circulaire. Il s'agit d'une méthode qui généralise
l'analyse en composantes principales car elle permet d'obtenir les relations dans les deux
sens entre l'espace de haute dimension et l'espace réduit. La contrainte circulaire permet
d'adapter la technique aux courbes fermées.
En ne considérant que l'espace articulaire,
la sémantique du mouvement n'est pas explicite: l'exécution d'une t\^ache
n'est que la conséquence de la reproduction de la trajectoire articulaire. 

Le cas contraire est illustré dans~\cite{herzog08}.
Les travaux se focalisent sur les mouvements d'un bras droit pour 
des gestes d'atteintes d'objets.
L'objectif étant d'effectuer la reconnaissance et l'imitation de mouvements d'atteintes
effectués par un humain sur un robot humanoïde.
L'espace d'observation est ici composé des trajectoires dans l'espace cartésiens
de points associés à chaque articulation (épaule, coude, pouce, doigt, poing).
Le choix de cet espace d'observation donne une interprétation aux mouvements considérés.
Les déplacements de ces points sont mis à l'échelle du robot, puis sont utilisés
pour le calcul des trajectoires articulaires
que le robot doit exécuter pour imiter un mouvement. Ainsi
l'algorithme d'apprentissage peut associer le mouvement du bras humain 
au mouvement du bras du robot.
D'une manière similaire, les positions, vitesses et accélérations d'un effecteur sont
observées dans~\cite{hersch08} pour généraliser et reproduire des mouvements.
Les mouvements de démonstrations sont appliqués directement sur le robot par un opérateur.
Des informations de visions sont utilisées pour généraliser les mouvements appris à
d'autres conditions initiales et gérer les perturbations.

Dans \cite{montecillo10}, les positions et orientations
de la tête, des mains, des pieds et du torse d'un humain sont observées. 
Ces positions sont transposées à un robot humanoïde 
pour reproduire le mouvement de l'humain en temps réel.
Toutes les trajectoires de t\^aches sont imitées sans faire
la distinction des mouvements pertinents ou non.

L'espace d'observation considéré dans~\cite{billard03}, est l'union de l'espace articulaire
d'un bras humain, et des positions dans l'espace cartésien d'objets à manipuler.
Dans~\cite{billard06, calinon07a, kwon08, eppner09}, l'espace articulaire et les positions
absolues et relatives entre les effecteurs et les objets à manipuler dans l'environnement sont utilisés
comme observations.
Ces espaces permettent de représenter d'une manière plus explicite,
les t\^aches effectuées puisque l'ajout des positions cartésiennes des éléments
permet de spécifier qu'il n'y a pas que les trajectoires articulaires
qui sont importantes dans le mouvement:  les déplacements des objets et des effecteurs
sont caractéristiques du mouvement. L'espace d'observation est donc spécialisé pour un certain
type d'activité.
Les variations de l'environnement seront prises en compte dans l'apprentissage
pour extraire des descriptions génériques des mouvements démontrés.
Dans~\cite{gribovskaya11}, l'apprentissage se base sur les vitesses et les forces
afin de se spécialiser au contexte de l'interaction physique avec un humain.

Dans tous les travaux présentés, le choix des espaces d'observations est directement lié
à la nature des mouvements. La spécialisation des espaces permet 
de décrire (et donc de reconnaître) plus efficacement les mouvements.
Notre approche consiste à généraliser la spécialisation 
en considérant tous les espaces dans lesquels une t\^ache
peut être exécutée.

\section{Reconnaissance du mouvement}
La reconnaissance du mouvement est un problème pouvant s'appliquer à de nombreux domaines
comme la vision ou l'interaction homme-machine et s'intègre naturellement aux applications d'imitations.
Dans le cas des mouvements humains, la reconnaissance peut porter sur plusieurs niveaux. 
Il peut s'agir de suivre un humain, d'estimer sa configuration articulaire,
reconnaître l'identité d'un humain en se basant sur ses mouvements ou encore interpréter 
ses actions et ses comportements.
Une revue complète et générale de l'analyse de mouvements
humains est présentée dans~\cite{moeslund06}.
Une autre synthèse des travaux sur la reconnaissance de l'action
exécutée dans un mouvement est présentée dans~\cite{krueger07}.

L'étude des mouvements et des t\^aches en biomécanique a amené au développement 
de la théorie des \emph{variétés non contrôlées}~\cite{scholz99}.
Pendant la réalisation d'une t\^ache, les degrés de liberté contribuant à la t\^ache d'un humain
seraient classés dans deux catégories: d'une part les degrés de liberté 
qui sont contrôlés par le système nerveux et d'autre part ceux qui ne le sont pas.
Les variables sont considérées comme étant contrôlées si elles sont stabilisées.
%La mesure de l'accomplissement d'une t\^ache est alors définie par un ratio
%de variance dans deux sous-espaces: le sous-espace non contrôlé et son sous-espace orthogonal.
%Une grande variabilité dans le sous-espace des degrés de liberté
%non contrôlé n'affecte pas les performances tandis
%qu'une grande variabilité dans son sous-espace orthogonal dégrade les performances.
Dans~\cite{jacquierbret09}, la théorie des variétés non contrôlées est appliquée
pour l'étude de mouvements d'atteintes d'un humain en présence d'un obstacle qui va
matérialiser des contraintes spatiales.
L'objectif est de comprendre comment le système nerveux central gère la redondance du bras pour une t\^ache 
d'atteinte en fonction de la présence ou non d'un obstacle. Les différentes 
expérimentations menées renforcent l'hypothèse selon laquelle le système nerveux central
adapte les synergies entre les différents degrés de liberté du bras en fonction
des contraintes spatiales (ici des obstacles) et des objectifs.
On peut noter la similitude entre les variétés non contrôlées et leurs sous-espaces
orthogonals avec le formalisme de la pile de t\^aches présenté dans le chapitre~\ref{chap:sot}. 

\subsection{Classification de mouvements}
Le problème de reconnaissance d'action peut se formuler comme un problème de classification.
Dans les travaux en vision présentés dans \cite{chan07}, 
des mouvements humains, issus de séquences d'images, sont classés pour dégager des comportements distincts
en utilisant des informations contextuelles aux mouvements du corps.
Ces informations sont inspirées de la biomécanique et sont
modélisées par un système de connaissance, à base de règles floues qui sont
appliquées aux trajectoires articulaires d'un humain.  Par
exemple, pour un mouvement de marche, les deux bras ne doivent pas être
en même temps devant le torse. Les classes de mouvements considérées sont 
la course et la marche. Elles correspondent à des comportements humains.
L'inconvénient majeur de la méthode est qu'il est difficile de trouver et de formuler
les règles qui définissent une classe de mouvements. Il est aussi 
difficile de gérer le grand nombre de règles si on accumule
tous les comportements.

Peu de techniques sont développées pour la reconnaissance d'actions simultanées.
La méthode présentée dans~\cite{mori02} est dédiée aux actions humaines simultanées.
Encore une fois, le défaut majeur de cette méthode est qu'une action est représentée 
par un ensemble d'heuristiques sélectionnées manuellement.
Par exemple, pour qu'une action appartienne à la classe \emph{levé de main},
le mouvement doit respecter deux règles: la position de la main doit être au
dessus d'un certain seuil, et la main doit se déplacer de bas en haut.
Toutes les heuristiques sont appliquées au mouvement observé pour dégager l'ensemble
des actions exécutées.

D'autres types de classes peuvent être considérés comme dans~\cite{drumwright03, jenkins04}. Ces classes
sont focalisées sur des classes sémantiques (un direct, un crochet, un uppercut, une garde).
Chaque classe est représentée par des exemples de trajectoires articulaires qui 
vont créer un espace par balayage.
Un classificateur bayesien est utilisé pour calculer la probabilité qu'un mouvement appartienne à une
classe particulière sur des trajectoires articulaires observées, en se basant sur les
espaces de chacune des classes sémantiques.
Des sous-espaces articulaires sont utilisés comme critères de classification
de pas de danse dans~\cite{campbell95}.

La classification présentée dans~\cite{kulic08} est une classification adaptative en ligne. Les classes évoluent
au fur et à mesure des observations. Chaque mouvement observé est encodé sous la forme d'une chaîne
de Markov cachée pour être classé.
Dans~\cite{takano05}, les classes de mouvements se basent sur les paramètres des modèles
de Markov cachés encodant les mouvements.

\subsection{Segmentation temporelle}
La segmentation temporelle de mouvements est utilisée pour définir la succession
de mouvements unitaires ou primitifs dans un mouvement observé. L'outil le
plus utilisé dans ce domaine est le modèle de Markov caché.
Cet outil a surtout été appliqué dans le domaine de la reconnaissance 
de parole~\cite{rabiner89}. Mais ses propriétés mathématiques
permettent de l'utiliser pour modéliser des données séquentielles génériques, dont le mouvement.

La segmentation de données de capture de mouvements est étudiée dans~\cite{barbic04}.
La segmentation est de haut niveau: chaque segment unitaire représente 
un comportement (marcher, courir, frapper\ldots). Trois approches y sont évaluées
en se basant sur les dimensions obtenues par analyse en composantes principales,
puis en se basant sur une analyse en composantes principales associée à un modèle probabiliste~\cite{tipping99} 
et enfin en utilisant des mélanges de modèles gaussiens.

Des techniques très simples sont considérées dans~\cite{field08} pour la segmentation
de mouvements. Celles-ci se basent sur les accélérations des corps en mouvement, les changements de directions
(les points où la vitesse devient nulle)
et les analyses en composantes principales probabilistes.
En animation, dans le cadre des bases de données de mouvements,
la segmentation automatique de mouvements peut être utilisée
pour annoter automatiquement des mouvements et retrouver certains types de motifs de mouvements~\cite{so06, beaudoin08}.
Dans le même domaine, une segmentation de mouvements est réalisée en recherchant les 
postures clefs menant à des changements de direction significatifs~\cite{so05}.

Dans~\cite{kulic08a}, la segmentation d'un flux
de données représentant des mouvements humains, se fait en ligne contrairement à la plupart
des méthodes de segmentations. En effet, il est important que l'apprentissage puisse se faire en ligne
afin de pouvoir tenir compte de la présence d'humains dans l'environnement du robot.
De plus l'approche est incrémentale et permet de
prendre en compte les mouvements déjà appris ou appris manuellement dans 
la segmentation. Cette segmentation se base sur des propriétés des vitesses angulaires des articulations.
Dans~\cite{chalodhorn09}, la segmentation est également effectuée en ligne, 
et se base sur la projection des données observées
dans un espace réduit. Un nouveau segment est créé lorsque les nouveaux points projetés ne correspondent
plus au modèle de prédiction de l'espace courant. Les différents segments représentent
des phases de marches (marche avant, pas sur le côté, virage).
Cette approche ressemble à notre méthode de reconnaissance de t\^ache présentée
dans le chapitre~\ref{chap:reco}, dans le sens où 
les données observées sont comparées dans un espace approprié à des comportements attendus.
La différence est que les espaces que l'on considère sont traités de manière parallèle
et sont directement liés à la sémantique du mouvement.

\section{Génération de mouvements}
L'étape finale en imitation est la génération du mouvement à partir
de la représentation des mouvements reconnus. Les primitives moteurs
permettent d'exprimer le contrôle dans un espace de dimension
inférieure à l'espace articulaire. En combinant
les primitives, il est alors possible d'obtenir une 
grande expressivité pour les mouvements générés. Les primitives
peuvent être tirées d'un système dynamique, et peuvent être
modifiées et commutées en ligne~\cite{degallier08}.
Dans~\cite{sugihara05}, les mouvements sont générés en connectant
par une fonction lisse et continue, une posture clef et la posture courante.
Le mouvement généré tient compte des contraintes dynamiques du robot
et la trajectoire du centre de gravité du robot est recalculée.

Dans~\cite{drumwright03} à la manière de~\cite{rose98},
le mouvement est généré en interpolant des exemples appartenant à une classe de mouvements.
Dans~\cite{calinon07a}, des mouvements sont généralisés par 
mélanges de gaussiennes, puis un mouvement est généré par optimisation
d'une fonction de mesure d'imitation. Cette phase d'optimisation
permet d'adapter le mouvement au contexte courant.

Bien que la technique de réduction de dimension utilisée dans~\cite{chalodhorn09} permette
d'obtenir directement des trajectoires articulaires gr\^ace à la connaissance des relations dans les deux
sens entre l'espace de haute dimension et l'espace réduit, une perte de précision
est possible à cause du processus d'encodage et décodage.
Ainsi l'application du mouvement sur le robot ne garantit aucune stabilité. C'est pour 
cette raison que le mouvement généré est ajusté par optimisation d'une fonction de coût
représentant des valeurs capteurs attendues en fonction de l'état du robot 
dans l'espace réduit. 


 \section{Approche proposée}
Notre approche pour la reconnaissance de 
mouvements s'inscrit dans la continuité des travaux réalisés au JRL et au LAAS-CNRS au sein
de l'équipe Gepetto.
Elle se focalise dans l'espace des t\^aches et par conséquent,
à la reconnaissance de t\^aches.
%Ceci permet
%d'utiliser directement la relation entre l'espace des t\^aches 
%et l'espace articulaire pour générer un mouvement.
Plutôt que de construire un modèle discriminant ou génératif en utilisant 
les outils statistiques, nous exploitons le modèle
de génération de mouvements. En effet, ce modèle permet
d'exprimer directement les corrélations non linéaires entre
les variables de mouvements qui sont inhérentes aux mouvements
humanoïdes. Les approches statistiques et les techniques de
réduction de dimensions ne considèrent généralement que les corrélations linéaires.
Nous utilisons le formalisme de la fonction de t\^ache pour projeter 
le mouvement observé dans les espaces de t\^aches, caractérisant un mouvement, 
ainsi que dans les espaces
nuls. Ce formalisme sera présenté dans le chapitre suivant.
\textbf{Contrairement à la majorité des travaux antérieurs qui s'intéressent
aux mouvements composés de séquences de primitives, nous explorons les 
mouvements complexes issus d'un empilement de contôleurs.}

Nous ne considérons pas les séquencements d'actions ni la généralisation de mouvements:
aucun apprentissage n'est réalisé. En effet, nous supposons que les modèles 
de comportements sont connus puisque ce sont directement les modèles 
de générations de mouvements utilisés pour contrôler le robot.
Notre méthode de reconnaissance de t\^ache est formulée comme un problème d'identification
de t\^aches actives parmis un ensemble connus de t\^aches possible. Notre méthode 
est capable de détecter des t\^aches exécutées en parallèle en projetant
les trajectoires observées dans des espaces caractéristiques.
La méthode est aussi capable de gérer les éventuels couplages de t\^aches gr\^ace
à l'utilisation d'un opérateur de projection dans les espaces nuls des t\^aches.

\section{Plan du manuscrit}
Le chapitre~\ref{chap:sot} introduit les généralités sur la 
cinématique inverse et le concept de la fonction de t\^aches.
Ces fonctions de t\^aches sont utilisées pour la génération de mouvements
gr\^ace à une hiérarchisation de contrôleurs. La méthode
de génération de mouvements utilisée tout au long des travaux présentés
est une implémentation de cette approche: la \emph{pile de t\^aches}.
La méthode de reconnaissance présentée dans ce manuscrit s'appuie sur les propriétés associées à 
la fonction de t\^aches. Ces propriétés sont utilisées pour manipuler
les observations de mouvements.

Il est important de se placer dans un contexte proches des applications
réelles. Ainsi, l'analyse des mouvements est précédée par une phase d'acquisition.
Dans nos travaux, l'acquisition des mouvements s'effectue 
par le biais d'un système de capture de mouvements optique.
Le chapitre~\ref{chap:imitation} présente les différentes 
technologies permettant d'enregistrer un mouvement, le traitement
des données collectées ainsi que des exemples d'applications
orientées vers la fonction de t\^aches.

Les techniques présentées dans ces deux chapitres sont exploitées
dans le chapitre~\ref{chap:reco} qui représente
le c\oe ur de la contribution apportée dans ces travaux:
une méthode de reconnaissance de t\^aches pour un robot humanoïde,
capable de reconnaître des t\^aches exécutées en parallèles et de manière précise.
Les différentes expériences en simulation et sur
un vrai robot validant la méthode y sont détaillées.
 

 %\section*{Introduction}
%\label{sec:intro}
%-------------------------------------------------------------------%
%-------------------------------------------------------------------%
\chapter{Pile de t\^aches}
\label{chap:sot}
%-------------------------------------------------------------------%
%-------------------------------------------------------------------%

La fonction de t\^ache~\cite{samson91} est une 
approche élégante pour décrire intuitivement des commandes pour un robot dans un espace
approprié. Une t\^ache permet donc de relier l'état d'un robot à un espace choisi. 
On peut alors prédire les mouvements dans l'espace des t\^aches connaissant
les mouvements du robot (fonction directe) ou bien trouver un mouvement (c'est-à-dire une commande)
du robot réalisant un objectif donné dans l'espace des t\^aches.
En se reposant sur la redondance du système, cette approche peut être étendue pour 
tenir compte d'un ensemble hiérarchisé de t\^aches~\cite{siciliano91}. 
Il est possible de ranger ces t\^aches sous la forme d'une pile
afin de construire efficacement une loi de commande générant des mouvements complexes~\cite{baerlocher04, mansard07}. 
Cette organisation en pile autorise des opérations sur les t\^aches (ajouts, suppressions et permutations)
qui préservent la continuité de la commande~\cite{keith09}.
Ce mode de composition offre une véritable versatilité: en effet,
les t\^aches qui composent une pile sont réutilisables indépendamment,
et peuvent être modifiées gr\^ace à leurs paramétrages. De plus
les t\^aches peuvent être exprimées dans les mêmes espaces que les données de
capteurs, et il est par conséquent facile de suivre l'évolution des t\^aches.

Ce chapitre présente d'abord le problème de la géométrie inverse
qui consiste à calculer une posture d'un corps poly-articulé en fonction
d'un objectif dans l'espace cartésien à atteindre. Les principes et les 
outils permettant de résoudre ce problème ont un lien direct avec la fonction de t\^ache.
Le formalisme de la fonction de t\^ache sera ensuite introduit, pour finir
sur la présentation de la pile de t\^aches. Cette pile de t\^aches, gr\^ace à 
ses propriétés, pourra \^etre utilisée pour générer une loi de commande
ou effectuer une reconnaissance de mouvements.

%Par la suite, nous considérons que le robot est commandé en vitesse $\mbf{\dot{q}}$,
%où $\mbf{q}$ est le vecteur de configuration articulaire du robot.                   
%Une t\^ache est définie par un vecteur
%$\mbf{e}$ et par un comportement référence $\mbf{\dot{e}^*}$ à éxécuter dans l'espace
%des t\^aches.
%La Jacobienne de la t\^ache est notée
%$\mbf{J}=\dpartial{\mbf{e}}{\mbf{q}}$. 
%Divers comportements typiques peuvent être choisi pour $\mbf{\dot{e}^*}$. Typiquement,
%le comportement qui sera utilisé ici est une décroissance exponentielle définie par
%\begin{equation}
%  \mbf{\dot{e}^*} = -\lambda\mbf{e}
%  \label{eq:lambda}
%\end{equation}
%où $\mbf{e} = \mbf{s} - \mbf{s^*}$ est l'erreur entre la valeur observée courante
%$\mbf{s}$ et sa référence $\mbf{s^*}$,
%et $\lambda>0$ est le gain qui permet d'ajuster la vitesse de régulation de
%$\mbf{e}$ à $0$.
%Par exemple, la caractéristique observée peut \^etre une position 3d $\mbf{p}$
%d'un effecteur du robot, à ramener à une position choisie $\mbf{p^*}$.
%Dans ce cas, $\mbf{J} = \dpartial{\mbf{p}}{\mbf{q}}$ est la Jacobienne
%articulaire du robot.
%
%La loi de contr\^ole est donnée par la solution aux moindres carrés~\cite{liegeois77}:
%\begin{equation}
%\dot{\mbf{q}} = \mbf{J}^+ \mbf{\dot{e}}^* + \mbf{P}\mbf{z}
%\label{eq:ltsq}
%\end{equation}
%où $\mbf{J}^+$ est l'inverse de $\mbf{J}$ au sens des moindres carrés,
%$\mbf{P} = \mbf{I} - \mbf{J}^+ \mbf{J}$ est l'opérateur de projection dans l'espace nul
%de $\mbf{J}$ et $\mbf{z}$ est un critère secondaire quelquonque. $\mbf{P}$ assure
%le découplage de la t\^ache principale et de $\mbf{z}$. 
%En utilisant $\mbf{z}$ comme une entrée secondaire, la loi de commande peut
%\^etre étendue recursivement à un ensemble de 
%\emph{n} t\^aches. Ces \emph{n} t\^aches
%sont ordonnées par ordre de priorités :
%la t\^ache d'ordre 1 étant celle de plus haute priorité,
%et la t\^ache d'ordre \emph{n}, celle de plus basse priorité,
%une $t\^ache_i$ ne doit pas perturber une $t\^ache_j$ si $i>j$.
%La formulation récursive de la loi de commande est proposée dans~\cite{siciliano91} :
%\begin{equation}
%\dot{\mbf{q}}_i = \dot{\mbf{q}}_{i-1} + (\mbf{J}_i \mbf{P}_{i-1}^{A})^+
%(\dot{\mbf{e}}^*_i - \mbf{J}_i \dot{\mbf{q}}_{i-1}) , \ \ i = 1 \ldots n
%\end{equation}
%\noindent avec $\dot{\mbf{q}}_0 = 0$ et $\mbf{P}_{i-1}^{A}$ est
%le projecteur dans l'espace nul de la Jacobienne augmentée
%$\mbf{J}_i^A = (\mbf{J}_1, \ldots \mbf{J}_i)$. La trajectoire des vitesses articulaires
%réalisant toutes les t\^aches est $\dot{\mbf{q}}^* = \dot{\mbf{q}}_n$.
%Une implémentation complète de cette approche est proposée dans~\cite{mansard07}
%sous le nom de \emph{pile de t\^aches} (SoT). 
%

 \section{Cinématique inverse}
\label{sec:inversekine}
Le problème de la cinématique inverse consiste à déterminer le mouvement à appliquer à
un corps poly-articulé pour satisfaire un objectif désiré dans un espace de travail (par exemple
l'espace cartésien).
Le champ d'application de la cinématique inverse s'étend au domaine de la robotique
pour le contr\^ole, en animation graphique et dans les jeux vidéo pour animer
des corps poly-articulés tels que des personnages ou des animaux, mais aussi
en bioinformatique pour modéliser les mouvements de protéines. La difficulté
de ce problème réside dans l'explosion combinatoire due au grand nombre
d'articulations à contrôler. Cette partie présente le problème de la cinématique inverse
ainsi que les principales méthodes permettant de le résoudre.

\subsection{Formulation du problème}
La cinématique est l'étude du mouvement des corps d'un point de vue strictement géométrique
et temporel. Les forces et les moments agissant sur ces corps ne sont pas considérés.
Un système cinématique est composé de corps reliés entre eux par des articulations possédant un
ou plusieurs degrés de liberté. Ces systèmes peuvent être
représentés sous la forme d'un arbre dont les branches
sont des sous-chaînes cinématique (voir la figure~\ref{fig:kinechain} pour un exemple de système cinématique).
\begin{figure}[t]
  \begin{center}
    \includegraphics[width=0.4\linewidth]{figures/chapitre3/RobotJacobienne.eps}
  \end{center}
  \caption[Un exemple de système cinématique de type humanoïde.]{Un exemple de système cinématique de type humanoïde. En fonction de la
  t\^ache, une sous-chaine cinématique peut être utilisé (par exemple une sous-chaine associé à un bras).}
  \label{fig:kinechain}
\end{figure}
Un système cinématique peut être décrit par sa configuration articulaire :
\begin{equation}
  \mathbf{q} = (q_0, \ldots, q_n) \in \mathbb{R}^n
  \label{eq:configArt}
\end{equation}
Ces données articulaires peuvent par exemple correspondre à des valeurs
angulaires dans le cas où l'articulation est un pivot.
L'ensemble de ces données articulaires représente l'espace articulaire.
Dans le cas d'un robot de type bras articulé industriel à six degrés de liberté ,
l'espace articulaire correspond au vecteur de dimension six dont les éléments
sont les valeurs angulaires de chaque articulation (voir la figure~\ref{fig:robot6}).
\begin{figure}[t]
  \begin{center}
    \includegraphics[width=0.3\linewidth]{figures/chapitre3/robot6.eps}
  \end{center}
  \caption{Un exemple de robot de type bras articulé industriel à six degrés de liberté.}
  \label{fig:robot6}
\end{figure}

Le problème de géométrie directe consiste à déterminer
la position d'un point appartenant au robot à partir d'une configuration
articulaire donnée (voir exemple dans la figure~\ref{fig:directKine}).
Tandis que le problème de géométrie inverse consiste à déterminer
une configuration articulaire correspondant à une position désirée.
\begin{figure}[t]
  \begin{center}
    \includegraphics[width=0.4\linewidth]{figures/chapitre3/directKine.ps}
  \end{center}
  \caption[Géométrie directe.]{Le problème de géométrie directe consiste à calculer 
  la position du point $\mbf{p}$ en fonction de la configuration du robot
  $\mbf{q} = [\theta_1, \theta_2]$.}
  \label{fig:directKine}
\end{figure}
Le problème qui consiste à déterminer par la relation inverse 
$\left[ \begin{array}{cc}
  \mbf{v}\\
  \boldsymbol\omega
\end{array}
\right] 
  \rightarrow \mbf{\dot{q}}$ 
une vitesse articulaire 
amenant un effecteur à une position précise dans l'espace cartésien en fonction de la configuration
articulaire actuelle, est
appelé la cinématique inverse (voir la figure~\ref{fig:ik}). 
\begin{figure}[t]
  \begin{center}
    \includegraphics[width=0.5\linewidth]{figures/chapitre3/ik.ps}
  \end{center}
  \caption[Cinématique inverse.]{Le problème de la cinématique inverse est de trouver les modifications à apporter
  à la configuration articulaire $\mbf{q}$ pour qu'un l'effecteur atteigne une position désirée $\mbf{p}^*$.}
  \label{fig:ik}
\end{figure}

\subsection{Résolution du problème de cinématique inverse}
Le problème de la cinématique inverse peut être résolu de plusieurs manières.
Pour les robots simples (avec peu de degrés de liberté), le calcul analytique est envisageable. Elle consiste à 
résoudre un système équations souvent non linéaires. Une fois résolues, le résultat est alors obtenu
directement en une seule étape. Il s'agit de l'approche classique
en robotique pour le contrôle des bras articulés possédant jusqu'à 
six degrés de liberté car la solution analytique est rapide à calculer~\cite{craig04}.
Malheureusement, dans le cas général d'une structure articulée avec beaucoup de degrés
de liberté, il n'existe pas de solutions analytique
calculable.

Il est aussi possible de résoudre le problème de cinématique inverse 
par apprentissage, via des exemples, de la relation $\left[ \begin{array}{cc}
  \mbf{v}\\
  \boldsymbol\omega
\end{array}
\right]  \rightarrow \mbf{\dot{q}}$~\cite{dsouza01}. 
Cette méthode est intéressante
lorsque le modèle du robot n'est pas connu avec exactitude. Elle
permet aussi de se restreindre aux configurations réalisables sur le 
robot et d'obtenir des mouvements dont l'allure n'est pas 
artificielle. C'est particulièrement important lorsqu'il s'agit
d'animer des personnages par exemple. Les inconvénients de cette méthode sont hérités des méthodes par apprentissage,
c'est-à-dire que la relation échantillonnée $ \left[ \begin{array}{cc}
  \mbf{v}\\
  \boldsymbol\omega
\end{array}
\right] \rightarrow \mbf{\dot{q}}$
est spécifique au robot considéré et la qualité de la modélisation dépend de la qualité
des données d'entrainement.

L'alternative la plus populaire est d'effectuer une résolution numérique.
La méthode est itérative, et consiste à linéariser le système
autour de l'état courant, puis d'effectuer une descente de gradient
afin de converger vers la solution.

\subsubsection{Formulation linéaire et non-linéaire}
Une des méthodes pour résoudre le problème de cinématique inverse consiste à donner
à un solveur non-linéaire une fonction de coût pour 
calculer des postures satisfaisant 
des contraintes représentant par exemple des contacts~\cite{escande06}.
Il est aussi possible de linéariser le problème et résoudre itérativement
en calculant des petits incréments de la configuration~\cite{nakamura90}.
Ces incréments font converger la configuration courante
de la chaîne cinématique vers une configuration satisfaisant les
objectifs. Ainsi, les itérations successives fournissent un chemin
entre la posture initiale et la posture satisfaisant les objectifs, et donc
ce chemin peut \^etre utilisé comme une commande pour un robot.

\subsubsection{Linéarisation}
La linéarisation est effectuée en calculant la jacobienne de la fonction:
\begin{equation}
  \mbf{f} \rightarrow \mbf{f}(\mbf{q}) = \mbf{p}
  \label{eq:simpleKine}
\end{equation}
\noindent où $\mbf{p}$ peut par exemple être la position de l'effecteur d'un robot.
La jacobienne est une matrice qui généralise la notion de dérivée de fonction
scalaire. Dans le cas de \eqref{eq:simpleKine}, la jacobienne $\mbf{J}$ s'écrit:
\begin{equation}
  \mbf{J} = \left(
  \begin{array}{ccc}
    \dpartial{f_1}{q_1} & \ldots & \dpartial{f_1}{q_n}\\
    \vdots & \ddots & \vdots \\
    \dpartial{f_m}{q_1} & \ldots & \dpartial{f_m}{q_n}
  \end{array}
  \right)
  \label{eq:Jacobian}
\end{equation}
La linéarisation permet l'approximation au premier ordre suivante:
\begin{equation}
  \Delta \mbf{p} = \mbf{J} \Delta \mbf{q}
  \label{eq:directKineLin}
\end{equation}

La jacobienne est ensuite inversée pour obtenir une approximation
locale de la fonction:
\begin{equation}
  \mbf{p} \rightarrow \mbf{q} = \mbf{f}^{-1}(\mbf{p})
  \label{eq:simpleInvKine}
\end{equation}
On a alors une approximation de l'inverse:
\begin{equation}
  \Delta \mbf{q} = \mbf{J}^{-1} \Delta \mbf{p} + o(\Vert \Delta\mbf{q} \Vert^2)
  \label{eq:ik}
\end{equation}
On pourra donc localement faire varier $\mbf{q}$ dans la direction de la solution.
Cette linéarisation pourra être utilisé pour résoudre de manière itérative le problème 
de cinématique inverse.
Malheureusement, la jacobienne n'est pas toujours inversible. Entres autres, $\mbf{J}$ n'est pas souvent carrée.

\subsubsection{Simplification du problème d'inversion}
Il est possible de contourner la difficulté de l'inversion de la jacobienne en simplifiant
le problème.
Une première idée est d'utiliser la transposée de la jacobienne plutôt 
que son inverse pour résoudre le problème de cinématique inverse~\cite{wolovich84}.
Ainsi contrairement à~\eqref{eq:ik} on choisit le prochain $\Delta \mathbf{q}$ comme suit :
\begin{equation}\label{IK:Tr:Geq1}
  \alpha \mbf{J}^{T} \Delta \mathbf{p}
\end{equation}
L'intuition, derrière cette approximation, est de faire l'analogie avec
la relation entre les couples moteurs et les forces dans l'espace cartésien.
Si $\mathbf{f}$ est la force qui
va amener le robot en direction de l'objectif, et $\boldsymbol\tau$ le moment articulaire interne
à la chaine cinématique nécessaire, cette relation est~:
\begin{equation*}
  \mbf{\tau} = \mbf{J}^T \mbf{f}
\end{equation*}
Ainsi une quantité représentant des déplacements angulaires est liée à une quantité représentant des 
déplacements dans l'espace par la transposée de la jacobienne.

Cette méthode est rapide car il n'y a pas d'opération d'inversion
à effectuer pour calculer la variation de $\mbf{q}$ à appliquer.
Cependant, la convergence n'est pas rapide, et en particulier
au voisinage de la solution, puisque 
les variations à appliquer de chaque articulations sont calculées sans tenir compte de l'influence
des autres articulations comme le montre~\eqref{eq:exTranspose} pour un exemple de robots
à deux degrés de liberté et un objectif de dimension deux.
\begin{equation}
  \left(
  \begin{array}{c}
    \Delta \theta_1\\
    \Delta \theta_2
  \end{array}
  \right)
  = \alpha \left( 
  \begin{array}{cc}
    \dpartial{x}{\theta_1} & \dpartial{y}{\theta_1}\\
    \dpartial{x}{\theta_2} & \dpartial{y}{\theta_2}
  \end{array}
  \right)
  \left(
  \begin{array}{c}
    \Delta x\\
    \Delta y
  \end{array}
  \right)
  = \alpha \left( 
  \begin{array}{c}
    \dpartial{x}{\theta_1} \Delta x + \dpartial{y}{\theta_1} \Delta y\\
    \dpartial{x}{\theta_2} \Delta x + \dpartial{y}{\theta_2} \Delta y
  \end{array}
  \right)
  \label{eq:exTranspose}
\end{equation}

Une autre méthode itérative appelée \emph{cyclic coordinate descent}
consiste à calculer séquentiellement les variations à appliquer à chaque articulation~\cite{luenberger84, wang91}.
Chaque itération est décomposée en $n$ étapes, pour une chaine articulaire en série
de $n$ degrés de liberté. Chaque étape consiste à trouver la variation
pour une articulation (en remontant de l'effecteur jusqu'à la base) qui va minimiser la distance entre l'effecteur
et l'objectif.
Ceci simplifie le calcul d'inversion puisqu'elle ne concerne plus qu'un vecteur
de dérivées partielles (une colonne de la jacobienne, voir~la figure~\ref{fig:ccd}).
La méthode est rapide, mais converge lentement. De plus, à cause de l'approche séquentielle,
les variations des premières articulations auront tendance à être plus grande
que les suivantes: les postures calculées peuvent donc paraître non naturelles.
Un autre problème est que la méthode est difficile à adapter pour des systèmes
à plusieurs effecteurs.
\begin{figure}[t]
  \begin{center}
    \includegraphics[width=0.5\linewidth]{figures/chapitre3/ccd.ps}
  \end{center}
  \caption[\emph{Cyclic coordinate descent}.]{Illustration de la méthode \emph{cyclic coordinate descent}, dans laquelle les articulations
  sont traités séquentiellement amenant ainsi à ne considérer qu'une colonne de la jacobienne
  par étape. L'inversion en est simplifiée.}
  \label{fig:ccd}
\end{figure}

Ces méthodes reposant sur la simplification de l'inversion de la jacobienne n'apportent pas de 
résultats satisfaisants pour la convergence et la généricité. C'est pour cela que la 
méthode la plus utilisée est l'utilisation de la pseudo inverse de la jacobienne décrite
dans le paragraphe suivant.

\subsubsection{Pseudo inverse de la jacobienne}
La pseudo inverse notée $\mbf{A}^+$, d'une matrice notée $\mbf{A}$ est une généralisation de l'inverse d'une matrice~\cite{benisrael03, moore20, penrose55}.
Elle peut être calculée pour n'importe quelle matrice, même celles qui ne sont 
pas carrées ou de rang plein. Elle possède certaines propriétés
de l'inverse d'une matrice. Le type de pseudo inverse le plus
utilisé est la pseudo inverse de Moore-Penrose qui est définie par les conditions
suivantes:
\begin{align}
  \mbf{A}\mbf{A}^+\mbf{A} & = \mbf{A}\\
  \label{eq:defPseudoInv2}
  \mbf{A}^+\mbf{A}\mbf{A}^+ & = \mbf{A}^+\\
  (\mbf{A}\mbf{A}^+)^T & = \mbf{A}\mbf{A}^+\\
  (\mbf{A}^+\mbf{A})^T & = \mbf{A}^+\mbf{A}
  \label{eq:defPseudoInv}
\end{align}
\noindent La pseudo inverse permet d'obtenir une solution minimale au sens des moindre carrés 
à la fois de la distance à la solution ($0$ si la solution est atteignable) et
de la norme du vecteur solution d'un système d'équation linéaire. Ainsi pour l'équation~\eqref{eq:directKineLin},
la pseudo inverse permet d'obtenir la solution homogène:
\begin{equation}
  \Delta \mbf{q}^* = \min \underset{\Delta \mbf{q}}\argmin \Vert \mbf{J} \Delta \mbf{q} - \Delta \mbf{p} \Vert^2
  = \mbf{J}^+ \Delta \mbf{p}
  \label{eq:lstsqsol}
\end{equation}

Une autre propriété intéressante de la pseudo inverse est que la matrice $(\mbf{I} - \mbf{J}^{+} \mbf{J})$ effectue une
projection dans le noyau de $\mbf{J}$. Cela signifie que pour tout vecteurs $\mbf{z}$,  
$\mbf{J}(\mbf{I} - \mbf{J}^{+} \mbf{J}) \mbf{z} = \textbf{0}$ (direct d'après~\eqref{eq:defPseudoInv2}). 
Cela permet d'obtenir d'autres solutions pour $\Delta \mathbf{q}$:

\begin{equation}\label{IK:Ps:EqNullspace}
  \Delta \mathbf{q} = \mbf{J}^{+} \Delta \mathbf{p} + (\mbf{I} -\mbf{J}^{+} \mbf{J}) \mbf{z}
\end{equation}

$\Delta \mathbf{q}$ minimise encore $\Vert \mbf{J} \Delta \mathbf{q} - \Delta \mathbf{p} \Vert^2$.
La minimisation de $\Vert \Delta \mbf{q} \Vert$ est alors perdue, sauf dans le cas $\mbf{z} = 0$.
Nous montrerons plus tard que cette propriété permet de gérer des commandes secondaires en utilisant
$\mbf{z}$ pour faire varier la solution selon des critères choisis.

\subsubsection{Calcul de la pseudo inverse}
Une méthode efficace pour calculer la pseudo inverse d'une matrice est d'effectuer
une décomposition en valeurs singulières (singular value decomposition, SVD) sur cette matrice. Les propriétés
de cette décomposition simplifient le calcul de la pseudo inverse.
La décomposition en valeurs singulières d'une matrice $\mbf{J}$ de rang $r$ 
consiste à exprimer $\mbf{J}$ sous la forme :
\begin{equation*}
  \mbf{J} = \mbf{U} \mbf{\Sigma} \mbf{V}^{T}
\end{equation*}
\noindent où $\mbf{U}$ et $\mbf{V}$ sont des matrices orthogonales et $\mbf{\Sigma}$ est de la
forme:
\begin{equation}
  \mbf{\Sigma} = \left( \begin{array}{cc}
    \mbf{D} & \textbf{0}\\
    \textbf{0} & \textbf{0}
  \end{array} \right)
  \label{eq:sigmaSVD}
\end{equation}
\noindent $\mbf{D}$ est diagonale carrée de dimension $r$ et $\textbf{0}$ sont des matrices ne contenant que
de éléments nuls. 
Si $\mbf{J}$ est de dimension $m \times n$, alors $\mbf{U}$ est de dimension $m \times m$, 
$\mbf{\Sigma}$ est de dimension $m \times n$ et $\mbf{V}$ est de dimension $n \times n$ (voir la figure~\ref{fig:svd}). 
\begin{figure}[t]
  \begin{center}
    \includegraphics[width=0.8\linewidth]{figures/chapitre3/svd.ps}
  \end{center}
  \caption{Décomposition en valeurs singulières d'une matrice $\mbf{J}$.}
  \label{fig:svd}
\end{figure}
Les seuls membres non nuls de la matrice $\mbf{D}$ sont les valeurs
$\sigma_i$ selon la diagonale avec
$\sigma_1 \geq \sigma_2 \geq \ldots \geq \sigma_r \geq 0$. 

Soient $\mbf{u}_i$ et $\mbf{v}_i$ les $i^{\mbox{\footnotesize\it ème}}$ colonnes de $\mbf{U}$ et $\mbf{V}$. 
Les $r$ premières colonnes de $\mbf{U}$ forment une base orthonormée de l'image de $\mbf{J}$, et 
les vecteurs $\mathbf{v}_{r+1}$, \ldots, $\mathbf{v}_n$ forment une base orthonormée du noyau de $\mbf{J}$. 
La décomposition en valeurs singulières existe toujours. En ne considérant que les $r$ premier
vecteur, $\mbf{J}$ peut être
réduite à :
\begin{equation}
  \mbf{J} = \sum_{i=1}^{r} \sigma_i \mathbf{u}_i
	\mathbf{v}_i^T
\end{equation}
%\subsection{Pseudo Inverse et SVD}
Les matrices $\mbf{U}$ et $\mbf{V}$ sont orthogonales, et donc
$\mbf{U}^{-1} = \mbf{U}^T$ et $\mbf{V}^{-1} = \mbf{V}^T$. $\mbf{D}$ est diagonale,
et donc son inverse est obtenu en remplaçant tout ses coefficients
non nuls par leurs inverses. Ainsi, 
le calcul de la pseudo inverse de $\mbf{J}$ est directe:
\begin{equation}
  \mbf{J}^+ = \mbf{V} \mbf{\Sigma}^+ \mbf{U}^T = \mbf{V} \left( \begin{array}{cc}
    \mbf{D}^{-1} & \textbf{0}\\
    \textbf{0} & \textbf{0}
  \end{array} \right) \mbf{U}^T  = \sum_{i=1}^r \sigma_{i}^{-1} \mathbf{v}_i \mathbf{u}_i^{T}
\end{equation}

\subsubsection{Singularités}
Les configurations singulières sont définies par l'ensemble:
\begin{equation}
  \left\{ q \in \mathbb{R}^n | \mathrm{rank}(\mbf{J}(\mbf{q})) < \mathrm{rank}(\mbf{J}(\mbf{q}_0)) \right\}
  \label{eq:singularity}
\end{equation}
\noindent où $\mbf{q}_0$ est la configuration articulaire initiale.
Pour ces configurations, la dimension de l'espace $\mathbb{I}(\mbf{q}) = \mathrm{Image}(\mbf{J}(\mbf{q}))$
diminue brusquement.
Le problème de la pseudo inverse se pose à la frontière extérieure des singularités.
En effet, la pseudo inverse est une fonction continue des matrices dans les ensembles
de rang constant. Lors de la perte de rang, une discontinuité de la fonction 
pseudo inverse se produit. Elle s'accompagne d'une limite infinie lorsqu'on se rapproche de la
frontière extérieure de la singularité.
La décomposition en valeurs singulières montre plus clairement ce problème:
lorsque les valeurs des coefficients $\sigma_i$ sont très faibles,
leurs inverses, nécessaires dans le calcul de la pseudo inverse, vont tendre
vers l'infini et rendre les solutions arbitrairement grandes autour de ces configurations.
Les problèmes apparaissent donc aux voisinages des singularités.

Dans l'exemple de la figure~\ref{fig:singularities},
le déplacement de l'effecteur pour un bras articulé à deux degrés de liberté 
en rotation est représenté par les bras de levier.
En singularité, les coefficients $\sigma_i$ sont nuls et donc $\dpartial{\mbf{f}}{q_1}$ et $\dpartial{\mbf{f}}{q_2}$
sont parallèles, $\mathbb{I}(q)$
perd brusquement une dimension et
l'objectif $\mbf{\dot{p}}^*$ n'est plus accessible par combinaisons
linéaire, alors que dans une autre position, $\mbf{\dot{p}}^*$ aurait été accessible.
\begin{figure*}
  \centering
  \subfigure[La posture est non singulière.]{
  \resizebox{.31\textwidth}{!} {
    \input{figures/chapitre3/singularity1.tex}
  }
  \label{fig:singularities:a}
  }
  \subfigure[Au voisinage de la singularité.]{
  \resizebox{.31\textwidth}{!} {
    \input{figures/chapitre3/singularity2.tex}
  }
  \label{fig:singularities:b}
  }
  \subfigure[En singularité.]{
  \resizebox{.31\textwidth}{!} {
    \input{figures/chapitre3/singularity3.tex}
  }
  \label{fig:singularities:c}
  }
  \caption[Singularités.]{Illustration du problème de singularité pour un robot à deux degrés de liberté en rotation.
  Dans la figure~\ref{fig:singularities:a}, il n'y a pas de singularité. La figure~\ref{fig:singularities:b}
  illustre le cas au voisinage de la singularité: les directions de déplacement
  dû aux variations des articulations $\dpartial{\mbf{f}}{q_1}$ et $\dpartial{\mbf{f}}{q_2}$ sont très proches. 
  La figure~\ref{fig:singularities:c}
  montre un exemple de singularité:
  l'objectifs $\mbf{p}^*$ n'est plus 
  accessible par combinaisons linéaire, alors que dans une autre position de départ,
  $\mbf{p}^*$ aurait été accessible.
  }
  \label{fig:singularities}
\end{figure*}

\subsubsection{Amortissement de la pseudo inverse}
%\subsection{Amortissement par les moindres carrés et SVD}
Pour gérer les singularités, il est possible de modifier la décomposition en valeurs singulières
afin d'introduire un facteur d'amortissement~\cite{nakamura86, wampler86}. Ce facteur 
va garantir que la norme de la solution reste sous une valeur prédéfinie dépendant uniquement
de la norme de $\mbf{p}$ (et indépendante de $\mbf{J}$).
Le problème~\eqref{eq:lstsqsol} est modifié pour limiter la norme de $\Delta \mbf{q}$:
\begin{equation}
  \Delta \mbf{q}^* = \min \underset{\Delta \mbf{q}}\argmin \Vert \mbf{J} \Delta \mbf{q} - \Delta \mbf{p} \Vert^2 
  - \eta^2 \Vert \Delta \mbf{q} \Vert^2
  \label{eq:dampedlstsqsol}
\end{equation}
\noindent $\eta$ est le facteur d'amortissement qui va permettre de faire un compromis
entre l'erreur de $\Vert \mbf{J} \Delta \mbf{q} - \Delta \mbf{p} \Vert^2$ et la norme
de la solution. Le cas $\eta = 0$ correspond à~\eqref{eq:lstsqsol}. Pour
un $\eta > 0$, il est prouvé que la solution de ce problème est:
\begin{equation*}
  \Delta \mbf{q}^* = \mbf{J}^T ( \mbf{J} \mbf{J}^T + \eta^2 \mbf{I} )^{+} \Delta \mbf{p} = \mbf{J}^{\dagger} \Delta \mbf{p}
\end{equation*}

Et finalement la décomposition en valeurs singulières de la pseudo inverse amortie $\mbf{J}^{\dagger}$ amène au résultat suivant:
\begin{equation}
 \mbf{J}^{\dagger} = \mbf{J}^T ( \mbf{J} \mbf{J}^T + \eta^2 \mbf{I} )^{+} = 
 \sum_{i = 1}^{k}{\frac{\sigma_i}{\sigma_i^2 + \eta^2} \mathbf{v}_i
	\mathbf{u}_i^T }
\end{equation}
On remarque que l'inverse des coefficients $\sigma_i$ sont remplacés par:
\begin{equation}
  \sigma_i^{\dagger} = \frac{\sigma_i}{\sigma^2 + \eta^2}
  \label{eq:dampingsvd}
\end{equation}
\noindent qui est continue et borné par $\frac{1}{2\eta}$ (voir la figure~\ref{fig:damp}). 
\begin{figure}[t]
  \begin{center}
    \resizebox{0.48\textwidth}{!}{
      \input{figures/chapitre3/plotDampEdit.tex}
    }
  \end{center}
  \caption{Illustration du comportement de $\sigma_i^{-1}$ et de $\sigma_i^{\dagger}$.}
  \label{fig:damp}
\end{figure}
Lorsque les $\sigma_i$ sont grands $\mbf{J}^{\dagger} \approx \mbf{J}^{+}$. Donc la méthode de l'amortissement 
par les moindres carrés agit comme celle de la pseudo inverse loin des
singularités, et adoucit les résultats de la pseudo inverse au voisinage des singularités. 

%%%%%%%%%%%%%%%%%%%%%%%%%%%
\FloatBarrier
\section{La fonction de t\^ache}
Dans la partie précédente, on ne s'est intéressé qu'aux relations entre des
positions et vitesses de l'effecteur et l'espace des configurations.
La fonction de t\^ache permet de généraliser ce concept à d'autres fonction que la
position de l'effecteur.
Une fonction de t\^ache permet de relier l'espace articulaire
à un espace de t\^ache:
\begin{equation}
  \begin{array}{cccc}
  \mathbf{e} : &\mathbb{R}^n  &\rightarrow & \mathbb{R}^m
  \\ &\mathbf{q} &\mapsto &\mathbf{e}(\mathbf{q})
\end{array}
  \label{eq:fwdKine}
\end{equation}

L'espace de la t\^ache est l'espace dans lequel est exprimé une t\^ache robotique.
L'espace de la t\^ache est choisi en fonction de la
nature de la t\^ache robotique à effectuer, et peut par exemple être l'espace cartésien ou encore un sous espace 
de l'espace des configurations. Un cas particulier est bien sûr
une paramétrisation de la position de l'effecteur, étudié dans le paragraphe~\ref{sec:inversekine}.

L'exécution d'une t\^ache peut être considérée comme la régulation à zéro de la fonction de t\^ache associée.
En prenant par exemple comme fonction de t\^ache $\mathbf{e}(\mathbf{q}) = \mathbf{s} - \mathbf{s}^*$
correspondant à l'erreur entre un signal $\mathbf{s}$ et sa valeur désirée $\mathbf{s}^*$.
Ainsi, la jacobienne $\mathbf{J}$ relie le vecteur erreur au vecteur
de configuration:
\begin{equation}
  \mbf{\dot{e}} = \dpartial{\mbf{e}}{\mbf{q}} \mbf{\dot{q}} = \mbf{J} \mbf{\dot{q}}
  \label{jacobianDef}
\end{equation}

Le comportement de la fonction de t\^ache doit être défini. Habituellement,
en robotique, ce comportement est une décroissance exponentielle:
\begin{equation}
  \mbf{\dot{e}^*} = -\lambda \mbf{e}
  \label{expoDecrease}
\end{equation}
où $\lambda$ est un paramètre positif permettant d'ajuster la vitesse
de convergence.
$\mbf{J}$ n'étant pas toujours inversible, sa pseudo-inverse $\mbf{J}^+$ est utilisée
pour inverser~\eqref{jacobianDef}:
\begin{equation}
  \mbf{\dot{q}} = \mbf{J}^+ \mbf{\dot{e}^*}
  \label{eq:invKine}
\end{equation}

La cinématique inverse est généralement utilisée pour résoudre le problème de
géométrie inverse consistant à trouver une configuration $\mbf{q}$ telle que
$\mbf{e}(\mbf{q}) = 0$. 
Pour qu'un robot puisse exécuter une t\^ache, la résolution
de~\eqref{eq:invKine} est itérative.

En choisissant~\eqref{expoDecrease} dans~\eqref{eq:invKine}, on retrouve
la méthode itérative de Newton-Raphson (voir figure~\ref{fig:NewtonRaphson})
consistant à faire converger $\mbf{e}(\mbf{q})$ vers $0$ de manière itérative.
En utilisant des gains $\lambda$ suffisamment petits, la suite de 
positions obtenue lors des itérations successives 
constitue de plus un chemin reliant la position initiale à la solution
problème de géométrie inverse. Ce chemin peut ensuite
être utilisé comme commande au robot.
\begin{figure}[t]
  \begin{center}
    \resizebox{0.6\textwidth}{!}{
    \input{figures/chapitre4/newtonRaphson.tex}
    }
  \end{center}
  \caption{La méthode de Newton-Raphson pour résoudre itérativement $x(q) =0$.}
  \label{fig:NewtonRaphson}
\end{figure}

%-------------------------------------------------------------------%
%-------------------------------------------------------------------%
\section{Redondance et hiérarchie de t\^aches}
\label{chap:Redondance}
%-------------------------------------------------------------------%
%-------------------------------------------------------------------%
La solution~\eqref{eq:invKine} peut être généralisée en utilisant le
formalisme de la redondance présenté dans~\cite{siciliano91}. Le robot devient redondant par rapport à une t\^ache
donnée en un point $\mbf{q}$ lorsque la dimension du tangeant à l'espace de configurations
en $\mbf{q}$ est supérieure
à la dimension de l'espace $\mathbb{I}(\mbf{q}) = \mathrm{Image}(\mbf{J}(\mbf{q}))$ de la t\^ache considérée. Ainsi la généralisation
de~\eqref{eq:invKine} donne:
\begin{equation}
  \mbf{\dot{q}} = \mbf{J}^+ \mbf{\dot{e}^*} + \mbf{P}\mbf{z} 
  \label{eq:controlLaw}
\end{equation}
où $\mbf{P} = \mbf{I} - \mbf{J}^+\mbf{J}$ est le projecteur orthogonal
dans l'espace nul de $\mbf{J}$, et $\mbf{z}$ est un vecteur représentant
une commande secondaire qui, gr\^ace au projecteur,
ne va pas perturber la t\^ache $\mbf{\dot{e}^*}$ en exploitant par exemple
les degrés de liberté non utilisés. Une hierarchie de t\^aches est ainsi
établie en définissant la t\^ache principale comme étant prioritaire par
rapport à une t\^ache qui va se réaliser gr\^ace à la commande secondaire.

\subsection{Loi de commande pour deux t\^aches}
La commande secondaire $\mbf{z}$ introduite dans~\eqref{eq:controlLaw} peut
être utilisée pour exécuter au mieux une seconde t\^ache en définissant $\mbf{z}$
comme étant la commande qui va réaliser la relation de référence de la t\^ache: $\mbf{\dot{e}_2^*} = \mbf{J}_2 \mbf{\dot{q}}$.
En introduisant~\eqref{eq:controlLaw} dans cette dernière équation on obtient:
\begin{equation}
  \mbf{\dot{e}_2^*} = \mbf{J}_2 \mbf{J}^+ \mbf{\dot{e}^*} + \mbf{J}_2 \mbf{P}\mbf{z}
  \label{eq:zCompute}
\end{equation}
Cette dernière équation permet de calculer $\mbf{z}$.
La commande associée pour réaliser deux t\^aches $\mbf{\dot{e}_1}$ et $\mbf{\dot{e}_2}$ (qui sera
de priorité inférieure à $\mbf{\dot{e}_1}$) ayant
comme consigne $\mbf{\dot{e}_1^*}$ et $\mbf{\dot{e}_2^*}$, s'écrit:
\begin{equation}
  \mbf{\dot{q}} = \mbf{J}_1^+ \mbf{\dot{e}_1^*} + \mbf{P}_1 (\mbf{J}_2 \mbf{P}_1)^+(\mbf{\dot{e}}_ 2^* - \mbf{J}_2\mbf{J}_1^+ \mbf{\dot{e}_1^*})
  \label{eq:controlLaw2taskExpand}
\end{equation}
Etant donné que $\mbf{P}_1$, l'opérateur de projection, est hermitien et idempotent, 
\eqref{eq:controlLaw2taskExpand} s'écrit plus simplement:
\begin{equation}
  \mbf{\dot{q}} = \mbf{J}_1^+ \mbf{\dot{e}_1^*} + (\mbf{J}_2 \mbf{P}_1)^+(\mbf{\dot{e}}_ 2^* - \mbf{J}_2\mbf{J}_1^+ \mbf{\dot{e}_1^*})
  \label{eq:controlLaw2taskSimple}
\end{equation}

Un couplage apparait donc entre les deux t\^aches: la réalisation de $\mbf{\dot{e}_2}$ 
est modifiée pour ne pas perturber $\mbf{\dot{e}_1}$.

\subsection{Loi de commande pour un nombre arbitraire de t\^aches}
La loi de commande calculée pour deux t\^aches est étendue à un nombre arbitraire de t\^aches
par récursion. Ces t\^aches sont ordonnées par priorités: une t\^ache
de priorité 1 étant la t\^ache de plus haute priorité, et une t\^ache de priorité $n$ la plus basse.
Ce qui équivaut à dire qu'une t\^ache $\mbf{\dot{e}}_i$ ne doit pas perturber une
t\^ache $\mbf{\dot{e}}_j$ si $i>j$. La formulation récursive
s'écrit~\cite{siciliano91} :
\begin{equation}
  \left\{
  \begin{array}{ll}
  \mbf{\dot{q}}_0 &= 0\\
  \dot{\mbf{q}}_i &= \dot{\mbf{q}}_{i-1} + (\mbf{J}_i \mbf{P}_{i-1}^{A})^+
(\dot{\mbf{e}}^*_i - \mbf{J}_i \dot{\mbf{q}}_{i-1}) , \ \ i = 1 \ldots n
\end{array}
  \right.
  \label{eq:controlLawNtask}
\end{equation}

\noindent où $\mbf{P}_{i-1}^{A}$ est le projecteur
dans l'espace nul de la jacobienne augmentée
$\mbf{J}_i^A = (\mbf{J}_1, \ldots \mbf{J}_i)$.
La vitesse articulaire
réalisant toutes les t\^aches est $\dot{\mbf{q}}^* = \dot{\mbf{q}}_n$.
Une formulation récursive pour le calcul du projecteur $\mbf{P}_{i-1}^{A}$ 
est proposée par~\cite{baerlocher98}.
\begin{equation}
  \left\{
      \begin{array}{rl}
        \mbf{P}_0^A &= \mbf{I}\\
        \mbf{P}_i^A &= \mbf{P}_{i-1}^A - (\mbf{J}_i \mbf{P}_{i-1}^A)^+(\mbf{J}_i \mbf{P}_{i-1}^A) , \ \ i = 1 \ldots n
      \end{array}
    \right.
    \label{eq:projector}
\end{equation}

\noindent où $\mbf{I}$ est la matrice identité, $\mbf{J}_i$ est la matrice jacobienne de
la t\^ache $i$. 
Une implémentation complète de cette approche est proposée et 
détaillée dans~\cite{mansard07}
sous le nom de \emph{pile de t\^aches}. La représentation
du mouvement par une pile de t\^aches permet d'ajouter des t\^aches,
d'en supprimer ou de permuter les ordres de priorités entre deux t\^aches
facilement. L'implémentation utilisée préserve la continuité de la loi de commande
durant ces opérations. Les contraintes, comme les limites articulaires,
peuvent \^etre prises en compte localement.


 \section{Applications de la pile de t\^aches}
Dans ce paragraphe les différents types de t\^aches robotiques utilisées dans la suite des travaux
sont décrites. Puis un exemple d'exécution de t\^aches en parallèle est donné pour
illustrer les effets des couplages des t\^aches.
Dans le chapitre suivant nous présentons un exemple applicatif de génération de 
mouvement dans le cadre de l'imitation sans reconnaissance d'un mouvement
humanoïde. 
\subsection{Exemples de t\^aches robotique}
\subsubsection{T\^ache position 6D}
Un repère qui est attaché à un corps du robot doit atteindre une position 6D désirée.
L'erreur correspond à la transformation homogène $\mbf{M}_e = \mbf{M}(\mbf{M}^*)^{-1}$ permettant de passer de la position
actuelle à la position désirée avec 
$\mbf{M} = \left( \begin{array}{cc} \mbf{R}& \mbf{t}\\ \mbf{0} & \mbf{1} \end{array} \right)$.
La figure~\ref{fig:feature6d} illustre ce type de t\^ache.
\begin{figure}[t]
  \begin{center}
    \resizebox{0.95\textwidth}{!}{
    \input{figures/chapitre3/feature6d.tex}
    }
  \end{center}
  \caption{Exemple d'une t\^ache de position 6D.}
  \label{fig:feature6d}
\end{figure}
Cette représentation matricielle est redondante car il suffit de trois variables
indépendante pour identifier une matrice de rotation dans le groupe des rotations
SO(3). Une représentation vectorielle dans le groupe SE(3) (\emph{Special Euclidian Group} qui 
est isomorphe à $\mathbb{R}^3 \times$SO(3) ) est donc préférée.
Le vecteur erreur $\mbf{e}$ est défini par:
\begin{equation}
  \mbf{e} = \left( \begin{array}{c} \mbf{t}_e \\ \mbf{u}_{\theta, e} \end{array} \right)
  \label{eq:feature6dE}
\end{equation}
\noindent où $\mbf{u}_\theta$ est le vecteur représentant une rotation équivalente
à une matrice de rotation $\mbf{R}$: sa direction correspond à l'axe de rotation
et sa norme $\Vert \mbf{u}_\theta \Vert$ correspond à la valeurs de l'angle de rotation par rapport à son axe (voir figure~\ref{fig:utheta}).
On note que pour un suivi de trajectoire, l'erreur $\mbf{e}(t)  = \left( \begin{array}{c} \mbf{t}_e(t) \\ \mbf{u}_{\theta, e}(t) \end{array} \right) $ doit être petit.
\begin{figure}[t]
  \begin{center}
    \resizebox{0.1\textwidth}{!}{
    \input{figures/chapitre3/utheta.tex}
    }
  \end{center}
  \caption{Représentation d'une rotation avec un vecteur.}
  \label{fig:utheta}
\end{figure}

Dans la suite du document, on notera $\mbf{e}_{rh}$ la t\^ache de prise droite,
asservissant le point de contrôle de la dernière articulation du poignet
(telle que donnée dans le modèle classique du robot), voir figure~\ref{fig:feature6d}.
Cette t\^ache met en \oe uvre les chaînes articulaires du \emph{free flyer}, du torse
et du bras droit. Symétriquement, la t\^ache de prise main gauche sera noté $mbf{e}_{lh}$.
On nommera les t\^aches de position 6D \emph{prise orientée}.

\subsubsection{T\^ache de position 3D et de rotation 3D}
Il est possible de ne sélectionner qu'une partie des dimensions 
d'une t\^ache de position 6D en supprimant les lignes de la jacobienne
correspondant aux dimensions que l'on ne veut pas considérer.
On peut ainsi définir des t\^aches de positions dans n'importe quel
sous-espace de dimensions inférieures.
Par convention dans la suite du manuscrit, sauf mention contraire, lorsque l'on parle de t\^ache de position 3D,
nous ne considérons que les composantes de translations: 
\begin{equation}
\mbf{e} = \mbf{t}_e
\label{eq:taskPosition}
\end{equation}
De même, lorsque l'on parle de t\^ache de rotation 3D,
   nous ne considérons que les composantes de rotations:
\begin{equation}
\mbf{e} = \mbf{u}_{\theta, e} 
\label{eq:taskRotation}
\end{equation}

Dans la suite du document, les t\^aches nommées \emph{prise} sont des t\^aches
de position en 3D. 

\subsubsection{T\^ache de position du centre de masse}
L'équilibre statique d'un robot humanoïde est vérifiée lorsque la 
projection au sol du centre de masse (com) est à l'intérieur du polygone de support du robot
(voir figure~\ref{fig:supportPoly}).
\begin{figure}[t]
  \begin{center}
    \resizebox{0.3\textwidth}{!}{
    \input{figures/chapitre3/supportPoly.tex}
    }
  \end{center}
  \caption{Le polygone de support est défini par les frontières des empreintes de pieds.}
  \label{fig:supportPoly}
\end{figure}
Il est donc intéressant de contrôler la position du centre de masse. Sa position est
calculée en fonction de la position du centre de masse de chacun des corps composant le robot:
\begin{equation}
  \mbf{p}_{\textsf{com}} = \frac{1}{\sum_i m_i} \sum_i m_i\mbf{p}_{\textsf{com}_i}
  \label{eq:com}
\end{equation}
La t\^ache de position du centre de masse correspond à un cas particulier 
de la t\^ache de position 3D: les colonnes de la jacobienne associée 
à cette t\^ache sont pondérées par les masses des corps du robot~\cite{boulic96, sugihara02}.
L'erreur est donc exprimé par:
\begin{equation}
e = \mbf{p}_{com} - \mbf{p}_{com}^*
\label{eq:taskCom}
\end{equation}

\subsubsection{T\^ache regard}
Cette t\^ache consiste à centrer un point dans le plan image d'une caméra
situé au niveau de la t\^ete du robot (voir figure~\ref{fig:featureVis}).
Nous décrivons brièvement le principe général dans ce paragraphe. Une description
complète, avec notamment le calcul de la jacobienne est présentée dans~\cite{espiau92, chaumette06, chaumette07}.
\begin{figure}[t]
  \begin{center}
    \resizebox{0.95\textwidth}{!}{
    \input{figures/chapitre3/featureVis.tex}
    }
  \end{center}
  \caption[T\^ache de regard.]{La t\^ache de regard centre un point $\mbf{p}^*$ de l'espace
  dans le plan image de la caméra situé au niveau de la tête du robot.}
  \label{fig:featureVis}
\end{figure}
Un point $\mbf{M} = \left( \begin{array}{c} X\\ Y\\ Z\end{array} \right)$ dans l'espace
est projeté dans l'image d'une caméra, ayant un centre optique $\mbf{C}$ et une 
distance focale $f$, en un point $m$ (voir figure~\ref{fig:cameraProj}).
La position du point $m = \left( \begin{array}{c} x\\ y \end{array} \right)$ est alors obtenue par la relation:
\begin{equation}
  \left( \begin{array}{c} x\\ y \end{array} \right) = 
    \left( \begin{array}{c} Xf/Z\\ Yf/Z \end{array} \right) 
  \label{eq:camProj3D}
\end{equation}
\begin{figure}[t]
  \begin{center}
    \resizebox{0.4\textwidth}{!}{
    \input{figures/chapitre3/cameraProj.tex}
    }
  \end{center}
  \caption{Un point $\mbf{M}$ dans l'espace se projète en un point $m$ dans l'image de la caméra.}
  \label{fig:cameraProj}
\end{figure}
Ainsi le vecteur erreur est défini par :
\begin{equation}
  \mbf{e} = \left( \begin{array}{c} x_e \\ y_e \end{array} \right)
    = \left( \begin{array}{c} x-x^* \\ y-y^* \end{array} \right)
  \label{eq:visError}
\end{equation}
Un exemple d'utilisation de cette t\^ache est présenté dans~\cite{mansard07a}.

\subsubsection{T\^ache double support}
En supposant que les deux pieds du robot sont posés au sol,
la t\^ache de double support consiste à conserver une transformation géométrique
constante entre le repère attaché au pied gauche, et le repère attaché au pied droit.
Ainsi les deux pieds resteront en contact avec le sol (voir figure~\ref{fig:doubleSupport}).
\begin{figure}[t]
  \begin{center}
    \resizebox{0.4\textwidth}{!}{
    \input{figures/chapitre3/doubleSupport.tex}
    }
  \end{center}
  \caption[T\^ache de double support]{La t\^ache double support permet de garder une transformation constante
  égale à $\mbf{M}^*$ entre les repères attachés aux deux pieds du robot $\mathcal{R}_d$ et $\mathcal{R}_g$.}
  \label{fig:doubleSupport}
\end{figure}
Il s'agit en fait d'un autre cas particulier de la t\^ache de position 6D. 
La transformation entre deux repères attachés
à deux corps du robot doit rester constante:
\begin{equation}
\mbf{e} = \left( \begin{array}{c} \tensor[^{r}]{\mbf{t}}{_{l}} \\ \tensor[^{r}]{\mbf{u}}{_{\theta, l}} \end{array} \right)
\label{eq:taskRelative}
\end{equation}
\noindent avec $\tensor[^{r}]{\mbf{t}}{_{l}}$ la translation entre un repère attaché
au pied droit et un repère attaché au pied gauche et $\tensor[^{r}]{\mbf{u}}{_{\theta, l}}$
la rotation entre un repère attaché au pied droit et un repère attaché au pied gauche.

\subsubsection{T\^ache de posture}
Une t\^ache de posture permet de définir un état référence dans l'espace de configuration:
\begin{equation}
  \mbf{e} = \mbf{q} - \mbf{q}^*
  \label{eq:postureTask}
\end{equation}
La figure~\ref{fig:posture} illustre une t\^ache de posture.
\begin{figure}[t]
  \begin{center}
    \resizebox{0.3\textwidth}{!}{
    \input{figures/chapitre3/posture.tex}
    }
  \end{center}
  \caption[T\^ache de posture]{Un exemple de t\^ache de posture, pour l'articulation d'une jambe. $\mbf{q}^*$ étant la 
  configuration désirée.}
  \label{fig:posture}
\end{figure}

La correction de l'erreur agit directement sur $\mbf{q}$ et la jacobienne est une matrice identité.
Le contrôle de la posture pour un robot humanoïde est 
étudié en détail dans le chapitre 4 de~\cite{sentis07}. Les t\^aches de posture
sont utilisées pour optimiser des critères de performances (comme les efforts des actionneurs)
et de reproduire des comportements humain.

\subsection{Couplage des t\^aches}
Gr\^ace au formalisme de la fonction de t\^ache
et de la représentation du mouvement sous forme
de pile de t\^aches il est possible de combiner 
de manière efficace des t\^aches indépendantes à réaliser
pour générer un mouvement pour un robot.
Un couplage entre des t\^aches apparait si pour les réaliser il est nécessaire
d'utiliser des sous-chaînes articulaires communes.
En considérant la pile de t\^aches illustrée dans la figure~\ref{fig:couplingSot},
les t\^aches de regard, et de position de mains gauche et droite  ont des
influences mutuelles.
\begin{figure}[t]
  \begin{center}
    \begin{tabular}{|c|}
      \hline
      Main gauche\\
      \hline
      Main droite\\
      \hline
      Regard\\
      \hline
      Centre de masse\\
      \hline
      Double support\\
      \hline
    \end{tabular}
  \end{center}
  \caption{La pile de t\^aches utilisé pour illustrer le couplage.}
  \label{fig:couplingSot}
\end{figure}

La figure~\ref{fig:fullSot} illustre le mouvement
complet issu de cette pile de t\^aches. Pour illustrer les 
couplages, les différentes t\^aches de la piles sont
isolées pour générer plusieurs mouvements. 
\begin{figure}[t]
  \begin{center}
    \subfigure{
    \resizebox{0.36\linewidth}{!}{
    \includegraphics[trim=250px 0px 250px 0px, clip=true]{figures/chapitre3/GRL1.ps}
    }
    }
    \subfigure{
    \resizebox{0.36\linewidth}{!}{
    \includegraphics[trim=250px 0px 250px 0px, clip=true]{figures/chapitre3/GRL2.ps}
    }
    }
    \subfigure{
    \resizebox{0.36\linewidth}{!}{
    \includegraphics[trim=250px 0px 250px 0px, clip=true]{figures/chapitre3/GRL3.ps}
    }
    }
    \subfigure{
    \resizebox{0.36\linewidth}{!}{
    \includegraphics[trim=250px 0px 250px 0px, clip=true]{figures/chapitre3/GRL4.ps}
    }
    }
  \end{center}
  \caption[Exemple de t\^aches couplées dans un mouvement.]{Un exemple de mouvement comportant des t\^aches couplées. Les t\^aches
  exécutées sont la t\^ache de double support, le positionnement du centre de masse,
  l'orientation du regard, le positionnement de la main droite et gauche.}
  \label{fig:fullSot}
\end{figure}

La pile de t\^aches de base comporte les t\^aches de double support
et de positionnement du centre de masse.
La t\^ache de regard est ajoutée. Pour exécuter 
cette dernière t\^aches, le robot utilise son torse pour se pencher en avant.
Les bras du robot sont donc entraînés par le torse.
Un mouvement au niveau notamment de la main droite
peut être observé alors qu'aucune t\^ache ne contrôle explicitement
le bras droit (voir figure~\ref{fig:gazeCoupling}).
\begin{figure}[t]
  \begin{center}
    \subfigure{
    \resizebox{0.36\linewidth}{!}{
    \includegraphics[trim=250px 0px 250px 0px, clip=true]{figures/chapitre3/G1.ps}
    }
    }
    \subfigure{
    \resizebox{0.36\linewidth}{!}{
    \includegraphics[trim=250px 0px 250px 0px, clip=true]{figures/chapitre3/G2.ps}
    }
    }
    \subfigure{
    \resizebox{0.36\linewidth}{!}{
    \includegraphics[trim=250px 0px 250px 0px, clip=true]{figures/chapitre3/G3.ps}
    }
    }
    \subfigure{
    \resizebox{0.36\linewidth}{!}{
    \includegraphics[trim=250px 0px 250px 0px, clip=true]{figures/chapitre3/G4.ps}
    }
    }
  \end{center}
  \caption[T\^aches de double support, de centre de masse et de regard.]{Le mouvement est généré à partir des t\^aches de double support,
  de positionnement du centre de masse et de la t\^ache de regard.}
  \label{fig:gazeCoupling}
\end{figure}
En ajoutant la t\^ache de position de la main droite avec une position
de référence à l'avant du robot, le mouvement de la main
droite est naturellement modifié. Cependant, le mouvement de la main gauche est aussi
modifié (figure~\ref{fig:gazeRightArm}). Le déplacement de la main gauche 
est causé par le couplage de la t\^ache main droite
et la t\^ache de positionnement du centre de masse en effet, lorsque la main droite
se déplace vers l'avant, le centre de masse se déplace aussi dans cette
direction. Pour compenser le déplacement du centre de masse causé par
le déplacement de la main droite, il est nécessaire de déplacer la main gauche.
Le mouvement du torse est également modifié par rapport au mouvement ne faisant intervenir que
le regard.
\begin{figure}[t]
  \begin{center}
    \subfigure{
    \resizebox{0.36\linewidth}{!}{
    \includegraphics[trim=250px 0px 250px 0px, clip=true]{figures/chapitre3/GR1.ps}
    }
    }
    \subfigure{
    \resizebox{0.36\linewidth}{!}{
    \includegraphics[trim=250px 0px 250px 0px, clip=true]{figures/chapitre3/GR2.ps}
    }
    }
    \subfigure{
    \resizebox{0.36\linewidth}{!}{
    \includegraphics[trim=250px 0px 250px 0px, clip=true]{figures/chapitre3/GR3.ps}
    }
    }
    \subfigure{
    \resizebox{0.36\linewidth}{!}{
    \includegraphics[trim=250px 0px 250px 0px, clip=true]{figures/chapitre3/GR4.ps}
    }
    }
  \end{center}
  \caption[Couplage de la main et du torse.]{La t\^ache de positionnement de la main droite perturbe
  la position du torse et de la main gauche.}
  \label{fig:gazeRightArm}
\end{figure}

%-------------------------------------------------------------------%
%-------------------------------------------------------------------%
%\subsection{Application de la fonction de t\^ache pour la reconnaissance et l'imitation}
%\label{chap:basics:sec:application}
%-------------------------------------------------------------------%
%-------------------------------------------------------------------%

%Nous avons vu précédemment que le formalisme de la fonction
%de t\^aches permettait de générer des mouvements en créant une loi
%de commande satisfaisant l'exécution d'un ensemble de t\^aches 
%rangées par ordre de priorité. 
%Dans ce paragraphe nous présentons une autre application
%du formalisme de la fonction de t\^aches, qui sera développé dans le chapitre 5,
%à savoir utiliser ce formalisme pour effectuer
%de la reconnaissance de t\^aches.
%
%Ici, la reconnaissance
%de t\^aches consiste à déterminer l'ensemble des t\^aches (ou contr\^oleurs)
%qui ont été utilisées pour générer le mouvement observé (voir la figure~\ref{fig:sotReco}).
%\begin{figure}[t]
%  \begin{center}
%    \includegraphics[width=\linewidth]{figures/chapitre3/sotReco.ps}
%  \end{center}
%  \caption{Un mouvement est observé, puis après analyse, les t\^aches qui sont les plus probablement actives sont sélectionnées
%  parmi un ensemble de t\^aches candidates. Les valeurs numériques associées à ces t\^aches sont calculées. Ainsi une
%  pile de t\^aches est complètement reconstruite, permettant de générer des lois de commande pour n'importe quel
%  robot.}
%  \label{fig:sotReco}
%\end{figure}
%Le mouvement observé est analysé en se basant sur la connaissance 
%du comportement des t\^aches ainsi que la connaissance de toutes les t\^aches qui
%sont susceptibles d'être utilisées.
%Gr\^ace à ce formalisme, l'imitation est une conséquence directe de la reconnaissance,
%puisqu'imiter un mouvement revient à exécuter une pile de t\^aches identique.

 \section*{Conclusion}

Dans ce chapitre nous avons présenté le formalisme sur lequel
se base la suite des travaux présentés dans ce manuscrit: le formalisme de la fonction de 
t\^aches. Les méthodes utilisées pour la résolution de la cinématique inverse
sont utilisés d'une manière analogue pour manipuler les fonctions de t\^aches.
Nous avons présenté une méthode pour construire efficacement des lois de commandes,
qui permettent de réaliser plusieurs t\^aches parallèles en
respectant un ordre de priorité, en les combinant sous
forme de pile de t\^aches. Les t\^aches de priorités inférieures
sont réalisées aux mieux en exploitant l'espace nul des t\^aches
plus prioritaires afin de ne pas perturber leurs réalisations.
Chaque t\^ache est exprimée dans l'espace le plus adaptée à celle-ci.
Cette caractéristique permet d'avoir une grande expressivité
pour le contr\^ole d'un robot.
Une t\^ache peut \^etre définie dans un espace identique
aux données issues d'un capteur. Ce qui permet par exemple de suivre
facilement l'évolution de l'exécution de la t\^ache.
Dans le chapitre suivant, nous présentons une application directe
de la structure de la pile de t\^aches pour effectuer
une imitation de mouvements non structurés sans reconnaissance.




 %\section*{Introduction}
%\label{sec:intro}
%-------------------------------------------------------------------%
%-------------------------------------------------------------------%
\chapter{Imitation sans reconnaissance}
\label{chap:imitation}
%-------------------------------------------------------------------%
%-------------------------------------------------------------------%

L'imitation sans reconnaissance consiste à exécuter un mouvement
en utilisant directement des données issues d'une démonstration. Aucune
reconnaissance de ce qui doit être reproduit n'est effectuée.
Pour obtenir les données à partir d'une démonstration, les
techniques de capture de mouvements sont utilisées pour encoder 
un mouvement dans un format adapté aux besoins.
Dans ce chapitre, nous présentons les 
principales technologies utilisées pour effectuer de la capture
de mouvements, puis nous présenterons un peu plus en détail le système utilisé
dans le cadre de nos travaux ainsi que les différentes étapes qui constituent
l'utilisation de la capture de mouvements:
calibration du système, définition de la structure à capturer, interprétation
et phase de post-traitements des données acquises.
Ensuite nous appliquons le principe de la pile de t\^aches, présenté
dans le chapitre précédent, pour reproduire un mouvement difficilement 
reconnaissable car n'obéissant pas à une loi particulière.
Nous prenons l'exemple du mouvement d'attente d'un humain, pour le reproduire sur
un robot HRP-2.
Ce type de mouvement peut se traduire par la réalisation de mouvements inconscients de l'humain
lorsqu'il exécute l'action \emph{ne rien faire}.


 %-------------------------------------------------------------------%
%-------------------------------------------------------------------%
\section{Capture de mouvements}
\label{chap:mocap}
%-------------------------------------------------------------------%
%-------------------------------------------------------------------%
La capture de mouvements consiste à enregistrer un mouvement indépendamment 
de l'aspect visuel, afin de l'encoder dans un format adapté aux besoins.
Par exemple le format pourra faciliter une analyse ou une édition de mouvements.
Le mouvement enregistré peut être issu de n'importe quel objet ou individu réel.
Ainsi, les données acquises peuvent avoir des applications très variées.
Les paragraphes suivants présentent quelques exemples d'applications de la capture de mouvements,
puis les difficultés générales rencontrées. 
Ensuite nous présentons les différentes technologies 
développées pour effectuer de la capture de mouvement et détaillons le 
processus de capture pour le système disponible dans ces travaux.

\subsection{Applications}
La capture de mouvements est tout d'abord utilisée pour analyser 
des mouvements. Dans le domaine biomédical pour l'analyse de mouvements de marche par exemple ou
pour l'analyse de performances sportives. La visualisation 3D
de la performance de l'athlète lui permet de déceler les points perfectibles de
sa technique plus facilement qu'avec une simple vidéo gr\^ace à la possibilité 
de changement d'angle de vue. La capture de mouvements peut aussi
servir de mécanisme d'entrée pour l'interaction homme-machine.
Un système de capture de mouvements peut aussi être utilisé comme un outil
de localisation rapide et précise d'objets ou d'obstacles dans l'environnement dans le
cadre d'une planification ou d'une replanification~\cite{baudouin11, stilman05}. 

Mais l'application la plus populaire des techniques de capture de mouvement
est l'animation graphique dans le cinéma ou les jeux vidéo afin de doter
des personnages virtuels de mouvements visuellement réalistes. En effet,
il est très difficile de créer des mouvements réalistes et précis
capable de traduire des caractéristiques subtiles qu'une personne produit.
La technique d'animation manuelle (sans capture) la plus directe consiste à définir les
positions et configurations d'objets à chaque instant de temps.
Il est possible de réduire la quantité de travail de cette technique gr\^ace à 
des outils informatique. Des configurations clefs sont définies, puis
le mouvement entre deux configurations successives est calculé par interpolation~\cite{burtnyk76, kovar03}.
Cependant, ces techniques nécessitent
un animateur doté d'une grande expérience pour identifier et reproduire
de façon convaincante les propriétés des mouvements.
L'importance de l'expérience dans le domaine de l'animation
est soulignée dans~\cite{lasseter87} où une analogie entre les principes d'animation 
traditionnelle dessinée à la main et l'animation 3D assistée par ordinateur est présentée.
Le travail d'animation reste dans tous les cas considérable.
C'est pour cette raison qu'il peut être intéressant de
transférer des mouvements réels vers l'objet ou le personnage à animer
en utilisant de la capture de mouvements. L'avantage direct est de pouvoir
être capable de reproduire beaucoup de mouvements sans chercher à 
trouver un modèle mathématique permettant de générer un mouvement particulier
et obtenir ainsi une grande expressivité théorique de mouvements.

\subsection{Difficultés générales de la technique}
Le transfert de mouvements n'est pas direct à cause, entres autres, des imperfections
des systèmes de capture de mouvements. Les sessions 
de capture de mouvements complexes sont souvent suivies d'étapes de 
post-traitements. Les données collectées peuvent
être bruitées de manière sensible ou présenter des discontinuités en
fonction des technologies utilisées. Par conséquent, ces données 
doivent être nettoyées si nécessaire. Ensuite, ces mouvements peuvent être
édités en considérant que le mouvement
est un signal~\cite{bruderlin95}. L'édition de mouvement peut être motivée par plusieurs raisons~\cite{gleicher00}.
Si on veut réutiliser un mouvement déjà enregistré
en y apportant des petits changements comme adapter le mouvement
à un personnage possédant des caractéristiques physiques différentes. Il s'agit du problème 
connu sous le nom de \emph{motion retargeting}~\cite{gleicher98}. 
Un des problèmes récurrent du \emph{motion retargeting} est que le mouvement
recallé perd certaines propriétés. Notamment, lors des mouvements de marceh, les pieds du personnage
perdent le contact avec le sol, ou glissent sur celui-ci. Une autre étape 
d'édition est donc nécessaire pour obtenir un mouvement cohérent.
D'autres changements peuvent être apportés pour varier 
des mouvements dupliqués pour animer une foule par exemple. 
L'édition peut aussi être motivée par la volonté 
de créer des mouvements physiquement impossible ou corriger des imperfections
issues de la prestation de l'acteur. Une autre étape délicate est l'association
des données éditées aux éléments virtuels à animer. Cette dernière étape
peut être très complexe. 
Par exemple l'animation de visage est considéré comme un problème à part entière et des
techniques de réductions de dimensions sont même utilisées~\cite{deng06}.

Malheureusement, dans la plupart des cas, l'édition est
un processus lourd de par la nature de la capture de mouvements.
Les données correspondent aux configurations (ou postures) des éléments
enregistrés, par conséquent, l'édition portera sur un important volume de données.
Le volume de données cache souvent les problèmes liés à des erreurs de capteurs.
Par exemple des erreurs de suivi ou des discontinuités trop importantes.
Ainsi, la correction des erreurs nécessite un temps non négligeable de 
post-traitements.
De plus ces données ne comportent aucune sémantique. Il s'agit de données
bas niveau, et donc il n'y a pas directement d'informations sur les propriétés importantes
du mouvement ni sur les motivations du mouvement. Il faut donc beaucoup de données
pour avoir une grande couverture de type de mouvements (mouvement exécuté par un personnage avec
un certain degré de fatigue, de blessures, de joie, à différentes vitesses\ldots), 
et un problème d'explosion combinatoire
peut ainsi apparaître~\cite{gleicher08}.

\subsection{Systèmes optiques avec marqueurs}
Plusieurs caméras sont utilisées simultanément pour acquérir la
position 3D de marqueurs spécifiques en émettant des rayons infrarouge se réfléchissant
sur les marqueurs. Les marqueurs se présentent
sous la forme de petites billes réfléchissantes légères, ainsi l'acteur n'est pas beaucoup gêné
par le port de marqueurs.
Le nombre de caméras nécessaire dépend du volume de capture souhaité et
de la puissance de calcul disponible pour traiter les données des caméras.
%La figure~\ref{fig:cortex} montre le logiciel Cortex de Motion Analysis dédié à ce types de système
%de capture de mouvements.
%\begin{figure}[t]
%  \begin{center}
%    \includegraphics[width=0.5\linewidth]{figures/chapitre4/cortexwalk.ps}
%  \end{center}
%  \caption{Logiciel Cortex est dédié à la capture de mouvement en utilisant un sytème optique à plusieurs
%  caméra infrarouges pour localiser des marqueurs réfléchissant.}
%  \label{fig:cortex}
%\end{figure}
Les systèmes optiques nécessitent une calibration: les caméras suivent
des objets de dimensions connues que le logiciel maître reconnait.
Puis en combinant les informations des caméras, la position des caméras dans l'espace 
est calculée. 
Pour estimer la position des marqueurs dans l'espace, 
les images provenant des caméras sont filtrées afin d'en extraire
les positions dans l'image de la caméra des marqueurs visibles.
Ces images filtrées sont utilisées pour effectuer une correspondance spatiale des marqueurs
(voir la figure~\ref{fig:triang}).
\begin{figure}[t]
  \begin{center}
    \resizebox{0.5\textwidth}{!}{
    \input{figures/chapitre4/epip.tex}
    }
  \end{center}
  \caption{Le point $\mbf{x}$ se projète dans le plan image des caméras $C$ et $C'$ respectivement en
  $\boldsymbol{\mu}$ et $\boldsymbol{\mu}'$.}
  \label{fig:triang}
\end{figure}
Un point dans l'espace 3D se projette en un point
dans le plan image 2D d'une caméra $C$ selon l'équation:
\begin{equation}
  \boldsymbol{\mu} = \mbf{P} \mbf{x}
  \label{eq:projCamera}
\end{equation}
\noindent où $\boldsymbol{\mu}$ est le vecteur de dimension $3$ des coordonnées homogènes
du point dans l'image de la caméra, $\mbf{P}$ est la matrice de projection de dimension $3\times4$
de la caméra et $\mbf{x}$ le vecteur de dimension $4$ des coordonnées homogènes du point dans l'espace
cartésien. La matrice $\mbf{P}$ est calculée lors de la phase de calibration de la caméra
et traduit la projection, la mise à l'échelle de l'image, la translation à l'origine (dans le plan image)
et la position et orientation de la caméra dans le repère du monde.

Cette équation de projection peut s'écrire à l'aide d'un produit vectoriel:
\begin{equation}
  \boldsymbol{\mu} \times \mbf{P} \mbf{x} = 
                        \left( 
			\begin{array}{c} 
			  \mu_1 \\
			  \mu_2 \\
			  1
			\end{array}
			\right) 
			\times 
			\left( 
			\begin{array}{c} 
			  \mbf{P}_1^T \\
			  \mbf{P}_2^T \\
			  \mbf{P}_3^T 
			\end{array} 
			\right)  \mbf{x} = \mbf{0}  
  \label{eq:prodVecProjCam}
\end{equation}

Le développement conduit à deux équations linéairement indépendantes:
\begin{align}
  (\mu_1 \mbf{P}_3^T - \mbf{P}_1^T) \mbf{x} & = 0\\
  (\mu_2 \mbf{P}_3^T - \mbf{P}_2^T) \mbf{x} & = 0\\
  \label{eq:syscam}
\end{align}
Les équations associées à chacune des caméras sont regroupées pour former un système
$\mbf{A}\mbf{x} = \mbf{0}$ dont la résolution sur $\mbf{x}$ par les moindres carrés (à cause du bruit)
fournit la position 3D du point observé.

Les données produites par la capture contiennent les positions 3D dans l'espace Cartésien
de chacun des marqueurs utilisés. Cependant, bien que la position
des marqueurs soit connue, le système n'a aucun autre indice que
la continuité des positions pour les identifier, contrairement aux systèmes 
électromagnétiques par exemple dans lesquels chaque émetteur possède sa propre signature.
Ceci peut poser problème en cas d'occlusion d'un marqueur qui pourrait être perdu dans la suite
du mouvement. Des solutions matérielles peuvent être utilisées pour 
surmonter ce problème, comme l'utilisation de marqueurs actifs (avec des diodes électroluminescentes par exemple)
pour distinguer les marqueurs. Cependant ces marqueurs sont plus encombrant,
souvent c\^ablés et nécessite des dispositifs d'alimentations embarqués.

Les données obtenues sont précises, il est relativement aisé de changer la configuration des marqueurs
en cas de besoin (occlusions, changement d'acteur ayant une taille différente).
La liberté des mouvements est importante par l'absence de c\^ables. 
L'espace de capture peut être grand (en fonction du nombre de caméra).
Il est également possible de construire un squelette virtuel calculé à partir des points 3D 
correspondant aux marqueurs~\cite{silaghi98}. Les données issues des
transformations géométriques des os du squelette peuvent ensuite être exportées vers des 
logiciels d'animation graphique.

En revanche, l'environnement dans lequel
se déroule la capture doit être contr\^olé: la lumière du soleil, les objets réfléchissants
apportent beaucoup de bruits. De plus la calibration doit être fréquente si les caméras ne sont pas 
fixe (utilisation de trépieds).

\subsection{Systèmes optiques sans marqueurs}
Il est également possible d'enregistrer des mouvements 
sous formes de données de positions avec de simples caméras via une phase
de post-traitement sur le ou les flux vidéos.
La précision de ces techniques est faible, en effet,
il peut être difficile d'estimer les rotations d'un membre humain à cause de
trop importantes ambiguïtés provenant de l'image ou du manque de points caractéristiques.
L'étude présenté dans~\cite{gleicher02} souligne la nécessité d'utiliser des hypothèses
supplémentaire sur ce qui est observé (dynamique des mouvements, limitations géométrique du monde\ldots).
Par exemple, une technique permettant
de générer des mouvements physiquement réaliste à partir d'une séquence 
vidéo monoculaire est présentée dans~\cite{wei10}. L'utilisation
d'un modèle physique contraint la reconstruction
du mouvement aux lois de la physique.
Mais elle permet aussi de calculer
les couples et les forces de contacts correspondant au mouvement observé.
Des informations de profondeurs, en plus des images 2D peuvent
aussi être utilisées pour la reconnaissance de pose humaine~\cite{shotton11}.

\subsection{Dispositifs de suivi électromagnétique}
Pour ces systèmes, la mesure des distances et des orientations se base sur
les champs magnétiques créés par un émetteur et captés par un recepteur~\cite{raab79}.
La figure~\ref{fig:magMocap} montre un exemple de placement de capteurs électromagnétiques
sur une actrice.
\begin{figure}[p]
  \begin{center}
    \includegraphics[width=0.4\linewidth]{figures/chapitre4/magMocap.ps}
  \end{center}
  \caption[Capteurs magnétiques.]{Un exemple de placement de capteurs issu d'un système de capture de mouvements électromagnétique (image tirée de~\cite{bodenheimer97}).}
  \label{fig:magMocap}
\end{figure}
Lorsqu'un courant est appliqué à un bobinage, un champ magnétique est créé (voir figure~\ref{fig:ringMagnetic}).
\begin{figure}[p]
  \begin{center}
    \resizebox{0.9\textwidth}{!}{
    \input{figures/chapitre4/ringMagnetic.tex}
    }
  \end{center}
  \caption[Fonctionnement d'un capteurs magnétique.]{Un courant appliqué à une spire crée un champ magnétique $\mbf{H}$, de composante radiale $H_r$,
  et tangentielle $H_\theta$.}
  \label{fig:ringMagnetic}
\end{figure}
\FloatBarrier
Les composantes radiale et tangentielle de ce champ magnétique sont décris par:
\begin{align}
  H_r & =  \frac{M}{2\pi r^3}\mathrm{cos}(\theta)\\
  H_\theta & =  \frac{M}{4\pi r^3}\mathrm{sin}(\theta)
  \label{eq:magField}
\end{align}
\noindent où $M = N I S$ est le moment magnétique, $N$ étant le nombre de spires
de la bobine, $I$ l'intensité du courant et $S$ la surface entouré par une spire.
Le récepteur mesure les champs magnétiques créés par l'émetteur afin de 
déterminer leurs distances et leurs orientations relatives.
Un couple émetteur/récepteur fournit donc des positions 6D.
La précision des positions calculées décroit avec l'augmentation de la distance
entre l'émetteur et le récepteur.
Historiquement, les recepteurs, placés sur l'objet à suivre sont c\^ablés.
Ceci rendait l'utilisation de ce type de matériel difficile pour des mouvements amples
ou complexes. Ce n'est que récemment que des systèmes sans fils sont apparus~\cite{vanacht07}.
Ce type de système est aussi sensible à l'environnement qui
ne doit pas comporter d'éléments métalliques car ils causent des
distorsions du champ magnétique qui faussent ainsi l'estimation
de la position. Cependant, ils présentent
l'avantage de ne pas avoir de problème d'occlusions.
De plus, il est possible d'estimer 
les positions de l'épaule, du coude et du poignet ainsi que le vecteur
de configuration de dimension sept associé au modèle d'un bras avec seulement deux capteurs 3D
en effectuant une phase de calibration~\cite{rezzoug10}.
\subsection{Dispositifs de suivi inertiel}
Des capteurs inertiels peuvent être utilisés pour effectuer de la 
capture de mouvements comme par exemple dans~\cite{liu11}.
La figure~\ref{fig:inertialMocap} montre un capteur inertiel.
\begin{figure}[t]
  \begin{center}
    \subfigure[]{
    \includegraphics[width=0.4\linewidth]{figures/chapitre4/inertialSensor.ps}
  \label{fig:inertialMocap}
    }
    \subfigure[]{
    \resizebox{.4\textwidth}{!} {
      \input{figures/chapitre4/accelerometre.tex}
    }
  \label{fig:accelerometre}
    }
  \end{center}
  \caption[Capteurs inertiels.]{(a)~Un capteur inertiel placé pour mesurer les mouvements d'un bras. Photo tirée de~\cite{vanacht07}. 
  (b)~L'accélération est déterminée à partir de la mesure de la force gr\^ace à la relation $\mbf{F} = m\mbf{a}$}
\end{figure}
Ils peuvent être composés d'accéléromètres
et de gyroscopes. Le suivi d'objets est effectué 
en déterminant les accélérations et les orientations de ces objets. 

Les accélérations sont déterminées en suivant la seconde loi de Newton reliant
la résultante des forces $\mbf{F}$ exercées sur un corps de masse $m$ à son accélération $\mbf{a}$:
$\mbf{F} = m \mbf{a}$. Les accéléromètres contiennent une masse connue attachée à un ressort.
Lorsqu'une accélération est appliquée à l'accéléromètre, l'inertie va pousser la masse 
à compresser le ressort. Cette force est ensuite convertie en signal électrique
gr\^ace par exemple à des capteurs piézoélectriques (voir figure~\ref{fig:accelerometre}).
%\begin{figure}[t]
%  \begin{center}
%  \end{center}
%  \label{fig:accelerometre}
%\end{figure}
L'accélération est ensuite intégrée deux fois pour déterminer la position.
De la même manière, les gyroscopes permettent de déterminer les positions angulaires
d'un objet en mesurant les forces centrifuges d'une masse en rotation. Cette fois-ci,
la force mesurée est proportionnelle à la vitesse angulaire. Une seule intégration
est donc nécessaire pour avoir une mesure de la position angulaire.

Les capteurs inertiels ont l'avantage d'être relativement facile à
mettre en \oe uvre mais les mesures dérivent.
Les erreurs de mesures s'accumulent et par conséquent, 
l'estimation de la position de l'objet diverge.
Le suivi d'objet par capteurs inertiels n'est donc efficace que 
pendant de courtes périodes.

\subsection{Systèmes de capture mécanique}
Il s'agit de systèmes articulés que l'utilisateur manipule. Les angles des articulations sont
directement mesurés gr\^ace à des encodeurs. Ils peuvent prendre plusieurs formes en fonction
du type de données considérés, comme par exemple un petit bras articulé pour obtenir
une position 6D d'un objet manipulé avec la main, un gant pour obtenir
les positions de mains ou de doigts ou encore
un exosquelette si l'on s'intéresse aux mouvements complet du corps humain (voir figure~\ref{fig:mecaMocap}).
Ces systèmes sont limités par les contraintes mécaniques de ces systèmes et sont très encombrant.
\begin{figure}[p]
  \begin{center}
    \subfigure[Phantom de SensAble.]{
    \resizebox{.27\textwidth}{!} {
    \includegraphics{figures/chapitre4/Phantom.eps}
    }
    }
    \subfigure[CyberGlove II de CyberGlove Systems.]{
    \resizebox{.27\textwidth}{!}{
    \includegraphics{figures/chapitre4/pg6.eps}
    }
    }
    \subfigure[Gypsy 7 de Animazoo.]{
    \resizebox{.27\textwidth}{!}{
    \includegraphics{figures/chapitre4/gypsy7_05.ps}
    }
    }
  \end{center}
  \caption[Systèmes de capture de mouvements mécaniques.]{(a)~Un bras articulé à six degrés de liberté. La position et l'orientation du stylet est obtenue
  gr\^ace aux mesures des angles articulaire. (b)~Le gant articulé permet de mesurer la configuration
  de la main, mais permet également de fournir des positions 6D de la main. (c)~L'exosquelette Gypsy 7
  qui permet de mesurer les valeurs articulaires correspondant aux mouvements d'un humain.}
  \label{fig:mecaMocap}
\end{figure}

\section{Système utilisé}
Le LAAS-CNRS est équipé du système de capture de mouvements de la société Motion Analysis.
Le système installé est composé de dix caméras infrarouge (voir figure~\ref{fig:cameraIR}): quatre caméras
standard (modèle \emph{Hawk} 640x480@200Hz), et six caméras haute résolution (modèle \emph{Eagle} 2352x1728@200Hz). 
Les données sont capturées à une fréquence de 200Hz. Les caméras sont pilotées par le 
logiciel nommé Cortex (voir figure~\ref{fig:cortex1}).
\begin{figure}[p]
  \begin{center}
    \includegraphics[width=0.14\linewidth]{figures/chapitre4/camIR.ps}
  \end{center}
  \caption{Une caméra infrarouge.}
  \label{fig:cameraIR}
\end{figure}
\begin{figure}[p]
  \begin{center}
    \includegraphics[width=0.75\linewidth]{figures/chapitre4/cortex1.ps}
  \end{center}
  \caption[Logiciel de capture de mouvements.]{Le logiciel de capture de mouvement utilisé au LAAS-CNRS: Cortex de Motion Analysis.}
  \label{fig:cortex1}
\end{figure}
Les caméras sont réparties autour de la zone d'expérimentation suivant le schéma de la figure~\ref{fig:cameraSetUp}.
Afin d'optimiser le placement des caméras, les caméras à haute résolution 
sont placés pour couvrir la longueur de la zone d'expérimentation, tandis que les 
caméras à basse résolution sont placées pour couvrir la largeur de la zone.
\begin{figure}[p]
  \begin{center}
    \resizebox{0.94\linewidth}{!}{
      \input{figures/chapitre4/cameraSetUp.tex}
    }
  \end{center}
  \caption{Placement des caméras dans la zone d'expérimentation.}
  \label{fig:cameraSetUp}
\end{figure}
La figure~\ref{fig:cameraCoverage} illustre les zones de couvertures des caméras.
\begin{figure}[p]
  \begin{center}
    \includegraphics[width=0.94\linewidth]{figures/chapitre4/cortex3.ps}
  \end{center}
  \caption{Zone de couverture des caméras.}
  \label{fig:cameraCoverage}
\end{figure}

D'une manière générale, la capture de mouvements se décompose en plusieurs étapes.
La qualité du résultat obtenu à chaque étape dépend directement de la précédente 
(voir la figure~\ref{fig:mocapFlow}).
\begin{figure}[p]
  \begin{center}
    \includegraphics[width=0.95\linewidth]{figures/chapitre4/mocapFlow.ps}
  \end{center}
  \caption{Les différentes étapes pour capturer un mouvement.}
  \label{fig:mocapFlow}
\end{figure}
Un exemple du processus complet de capture de mouvement est présenté dans~\cite{bodenheimer97}.

\subsection{Calibration}
La calibration permet d'associer les positions dans le monde réel
et les positions dans l'image des caméras. C'est lors de cette phase que les
paramètres des caméras sont calculés, ces paramètres
dépendent des positions et orientations, des focales de l'objectif,
des facteurs d'echelle suivant les deux axes de l'image et des translations
d'origine de l'image des caméras.
Les matrices de projections des caméras dépendent directement
de ces paramètres.
Dans le cas de notre système, celle-ci se décompose en deux parties. 
\subsubsection{Calibration statique}
La calibration statique permet de définir un repère fixe
au volume étudié et de déterminer une première estimation
des paramètres d'au moins une partie des caméras. Une équerre de précision équipée de quatre marqueurs dont les positions
relatives sont parfaitement connues est utilisée (voir la figure~\ref{fig:Lframe}).
\begin{figure}[t]
  \begin{center}
    \subfigure[]{
    \resizebox{.41\textwidth}{!} {
    \includegraphics[width=0.4\linewidth]{figures/chapitre4/lframe.ps}
    }
  \label{fig:Lframe}
    }
    \subfigure[]{
    \resizebox{.41\textwidth}{!} {
    \includegraphics[width=0.4\linewidth]{figures/chapitre4/wand.ps}
    }
  \label{fig:wand}
    }
  \end{center}
  \caption[Outil de calibration.]{(a)~L'équerre de calibration définit un repère fixe dans lequel les données seront exprimées. (b)~La 
  baguette de précision qui est utilisée pour balayer la zone d'expérimentation.}
\end{figure}
Les paramètres des caméras
qui ont les quatre marqueurs dans leurs champs de vision ont alors une bonne première 
estimation qui sera raffiné par optimisation dans l'étape suivante.
Les autres caméras auront une mauvaise estimation de leurs paramètres, mais ils seront corrigés
lors de l'étape suivante.

\subsubsection{Calibration dynamique}
Elle consiste à balayer, de la manière la plus
homogène possible, tout l'espace d'expérimentation avec une baguette équipée de trois marqueurs
dont les positions relatives sont connues avec précision (voir figure~\ref{fig:wand}).
Les données collectées sont utilisées pour calculer de manière plus précise
les paramètres de toutes les caméras par optimisation.
La couverture des données dans l'espace permet d'obtenir une homogénéisation
sur la correspondance entre les points dans l'image des caméras et dans 
l'espace. C'est pour cette raison qu'il est important de balayer l'espace de manière homogène (ne pas se concentrer
sur une zone ou en négliger).

\subsection{Placement des marqueurs}
Idéalement pour capturer des mouvements humains,
les marqueurs doivent être placés au centre des articulations, ou au plus proche possible
pour éviter que lors d'une rotation d'une articulation, le marqueur associé
à celle-ci translate ou bouge par rapport à ce centre. 
La configuration des marqueurs ne doit présenter aucune 
symétrie afin que le suivi temporel des marqueurs ne soit pas perturbé. En cas de symétrie,
des permutations d'identifiant de marqueurs peuvent apparaitre.

Nous donnons un exemple de placement de marqueurs qui peut être modifié selon
les besoins dans la figure~\ref{fig:markerSet}. Dans cet exemple,
des petites planches sont placées sur les poignets
pour les alonger. De cette façon, les trois marqueurs placés sur les mains ne forment pas
un triangle trop proche d'un triangle équilatéral, et sont plus facile à suivre par le logiciel
de capture de mouvements. Le nombre de marqueurs peut être plus grand pour introduire des données
redondantes pour compenser les données des marqueurs dont on sait à l'avance
qu'ils vont être être occlus.
\begin{figure}[p]
  \begin{center}
    \includegraphics[width=0.9\linewidth]{figures/chapitre4/markerSet.ps}
  \end{center}
  \caption[Placements des marqueurs.]{Exemple de configuration de marqueurs pour capturer des mouvements humains. Des
  petites planches sont utilisées pour allonger les poignets.}
  \label{fig:markerSet}
\end{figure}

\subsection{Squelette virtuel}
\subsubsection{Définition du squelette}
Le format de données calculées par défaut par le système est la liste des 
coordonnées 3D dans le repère de la zone de capture
de tous les marqueurs. Pour des mouvements humains,
il est plus intéressant de travailler en utilisant des corps et leurs transformations
dans l'espace. Pour cela, un squelette est construit en hiérarchisant un ensemble
de corps sous la forme d'un arbre. Chaque repère associé à un corps est défini par un ensemble de trois marqueurs,
définissant l'origine, la direction de l'axe de la longueur et la direction de l'axe 
plan (voir figure~\ref{fig:bone}).
\begin{figure}[p]
  \begin{center}
    \subfigure[]{
    \includegraphics[width=0.5\linewidth]{figures/chapitre4/skelcrop.ps}
    }
    \hspace{50px}
    \subfigure[]{
    \includegraphics[width=0.2\linewidth]{figures/chapitre4/bone.ps}
    }
  \end{center}
  \caption[Squelette virtuel.]{(a)~Un squelette virtuel pour un humain. (b)~Définition du repère associé à un corps à partir de trois marqueurs.}
  \label{fig:bone}
\end{figure}
La figure~\ref{fig:bone} illustre un exemple de squelette associé à la configuration
des marqueurs présentée plus tôt.
%\begin{figure}[p]
%  \begin{center}
%  \end{center}
%  \label{fig:skelMocap}
%\end{figure}

\subsubsection{Interprétation des données}
Les données relatives au squelette sont exportées au format développé par Motion Analysis appelé
HTR (hierarchical translation-rotation). Le fichier de sortie se décompose en quatre parties:
\begin{itemize}
  \item \emph{Header}: Informations générales.
  \item \emph{Segment names \& Hierarchy}: Graphe de hiérarchie des corps du squelette, avec pour n\oe ud racine \emph{GLOBAL}.
  \item \emph{Base position}: Composantes de transformation initiale des corps relativement à son parent, 
    qu'on note dans la suite $\mbf{T}_{j,0}$ et $\mbf{R}_{j,0}$.
  \item \emph{Frame data}: Composantes de transformation à l'instant $Fr=i$ qu'on note $\mbf{T}_j(i)$, $\mbf{R}_j(i)$ et
    $\mbf{S}_j(i)$.
\end{itemize}
La figure~\ref{fig:htr1} illustre un extrait d'un fichier \emph{htr}.
\begin{figure}[p]
  \begin{center}
    \includegraphics[width=0.8\linewidth]{figures/chapitre4/htrAll.ps}
  \end{center}
  \caption{Extrait d'un fichier HTR.}
  \label{fig:htr1}
\end{figure}
%\begin{figure}[t]
%  \begin{center}
%    \includegraphics[width=0.8\linewidth]{figures/chapitre4/htr2.ps}
%  \end{center}
%  \caption{Extrait de la quatrième partie d'un fichier HTR.}
%  \label{fig:htr2}
%\end{figure}
La transformation $\mbf{M}_j$ appliquée à un corps $j$ par rapport
à son parent à l'instant $Fr=i$ se calcule gr\^ace à l'équation:
\begin{equation}
  \mbf{M}_j(i) = \mbf{T}_{j,0} \mbf{T}_j(i) \mbf{R}_{j,0} \mbf{R}_j(i) \mbf{S}_j(i)
  \label{eq:localMocapTransf}
\end{equation}
\noindent où $\mbf{T}_j(i)$ représente la composition des transformations de translation 
$\left[tx_{j,0}~ty_{j,0}~tz_{j,0}\right]^T$
issues de la partie \emph{Base~position} et $\left[tx_j(i)~ty_j(i)~tz_j(i)\right]^T$ issues de la partie \emph{Frame~data}.
\begin{align}
  \mbf{T}_{j,0} & = \left( \begin{array}{cccc}
    1 & 0 & 0 & tx_{j,0}\\
    0 & 1 & 0 & ty_{j,0}\\
    0 & 0 & 1 & tz_{j,0}\\
    0 & 0 & 0 & 1
		    \end{array}
	      \right)
  \label{eq:mocapTj0}\\
  \mbf{T}_j(i) & = \left( \begin{array}{cccc}
    1 & 0 & 0 & tx_j(i)\\
    0 & 1 & 0 & ty_j(i)\\
    0 & 0 & 1 & tz_j(i)\\
    0 & 0 & 0 & 1
		    \end{array}
	      \right)
  \label{eq:mocapTji}
\end{align}
\noindent $\mbf{R}_{j,0}$ est la composition de rotation calculée à partir des angles de la partie
\emph{Base~position} selon l'ordre des angles d'Euler défini dans \emph{Header}. Dans
l'exemple choisi, l'ordre est $ZYX$
donc $\mbf{R}_{j,0} = \mbf{R}_z \mbf{R}_y \mbf{R}_x$. De même $\mbf{R}_j(i)$ est la composition de rotation
à partir des données de la partie \emph{Frame~data}. Enfin $\mbf{S}_j(i)$ est la matrice de mise à l'échelle:
\begin{equation}
  \mbf{S}_j(i) =  \left( \begin{array}{cccc}
		      SF_j(i) & 0 & 0 & 0\\
		      0 & SF_j(i) & 0 & 0\\
		      0 & 0 & SF_j(i) & 0\\
		      0 & 0 & 0 & 1
		    \end{array}
	      \right)  
  \label{eq:mocapSi}
\end{equation}
\noindent où $SF_j(i)$ est le facteur d'échelle à l'instant $Fr=i$.

La transformation $\tensor[^{W}]{\mathbf{M}}{_{n}}$ dans le repère de la zone de 
capture (repère défini par l'équerre lors de la calibration) d'un corps $n$ ayant comme parent
un corps $n-1$ est obtenu gr\^ace à l'équation:
\begin{equation}
  \tensor[^{W}]{\mathbf{M}}{_{n}} = \prod_{j=0}^{n} \mbf{M}_j
  \label{eq:globalMocap}
\end{equation}

\FloatBarrier
\subsection{Post-traitement des données}
Bien que les caméras soient capables de fournir des données propres,
les contraintes liées à l'environnement perturbent l'interprétation des données.
Ces contraintes peuvent par exemple provenir des occlusions de marqueurs, d'un marqueur
endommagé, des erreurs logiciels (mauvais étiquetage de marqueurs), 
du bruit numérique ou d'une mauvaise calibration. 
Les données collectées peuvent présenter des problèmes de discontinuités ou du bruit parasite.
Cortex dispose d'outils qui permettent de corriger les intervalles de temps où un marqueur 
n'est plus étiqueté, en se basant
sur la configuration géométrique des marqueurs (fonction \emph{rectify}) et des outils d'interpolations.
Les parties manquantes dans les trajectoires peuvent être interpolées linéairement (fonction \emph{linear join}), 
à l'aide de splines cubiques (fonction \emph{cubic join})
ou encore en générant des données en fonction de la configuration géométrique de trois autres marqueurs 
dont les trajectoires sont complètes (fonction \emph{virtual join}).
La figure~\ref{fig:interpolMocap} illustre les méthodes
d'interpolation linéaire et par splines cubiques, et la figure~\ref{fig:virtualJoin}
illustre la méthode \emph{virtual join}. 
\begin{figure}[p]
  \begin{center}
    \subfigure[]{
    \includegraphics[width=0.10\linewidth]{figures/chapitre4/lineJoin.ps}
    \label{fig:interpolMocap}
    }
    \hspace{50px}
    \subfigure[]{
    \includegraphics[width=0.55\linewidth]{figures/chapitre4/joinvirtual.ps}
    \label{fig:virtualJoin}
    }
  \end{center}
  \caption[Interpolation de la position de marqueurs.]{(a)~La trajectoire incomplète d'un marqueur est interpolé linéairement ou à l'aide d'une spline cubique.
  (b)~La trajectoire incomplète d'un marqueur est reconstruite à partir
  de sa configuration géométrique par rapport à trois autres marqueurs.}
\end{figure}
%\begin{figure}[t]
%  \begin{center}
%  \end{center}
%  \label{fig:virtualJoin}
%\end{figure}
Cette dernière technique 
est intéressante: si on sait que le mouvement à capturer va  entrainer des occlusions sur un ou plusieurs
marqueurs particuliers, alors on peut ajouter des marqueurs redondants sur le corps en question
pour pouvoir lors de la phase de post-traitement reconstruire les trajectoires des 
marqueurs masqués.
La figure~\ref{fig:BadData} illustre des données incomplètes d'un marqueur
situé en bas du poignet droit. On remarque également que le signal est légèrement bruité.
\begin{figure}[p]
  \begin{center}
  \resizebox{.75\textwidth}{!} {
    \input{figures/chapitre4/RwristBGap.tex}
    }
  \end{center}
  \caption[Discontinuités de la trajectoire d'un marqueur.]{La composante en $x$ de la trajectoire du marqueur placé en bas du poignet droit présente
  des discontinuités et du bruit à hautes fréquences.}
  \label{fig:BadData}
\end{figure}
La figure~\ref{fig:RwristVjoin} illustre l'interpolation de type \emph{virtual join} de la trajectoire de ce marqueur
en utilisant les positions des marqueurs situés au coude et en haut du poignet droit.
\begin{figure}[p]
  \begin{center}
  \resizebox{.75\textwidth}{!} {
    \input{figures/chapitre4/RwristBGapVjoin.tex}
    }
  \end{center}
  \caption{Interpolation de la trajectoire du marqueur en bas du poignet droit.}
  \label{fig:RwristVjoin}
\end{figure}
L'étiquetage de marqueurs peut aussi être permuté manuellement si les marqueurs ont été mal identifiés.
Il est possible de lisser les trajectoires des marqueurs selon deux méthodes: un filtre de moyenne glissante ou
un filtre de Butterworth.
Le filtre de Butterworth est un filtre passe bande paramétrable possédant
de bonnes propriétés pour l'étude des mouvements biomécaniques. Ce filtre permet 
de retirer les composantes du mouvement qui ont des fréquences trop élevées pour avoir été réalisés
par un humain.

\section{Imitation: Reproduction de mouvements capturés}
La pile de t\^aches permet de générer un mouvement à partir d'une référence générique.
Les outils de capture de mouvements peuvent donc être utilisés pour fournir 
une référence n'étant pas issue d'un modèle mathématique. Il est ainsi possible de générer
des mouvements qui ne représentent pas forcement l'exécution d'une t\^ache robotique et son donc plus expressif.
Ce paragraphe décrit comment des données issues de la capture de mouvement d'un humain
peuvent être utilisées comme trajectoire de référence pour une t\^ache qui
doit être exécuté par un robot.
Dans tout les cas, un
changement de repère est nécessaire pour pouvoir calculer l'erreur du suivi. On pourra utiliser par
exemple comme référentiel commun un repère dont l'axe 
$\boldsymbol{z}$ est normal au sol, la direction de l'axe $\boldsymbol{y}$
est définie par la direction du vecteur $\left( \mbf{p}_{\textsf{Rwaist}} - \mbf{p}_{\textsf{Lwaist}} \right)$
où $\mbf{p}_{\textsf{Rwaist}}$ (respectivement $\mbf{p}_{\textsf{Lwaist}}$) est 
la position d'un point situé sur le côté droit (respectivement gauche) du bassin.
Un deuxième changement de repère sera nécessaire si pour une posture \emph{identique},
les repères attachés aux corps de l'humain et du robot n'ont pas des transformations identiques par
rapport au référentiel d'origine (voir figure~\ref{fig:changeRepere}).
\begin{figure}[t]
  \begin{center}
    \resizebox{0.95\textwidth}{!}{
    \input{figures/chapitre4/refChange.tex}
    }
  \end{center}
  \caption[Changement de repères du squelette de la capture de mouvement vers le robot.]{Un changement de repère est nécessaire: les repères associés aux bras ne sont pas concordant.
  La transformation homogène $\tensor[^{\textsf{m}_j}]{M}{_{\textsf{r}_j}}$ est calculée en utilisant 
  des postures concordantes sur le squelette de la capture de mouvements et le modèle du robot.}
  \label{fig:changeRepere}
\end{figure}

\subsection{T\^aches et suivi de trajectoires}
D'une manière générale, les trajectoires des t\^aches sont directement
reliées à leurs formulations mathématiques $\mbf{e} = \mbf{s}(t) - \mbf{s}^*(t)$.
Par exemple, pour attraper
une balle, la main doit se déplacer en direction de la balle: $\mbf{e}(t) = \mbf{p}_{\textsf{main}}(t) - \mbf{p}_{\textsf{balle}}^*$. 
En revanche, certaines t\^aches ont des trajectoires qui ne suivent pas un motif
particulier.
Si $\mbf{s}^*$ est observable, alors des t\^aches de suivi de trajectoires peuvent être
utilisées dans une pile de t\^aches pour reproduire les parties choisies d'un mouvement.
Nous considérons qu'il y a une imitation lorsque le mouvement doit être adapté 
à une structure cinématique différente de la structure sur laquelle le mouvement a été observé.

Si nous possédons des connaissances sur l'action observée, nous pouvons choisir
explicitement des t\^aches pertinentes à reproduire. Par opposition, des t\^aches apparaissent
implicitement si par exemple les trajectoires articulaires sont strictement recopiées.
Par exemple considérons le cas du mouvement
d'attente. Ce mouvement est défini comme étant 
le mouvement inconscient qu'un humain produit
lorsqu'il \emph{ne fait rien}.
Il semble très difficile
de définir simplement ce mouvement sous forme d'une combinaison de t\^ache,
c'est-à-dire d'un modèle causal,
sans donner une interprétation de ses caractéristiques, par exemple en remontant
à la structure élastique de l'actionnement humain.
Nous considérons que le mouvement d'attente d'un humain se traduit
par les mouvements de ses bras, du torse et de la tête.
Par conséquent le mouvement d'attente peut être représenté par une pile de
t\^aches constituée de t\^aches de suivi de trajectoire des mains, et de la t\^ete.
Le torse n'est pas choisi dans la pile car les chaînes articulaires
contrôlant les mains et la t\^ete contiennent le torse.
Le couplage des t\^aches entrainera le torse d'une manière cohérente dans le mouvement 
des autres membres.
On note que l'utilisation de la capture de mouvements nous permet de générer 
un mouvement expressif où le robot \emph{ne fait rien}. Alors
qu'en ne considérant l'action \emph{ne rien faire} d'un point de vue 
strictement robotique, la commande générée correspondrait à une vitesse
nulle sur chaque articulation.

\subsection{T\^aches possibles et mesures des trajectoires de références}
%\subsubsection{Position des mains}
Pour une t\^ache d'atteinte d'une position 3D dans l'espace cartésien,
la trajectoire de la position mesurée $\hat{\mbf{p}}$ d'un marqueur peut être utilisée comme le
signal de référence pour une t\^ache de suivi en position 3D (voir figure~\ref{fig:handDemo}).
\begin{figure}[t]
  \begin{center}
    \resizebox{0.7\textwidth}{!}{
    \input{figures/chapitre4/handCrop.tex}
    }
  \end{center}
  \caption[T\^ache de suivi de trajectoire.]{La position mesurée $\hat{\mbf{p}}$ d'un marqueur est utilisé comme
  référence pour une t\^ache de suivi de position 3D.}
  \label{fig:handDemo}
\end{figure}
%\subsubsection{Orientation du regard}

L'orientation du regard peut être approchée par l'orientation de la tête.
En utilisant les positions de trois marqueurs placés sur la tête, on définit un repère et la matrice
de transformation de rotation par rapport au repère du monde représentera l'orientation 
de la tête dans ce repère (voir figure~\ref{fig:gazeDemo}). 
\begin{figure}[t]
  \begin{center}
    \resizebox{0.7\textwidth}{!}{
    \input{figures/chapitre4/gazeDemo.tex}
    }
  \end{center}
  \caption[Mesure de l'orientation de la t\^ete.]{Un repère est associé à la tête à partir de trois marqueurs. La matrice de transformation du repère
  d'origine $\mathcal{R}_{\textsf{w}}$ vers le repère associé à la tête $\mathcal{R}_{\textsf{head}}$ 
  $\tensor[^{\textsf{w}}]{\mathbf{M}}{_{\textsf{head}}}$ donne l'orientation de la tête.}
  \label{fig:gazeDemo}
\end{figure}
%\subsubsection{Approximation de la trajectoire du centre de masse}
La position du centre de masse est utilisé comme un critère de stabilité statique:
sa projection sur le sol doit rester à l'intérieur du polygone de support défini par
les deux pieds d'un humanoïde. Une t\^ache de suivi de position du centre
de masse nécessite la connaissance d'une trajectoire de référence.
Le problème est que la position du centre de masse n'est pas directement observable
avec les outils de capture de mouvements. Il est cependant possible de
faire une approximation de sa trajectoire projetée au sol pour obtenir
une trajectoire de référence satisfaisant la contrainte d'équilibre statique. Dans~\cite{montecillo10},
une approximation de la trajectoire du centre de masse projetée au sol se basant 
sur les mouvements de la tête est proposée afin d'imiter des mouvements de transfert de point
d'appui:
\begin{itemize}
  \item lors d'une phase de double support, la projection au sol du centre de masse évolue suivant 
    un axe défini par la position des deux pieds, et son déplacement est proportionnel 
    à la projection orthogonale du vecteur vitesse de la tête projeté au sol sur l'axe défini par les pieds,
  \item lors d'une phase de simple support, la projection du centre de masse se déplace
    suivant le vecteur vitesse de la tête projeté au sol.
\end{itemize}
Cette approximation est justifiée par l'importance du mouvement de la tête dans des mouvements humains
qui est souligné dans les études en neurosciences~\cite{sreenivasa09}.
La figure~\ref{fig:comApprox} illustre cette approximation.
\begin{figure}[t]
  \begin{center}
    \subfigure{
    \resizebox{.47\textwidth}{!} {
    \input{figures/chapitre4/doubleSupportCom.tex}
    }
    }
    \subfigure{
    \resizebox{.47\textwidth}{!} {
    \input{figures/chapitre4/simpleSupportCom.tex}
    }
    }
  \end{center}
  \caption[Approximation de la trajectoire du centre de masse.]{Une approximation du déplacement du centre de masse lors d'un mouvement est effectuée
  en fonction de la vitesse de déplacement de la tête pour le cas d'un mouvement en
  double support et en simple support.}
  \label{fig:comApprox}
\end{figure}
\FloatBarrier

 
%-------------------------------------------------------------------%
%-------------------------------------------------------------------%
\subsection{Application au mouvement d'attente}
\label{chap:activeWait}
%-------------------------------------------------------------------%
%-------------------------------------------------------------------%
Lorsqu'un humain attend, l'intégralité de son corps est en mouvement.
Cependant, il n'est pas possible d'exprimer facilement une t\^ache d'attente
(au sens fonction de t\^ache).
C'est pour cette raison que pour reproduire ce mouvement, nous séléctionnons
un ensemble de t\^aches qui vont, par combinaison, reproduire
les caractéristiques du mouvement.
\subsubsection{Réplication du mouvement}
Les t\^aches de suivi de trajectoire des mains
et de l'orientation de la tête, présentées dans le paragraphe précédent,
sont choisies comme étant les t\^aches caractérisant un mouvement 
d'attente. Les trajectoires correspondantes sont donc mesurées sur un humain
à l'aide du système de capture de mouvement, et serviront de trajectoires
de références.
Pour garantir la stabilité du robot, deux t\^aches
sont ajoutées en haute priorité (\emph{double support} et \emph{centre de masse}): 
les deux pieds sont contraints à rester au sol, et le centre de masse
est fixe en $x$ et en $y$ (la hauteur n'est pas contrainte).
Pour éviter les postures trop proches des singularités,
une t\^ache de posture est utilisée en faible priorité: lors du mouvement, le robot
gardera une posture proche de la configuration de repos (configuration \emph{half-sitting}).
La figure~\ref{fig:sotActiveWait} illustre la pile de t\^aches utilisée pour réaliser
un mouvement d'attente.
\begin{figure}[t]
  \begin{center}
    \begin{tabular}{|c|}
      \hline
      Posture \emph{Half-sitting} (30D)\\
      \hline
      Main droite (3D)\\
      \hline
      Main gauche (3D)\\
      \hline
      Tête (6D)\\
      \hline
      Centre de masse (2D)\\
      \hline
      Double support (6D)\\
      \hline
    \end{tabular}
  \end{center}
  \caption{La pile de t\^aches qui va permettre de reproduire le mouvement d'attente.}
  \label{fig:sotActiveWait}
\end{figure}
Cette pile de t\^aches sera exécutée par un robot HRP-2.

%L'outil de capture de mouvement est donc utilisé pour obtenir
%des trajectoires de références correspondant
%aux trajectoires 3D des mains, et de la trajectoire 6D (position
%et orientation) de la tête d'un humain.

\subsubsection{Edition du mouvement}
Pour donner un aspect aléatoire au mouvement exécuté par le robot,
plusieurs démonstrations de mouvements d'attente sont enregistrées.
En ce qui concerne le mouvement d'attente, les positions atteintes
par les différents membres ne sont pas importantes. Ce sont les
déplacements de ces membres qui caractérisent ce mouvement.
Il n'est donc pas capital que le suivi de trajectoire soit parfaitement accompli.
Toutes les trajectoires sont éditées afin que les positions de début et de
fin soient identiques pour chaque ensemble de trajectoires. Ainsi n'importe quel séquencement
des trajectoires reste continu.
Pour chaque t\^ache, une des trajectoires enregistrées 
correspondant à cette t\^ache est choisie aléatoirement
puis est définie comme trajectoire de référence. Lorsque la trajectoire 
a été exécutée en totalité, une nouvelle trajectoire est choisie comme référence et ainsi de suite (voir
figure~\ref{fig:activeLoop}).
\begin{figure}[t]
  \begin{center}
    \resizebox{0.6\textwidth}{!}{
    \input{figures/chapitre4/activeWaitOpPoint.tex}
    }
  \end{center}
  \caption[Reproduction du mouvement d'attente.]{Pour chaque t\^ache de suivi, des trajectoires sont choisies comme
  référence parmi un ensemble de trajectoires enregistrées.}
  \label{fig:activeLoop}
\end{figure}
La figure~\ref{fig:activeWaitStrips} illustre le mouvement d'attente réalisé par le robot
HRP-2 en utilisant la pile de t\^aches, et les vidéos associées sont 
disponibles\footnote{\url{http://homepages.laas.fr/shak/videos/}}.
\begin{figure}[p]
  \begin{center}
    \subfigure{
    \includegraphics[width=0.3\linewidth]{figures/chapitre4/ActiveWait/aw1.ps}
    }
    \subfigure{
    \includegraphics[width=0.3\linewidth]{figures/chapitre4/ActiveWait/aw2.ps}
    }
    \subfigure{
    \includegraphics[width=0.3\linewidth]{figures/chapitre4/ActiveWait/aw3.ps}
    }
    \subfigure{
    \includegraphics[width=0.3\linewidth]{figures/chapitre4/ActiveWait/aw4.ps}
    }
    \subfigure{
    \includegraphics[width=0.3\linewidth]{figures/chapitre4/ActiveWait/aw5.ps}
    }\\

    \subfigure{
    \includegraphics[width=0.3\linewidth]{figures/chapitre4/ActiveWait/awz2.ps}
    }
    \subfigure{
    \includegraphics[width=0.3\linewidth]{figures/chapitre4/ActiveWait/awz3.ps}
    }
    \subfigure{
    \includegraphics[width=0.3\linewidth]{figures/chapitre4/ActiveWait/awz4.ps}
    }
  \end{center}
  \caption{Le mouvement d'attente réalisé par le robot HRP-2.}
  \label{fig:activeWaitStrips}
\end{figure}

\subsection{Adaptation du mouvement au modèle}
Les travaux présentés dans~\cite{shin01} introduisent la notion
d'\emph{importance} d'un effecteur. Il s'agit d'un critère qui permet de
déterminer si la position d'un effecteur est plus importante que la configuration articulaire associée
à sa chaîne cinématique.
L'importance des effecteurs est déterminée en analysant la posture d'un personnage
en relation avec son environnement: lorsqu'un effecteur touche
un objet (distance nulle), la position de l'effecteur est considérée comme étant plus importante
que la configuration articulaire. Par exemple, lorsqu'une main touche un objet,
il est important de conserver ce contact et donc la position de la main doit
être recopiée en priorité. Par contre si la main est en l'air alors,
la configuration articulaire est plus importante car il est possible que la position
de la main ne soit pas à portée du personnage, à cause de la structure différente,
qui reproduit le mouvement. La figure~\ref{fig:ballet} illustre un exemple
de mouvement où en fonction de l'effecteur, la configuration articulaire
ou la position doit être respectée.
\begin{figure}[p]
  \begin{center}
    \resizebox{0.4\textwidth}{!}{
    \input{figures/chapitre4/ballet.tex}
    }
  \end{center}
  \caption[Que faut-il reproduire?]{Pour reproduire le mouvement de la danseuse, il est plus important
  de respecter la position de la main droite touchant la barre que la configuration articulaire associée.
  En ce qui concerne le pied gauche, il est plus important de respecter la configuration articulaire pour
  conserver l'allure du mouvement.}
  \label{fig:ballet}
\end{figure}

Ce paragraphe s'appuie sur le travail réalisé par~\cite{ramos11} dans le 
cadre de l'édition de mouvement dynamique issu d'un mouvement humain pour l'adapter
à la dynamique d'un robot. L'exemple présenté illustre également l'importance
du choix des t\^aches à reproduire. L'utilisation de la dynamique dans la génération de 
mouvements a pour but de rendre le mouvement exécuté le plus réaliste possible. 

Considérons un mouvement de yoga réalisé par
un humain est enregistré par capture de mouvements (voir figure~\ref{fig:yogaL}).
\begin{figure}[t]
  \begin{center}
    \subfigure{
    \includegraphics[width=0.25\linewidth]{figures/chapitre4/yogaL1.ps}
    }
    \subfigure{
    \includegraphics[width=0.25\linewidth]{figures/chapitre4/yogaL2.ps}
    }
    \subfigure{
    \includegraphics[width=0.25\linewidth]{figures/chapitre4/yogaL3.ps}
    }
    \subfigure{
    \includegraphics[width=0.25\linewidth]{figures/chapitre4/yogaL4.ps}
    }
    \subfigure{
    \includegraphics[width=0.25\linewidth]{figures/chapitre4/yogaL.ps}
    }
  \end{center}
  \caption{Un mouvement de yoga enregistré gr\^ace au système de capture de mouvements.}
  \label{fig:yogaL}
\end{figure}
Si le mouvement
est converti en trajectoire articulaire et est appliqué directement 
sur le robot en cinématique, alors plusieurs problème liés 
à la différence de structures apparaissent. Le pied d'appui décolle du sol
(problème connu en animation sous le nom de \emph{foot skating} ou \emph{foot sliding}), 
les mains rentrent en collision puis transpercent le torse (voir figure~\ref{fig:yogaKine}). 
\begin{figure}[t]
  \begin{center}
    \subfigure{
    \includegraphics[width=0.3\linewidth]{figures/chapitre4/yogaHRP2Kine1.ps}
    }
    \subfigure{
    \includegraphics[width=0.3\linewidth]{figures/chapitre4/yogaHRP2Kine2.ps}
    }
    \subfigure{
    \includegraphics[width=0.3\linewidth]{figures/chapitre4/yogaHRP2Kine3.ps}
    }
    \subfigure{
    \includegraphics[width=0.3\linewidth]{figures/chapitre4/yogaHRP2Kine4.ps}
    }
    \subfigure{
    \includegraphics[width=0.3\linewidth]{figures/chapitre4/yogaHRP2Kine5.ps}
    }
  \end{center}
  \caption[Mouvement de yoga en cinématique.]{Le mouvement de yoga reproduit en cinématique en suivant les trajectoires articulaires
  calculées à partir des données issues de la capture de mouvements. Plusieurs problèmes apparaissent:
  le pied d'appui perd son contact avec le sol, les mains rentrent en collision et transpercent le torse.}
  \label{fig:yogaKine}
\end{figure}

En rejouant la trajectoire mais cette fois ci en respectant la dynamique du robot 
(en utilisant une extension de la pile de t\^aches en dynamique~\cite{saab11}), le 
pied d'appui est respecté mais le résultat n'est plus le mouvement de yoga puisque le robot perd
l'équilibre à cause de ses propres contraintes dynamiques différente de l'humain (voir figure~\ref{fig:yogaDynFall}).

\begin{figure}[t]
  \begin{center}
    \subfigure{
    \includegraphics[width=0.3\linewidth]{figures/chapitre4/yogaDynFall1.ps}
    }
    \subfigure{
    \includegraphics[width=0.3\linewidth]{figures/chapitre4/yogaDynFall2.ps}
    }
    \subfigure{
    \includegraphics[width=0.3\linewidth]{figures/chapitre4/yogaDynFall3.ps}
    }
    \subfigure{
    \includegraphics[width=0.3\linewidth]{figures/chapitre4/yogaDynFall4.ps}
    }
    \subfigure{
    \includegraphics[width=0.3\linewidth]{figures/chapitre4/yogaDynFall5.ps}
    }
  \end{center}
  \caption[Mouvement de yoga en dynamique.]{Le mouvement de yoga reproduit en dynamique. Le robot perd l'équilibre à cause de ses contraintes
  dynamiques.}
  \label{fig:yogaDynFall}
\end{figure}

En analysant le mouvement initial, on peut déterminer les caractéristiques importantes 
du mouvement. Dans l'exemple du mouvement de yoga, la position relative d'une main par rapport à l'autre
est importante, de même la position du centre de masse (et donc l'équilibre) doivent être contrôlées correctement.
En choisissant ces caractéristiques comme étant les t\^aches à exécuter pour reproduire le mouvement,
le mouvement généré devient correct (voir figure~\ref{fig:yogaDyn}).
\begin{figure}[t]
  \begin{center}
    \subfigure{
    \includegraphics[width=0.3\linewidth]{figures/chapitre4/yogaDyn1.ps}
    }
    \subfigure{
    \includegraphics[width=0.3\linewidth]{figures/chapitre4/yogaDyn2.ps}
    }
    \subfigure{
    \includegraphics[width=0.3\linewidth]{figures/chapitre4/yogaDyn3.ps}
    }
    \subfigure{
    \includegraphics[width=0.3\linewidth]{figures/chapitre4/yogaDyn4.ps}
    }
    \subfigure{
    \includegraphics[width=0.3\linewidth]{figures/chapitre4/yogaDyn5.ps}
    }
  \end{center}
  \caption[Mouvement de yoga valide en dynamique.]{Le mouvement de yoga reproduit en dynamique après édition du mouvement
  original pour définir correctement les t\^aches caractéristiques: la posture, la distance relative entre les
  deux mains, et le contrôle du centre de masse. Le mouvement généré est fidèle à la démonstration (sans collision ni
  perte d'équilibre).}
  \label{fig:yogaDyn}
\end{figure}
Enfin la figure~\ref{fig:yogaR} illustre le mouvement appliqué sur le vrai robot.
\begin{figure}[t]
  \begin{center}
    \subfigure{
    \includegraphics[trim=0px 300px 0px 300px, clip=true, width=0.3\linewidth]{figures/chapitre4/treepose1.ps}
    }
    \subfigure{
    \includegraphics[trim=0px 300px 0px 300px, clip=true, width=0.3\linewidth]{figures/chapitre4/treepose2.ps}
    }
    \subfigure{
    \includegraphics[trim=0px 300px 0px 300px, clip=true, width=0.3\linewidth]{figures/chapitre4/treepose3.ps}
    }
    \subfigure{
    \includegraphics[trim=0px 300px 0px 300px, clip=true, width=0.3\linewidth]{figures/chapitre4/treepose4.ps}
    }                                                     
    \subfigure{
    \includegraphics[trim=0px 300px 0px 300px, clip=true, width=0.3\linewidth]{figures/chapitre4/treepose5.ps}
    }
  \end{center}
  \caption[Mouvement de yoga sur le robot.]{Le mouvement de yoga reproduit sur le robot HRP-2.}
  \label{fig:yogaR}
\end{figure}

\FloatBarrier

 \section*{Conclusion}

Dans ce chapitre, nous avons présenté différents outils existant pour
acquérir des mouvements réels en détaillant les avantages et inconvénients des différentes 
techniques. Les techniques de capture de mouvements permettent
d'analyser les données récoltées afin d'étudier un mouvement, mais aussi
d'animer des objets ou des personnages virtuels en s'inspirant 
de mouvements réels. La possibilité d'éditer 
les données représentant les mouvements permet d'adapter un mouvement
à différente morphologie ou de créer des variations de mouvements.
Nous avons aussi souligné l'importance
du choix des t\^aches qui une fois exécutées, vont permettre de conserver les
caractéristiques du mouvement.
Nous avons présenté comment un système de capture de mouvements a permit
de fournir des trajectoires de références pour des t\^aches de suivi de trajectoires,
fournissant ainsi un moyen d'élargir l'expressivité des mouvements générés pour le robot.
Ces t\^aches peuvent ensuite être utilisées, via une pile de t\^aches, pour reproduire sur un robot HRP-2
un mouvement difficilement définissable par un modèle de génération: le mouvement
d'attente. Dans le chapitre suivant le problème de reconnaissance de t\^aches
est introduit, et les données issues de la capture de mouvements seront utilisées 
pour analyser des trajectoires associées à un mouvement.


 %\section*{Introduction}
%\label{sec:intro}
%-------------------------------------------------------------------%
%-------------------------------------------------------------------%
\chapter{Reconnaissance de t\^ache en robotique par commande inverse}
\label{chap:reco}
%-------------------------------------------------------------------%
%-------------------------------------------------------------------%
Nous présentons dans ce chapitre une méthode de reconnaissance de t\^aches
particulièrement adaptée à la reconnaissance de t\^aches exécutées en parallèle.
%Ce chapitre pr\'esente une m\'ethode exploitant les 
%outils du formalisme de la fonction de t\^ache pour
%effectuer de la reconnaissance de t\^ache. La m\'ethode
%repose sur une analyse du mouvement semblable \`a de
%la r\'etro-ing\'enierie.
%
L'originalité de la méthode présentée ici est d'utiliser les propriétés de la fonction de t\^aches,
classiquement utilisée pour la génération de mouvements,
pour effectuer la reconnaissance de t\^aches. L'idée principale est d'effectuer
une rétro-ingénierie sur un mouvement observé, connaissant l'ensemble 
des t\^aches qui peuvent appara\^itre dans ce mouvement et en utilisant les 
trajectoires engendrées par les lois de commandes dans 
les espaces de t\^aches comme étant des trajectoires caractéristiques.
Sous l'hypothèse que le mouvement a été généré en empilant un ensemble
de contrôleurs, le mouvement est traité afin de rechercher les comportements
connus dans chaque espace des t\^aches.
Nous appelons cette approche de rétro-ingénierie la \emph{commande inverse}.
Alors que toutes les approches présentées dans le paragraphe suivant ne peuvent reconnaitre que
des t\^aches non parallèles, nous nous appuyons sur les principes de la redondance
en contrôle pour reconnaitre des ensembles de t\^aches parallèles.
Lors de la phase de reconnaissance, le mouvement est projeté dans 
des espaces orthogonaux aux espaces des t\^aches déjà détectées.
Cela assure un découplage efficace des t\^aches effectuées par le robot.

Nous avons introduit dans le chapitre~\ref{chap:sot} les fondements sur lesquels 
les travaux présentés s'appuient : le formalisme de la fonction de t\^aches.
L'algorithme proposé pour effectuer l'analyse de mouvement est présenté dans les
paragraphes suivants.
Enfin, des expérimentations validant la méthode sont présentées en simulation
et sur le robot HRP-2.

\section{Etat de l'art}
La reconnaissance de t\^aches est un probl\`eme qui est apparu assez
t\^ot en robotique. Dans la communaut\'e de la vision, ce probl\`eme
est g\'en\'eralement consid\'er\'e pour les mouvements non-structur\'es:
aucune hypoth\`eses ne sont faites, que ce soit sur la forme ou la rigidit\'e
des corps qui bougent.

Cependant, les informations sont issues de l'environnement et la reconnaissance
est principalement faite \`a partir du contexte, par exemple, les points
saillants en temps et en position (le flux de vid\'eo 2D est considéré comme 
étant une fonction 3D)~\cite{laptev05}. Ces points sont appris d'une base de donn\'ees,
puis associ\'es pendant une d\'emonstration. L'environnement est aussi
utilis\'e pour effectuer une extraction d'arri\`ere plan pour 
d\'egager une silhouette et effectuer, par exemple, une reconnaissance 
de d\'emarche~\cite{liu05}.

La structure du corps poly-articul\'e est souvent suppos\'ee connue et est
utilis\'ee pour estimer sa pose. Par exemple l'estimation d'une
trajectoire de posture d'un humano\"ide dans~\cite{zordan03} est ex\'ecut\'ee
en utilisant des donn\'ees issues d'un syst\`eme de capture de mouvements pour 
guider un mod\`ele physique connu.
Dans le reste de ce chapitre, nous consid\'erons que le modèle cinématique du corps poly-articul\'e
est connu.

%In the quest of robot autonomy, research and development in Robotics is
%dominated by the stimulating competition between abstract symbol manipulation and physical signal
%processing, between discrete data structures and continuous variables.
%Indeed finding a proper way to relate the discrete space of symbols and the continuous space
%of controllers is a challenge.\\
%%%
Les outils statistiques ont \'et\'e appliqu\'es avec succ\`es \`a la 
reconnaissance d'actions et \`a l'analyse de mouvements~\cite{schaal03}.
Ces outils sont utilis\'es pour cr\'eer des symboles, et par extension,
les d\'etecter dans un mouvement.
Par exemple, une m\'ethode pour le contr\^ole bas\'e comportement est 
pr\'esentée dans~\cite{drumwright03, drumwright04}. 
Les comportements sont d\'efinis comme \'etant des symboles de mouvements
(par exemple un direct, un crochet, un uppercut, une garde).
Les comportements sont mod\'elis\'es par apprentissage \`a partir d'une 
s\'erie d'exemples.
Une r\'eduction de dimension est alors appliqu\'ee pour avoir une classification
significative. La reconnaissance est g\'er\'ee par
un classifieur bayésien qui reconnait une trajectoire
dans l'espace articulaire ou l'espace cart\'esien. 
L'imitation est une extension de la reconnaissance.
Elle est effectuée en interpolant des exemples connus pour obtenir
des trajectoires réalisables.
L'introduction des processus de décision markoviens partiellement observables 
(partially observable Markov decision process, POMDP) ou des inférences bayésiennes~\cite{pearl88} 
ont relancé le domaine de la modelisation d'action~\cite{kaelbling98} ces dernières années. 
De telles techniques sont appliquées à l'apprentissage de primitives motrices~\cite{peters08} 
et à la segmentation de mouvements~\cite{calinon10, inamura04}. 
Les modèles de Markov cachés (HMMs) ont été largement utilisés dans des travaux antérieurs,
par exemple, pour exécuter une reconnaissance d'action ou de démarche~\cite{gu10}
ou pour générer des mouvements proches du mouvement humain~\cite{kwon08}. 
Dans ces travaux, des données issues de captures de mouvements humain sont utilisées
pour construire une base dans l'espace articulaire pour chaque classe de mouvement
en utilisant une technique de réduction de dimension.
En parallèle, des HMMs sont entrainés pour capturer les caractéristiques relatives
à la classe de mouvement dans l'espace des t\^aches.
La génération est obtenue en trouvant la combinaison linéaire optimale des éléments 
de la base qui va maximiser la probabilité du HMM entrainé.
Bien que l'espace des t\^aches soit considéré comme étant un espace où les caractéristiques
doivent être extraites, la méthode est limitée à un espace de t\^ache spécifique par mouvement
généré.
Dans~\cite{liang09} des modèles de Markov à tailles variable sont utilisés
pour apprendre des actions atomiques d'humain. Une séquence d'actions atomiques représente
un comportement complet.
D'une manière générale, l'efficacité des techniques de reconnaissance basé sur les outils 
statistiques est dépendante de la qualité de la base de données construite lors de la phase 
d'apprentissage.
Plusieurs démonstrations pour chaque cas particulier sont nécessaire afin
d'extraire les invariants qui vont discriminer les t\^aches.
La couverture des données d'apprentissage peut aussi limiter l'efficacité de la reconnaissance.
Enfin, l'ensemble des démonstrations doivent être associées
aux bons symboles, ce qui est généralement donné à l'algorithme
d'apprentissage.

D'une autre façon, la reconnaissance peut se baser sur des critères spécifiques qui
sont des a priori donnés au système.
Dans~\cite{nakaoka07}, seules les trajectoires du robot sont utilisées pour
distinguer les différentes phases de mouvement. Dans ces travaux là, une t\^ache
est un mouvement corps complet du robot durant un segment temporel.
Le mouvement global est une séquence de t\^aches. Chaque t\^ache possède 
ses propres paramètres appelés \emph{skill parameters}.
La méthode de reconnaissance de t\^ache est décomposée en deux étapes:
d'abord, étiqueter tous les segments temporels du mouvement 
observé avec les noms des t\^aches.
La seconde est l'estimation des \emph{skill parameters} pour chaque segment.
Chaque t\^ache est détectée par l'analyse de trajectoires projetées
dans un espace spécifique. Par exemple, la t\^ache \emph{faire un pas} est détectée
en analysant la trajectoire d'un pied;
une t\^ache de \emph{squat} est détectée en analysant la trajectoire verticale
du bassin. Le critère utilisé pour la détection et le choix des espaces de projections
sont ad-hoc, construits manuellement pour un mouvement particulier qui doit être imité par
le robot.
D'une manière similaire, ~\cite{muhlig09}
utilise un ensemble d'espaces spécifiques dans lequel le mouvement 
observé est projeté. Des critères ad-hoc
%Observations of a movement from sensors (camera, or motion capture system)
sont utilisés pour choisir automatiquement l'ensemble des espaces de t\^aches qui va 
décrire le mieux un mouvement donné, dans le but de focaliser la technique 
d'apprentissage dans ce nouvel espace. Le critère utilisé pour la
sélection de l'espace de t\^ache est exprimé par
des fonctions de coût inspirées de neurosciences:
la saillance de l'objet manipulé, la variance de la dimension d'un espace à travers
plusieurs démonstrations, et quelques heuristiques qui indiquent que les mouvements
fatiguants ou inconfortables sont provoqués par l'exécution de t\^aches. Ces heuristiques
règlent le problème de détection de t\^aches d'immobilité.
Cette méthode présentée ajoute des informations de plus haut niveau que les analyses
purement statistiques et s'appuie sur les espaces de t\^aches comme 
étant des espaces appropriés pour représenter le mouvement. Cependant, l'efficacité
de la sélection de t\^aches dépend de la puissance des heuristiques.

L'approche commune de ces travaux est de projeter le mouvement observé dans un espace réduit
particulier, dans lequel la reconnaissance est plus facile. 
Ces espaces sont choisis arbitrairement~\cite{nakaoka07}, 
séléctionnés automatiquement~\cite{muhlig09}, ou obtenus par apprentissage~\cite{peters08}. 
L'analyse en composantes principales (PCA) est une technique de réduction de dimension qui 
permet de trouver des espaces réduits. 
Malheureusement cette technique ne tient pas compte des corrélations non linéaires
entre les différents espaces réduits, et donc la technique est inadapté aux mouvements humanoïdes
qui comporte de telles corrélations non linéaires.
Il existe des techniques d'analyse en composantes principales non linéaires. Ces 
techniques sont plus adaptées aux mouvements humanoïdes~\cite{chalodhorn09}.
Cependant les espaces réduits sont dégagés automatiquement, or, en contrôle,
des espaces de tailles réduites, présentant des corrélations non linéaires, sont construit pour définir des objectifs et
moduler le comportement du robot.
Par exemple l'approche de la fonction de t\^aches~\cite{samson91} 
exprime des commandes génériques dans un espace de t\^ache donné de dimension $n$.
L'approche a été étendue pour gérer un ensemble hiérarchique de t\^aches~\cite{siciliano91, nakamura87}
en utilisant la redondance du système.
Ces espaces de t\^ache représentent ainsi des espaces parfaitement appropriés pour
effectuer des analyses et reconnaître des motifs particuliers.
Nous proposons ainsi d'utiliser directement ces contrôleurs connus pour la génération
de mouvement plutôt que d'utiliser des primitives déterminées automatiquement a posteriori.
\medskip
%In this work the control law is obtained by a \emph{stack of tasks}
%(SoT)~\cite{mansard07} ie hierarchical inverse kinematics
%\cite{siciliano91}.\\                                       

%In this paper another point of view is taken: the control theory based
%one. Control theory constitutes the second corpus that accompanies
%robotics development. Originating from mechanics and applied
%mathematics, it focuses on robot motion control~\cite{murray94,
%siciliano10}. Among all the
%contributions on linear and non-linear systems, robot control theory has
%provided efficient concepts for motion generation. The research
%initiated by A. Li\'egeois~\cite{liegeois77} on redundant robots (i.e. robots that have
%more degrees of freedom than necessary to perform a given task), and
%then developed by Y. Nakamura~\cite{nakamura91}, B. Siciliano, J.J. Slotine~\cite{siciliano91} and O.
%Khatib~\cite{khatib87} introduce mathematical machinery based on linear algebra and
%numerical optimization that allows for clever ways to model the symbolic
%notion of \emph{task}~\cite{samson91}.\\

%Those \emph{tasks} are defined by their tasks spaces (eg. position of the hand
%in an arbitrary frame\cite{nakamura86a,khatib87}, or position of a visual feature in the image
%plane \cite{espiau92,hutchinson96a}), a reference behavior in those tasks spaces
%(eg. exponential decrease to zero) and by the differential link between
%the task space and the actuator space, typically the task Jacobian.
%Given a set of active task, the corresponding control law can be
%obtained by inverting the equation of motion of the robot. 

%In this work, the analyzed motion is not seen as a sequence of motion primitives but as a
%superimposition of controllers. The recognition of sequences
%is out of the scope of the paper.
%The task-function is generally used to generate a motion\cite{siciliano91, mansard07}, and

Dans le chapitre~\ref{chap:sot}, le formalisme de la fonction de t\^aches 
a été présenté. Chaque t\^ache peut \^etre utilisée pour générer un
modèle de mouvement dans diverses situations.
Dans ce sens, une t\^ache est à la fois un contr\^oleur qui peut 
générer un mouvement, mais également un descripteur du mouvement
exécuté.
Ce descripteur est quasi symbolique, mais contient aussi des paramètres
additionnels qui caractérisent la façon dont il est exécuté : par exemple,
une t\^ache régie par une décroissance exponentielle est paramétrée par
$\lambda$ qui caractérise la vitesse de décroissance. Dans ce chapitre, 
nous proposons d'utiliser les t\^aches comme un ensemble de descripteurs
pour reconnaitre un mouvement observé, en identifiant l'ensemble
des t\^aches qui ont été utilisées pour générer ce mouvement (voir la figure~\ref{fig:taskReco}).

\begin{figure}[t]
  \begin{center}
    \includegraphics[width=0.6\linewidth]{figures/chapitre5/taskReco.ps}
  \end{center}
  \caption[La reconnaissance de t\^aches.]{Nous proposons de déterminer automatiquement quelles sont les t\^aches,
  parmi un ensemble connu, qui ont servi à générer le mouvement observé.}
  \label{fig:taskReco}
\end{figure}

L'ensemble des t\^aches identifiées est utilisé pour caractériser le mouvement observé,
par exemple pour distinguer deux mouvements visuellement proches.

Nous présentons d'abord les hypothèses considérées pour l'application
de la méthode de reconnaissance de t\^aches, ensuite, les différentes
étapes constituant la reconnaissance de t\^aches sont présentées.
Ensuite, des expérimentations sont effectuées en simulation pour illustrer le comportement
de la méthode dans des conditions idéales, puis en utilisant
un système capture de mouvement pour acquérir les mouvements à analyser joués par un vrai robot.
Ces expérimentations illustrent la précision et la robustesse aux bruits de la reconnaissance de t\^aches en 
distinguant des paires de mouvements visuellement proches. Enfin,
des expérimentations prospectives sur la validité de la méthode
pour l'analyse du mouvement humain sont présentées.

 %-------------------------------------------------------------------%
%-------------------------------------------------------------------%
\section{M\'ethode}
\label{chap:taskReco}
%-------------------------------------------------------------------%
%-------------------------------------------------------------------%
\subsection{Hypothèses}
On suppose que le modèle de comportement de toute les t\^aches potentiellement utilisées sont connus
($\mbf{e}$, $\mbf{\dot{e}^*}$, $\mbf{J}$). Le modèle du robot
ainsi que toutes les t\^aches susceptibles d'intervenir
dans le mouvement sont connus. L'ensemble de ces t\^aches est appelé le \emph{lot de t\^aches}
et représente les capacités du robot.
Le mouvement observé est supposé avoir été généré en utilisant un sous-ensemble
inconnu du lot de t\^aches.
On suppose aussi que les t\^aches intervenant dans le mouvement observé sont compatibles
au sens des projections~$\mbf{P}$ définies dans~\eqref{eq:projector},
c'est-à-dire qu'il n'y a pas de singularité algorithmique~\cite{chiaverini97}. Enfin
l'ensemble des t\^aches actives ne changent pas d'état pendant le mouvement
(une pile de t\^aches fixe est valide pendant toute la durée du mouvement).
Enfin, le mouvement observé est donné sous la forme de trajectoire articulaire: $\mbf{\hat{\dot{q}}}(t)$.
Dans notre cas, cette trajectoire articulaire sera reconstruite à partir de données issues
d'un système de capture de mouvements comme décrit dans le paragraphe~\ref{sec:xpset}.

\subsection{Présentation générale}
\label{sec:alg1:selec}
L'idée générale est de reconstruire de manière itérative
la pile de t\^aches qui aurait généré le mouvement observé.
La trajectoire articulaire observée est projetée dans 
chacun des espaces de t\^aches issu d'un lot de t\^aches connus.
Les trajectoires projetées sont ensuite comparé par optimisation à une trajectoire
théorique calculée gr\^ace au modèle de génération.
Cette comparaison correspond à une fonction de coût dont le score
va servir de critère de selection. La t\^ache qui donne la trajectoire
projetée de plus faible score est sélectionnée comme étant une
t\^ache intervenant dans le mouvement.

Les effets de la t\^ache sélectionnée sont annulés en projetant la trajectoire
articulaire dans l'espace nul de la t\^ache. Le mouvement résultant est alors utilisé
pour l'itération suivante.

Si à l'itération $n$ le mouvement résultant de l'annulation d'une t\^ache est nul, alors
tous les mouvements présent dans le mouvement initial ont été justifiés 
par l'exécution des t\^aches sélectionnées. Par conséquent,
la pile de t\^aches qui aurait généré le mouvement observé
est constituée des t\^aches sélectionnées.
\newcommand{\shOUTPUT}{\textbf{Output: }}
\newcommand{\shINPUT}{\textbf{Input: }}

\begin{algorithm}[t]
  \caption{Algorithme de sélection de t\^aches}
  \label{alg:taskSelection}
\begin{algorithmic}[1]
  \STATE \shINPUT $\mathbf{\hat{\dot{q}}}(t)$
\STATE \shOUTPUT $activePool$
\STATE $\mathbf{P}_A\mathbf{\dot{q}}(t)\gets \mathbf{\hat{\dot{q}}}(t)$
\WHILE{$\int \Vert \mathbf{P}_A\mathbf{\dot{q}}(t) \Vert ^2 dt > \epsilon$ }
  \FOR{task $i = 1..n$}
    \STATE $r_i \gets \mathrm{taskFitting}(i, activePool)$
  \ENDFOR
  \STATE $i_{select} \gets \mathrm{argmin}(r_i)$
  \STATE $activePool.\mathrm{push}(i_{select})$
  \STATE $\mathbf{P}_A\mathbf{\dot{q}}(t) \gets \mathrm{projection}(i_{select}, \mathbf{P}_A\mathbf{\dot{q}}(t))$
\ENDWHILE
\end{algorithmic}
\end{algorithm}
L'algorithme est présenté dans l'algorithme~\ref{alg:taskSelection}.
La trajectoire de vitesse articulaire du mouvement observé est notée
$\mathbf{\hat{\dot{q}}}(t)$.
$\mbf{P}_A\mbf{\dot{q}}(t)$ représente les résultats des mouvements projetés dans les espaces nuls des t\^aches
sélectionnées. $A$ est l'ensemble des t\^aches qui ont été selectionnées.
Avant la première itération, $\mbf{P\dot{q}}(t)$ est initialisée avec la trajectoire  
à analyser. Ensuite, à chaque itération, cette trajectoire est projetée
dans l'espace nul de la t\^ache sélectionnée.
$r_i$ représente le score de la fonction de coût de l'optimisation. $activePool$ représente
l'ensemble des t\^aches sélectionnées lors du déroulement de l'algorithme. 

Dans le cas où le mouvement observé a été exactement
généré par la pile de t\^aches, la trajectoire articulaire $\mathbf{P\dot{q}}$ après les
projections dans l'espace nul de toute les t\^aches est nulle.
Cependant, en présence de bruits (comme par exemple un bruit issu 
de vrais capteurs lors de l'acquisition du mouvement),
un résidu est systématiquement obtenu. Il faut donc utilisé une valeur
seuil comme critère d'arrêt: la boucle s'arrête lorsque le résidu
des projections passe en dessous du niveau du bruit de la chaine d'acquisition.
$\epsilon$ représente ce seuil: lorsque la norme du mouvement
est inférieure à ce seuil, l'algorithme s'arrête.

La section suivante décrit les deux fonctions principales de l'algorithme:
\begin{itemize}
  \item $\textsf{projection}(i, \mathbf{\dot{q}}(t))$ calcule la projection
de la vitesse $\dot{\mathbf{q}}(t)$ dans l'espace nul de la t\^ache $i$.
  \item $\textsf{taskFitting}(i, activePool)$ s'occupe de l'ajustement de 
courbe du mouvement observé et du mouvement théorique et calcule donc un score. 
Ce processus est détaillé dans la section~\ref{sec:alg2:proj}.
\end{itemize}

\subsection{Projection du mouvement}
\label{sec:alg2:nullification}
%\subsubsection{General case}
Une reconstruction des trajectoires dans l'espace des t\^aches à partir des trajectoires 
angulaires peut être obtenue en multipliant cette trajectoire articulaire par la 
jacobienne de la t\^ache calculée en chaque point.

Afin d'annuler les effets d'une t\^ache détectée $i$, 
la trajectoire articulaire est projetée dans l'espace nul
de cette t\^ache en la multipliant par le projecteur dans l'espace nul
de toute les t\^aches à annuler.
\`A chaque instant $t$ du mouvement :
\begin{equation}
  \mbf{P}_A\mbf{\dot{q}}(t) \leftarrow \mbf{P}_{A+i}(t) \mbf{P}_A\mbf{\dot{q}}(t) 
  \label{eq:projection}
\end{equation}
où $A$ est l'ensemble des t\^aches déjà sélectionnées et $\mbf{P}_A(t)$ est mis à jour:
\begin{equation*}
  \mbf{P_{A+i}}(t) = \mbf{P}_A(t) - (\mbf{J_{i}}(t)\mbf{P}_A(t))^+(\mbf{J_{i}}(t)\mbf{P}_A(t))
  \label{eq:projectionupdate}
\end{equation*}
Le projecteur $\mbf{P}_A(t)$ est initialisé à $\mbf{P}_A^0(t) = \mbf I$. Le mouvement
restant après projection $\mbf{P}_A\mbf{\dot{q}}(t)$ est analysé pour détecter
les t\^aches potentiellement restantes.

L'opération de projection va annuler les effets de l'espace
de la t\^ache sélectionnée dans le mouvement. Dans l'espace articulaire,
ces effets sont les suivants: d'une part, la composante de mouvement
qui est indépendante des autres t\^aches est complètement annulée, et d'autre part, la composante 
de mouvement qui est couplée avec d'autres t\^aches est modifiée.
Le premier effet est bénéfique car il permet d'éviter
de fausses détections causées par un mouvement présent dans 
l'espace de la t\^ache sélectionnée.
Cependant, la partie couplée doit \^etre traitée avec attention
à cause de sa modification comme expliqué dans l'exemple suivant:

%\medskip
%\subsubsection{Example with two tasks}
En considérant un mouvement composé de deux t\^aches arbitraires $\mbf{e}_a$ et $\mbf{e}_b$.
La loi de commande est donnée par:
\begin{equation}
  \mbf{\dot{q}} = \mbf{J}_a^+\mbf{\dot{e}}_a^* - (\mbf{J}_b\mbf{P}_a)^+ (\mbf{\dot{e}}_b^* - \mbf{J}_b\mbf{J}_a^+\mbf{\dot{e}}_a^*)
  \label{eq:controlLaw2Tasks}
\end{equation}

Si $\mbf{e}_a$ est détectée en premier, \eqref{eq:projection} est appliquée avec
$i_{select} = a$. Le produit de~(\ref{eq:controlLaw2Tasks}) par $\mbf{P}_a$
annule le mouvement dans l'espace de la t\^ache $a$:
\begin{equation}
  \mbf{P}_a \mbf{\dot{q}} = \underbrace{\mbf{P}_a \mbf{J}_a^+\mbf{\dot{e}}_a^*}_{\mbf{0}} + \mbf{P}_a (\mbf{J}_b\mbf{P}_a)^+ (\mbf{\dot{e}}_b^* - \mbf{J}_b\mbf{J}_a^+\mbf{\dot{e}}_a^*)  
  \label{eq:cancel1_2:1}
\end{equation}
Le premier terme est nul par définition de $\mbf{P}_a$.
Le mouvement provoqué par $\mbf{P}_a \mbf{\dot{q}}$ dans l'espace de la t\^ache $b$ est 
obtenue par son produit avec $\mbf{J}_b$:
\begin{equation}
  \mbf{J}_b \mbf{P}_a \mbf{\dot{q}} = \mbf{\dot{e}}_b^* -
  \mbf{J}_b  \mbf{J}_a^+ \mbf{\dot{e}}_a^*  
  \label{eq:J2P1q}
\end{equation}
puisque $\mbf{J}_b\mbf{P}_a(\mbf{J}_b\mbf{P}_a)^+ = \mbf{I}$ par hypothèse.
Le premier terme est la composante indépendante de la seconde t\^ache, et
le second terme est la composante couplée avec la t\^ache $a$.
Par conséquent, la projection dans l'espace nul de la t\^ache découverte à la première itération
va faire apparaitre le terme de couplage dont il faudra tenir compte pour la reconnaissance d'autres t\^aches.
Ce couplage sera traité dans la prochaine section.

\subsection{Ajustement de t\^ache par optimisation} \label{sec:alg2:proj}
$\medoth{}(t)$ représente la trajectoire causée par le mouvement à analyser courant $\mbf{P}_A\mbf{\dot{q}}$ 
dans l'espace de la t\^ache observée $\mbf{e}$. La quantification de la pertinence
d'une t\^ache donnée $\mbf{e}$, par rapport au mouvement,
est obtenue en appliquant une optimisation par les moindre carrés
entre le véritable mouvement observé projeté dans l'espace de la t\^ache et le comportement de 
référence d'une t\^ache dont les paramètres sont inconnus:
\begin{equation}
  \mbf{\hat x} = \underset{\mbf{x}}\argmin \frac{\int \Vert \medoth{}(t) - \medot{\mbf x}(t) \Vert^2 \mathrm{dt}}{\int \Vert \medoth{}(t) \Vert^2 \mathrm{dt}}
\label{optimProblem}
\end{equation}

\noindent où $\medoth{}(t)$ est la trajectoire observée et $\medot{\mbf{x}}(t)$ est la trajectoire
générée par le modèle de t\^ache en utilisant les paramètres $\mbf{x}$.
Le score est donc le résidu de l'optimisation après avoir essayé d'obtenir la meilleure correspondance 
entre la trajectoire observée et le modèle théorique dans l'espace des t\^aches.

Dans quelques cas, la trajectoire $\medoth{}$ peut \^etre observée directement.
Par exemple, si l'espace de t\^ache est la position 3D d'une main, une observation
directe est possible. Cependant, en général, une observation directe n'est pas possible.
Typiquement, le centre de masse du robot est difficile à observer avec précision dans le cadre d'une t\^ache
de maintien du centre de masse dans le polygone de sustentation.
De plus, il n'est pas possible d'observer directement l'effet d'une succession de projections.
$\medoth{}$ est alors obtenue gr\^ace au modèle géométrique du robot et
de sa trajectoire articulaires lors du mouvement: $\mbf{J}_i\mbf{\hat{\dot{q}}}$.
Cependant, procéder de cette manière amène à~(\ref{eq:J2P1q}), où le couplage
des t\^aches précédemment sélectionnées apparait.
Le mouvement projeté est alors augmenté par un terme qui va compenser ce couplage.
\begin{equation}
  \medoth{}(t) = \mbf{J}_i\mbf{P}_A \mbf{\dot{q}}(t) + \mbf{J}_i\mbf{J}_A^+\medoth{A}(t)
  \label{eq:compensation}
\end{equation}
où $A$ est l'ensemble des t\^aches qui a déjà été détecté et $i$ est la t\^ache candidate.
$\mbf{P}_A$ est le projecteur dans l'espace nul des t\^aches $A$, $\mbf{J}_A$ est 
la jacobienne des t\^aches $A$. Le terme $\mbf{J}_i\mbf{J}_A^+\medoth{A}(t)$ représente la 
composante du mouvement de la t\^ache $i$ couplé aux t\^aches de l'ensemble $A$.
Tous les termes correspondant à $A$ sont connus puisqu'ils ont été identifiés
dans la précédente itération de l'algorithme.

\medskip
En pratique, l'observation est échantillonnée et l'intégrale dans~\eqref{optimProblem} est une somme.
Le problème d'optimisation~(\ref{optimProblem}) est en général un problème non
linéaire. Pour le résoudre numériquement, le solveur CFSQP
a été utilisé~\cite{lawrence97}. Le résultat de l'optimisation produit
en même temps le résidu utilisé comme critère de sélection de la t\^ache la plus probable,
et les valeurs numériques associées à la t\^ache (par exemple le gain et la position désirée, quand il
s'agit de t\^aches proportionnelles).

%The task reference behavior $\mbf{e_x}$ can, for example, be defined by an exponential regulation.
%In that case, the characteristic trajectory in the task
%space can be obtained by analytical integration of the differential equation $\mbf{\dot{e}} = -\lambda\mbf{e}$ :
%it is an exponential decrease with three input parameters :
%
%\begin{equation}
%p_{\mbf x}(t) = x_1 \mathrm{e}^{(-x_2 t)} + x_3
%\label{eqTask}
%\end{equation}
%
%The vector of parameters $(x_1, x_2, x_3)$ has then a material interpretation:
%$x_3$ is the value at convergence ($x_3=s^*$). $x_1+x_3$ is the initial value ($x_1=s(0)-s^*$).
%And $x_2$ is the gain ($x_2=\lambda$).

\subsection{Invariance à l'ordre d'annulation}
La précédente formulation de détection-projection permet d'annuler les effets de
bords des t\^aches précédemment détectées sans introduire de couplage venant des t\^aches
non détectées. Nous montrons maintenant que l'ordre dans lequel les t\^aches sont détectées
et les projections suivantes n'affectent pas la détection des t\^aches restantes.

Considérons un mouvement observé ayant été généré par une pile de t\^aches à
deux étages où la t\^ache $a$ est prioritaire
par rapport à la t\^ache $b$~(\ref{eq:controlLaw2Tasks}).
\`A partir du mouvement observé initiallement, noté $\mbf{\dot{q}}$, on observe une trajectoire dans l'espace
de la t\^ache $i$. La t\^ache est ensuite supprimé du mouvement. Le mouvement
résultant est $\mbf{P}_i \mbf{\dot{q}}$. On observe alors une trajectoire dans l'espace de la tâche $j$.
Nous montrons dans la suite que les observations dans les espaces des t\^aches sont invariablement les mêmes
pour $i=a$ et $j=b$ ou pour $i=b$ et $j=a$. C'est-à-dire que 
les deux t\^aches ayant servies à générer le mouvement
peuvent \^etre détectées (et supprimées) dans n'importe quel
ordre.

Le mouvement de référence de l'espace de t\^ache $i$
est noté $\medotc{i}$. Le mouvement observé par la t\^ache $i$ dans le mouvement
initial $\mqdot$ est noté $\medoth{i}$, tandis que le mouvement observé par la t\^ache
$j$ après avoir supprimé le mouvement issu de la t\^ache $i$ $\mP{j} \mqdot$ est noté~$\medothp{j}{i}$.

\begin{proposition}
  Soit un mouvement généré par une pile de t\^aches à deux étages. Alors
$\medoth{i} = \medothp{i}{j} = \medotc{i}$, pour $i=a$ et $j=b$, et réciproquement, pour $i=b$ et $j=a$.
\end{proposition}
%\subsubsection{Direct implication: removing Task $a$, then Task $b$}
\noindent\textbf{Preuve:}
La preuve se fait en deux parties, en considérant que $a$ est annulée en premier ($i=a$ et $j=b$),
puis que $b$ est annulée en premier ($i=b$ et $j=a$).\\

\noindent\textbf{Détection de la t\^ache $a$, puis de la t\^ache $b$:}

  Cette implication est immédiate.
\`A la première itération, l'observation dans l'espace de t\^ache $a$ est directement
$\medoth{a} = \mbf{J}_a\mbf{\hat{\dot{q}}} = \medotc{a}$.  
Après la suppression de la t\^ache $a$, l'observation dans l'espace de la t\^ache $b$
est obtenue à partir de \eqref{eq:compensation}:
\begin{align*}
 \medothp{b}{a} &=  \mbf{J}_b\mbf{\hat{\dot{q}}} +  \mbf{J}_b \mbf{J}_a^+ \medoth{a}\\
 &= \mbf{J}_b(\mbf{J}_b \mbf{P}_a)^+ (\medotc{b} - \mbf{J}_b \mbf{J}_a^+\medotc{a}) + \mbf{J}_b \mbf{J}_a^+ \medotc{a}
\end{align*}
impliquant directement $\medothp{b}{a}=\medotc{b}$, puisque $\mbf{J}_b(\mbf{J}_b \mbf{P}_a)^+ = \mbf{J}_b \mbf{P}_a(\mbf{J}_b \mbf{P}_a)^+ = \mbf{I}$ et $\medoth{a} = \medotc{a}$.

L'observation obtenu dans l'espace de t\^ache $b$ est indépendante de la projection
dans l'espace nul de la t\^ache $a$.\\

\noindent\textbf{Détection de la t\^ache $b$, puis de la t\^ache $a$:}

\`A la première itération, l'observation dans l'espace de la t\^ache $b$ est directement
\begin{equation*}
\medoth{b}  = \mJ{b} \mJp{a} \medotc{a} + \mJ{b} ( \mJ{b} \mP{a} )^+ ( \medotc{b} - \mJ{b} \mJp{a} \medotc{a} )
\end{equation*}
ce qui implique $\medoth{b} = \medotc{b}$ puisque $\mJ{b} ( \mJ{b} \mP{a} )^+ = \mbf{I}$.

La projection du mouvement initial~(\ref{eq:controlLaw2Tasks}) dans l'espace nul de
la t\^ache $b$ à la fin de la première itération conduit au calcul de la trajectoire 
dans l'espace de la t\^ache $a$ suivant:
%\begin{equation}
%  \mbf{P}_b \mbf{\hat{\dot{q}}} = \mbf{P}_b \mbf{J}_a^+\mbf{\hat{\dot{e}}}_a + 
%  \mbf{P}_b (\mbf{J}_b \mbf{P}_a)^+ (\mbf{\hat{\dot{e}}}_b - \mbf{J}_b \mbf{J}_a^+ \mbf{\hat{e}}_a)
%  \label{eq:proof:P2q}
%\end{equation}
\begin{align*}
\medothp{a}{b} =& \mJ{a} \mP{b} \mJp{a} \medotc{a} +
 \mJ{a} \mP{b} (\mJ{b} \mP{a} )^+ ( \medotc{b} - \mJ{b} \mJp{a} \medotc{a})
+ \mJ{a} \mJp{b} \medoth{b} \\
=& (\mJ{a} \mP{b} \mJp{a} - \mJ{a} \mP{b} (\mJ{b} \mP{a} )^+ \mJ{b} \mJp{a}) \medotc{a} \\
&+ (\mJ{a} \mP{b} (\mJ{b} \mP{a} )^+ + \mJ{a} \mJp{b}) \medotc{b}
\end{align*}
Nous montrons que cette somme est en réalité simplement égal à $\medotc{a}$
en montrant d'abord que le coefficient gauche du premier terme 
$(\mJ{a} \mP{b} \mJp{a} - \mJ{a} \mP{b} (\mJ{b} \mP{a} )^+ \mJ{b} \mJp{a})$, 
qu'on note $\mbf{C}_1$, est égal à l'identité.
Puis en montrant que le second coefficient gauche $(\mJ{a} \mP{b} (\mJ{b} \mP{a} )^+ + \mJ{a} \mJp{b})$,
qu'on note $\mbf{C}_2$, est nul.
$\mbf{C}_1$ peut s'écrire en développant l'operateur 
de projection $\mP{b}$ en utilisant $\mbf{P} = \mbf{I} - \mbf{J}^+ \mbf{J}$:
\begin{align*}
\mbf{C}_1 =& \mJ{a} \mJp{a} -   \mJ{a} \mJp{b} \mJ{b} \mJp{a} \\
&- \mJ{a} (\mJ{b} \mP{a} )^+ \mJ{b} \mJp{a}  + \mJ{a} \mJp{b} \mJ{b} (\mJ{b} \mP{a} )^+ \mJ{b} \mJp{a}
\end{align*}
Le premier terme est l'identité par hypothèse sur la t\^ache $a$, car
$(\mJ{b} \mP{a} )^+ = \mP{b}(\mJ{b} \mP{a} )^+$, le troisième terme est nul par définition. 
Par hypothèse sur les t\^aches $a$ et $b$, $(\mJ{b} \mP{a} )^+$ 
est l'inverse généralisé de $\mJ{b}$, et donc $\mJ{b} (\mJ{b} \mP{a} )^+ \mJ{b} = \mJ{b}$,
le quatrième terme est égal au second et se simplifie. Donc $\mbf{C}_1 = \mbf{I}$.

De la même façon pour $\mbf{C}_2$, le projecteur $\mP{b}$ est developpé:
\[
\mbf{C}_2 = \mJ{a} (\mJ{b} \mP{a} )^+ - \mJ{a} \mJp{b}\mJ{b} (\mJ{b} \mP{a} )^+ + \mJ{a} \mJp{b}
\]
Comme précédemment, le premier terme est nul, $(\mJ{b} \mP{a} )^+$ 
est l'inverse généralisé de $\mJ{b}$, le second terme est égal au troisième et se simplifie.

Ceci prouve que le même ajustement est obtenu indépendamment de la t\^ache qui a été
détectée la première. ~\QED 

%\end{proof}

\medskip
La preuve peut \^etre étendue en utilisant les m\^emes arguments 
pour un ensemble de $n$ t\^aches.
D'une façon similaire, il est trivial de prouver que quelque soit l'ordre
de détection, le mouvement calculé après toutes les projections est nul.


 %-------------------------------------------------------------------%
%-------------------------------------------------------------------%
%\section{Exp\'erimentations: Reconnaissance de mouvements effectués par le robot HRP-2}
\label{chap:xpRobot}
%-------------------------------------------------------------------%
%-------------------------------------------------------------------%

Pour valider la m\'ethode présentée plus haut, plusieurs exp\'erimentations
ont \'et\'e effectu\'ees sur le robot HRP-2 en simulation
et sur le vrai robot. Ces exp\'erimentations consistent 
\`a discriminer des paires de mouvements visuellement proches, mais 
faisant intervenir des t\^aches diff\'erentes: les s\'emantiques
de ces mouvements sont diff\'erentes.

\section{Résultats en simulation: reconnaissance de t\^aches effectuées par le robot HRP-2}
\label{sec:simu}
Ce paragraphe détaille une série d'expérimentations effectuées en simulation pour 
valider l'algorithme de reconnaissance.
La simulation nous permet d'observer le comportement nominal de l'algorithme
en absence de bruits issus de capteurs.
La première expérimentation valide simplement l'étape de projection 
du mouvement~(paragraphe \ref{sec:prelimValid}). 
%Fig.~\ref{fig:snapshotXpqdot} and Fig.~\ref{fig:xp3Pqdot} illustrate the effect of the projections.
La seconde partie réunit un ensemble d'expérimentations qui valide l'algorithme de reconnaissance 
de t\^aches~(paragraphe \ref{sec:distinc}).
%Fig.~\ref{fig:XP2RFit} and Fig.~\ref{fig:XP2RLFit} shows how the fitting behaves. 
%Fig.~\ref{fig:RbeforeAfterProj} shows how a projected trajectory evolves after other projections.
Pour chaque expérimentation de cet ensemble, l'algorithme de reconnaissance de t\^aches
est appliqué à deux mouvement visuellement proches:
ces deux mouvements ont été artificiellement construits pour présenter une
ambigu\"ité visuelle lorsqu'on les compare entre eux. La présence de cette
ambiguïté permet d'illustrer l'efficacité de notre algorithme
vis à vis de la précision de la reconnaissance.
Tous les mouvements qui ont servis aux expérimentations sont résumés dans le tableau~\ref{tab:motion}.
La description des t\^aches utilisées pour construire ces mouvements est détaillée ci-dessous.

\subsection{Protocole d'expérimentations}
Les mouvements de référence ont été générés en utilisant le modèle du robot humanoïde
HRP-2. Celui-ci présente 30 degrés de liberté actionnés, plus six degrés de liberté 
pour la base flottante.
Tous les mouvements démarrent de la configuration de repos \emph{half-sitting} (comme illustrée
dans la figure~\ref{fig:halfSit}),
et le centre de masse est situé à zéro dans le repère du bassin lorsque 
celui-ci est la racine de l'arbre cinématique. Cette configuration a pour particularité de ne pas présenter
de singularité.
En cinématique inverse, la position de la base flottante sous-actuée est résolue en contraignant le pied gauche
à rester sur le sol.
Les figure~\ref{fig:spotDiff1},~\ref{fig:spotDiff2} et~\ref{fig:spotDiff3} montrent les postures finales
des mouvements utilisées dans les expérimentations.
\begin{figure}[t]
\begin{center}
\includegraphics[width=0.3\linewidth]{figures/chapitre5/halfSit.ps}
\end{center}
\caption[Pose \emph{half-sitting}.]{Tout les mouvements de référence démarrent à partir de la posture de repos \emph{half-sitting}.}
\label{fig:halfSit}
\end{figure}
\begin{figure}[p]
  \begin{center}
    \includegraphics[width=0.7\linewidth]{figures/chapitre5/spotDiff.ps}
  \end{center}
  \caption[Mouvements d'atteintes.]{Gauche: La posture finale du \emph{mouvement~2.a}; Droite: La posture finale du \emph{mouvement~2.b}.}
  \label{fig:spotDiff1}
\end{figure}
\begin{figure}[p]
\begin{center}
\includegraphics[width=0.50\linewidth]{figures/chapitre5/simu/3/final3.ps}
\end{center}
\caption[Mouvements prise et prise orientée.]{Gauche: La posture finale du \emph{mouvement~3.a}; Droite: La posture finale du \emph{mouvement~3.b}.
La différence entre ces deux mouvements est difficile à percevoir: l'orientation de la main droite
est contrainte à être parallèle au sol dans le \emph{mouvement~3.b}.}
\label{fig:spotDiff2}
%\vspace{-3pt}
\end{figure}
\begin{figure}[p]
\begin{center}
\includegraphics[width=0.50\linewidth]{figures/chapitre5/spotDiff3.ps}
\end{center}
\caption[Mouvements de prise orientée et regard.]{Gauche: La posture finale du \emph{mouvement~4.a}; Droite: La posture finale du \emph{mouvement~4.b}.}
\label{fig:spotDiff3}
\end{figure}

L'ensemble des t\^aches considérées dans les expérimentations sur HRP-2 est : 

\begin{itemize}
  \item \emph{Com} : t\^ache~\eqref{eq:taskCom}, le centre de masse du robot est contraint pour maintenir un équilibre statique (3 DDL)
  \item \emph{Regard} : t\^ache~\eqref{eq:visError}, le robot regarde un point dans l'espace Cartésien, le point de contrôle se situe
à 25cm au dessus du centre de la dernière articulation du cou (2 DDL)
  \item \emph{Double support} : t\^ache~\eqref{eq:taskRelative}, la transformation entre le repère attaché au pied droit et le repère 
  attaché au pied gauche du robot doit être constante (6 DDL)
  \item \emph{Prise gauche/droite} : t\^ache~\eqref{eq:taskPosition}, un repère attaché à la main gauche ou droite du robot atteint un point défini dans l'espace Cartésien, les points de contrôle se situent aux centres des articulations terminales des bras (3 DDL)
  \item \emph{Prise orientée} : t\^ache~\eqref{eq:feature6dE}, similaire à la t\^ache de \emph{prise}, mais la position
    désirée doit \^etre atteinte avec une orientation de la main définie (6 DDL)
  \item \emph{Tête} : t\^ache~\eqref{eq:feature6dE}, un repère attaché à la t\^ete du robot est contrainte en position et en orientation, le point de contrôle se situe au centre de la dernière articulation de la tête (6 DDL) 
  \item \emph{Torse} : t\^ache~\eqref{eq:taskRotation}, un repère attaché au torse du robot est contraint en orientation, le point de contrôle se situe au centre de la dernière articulation associée au torse (3 DDL)
\end{itemize}
\begin{table}[h]
  \centering
    \begin{tabular}{|c|c|c|}
      \hline
      & (a) & (b) \\
      \hline
      & &
      \\
      Mouvement 1 & 
	\begin{tabular}{|c|}
          \hline
          \emph{Prise droite}\\
          \hline
          \emph{Prise gauche}\\
          \hline
          \emph{Regard}\\
	  \hline
	  \emph{Com}\\
	  \hline
	  \emph{Double support}\\
	  \hline
	\end{tabular} & \\
      & &
      \\
      \hline
      & &
      \\
      Mouvement 2 & 
        \begin{tabular}{|c|}
          \hline
          \emph{Prise droite}\\
	  \hline
	  \emph{Com}\\
	  \hline
	  \emph{Double support}\\
	  \hline
	\end{tabular}  
    &
	\begin{tabular}{|c|}
	  \hline
	  \emph{Prise gauche}\\
	  \hline
	  \emph{Prise droite}\\
	  \hline
	  \emph{Com}\\
	  \hline
	  \emph{Double support}\\
	  \hline
	\end{tabular}  
    \\
    & &
    \\
    \hline
    & &
    \\
    Mouvement 3 & 
        \begin{tabular}{|c|}
	  \hline
	  \emph{Regard}\\
	  \hline
	  \emph{Prise droite}\\
	  \hline
	  \emph{Com}\\
	  \hline
	  \emph{Double support}\\
          \hline
        \end{tabular}  
    &
	\begin{tabular}{|c|}
	  \hline
	  \emph{Regard}\\
	  \hline
	  \emph{Prise orientée droite}\\
	  \hline
	  \emph{Com}\\
	  \hline
	  \emph{Double support}\\
	  \hline
	\end{tabular}  
    \\
    & &
    \\
    \hline
    & &
    \\
    Mouvement 4 & 
	\begin{tabular}{|c|}
	  \hline
	  \emph{Regard}\\
	  \hline
	  \emph{Com}\\
	  \hline
	  \emph{Double support}\\
	  \hline
	\end{tabular}  
    &
	\begin{tabular}{|c|}
	  \hline
	  \emph{Prise orienté gauche}\\
	  \hline
	  \emph{Com}\\
	  \hline
	  \emph{Double support}\\
	  \hline
	\end{tabular} 
      \\
    & &
    \\
    \hline
    & &
    \\
    Mouvement 5 &
    \begin{tabular}{|c|}
      \hline
      \emph{Prise droite}\\
      \hline
      \emph{Regard}\\
      \hline
      \emph{Com}\\
      \hline
      \emph{Double support}\\
      \hline
    \end{tabular}
    &
    \begin{tabular}{|c|}
      \hline
      \emph{Prise droite}\\
      \hline
      \emph{Torse}\\
      \hline
      \emph{Com}\\
      \hline
      \emph{Double support}\\
      \hline
    \end{tabular}
    \\
    & &
    \\
    \hline
  \end{tabular}
  \caption{Tableau des mouvements et des t\^aches considérés.}
  \label{tab:motion}
\end{table}

\FloatBarrier
\subsection{Expérimentation 1 : Validation préliminaire}
\label{sec:prelimValid}
Dans cette expérimentation, un mouvement de base est généré.
La reconnaissance de t\^ache par ajustement et le critère d'arrêt de l'algorithme sont validés ici.
Le mouvement de référence est le \emph{mouvement~1.a} (voir le tableau~\ref{tab:motion}).
Le mouvement est donné au programme de détection qui va sélectionner
les t\^aches qui correspondent le mieux par optimisation.

La figure~\ref{fig:snapshotXpqdot} montre des vignettes du mouvement original
et du mouvement résultant des projections successives dans les espaces
nuls des t\^aches détectées:
\emph{prise droite}, \emph{com}, \emph{regard}, \emph{double support} et \emph{prise gauche}.
Le mouvement initial est visualisé en intégrant un champs de vecteurs. Les
projections dans les espaces nuls vont modifier ces vecteurs que l'on intègre pour visualiser
les mouvements projetés. Ces mouvements projetés n'ont donc pas 
de sens physique et seule la quantité de mouvements reste valide.
Ces quantités de mouvements permettent d'avoir une bonne
intuition sur les effets des projections dans les espaces nuls.
Chaque projection annule une partie du mouvement, et le mouvement du robot devient nul quand 
toutes les projections sont appliquées, ce qui signifie que toutes les t\^aches
intervenant dans le mouvement ont été détectées.
\begin{figure*}[p]
\centering
\begin{tabular}{c@{}c@{}c@{}c@{}c@{}c@{}c}
(a)&
\parbox[c]{2.2cm}{\includegraphics[width=\linewidth]{figures/chapitre5/Pqdot0_0.png.ps}} &
\parbox[c]{2.2cm}{\includegraphics[width=\linewidth]{figures/chapitre5/Pqdot0_99.png.ps}} &
\parbox[c]{2.2cm}{\includegraphics[width=\linewidth]{figures/chapitre5/Pqdot0_199.png.ps}} &
\parbox[c]{2.2cm}{\includegraphics[width=\linewidth]{figures/chapitre5/Pqdot0_299.png.ps}} &
\parbox[c]{2.2cm}{\includegraphics[width=\linewidth]{figures/chapitre5/Pqdot0_399.png.ps}} &
\parbox[c]{2.2cm}{\includegraphics[width=\linewidth]{figures/chapitre5/Pqdot0_499.png.ps}}\\

(b)&
\parbox[c]{2.2cm}{\includegraphics[width=\linewidth]{figures/chapitre5/Pqdot1_0.png.ps}} &
\parbox[c]{2.2cm}{\includegraphics[width=\linewidth]{figures/chapitre5/Pqdot1_99.png.ps}} &
\parbox[c]{2.2cm}{\includegraphics[width=\linewidth]{figures/chapitre5/Pqdot1_199.png.ps}} &
\parbox[c]{2.2cm}{\includegraphics[width=\linewidth]{figures/chapitre5/Pqdot1_299.png.ps}} &
\parbox[c]{2.2cm}{\includegraphics[width=\linewidth]{figures/chapitre5/Pqdot1_399.png.ps}} &
\parbox[c]{2.2cm}{\includegraphics[width=\linewidth]{figures/chapitre5/Pqdot1_499.png.ps}}\\

(c)&
\parbox[c]{2.2cm}{\includegraphics[width=\linewidth]{figures/chapitre5/Pqdot2_0.png.ps}} &
\parbox[c]{2.2cm}{\includegraphics[width=\linewidth]{figures/chapitre5/Pqdot2_99.png.ps}} &
\parbox[c]{2.2cm}{\includegraphics[width=\linewidth]{figures/chapitre5/Pqdot2_199.png.ps}} &
\parbox[c]{2.2cm}{\includegraphics[width=\linewidth]{figures/chapitre5/Pqdot2_299.png.ps}} &
\parbox[c]{2.2cm}{\includegraphics[width=\linewidth]{figures/chapitre5/Pqdot2_399.png.ps}} &
\parbox[c]{2.2cm}{\includegraphics[width=\linewidth]{figures/chapitre5/Pqdot2_499.png.ps}}\\

(d)&
\parbox[c]{2.2cm}{\includegraphics[width=\linewidth]{figures/chapitre5/Pqdot3_0.png.ps}} &
\parbox[c]{2.2cm}{\includegraphics[width=\linewidth]{figures/chapitre5/Pqdot3_99.png.ps}} &
\parbox[c]{2.2cm}{\includegraphics[width=\linewidth]{figures/chapitre5/Pqdot3_199.png.ps}} &
\parbox[c]{2.2cm}{\includegraphics[width=\linewidth]{figures/chapitre5/Pqdot3_299.png.ps}} &
\parbox[c]{2.2cm}{\includegraphics[width=\linewidth]{figures/chapitre5/Pqdot3_399.png.ps}} &
\parbox[c]{2.2cm}{\includegraphics[width=\linewidth]{figures/chapitre5/Pqdot3_499.png.ps}}\\

(e)&
\parbox[c]{2.2cm}{\includegraphics[width=\linewidth]{figures/chapitre5/Pqdot4_0.png.ps}} &
\parbox[c]{2.2cm}{\includegraphics[width=\linewidth]{figures/chapitre5/Pqdot4_99.png.ps}} &
\parbox[c]{2.2cm}{\includegraphics[width=\linewidth]{figures/chapitre5/Pqdot4_199.png.ps}} &
\parbox[c]{2.2cm}{\includegraphics[width=\linewidth]{figures/chapitre5/Pqdot4_299.png.ps}} &
\parbox[c]{2.2cm}{\includegraphics[width=\linewidth]{figures/chapitre5/Pqdot4_399.png.ps}} &
\parbox[c]{2.2cm}{\includegraphics[width=\linewidth]{figures/chapitre5/Pqdot4_499.png.ps}}\\

(f)&
\parbox[c]{2.2cm}{\includegraphics[width=\linewidth]{figures/chapitre5/Pqdot5_0.png.ps}} &
\parbox[c]{2.2cm}{\includegraphics[width=\linewidth]{figures/chapitre5/Pqdot5_99.png.ps}} &
\parbox[c]{2.2cm}{\includegraphics[width=\linewidth]{figures/chapitre5/Pqdot5_199.png.ps}} &
\parbox[c]{2.2cm}{\includegraphics[width=\linewidth]{figures/chapitre5/Pqdot5_299.png.ps}} &
\parbox[c]{2.2cm}{\includegraphics[width=\linewidth]{figures/chapitre5/Pqdot5_399.png.ps}} &
\parbox[c]{2.2cm}{\includegraphics[width=\linewidth]{figures/chapitre5/Pqdot5_499.png.ps}}\\
\\
Itération & 0 & 99 & 199 & 299 & 399 & 499\\
\end{tabular}
\caption[Illustrations des projections successives.]{Le mouvement généré par une pile de t\^aches contenant :
\emph{Prise droite}, \emph{Prise gauche}, \emph{Com}, \emph{Regard}, \emph{Double support} 
est représenté dans la ligne (a).
Les autres lignes représentent les projections successives du mouvement dans les espaces
nuls des t\^aches: (b) Prise droite, (c) Com, (d) Regard, (e) Double support, (f) Prise gauche.}
\label{fig:snapshotXpqdot}
\end{figure*}
Comme nous pouvons le constater dans la seconde ligne, le mouvement de la main droite est annulé.
L'annulation de \emph{com} à partir de la troisième ligne est plus difficile à percevoir.
On peut noter la modification du mouvement des jambes. Cette modification correspond au fait que pour
exécuter la t\^ache \emph{Com}, le robot utilise ses jambes.
À la quatrième ligne, l'annulation du mouvement de la t\^ete est très claire.
On note que le mouvement du torse est altéré. Le mouvement du torse est donc justifié par
la t\^ache de regard.
À la cinquième
ligne, l'annulation de la t\^ache \emph{double support} supprime la compensation
faite avec le pied droit. Enfin, comme on peut le voir dans la dernière ligne,
l'annulation de toutes les t\^aches mène à un mouvement nul.
Tous les mouvements du robot ont donc été justifiés.
La figure~\ref{fig:xp3Pqdot} montre l'évolution de la norme du mouvement,
définie par la somme des normes au carré sur le temps de la trajectoire articulaire du robot.
\begin{figure}[t]
\begin{center}
%\includegraphics[height=0.9\linewidth, angle = -90]{figures/chapitre5/PqdotNorms.ps}
\resizebox{.48\textwidth}{!} {
      \input{figures/chapitre5/PqdotNormsProjection.tex}
    }
\end{center}
\caption[Évolution de la norme du mouvement.]{L'évolution de la norme du mouvement après avoir projeté successivement le mouvement dans les
	espaces nuls des t\^aches.}
\label{fig:xp3Pqdot}
\end{figure}
Chaque projection fait strictement décroître la norme de la vitesse
(\ie~la quantité de mouvement). Après cinq projections, le mouvement est complètement
annulé, ce qui confirme que toutes les t\^aches actives ont été découvertes. L'algorithme
s'arrête donc.
\FloatBarrier

\subsection{Expérimentation 2 : Distinction entre deux mouvements proches}
\label{sec:distinc}
En ce qui concerne la détection de mouvement, \^etre capable
de différencier deux mouvements visuellement proches faisant tout de même
intervenir des t\^aches différentes est un problème intéressant.
Les algorithmes pour les systèmes anthropomorphes utilisent plut\^ot le contexte
pour la desambigüisation.
Le travail présenté ici montre que le critère d'ajustement mêlé à la représentation
du mouvement sous forme de pile de t\^aches est suffisant pour distinguer des mouvements visuellement
proches. Trois paires de mouvements ambigus vont \^etre présentées pour illustrer 
la capacité de distinction de mouvement de notre méthode.
Le modèle de comportement de t\^ache choisi est une décroissance exponentielle dont 
les paramètres seront les paramètres de l'optimisation du problème~\ref{optimProblem}.


\subsubsection{Mouvements d'atteintes}
\label{sec:distinc1}
Dans ce paragraphe, deux mouvements sont considérés: 
\emph{mouvement~2.a} et \emph{mouvement~2.b}.
Le premier est un mouvement d'atteinte lointain avec la main droite.
Ce mouvement d'atteinte de la main droite a une influence sur la main gauche par
le biais de la t\^ache \emph{com}: pour retrouver son équilibre,
le robot met sa main gauche en arrière.
Le second mouvement est la même t\^ache d'atteinte de la main droite, mais
une seconde t\^ache est ajoutée. Il s'agit d'une t\^ache d'atteinte de main gauche.
La position désirée pour cette dernière t\^ache a été définie artificiellement comme 
étant la position finale de la main gauche obtenue lors du premier mouvement.

Les état finaux du robot pour les deux mouvements sont illustrés dans la figure~\ref{fig:spotDiffbis}.
\begin{figure}[t]
  \centering
  \subfigure{
  \includegraphics[trim=220px 10px 220px 0px, width=0.43\linewidth, clip=true]{figures/chapitre1/Lonly.ps}
  }
  \subfigure{
  \includegraphics[trim=220px 10px 220px 0px, width=0.43\linewidth, clip=true]{figures/chapitre1/RL.ps}
  }
  \caption[Mouvements d'atteintes.]{États finaux des le \emph{mouvement~2.a} et le \emph{mouvement~2.b}.}
  \label{fig:spotDiffbis}
\end{figure}
Une vidéo montrant les deux mouvements appliqués sur le robot 
HRP-2 est disponible\footnote{\url{http://homepages.laas.fr/shak/videos/}}.
Les deux mouvements se ressemblent, et il est très difficile à l'\oe{}il nu
de dire quel mouvement implique les deux t\^aches d'atteinte sans le contexte.
Dans le premier cas, le mouvement de la main gauche est dû à aux t\^aches \emph{com}
et \emph{prise droite}, étant donné que c'est un effet secondaire pour compenser
l'équilibre de la main droite.
Dans le second cas, le mouvement de la main gauche est découplé,
puisque la main gauche a été dirigée par son propre objectif.
Cependant, dans l'espace de t\^aches approprié, ces mouvements apparaissent
clairement différents.
La figure~\ref{fig:XP2RFit} montre un exemple de résultats de l'ajustement de 
modèles de t\^aches pour la main droite et la main gauche appliqué au \emph{mouvement~2.a}.
\begin{figure*}[p]
\centering
\subfigure{
%\includegraphics[width=0.9\linewidth]{figures/chapitre5/projRHNull.eps}
  \resizebox{.47\textwidth}{!} {
      \input{figures/chapitre5/simu/2a/taskRhandNormInvJerr_0}
      }      }
      \subfigure{
  \resizebox{.47\textwidth}{!} {
      \input{figures/chapitre5/simu/2a/taskLhandNormInvJerr_0}
      }            }
\caption[Ajustement de modèle de t\^ache sur le \emph{mouvement~2.a}.]{L'ajustement de modèle de t\^ache sur le \emph{mouvement~2.a} sur les t\^aches de la 
main droite et gauche. La variable $r$ est le résidu,                                                         
\ie~la distance entre deux courbes. Le mouvement de la main droite est correctement ajusté par
le modèle de la t\^ache, montrant que la t\^ache est active. Le mouvement
de la main gauche n'est pas ajusté correctement, puisque la t\^ache est inactive.
Les deux cas sont facilement distinguables gr\^ace à la valeur du résidu.} 
\label{fig:XP2RFit}
\end{figure*}
Le résidu de l'optimisation est élevé puisque l'ajustement n'est pas possible sur une 
trajectoire qui ne respecte pas le modèle (en particulier, le résidu de la t\^ache \emph{prise gauche} est 
bien plus élevé que celui associé à la t\^ache \emph{prise droite}).
Les résultats de l'algorithme de détection sont résumé dans le 
Tableau~\ref{tab:spotDiff1}.
\begin{table}[t]
\centering
\begin{tabular}{|c|c|c|c|}
\hline
Référence & Détectées & $\int \Vert \dot{q}(t) \Vert ^2 dt$ & $\int \Vert P\dot{q}(t) \Vert ^2 dt$ \\
\hline
\begin{tabular}{c}
Com\\
Prise droite\\
Double support\\
\end{tabular}

&

\begin{tabular}{c}
Com\\
Prise droite\\
Double support\\
\end{tabular}

& 0.364398 & 0.00159355 \\
\hline
\begin{tabular}{c}
Com\\
Prise gauche\\
Prise droite\\
Double support\\
\end{tabular}

&
\begin{tabular}{c}
Com\\
Prise gauche\\
Prise droite\\
Double support\\
\end{tabular}

& 0.538329  & 0.0035343 \\
\hline
\end{tabular}
\caption[Résumé des résultats pour \emph{mouvement~2.a} et du \emph{mouvement~2.b}.]{Résultats de l'algorithme de sélection de t\^aches lors de l'analyse du \emph{mouvement~2.a} et du \emph{mouvement~2.b}.}
\label{tab:spotDiff1}
%\vspace{-20pt}
\end{table}
La première colonne liste les t\^aches 
utilisées dans le mouvement de référence,
la seconde colonne liste les t\^aches sélectionnées par l'algorithme,
la troisième indique la norme du mouvement de référence (quantité de mouvement initialement observée),
et la dernière colonne montre la norme du mouvement de référence projeté dans l'espace nul
des t\^aches sélectionnées.
La quantité de mouvement final est très faible pour les deux mouvements comparé au seuil
défini comme critère d'arrêt 
$\int \Vert P\dot{q}(t) \Vert ^2 dt > \epsilon$
qui est fixé à $\epsilon = 0.07$.

Enfin la figure~\ref{fig:RbeforeAfterProj} illustre comment les normes de la vitesse articulaire
correspondant aux t\^aches \emph{prise droite} et \emph{prise gauche} évoluent
après les projections dans les espaces nuls. 
\begin{figure*}[p]
\centering
  \subfigure[Mouvement 2.a]{
  \resizebox{.47\textwidth}{!} {
      \input{figures/chapitre5/simu/2a/RbeforeAfterProj.tex}
    }
  \label{fig:RbeforeAfterProj:2a}
  }
  \subfigure[Mouvement 2.b]{
  \resizebox{.47\textwidth}{!} {
      \input{figures/chapitre5/simu/2b/RbeforeAfterProj.tex}
  }
  \label{fig:RbeforeAfterProj:2b}
  }
  \caption[Evolution de la trajectoire $\dot{\mbf{q}}$.]{Evolution de la trajectoire $\dot{\mbf{q}}$ projetée dans les espaces des t\^aches
  \emph{prise droite} (en ligne pleine) et 
  \emph{prise gauche} (en pointillés) après les projections successives du mouvement dans l'espace
  nul des t\^aches \emph{prise droite} et l'espace nul de la t\^ache \emph{com}.
  Dans le \emph{mouvement~2.a}, une grande partie du mouvement du bras gauche
  est dû à la t\^ache \emph{com}: la suppression de la t\^ache \emph{com} va annuler presque tout
  le mouvement du bras gauche.
  Dans le \emph{mouvement~2.b}, le mouvement du bras gauche n'est pas seulement issu de
  la t\^ache \emph{com}, mais principalement de la t\^ache \emph{prise gauche}.
  La suppression de la t\^ache \emph{com} va seulement annuler une petite partie
  du mouvement du bras gauche.}
\label{fig:RbeforeAfterProj}
\end{figure*}
La projection du \emph{mouvement~2.a} dans l'espace nul de la t\^ache
\emph{prise droite} va faire décroître la norme de la vitesse articulaire théorique
associée tout en laissant la norme de la vitesse articulaire théorique de la t\^ache
\emph{prise gauche} inchangée. La figure~\ref{fig:RbeforeAfterProj:2a} montre
que la vitesse articulaire théorique associée à la t\^ache \emph{prise gauche}
est diminuée après la projection du mouvement dans l'espace nul de la t\^ache \emph{com}.
Ceci explique que le bras gauche du robot a été déplacé par la t\^ache \emph{com}.
De plus, cette projection empêche la t\^ache \emph{prise gauche} d'être détectée par l'algorithme
dans des itérations futures à cause de mouvements parasites.
Cependant, après la projection du \emph{mouvement~2.b} dans l'espace nul de la t\^ache \emph{prise droite}
et la t\^ache \emph{com}, la norme de la trajectoire de la vitesse articulaire associée à la t\^ache
\emph{prise gauche} reste significative (figure~\ref{fig:RbeforeAfterProj:2b}). 
Ceci signifie que la t\^ache \emph{com} n'a que peu d'influence sur le mouvement du
bras gauche, et que le mouvement de ce bras est dû à une autre t\^ache.
Par conséquent, l'algorithme de sélection de t\^ache va continuer à chercher la t\^ache
qui a contrôlée le bras gauche.
Après la détection des deux t\^aches principales (pour le \emph{mouvement~2.a}) et des 
trois t\^aches principales (pour le \emph{mouvement~2.b}), 
la norme de la dernière trajectoire de la vitesse articulaire n'est pas nulle parce que
l'algorithme de sélection de t\^ache n'est pas terminé et d'autres t\^aches n'ont pas encore
été sélectionnées.
La t\^ache de \emph{double support} est alors détectée mais les courbes associées 
ne sont pas tracées par souci de clarté, car elles sont quasi-nulles.
D'autre part, la figure~\ref{fig:XP2RLFit} montre l'ajustement de modèle de t\^ache pour le \emph{mouvement~2.b}.
\begin{figure*}[p]
\centering
%\includegraphics[width=0.9\linewidth]{figures/chapitre5/projRHNull.eps}
\subfigure{
  \resizebox{.47\textwidth}{!} {
      \input{figures/chapitre5/simu/2b/taskRhandNormInvJerr_0}
    }      
    }
\subfigure{
  \resizebox{.47\textwidth}{!} {
      \input{figures/chapitre5/simu/2b/taskLhandNormInvJerr_0}
    }
    }
\caption[Ajustement de modèle de t\^ache pour le \emph{mouvement~2.b}.]{L'ajustement de modèle de t\^ache pour le \emph{mouvement~2.b} pour les t\^aches de 
\emph{prise droite} et \emph{prise gauche}. Les deux mouvements de la main gauche et de
la main droite sont correctement ajustés par le modèle, avec un petit résidu: 
les t\^aches sont bien détectées.}
\label{fig:XP2RLFit}
\end{figure*}
\FloatBarrier
\subsubsection{Prise VS Prise orientée}
\label{sec:distinc2}
\begin{table}[t]
  \centering
  \begin{tabular}{|c|c|c|c|}
    \hline
    Référence & Détectées & $\int \Vert \dot{q}(t) \Vert ^2 dt$ & $\int \Vert P\dot{q}(t) \Vert ^2 dt$ \\
    \hline
    \begin{tabular}{c}
      Com\\
      Regard\\
      Prise\\
      Double support\\
    \end{tabular}
    &
    \begin{tabular}{c}
      Com\\
      Regard\\
      Prise\\
      Double support\\
    \end{tabular}
    & 0.619266 & 0.00245631 \\
    \hline
    \begin{tabular}{c}
      Com\\
      Regard\\
      Prise orientée\\
      Double support\\
    \end{tabular}
    &
    \begin{tabular}{c}
      Com\\
      Regard\\
      Prise\\
      Prise orientée\\
      Double support\\
    \end{tabular}
    & 0.717041 & 0.00344557\\
    \hline
  \end{tabular}
  \caption[Résultats de l'algorithme pour l'analyse du \emph{mouvement~3.a} et du \emph{mouvement~3.b}.]{Résultats de l'algorithme de sélection de t\^ache 
  pour l'analyse du \emph{mouvement~3.a} et du \emph{mouvement~3.b}.}
  \label{tab:spotDiff2}
  %\vspace{-3pt}
\end{table}
\begin{figure*}[t]
\centering
\subfigure{
%\includegraphics[width=0.9\linewidth]{figures/chapitre5/projRHNull.eps}
  \resizebox{.48\textwidth}{!} {
      \input{figures/chapitre5/simu/3a/taskRhandScrewNormInvJerr_0}
      }      }
      \subfigure{
  \resizebox{.48\textwidth}{!} {
      \input{figures/chapitre5/simu/3b/taskRhandScrewNormInvJerr_0}
    }
    }
\caption[L'ajustement de modèle de t\^ache \emph{prise orientée} échoue.]{L'ajustement de modèle de t\^ache \emph{prise orientée}
échoue, car le résidu est plus élevé, pour le \emph{mouvement~3.a} et réussi pour le \emph{mouvement~3.b}.}
\label{fig:spotDiff3:screw}
\end{figure*}
Dans ce paragraphe, le \emph{mouvement~3.a} et le \emph{mouvement~3.b} sont analysés.
Les deux mouvements partagent la même position de but pour la main droite.
La seule différence entre les deux démonstrations est la présence d'une contrainte 
d'orientation sur la main droite dans le \emph{mouvement~3.b}.
Les postures finales de ces mouvements sont illustrées dans la figure~\ref{fig:spotDiff2}.
Le tableau~\ref{tab:spotDiff2} résume les résultats de l'algorithme de sélection de t\^ache
qui s'est déroulé avec succès.
Lors de l'analyse du \emph{mouvement~3.b}, la t\^ache de \emph{prise} a également
été sélectionnée. Ce qui est parfaitement logique car dans ce cas précis,
la t\^ache \emph{prise} est une sous-t\^ache de \emph{prise orientée}: la t\^ache \emph{prise}
est identique à la t\^ache \emph{prise orientée} sans la contrainte d'orientation.
Par contre, lors de l'analyse du \emph{mouvement~3.a}, la t\^ache
\emph{prise orientée} n'est pas sélectionnée.
La figure~\ref{fig:spotDiff3:screw} montre que l'ajustement de la t\^ache
\emph{prise orientée} échoue pour le \emph{mouvement~3.a} mais 
réussi pour le \emph{mouvement~3.b}.
Comme précédemment,
les projections annulent les mouvements résiduels qui pourraient mener à de futures détections erronées.
Le mouvement résiduel après les projections successives sont finalement proches de zéro, ce qui 
prouve que toutes les t\^aches ont été détectées.

\subsubsection{Prise orientée VS Regard}
\label{sec:distinc3}
Les deux mouvements considérés sont le \emph{mouvement~4.a} et le \emph{mouvement~4.b}.
Le \emph{mouvement~4.a} peut \^etre décrit par le scénario suivant: 
un objet se trouve devant le robot, et crée une occlusion dans le champ de vision du robot.
Pour se débarrasser de cette occlusion, le robot se penche sur sa droite.
Lorsque le robot se penche, la main gauche est entrainée par le torse par un effet
secondaire involontaire.
La position et l'orientation de cette main sont enregistrées comme étant l'état
désiré pour la t\^ache de prise orientée dans le \emph{mouvement~4.b}.
Dans le \emph{mouvement~4.b}, le regard n'est pas contr\^olé. Le mouvement de la t\^ete
est cette fois-ci, un effet secondaire du mouvement de la main gauche.
Les postures finales du robot pour ces deux mouvements sont illustrées dans la figure~\ref{fig:spotDiff3},
et les résultats des analyses sont résumés dans le Tableau~\ref{tab:spotDiff3}.
\begin{table}[t]
\centering
\begin{tabular}{|c|c|c|c|}
\hline
Référence & Détectée & $\int \Vert \dot{q}(t) \Vert ^2 dt$ & $\int \Vert P\dot{q}(t) \Vert ^2 dt$ \\
\hline
\begin{tabular}{c}
Com\\
Regard\\
Double support\\
\end{tabular}

&

\begin{tabular}{c}
Com\\
Regard\\
Double support\\
\end{tabular}

& 0.534478 & 0.0545944 \\
\hline
\begin{tabular}{c}
Com\\
Prise orientée\\
Double support\\
\end{tabular}

&
\begin{tabular}{c}
Com\\
Prise orientée\\
Double support\\
\end{tabular}

& 0.558084 & 0.0023297\\
\hline
\end{tabular}
\caption[Résultats de l'algorithme pour l'analyse du \emph{mouvement~4.a} et du \emph{mouvement~4.b}.]{Résultats de l'algorithme de sélection de t\^aches à partir de l'analyse du \emph{mouvement~4.a} 
et du \emph{mouvement~4.b}.}
\label{tab:spotDiff3}
%\vspace{-12pt}
\end{table}

Comme précédemment, les bonnes t\^aches ont été détectées pour chaque mouvement.
Le résidu après les projections de toutes les t\^aches détectées étant très bas, 
toutes les t\^aches ont été correctement détectées et soustraites au mouvement.\\

Pour conclure, nous venons de montrer expérimentalement que, sans bruits, l'algorithme de détection
se comporte parfaitement, \ie~que toute les t\^aches actives dans le mouvement analysé
sont détectées, sans faux positifs, et la suppression des effets de ces t\^aches, via les projections
du mouvement original dans les espaces nuls, conduit à un mouvement quasi nul. En effet, le mouvement
n'est pas totalement annulé à cause du bruit numérique. Dans le paragraphe suivant
nous réalisons des expérimentations dans un cadre réaliste en utilisant 
un vrai robot et de véritables capteurs qui fourniront des signaux bruités,
correspondant aux trajectoires articulaires du robot.

\section{Expérimentations sur le robot}
\label{sec:real}
Dans ce paragraphe, nous démontrons la validité de l'algorithme de
reconnaissance de t\^ache dans un cadre réaliste en analysant des signaux
bruités. Cette fois, le mouvement de référence est joué sur un véritable
robot HRP-2 et est observé en utilisant un système de capture de mouvement (figure~\ref{fig:hrp2Markers}).
Comme dans les expérimentations en simulations, l'algorithme de sélection de t\^aches est appliqué
sur des paires de mouvements visuellements proches.
Dans la suite de ce paragraphe, nous détaillons le processus d'acquisition 
des trajectoires articulaires par le biais du système de capture de mouvements.
Puis, deux paires de mouvements seront analysées par l'algorithme de reconnaissance
de t\^aches.

\subsection{Protocol expérimental}\label{sec:xpset}
Un mouvement est généré par la pile de t\^aches, puis est exécuté sur le robot HRP-2 équipé de marqueurs sur 
chacun de ses corps (voir figure~\ref{fig:hrp2Markers}).
\begin{figure}[t]
  \centering
  \begin{tabular}{cc}
    \includegraphics[height=0.4\linewidth]{figures/chapitre5/hrp2Markers.ps} &
    \includegraphics[height=0.4\linewidth]{figures/chapitre5/skel.ps} \\
  \end{tabular}
  \caption{Ensemble des marqueurs, et squelette virtuel pour le robot HRP-2.}
  \label{fig:hrp2Markers}
\end{figure}
Le système de capture de mouvements utilisé est composé de 10 caméras 
infrarouges, et enregistre les données à une fréquence de 200Hz.
Les données collectées à partir de ces marqueurs sont utilisés pour
construire un squelette virtuel qui est ajusté à la hiérarchie cinématique du
robot (figure~\ref{fig:hrp2Markers}).
Le système de capture de mouvements fournit une trajectoire pour chacun des
corps du robot.
Les données sont obtenues sous forme de trajectoires 6D de chaque corps
du robot dans l'espace, sans aucune contrainte articulaire.

L'analyse du mouvement est effectuée sur la trajectoire articulaire.
Par conséquent, les trajectoires articulaires doivent être calculées à partir des
données issues de la capture de mouvements.
Les trajectoires articulaires sont calculées de manière classique, par minimisation
de la distance entre les matrices de transformations $\mTransfMatrix{W}{R}{r_j}(\mbf{q})$
associées à l'origine des articulations du robot et des matrices de transformations mesurées
$\tensor[^{W}]{\mathbf{\hat{R}}}{_{m_j}}(t)$ par le système de capture de mouvement,
en tenant compte des limites articulaires du robot. 
\begin{eqnarray}
  \mbf{\hat{q}}(t) =  & \underset{\mbf{q}}\argmin & \sum_j^m \Vert \tensor[^{W}]{\mathbf{\hat{R}}}{_{r_j}}(t) \ominus \tensor[^{W}]{\mathbf{R}}{_{r_j}}(\mbf{q}) \Vert ^2\\
    & \text{s.t.} & q_{i\mathrm{min}} \leq q_i \leq q_{i\mathrm{max}}, \; i = 1..n
  \label{mocapOpti}
\end{eqnarray}
où $m$ est le nombre de corps, $n$ le nombre
d'articulations, $\mbf{q}$ est le vecteur de configuration du robot,
$\ominus$ est l'opérateur de distance dans SO(3), 
$\tensor[^{W}]{\mathbf{R}}{_{r_j}}(\mbf{q})$ est calculée en utilisant le modèle cinématique du robot, 
et $\tensor[^{W}]{\mathbf{\hat{R}}}{_{m_i}}$ est obtenue par:
\begin{equation}
  \tensor[^{W}]{\mathbf{\hat{R}}}{_{r_j}}(t) = \tensor[^{W}]{\mathbf{R}}{_{W_{C}}} \times \tensor[^{W_{C}}]{\mathbf{\hat{R}}}{_{m_j}}(t) \times \tensor[^{m_j}]{\mathbf{R}}{_{r_j}}    
  \label{mocapMatrix}
\end{equation}
où $\tensor[^{W}]{\mathbf{R}}{_{W_{C}}}$ est 
la matrice de transformation entre le repère du monde
et le repère de la zone d'expérimentation de la capture de mouvements.
$\mTransfMatrix{m_j}{R}{r_j}$ sont des 
transformations constantes dû à la différence entre un modèle de squelette quelconque de la capture de
mouvement et du modèle cinématique du robot utilisé lors de la phase de calibration.
Cette phase de calibration consiste à calculer les transformations $\tensor[^{W}]{\mathbf{R}}{_{W_{C}}}$
et $\tensor[^{m_j}]{\mathbf{R}}{_{r_j}}$ à partir d'une posture connue.
$\tensor[^{W_{C}}]{\mathbf{\hat{R}}}{_{m_j}}$
est la matrice de transformation mesurée associée à l'origine 
du corps virtuel $i$ dans le repère de référence du système de capture de mouvements.

Les trajectoires articulaires obtenues sont utilisées pour effectuer la reconnaissance de t\^aches
par retro-ingénierie.

Sur le véritable robot, une flexibilité est introduite après l'articulation de la cheville.
Elle interfère avec le mouvement au tout début car l'accélération angulaire
au niveau des articulations est importante. La flexibilité n'est pas modélisée,
ni dans l'algorithme de contr\^ole, ni dans la méthode de détection.
Pour ne pas \^etre influencé par la flexibilité, les 100 premières millisecondes
des mouvements analysés sont ignorées. C'est dans cette phase que les effets de la
flexibilité sont les plus importants dûs à l'accélération initiale typique d'une t\^ache
de régulation proportionnelle en cinématique inverse.
Dans les deux prochains paragraphes, nous considérons que les mesures sont directement
les trajectoires $\mbf{\hat{q}}$ données par~\eqref{mocapOpti}.

\subsection{Prise VS Maintien de l'équilibre}
Cette expérience correspond à celle effectuée en simulation dans le paragraphe~\ref{sec:distinc1}:
le premier mouvement est un mouvement de la main gauche produit par le couplage entre
la prise lointaine de la main droite, et la t\^ache \emph{com} (\emph{mouvement~2.a}). 
Le second mouvement correspondant à une double prise, a été construit pour 
présenter une ambiguïté avec le premier mouvement~(\emph{mouvement~2.b} dans le tableau~\ref{tab:motion}).
Le tableau~\ref{tab:spotDiffReal1} présente les résultats de l'algorithme de reconnaissance
pour le \emph{mouvement~2.a} et le \emph{mouvement~2.b} avec le critère d'arrêt égal à $ \epsilon = 0.07$.
\begin{table}[t]
  \centering
  \begin{tabular}{|c|c|c|c|}
    \hline
    Référence & Détectée & $\int \Vert \dot{q}(t) \Vert ^2 dt$ & $\int \Vert P\dot{q}(t) \Vert ^2 dt$ \\
    \hline
    \begin{tabular}{c}
      Com\\
      Prise droite\\
      Double support\\
    \end{tabular}

    &

    \begin{tabular}{c}
      Com\\
      Prise droite\\
      Double support\\
    \end{tabular}

    & 0.104835 & 0.0482885 \\
    \hline
    \begin{tabular}{c}
      Com\\
      Prise gauche\\
      Prise droite\\
      Double support\\
    \end{tabular}

    &
    \begin{tabular}{c}
      Com\\
      Prise gauche\\
      Prise droite\\
      Double support\\
    \end{tabular}

    & 0.142293  & 0.0541836 \\
    \hline
  \end{tabular}
  \caption[Résultats de l'algorithme sur le vrai robot.]{Résultats de l'algorithme de sélection de t\^aches pour l'analyse du \emph{mouvement~2.a} et du \emph{mouvement~2.b} joués sur le vrai robot.}
  \label{tab:spotDiffReal1}
\end{table}

En ce qui concerne le \emph{mouvement~2.a}, l'ordre d'extraction des t\^aches est:
\emph{prise droite}, \emph{double support} et \emph{com}.
Tandis que pour le \emph{mouvement~2.b},
les t\^aches extraites sont: \emph{prise droite},  \emph{com}, \emph{double support} et \emph{prise gauche}.

Les figure~\ref{fig:exp1:headFit} et \ref{fig:exp1:taskLhand} montrent l'ajustement
du modèle de t\^ache à la première itération de l'algorithme pour les t\^aches \emph{tête} et \emph{prise gauche}. 
Comme prévu, l'ajustement est correct seulement pour la t\^ache 
\emph{prise gauche} dans le \emph{mouvement~2.b} tandis que cette t\^ache est rejetée dans
le \emph{mouvement~2.a} à cause de son ajustement incorrect.
\begin{figure*}[t]
  \centering
  \subfigure[Mouvement 2.a]{
  \resizebox{.48\textwidth}{!} {
  \input{figures/chapitre5/realRobot/2a/taskHeadNormInvJerr_0.tex}
  }
  \label{fig:exp1:headFit:R}
  }
  \subfigure[Mouvement 2.b]{
  \resizebox{.48\textwidth}{!} {
  \input{figures/chapitre5/realRobot/2b/taskHeadNormInvJerr_0.tex}
  }
  \label{fig:exp1:headFit:RL}
  }
  \caption[T\^ache non sélectionnée.]{Ajustement des modèles de t\^ache à la première itération de l'algorithme de sélection de
  t\^aches pour les t\^aches \emph{tête} dans le \emph{mouvement~2.a} et \emph{mouvement~2.b} 
  Pour les deux mouvements, $r$ est élevé: la t\^ache \emph{tête}, qui n'est pas active, n'est
  pas sélectionnée.}
  \label{fig:exp1:headFit}
\end{figure*}
\begin{figure*}[t]
  \centering
  \subfigure[Mouvement 2.a]{
  \resizebox{.48\textwidth}{!} {
  \input{figures/chapitre5/realRobot/2a/taskLhandNormInvJerr_0.tex}
  }                           
  \label{fig:exp1:taskLhand:R}
  }
  \subfigure[Mouvement 2.b]{
  \resizebox{.48\textwidth}{!} {
  \input{figures/chapitre5/realRobot/2b/taskLhandNormInvJerr_0.tex}
  }
  \label{fig:exp1:taskLhand:RL}
  }
  \caption[La t\^ache \emph{prise gauche} n'est pas pertinente dans le \emph{mouvement~2.a}.]{Ajustement des modèles de t\^ache à la première itération de l'algorithme de sélection de t\^aches
  pour la t\^ache \emph{prise gauche} dans le \emph{mouvement~2.a} et \emph{mouvement~2.b} 
  La t\^ache \emph{prise gauche} n'est pas pertinente dans le \emph{mouvement~2.a}, 
  mais est sélectionnée dans le \emph{mouvement~2.b}.}
  \label{fig:exp1:taskLhand}
\end{figure*}

Bien que la t\^ache \emph{tête} ne soit impliquée dans aucun des mouvements,
la tête n'est pas fixe dans l'espace à cause du mouvement 
du torse et des jambes causés par les t\^aches actives.
Dans le \emph{mouvement~2.b}, la t\^ache candidate \emph{t\^ete} impliquerait moins de mouvement 
que dans le \emph{mouvement~2.a}. La raison est que lorsque les deux 
mains opposées du robot sont contraintes, le torse n'a pas beaucoup de liberté. 
Cependant, la t\^ache \emph{tête} n'est pas gardée pour aucun des deux mouvements (le résidu $r$ est élevé).

L'évolution du mouvement projeté dans les espaces de la t\^ache 
\emph{prise gauche} et \emph{prise droite}
après les projections successives dans les espaces nuls
automatiquement sélectionnées est illustrée dans la 
figure~\ref{fig:exp1:Evolution2L}. Pour le \emph{mouvement~2.a},
le mouvement du bras gauche est annulé après avoir annulé la t\^ache
\emph{com}.
\begin{figure*}[p]
  \centering
  \subfigure[Mouvement 2.a]{
  \resizebox{.47\textwidth}{!} {
  \input{figures/chapitre5/realRobot/2a/RevolutionLH.tex}
  }                           
  \label{fig:exp1:Evolution2L:a}
  }
  \subfigure[Mouvement 2.b]{
  \resizebox{.47\textwidth}{!} {
  \input{figures/chapitre5/realRobot/2b/RevolutionLH.tex}
  }
  \label{fig:exp1:Evolution2L:b}
  }
  \caption[Evolution du mouvement après annulation de t\^aches.]{Norme du mouvement associés aux t\^aches \emph{prise gauche} (\ie mouvement reconstruit de
  la main gauche).
  Le mouvement est nul après avoir retiré la t\^ache \emph{com} dans le 
  \emph{mouvement~2.a}. Dans ce cas, cela signifie que le mouvement de la main 
  gauche provenait de la t\^ache \emph{com}.
  Tandis que dans le \emph{mouvement~2.b}, le mouvement de la main gauche
  n'est pas annulé après avoir retiré la t\^ache 
  \emph{com}. Le mouvement devient nul uniquement après avoir retiré la t\^ache
  \emph{prise gauche}: le mouvement de la main gauche provenait de la t\^ache
  \emph{prise gauche}.}
  \label{fig:exp1:Evolution2L}
\end{figure*}
Dans le \emph{mouvement~2.b}, le mouvement de la main gauche n'est pas annulé par 
la projection dans l'espace nul de la t\^ache 
\emph{com}, mais devient nul après projection dans l'espace nul de la t\^ache
\emph{prise gauche}.
La figure~\ref{fig:exp1:Evolution2R} montre comment évolue le mouvement reconstruit de la main
droite lorsque le mouvement dû à une t\^ache est enlevé. On peut voir
que le bras droit bouge à cause des t\^aches \emph{prise droite} et
\emph{com}.
\begin{figure*}[p]
  \centering
  \subfigure[Mouvement 2.a]{
  \resizebox{.47\textwidth}{!} {
  \input{figures/chapitre5/realRobot/2a/RevolutionRH.tex}
  }                           
  \label{fig:exp1:Evolution2R:a}
  }
  \subfigure[Mouvement 2.b]{
  \resizebox{.47\textwidth}{!} {
  \input{figures/chapitre5/realRobot/2b/RevolutionRH.tex}
  }
  \label{fig:exp1:Evolution2R:b}
  }
  \caption[Annulations succéssives de t\^aches.]{Mouvement reconstruit de la main droite après les projections successives.
  Le mouvement de la main droite est une conséquence des t\^aches
  \emph{prise droite} et \emph{com}, comme c'est expliqué par le fait 
  que le mouvement de la main droite est annulé après la suppression
  des t\^aches \emph{prise droite} et \emph{com}.}
  \label{fig:exp1:Evolution2R}
\end{figure*}
La figure~\ref{fig:exp1:PqdotNorms} montre l'évolution de la quantité de mouvement
après chaque sélection de t\^ache dans les deux mouvements.
La quantité décroit au fur et à mesure des projections, jusqu'à ce que le mouvement
restant soit principalement du bruit issu des capteurs.
\begin{figure*}[p]
  \centering
  \subfigure[Mouvement 2.a]{
  \resizebox{.47\textwidth}{!} {
    \input{figures/chapitre5/realRobot/2a/PqdotNormsR}
  }
  \label{fig:exp1:PqdotNormsR}
  }
  \subfigure[Mouvement 2.b]{
  \resizebox{.47\textwidth}{!} {
    \input{figures/chapitre5/realRobot/2b/PqdotNormsRL}
  }
\label{fig:exp1:PqdotNormsRL}
}
\caption[Évolution de la norme du mouvement après annulations successives.]{Evolution de la norme du mouvement après projections successives
du mouvement dans les espaces nuls des t\^aches pour le 
\emph{mouvement~2.a} et \emph{mouvement~2.b}.}
\label{fig:exp1:PqdotNorms}
\end{figure*}

\subsection{Regard VS torse}
Les mouvements considérés sont les 
\emph{mouvement~5.a} et \emph{mouvement~5.b}.
Dans le premier cas, le robot regarde sa main droite tout en effectuant une prise avec
celle-ci.
Dans le second cas, le robot effectue une prise identique, mais doit maintenir
son torse dans la posture initiale. La figure~\ref{fig:motion5}
montre la posture finale du robot pour ces deux mouvements.
\begin{figure}[t]
  \centering
  \begin{tabular}{cc}
    \includegraphics[width=0.40\linewidth]{figures/chapitre5/realRobot/5a/5aFinal1.ps} &
    \includegraphics[width=0.40\linewidth]{figures/chapitre5/realRobot/5b/5bFinal1.ps} \\
  \end{tabular}
  \caption{Posture finale pour le \emph{mouvement~5.a} et le \emph{mouvement~5.b}.}
  \label{fig:motion5}
\end{figure}
%%%%%%%%%%%%%%%%%%%%%%%%%%%%%%%%%%%%%%%5
Le tableau~\ref{tab:result5} résume les résultats de l'algorithme de sélection de t\^aches.
Malgré le bruit, les t\^aches sont correctement détectées. Les deux mouvements sont
différenciés correctement.
Enfin, le résidu très faible après l'annulation de toutes les t\^aches
indique qu'il n'y a pas d'autres t\^aches à reconnaitre. 
\begin{table}[t]
  \centering
  \begin{tabular}{|c|c|c|c|}
    \hline
    Référence & Détectée & $\int \Vert \dot{q}(t) \Vert ^2 dt$ & $\int \Vert P\dot{q}(t) \Vert ^2 dt$ \\
    \hline
    \begin{tabular}{c}
      Com\\
      Regard\\
      Prise droite\\
      Double support\\
    \end{tabular}

    &

    \begin{tabular}{c}
      Com\\
      Regard\\
      Prise droite\\
      Double support\\
    \end{tabular}

    & 0.320001 & 0.0785168 \\
    \hline
    \begin{tabular}{c}
      Torse\\
      Com\\
      Prise droite\\
      Double support\\
    \end{tabular}

    &
    \begin{tabular}{c}
      Torse\\
      Com\\
      Prise droite\\
      Double support\\
    \end{tabular}

    & 0.216742 & 0.0516134 \\
    \hline
  \end{tabular}
  \caption{Résultats de l'algorithme de sélection de t\^aches à partir de l'analyse des \emph{mouvement~5.a} 
  et \emph{mouvement~5.b} joués sur le vrai robot.}
  \label{tab:result5}
\end{table}
La figure~\ref{fig:exp6:PqdotNorms5} montre l'évolution de la norme
du mouvement après les projections dans les espaces nuls des t\^aches: 
\emph{Prise droite},  \emph{regard}, \emph{double support} and \emph{com} pour
le \emph{mouvement~5.a}, et les t\^aches \emph{prise droite}, \emph{double support}, \emph{com}
et \emph{torse} pour le \emph{mouvement~5.b}.
\begin{figure*}[t]
  \centering
  \subfigure[Mouvement 5.a]{
  \resizebox{.46\textwidth}{!} {
    \input{figures/chapitre5/realRobot/5a/PqdotNorms5a}
  }
  }
  \subfigure[Mouvement 5.b]{
  \resizebox{.46\textwidth}{!} {
    \input{figures/chapitre5/realRobot/5b/PqdotNorms5b}
  }
}
\caption[Évolution de la norme du mouvement.]{Évolution de la norme du mouvement après avoir projeté avec succès
le mouvement dans les espaces nuls des t\^aches pour les mouvements
\emph{mouvement~5.a} et \emph{mouvement~5.b}.}
\label{fig:exp6:PqdotNorms5}
\end{figure*}
Comme dans les précédentes expérimentations, la projection fait strictement décroitre
la quantité de mouvement à chaque itération de l'algorithme. La quantité de mouvement restant 
à la fin de l'algorithme est clairement dû au bruit.
\FloatBarrier

 %-------------------------------------------------------------------%
%-------------------------------------------------------------------%
\section{Exp\'erimentation sur l'humain}
\label{chap:xpHumain}
%-------------------------------------------------------------------%
%-------------------------------------------------------------------%
La méthode de reconnaissance de t\^aches présentée dans ce chapitre donne
de très bon résultats lorsque l'on connaît à la fois le modèle
des t\^aches mis en \oe uvre, et le modèle géométrique 
du robot pour calculer les différentes opération de projections. 
La reconnaissance de t\^aches pour le robot humanoïde se basait sur la connaissance
parfaite du comportement du robot lors de l'exécution d'une t\^ache.
Nous avions appliqué comme exemple de comportement, une décroissance exponentielle.
On souhaite maintenant étudier la possibilité
d'appliquer la méthode de reconnaisssance sur des mouvements humains en
adaptant les modèles de comportement de t\^ache et en utilisant 
des modèles de génération de mouvement humains.
En effet, la méthode n'est pas 
dépendante du comportement des t\^aches à retrouver.
Deux points sont donc à vérifier, correspondant
à la sélection de t\^aches d'une part (paragraphe~\ref{sec:alg2:proj}),
et d'autre part l'annulation du mouvement provoqué par l'exécution
d'une t\^ache (paragraphe~\ref{sec:alg2:nullification}).
Dans le cadre de cette thèse, seul le premier point 
a été validé en se plaçant dans le cas particulier du 
mouvement d'atteinte, dont le modèle de comportement chez l'humain 
est bien connu. 
Il s'agit du modèle du minimum jerk qui sera utilisé pour caractériser 
la trajectoire de l'effecteur atteignant sa cible.
La généralisation à d'autres types de t\^aches
ainsi que le second point
sont laissés en perspectives.

\subsection{Comportement d'une t\^ache d'atteinte par un humain}
Nous nous focalisons sur  un type de t\^ache particulier:
la t\^ache d'atteinte.
En effet, des études ont mis en évidence le profil  
de trajectoires humaines pour les mouvement d'atteinte.
Celles-ci respectent un profil de type \emph{minimum jerk}~\cite{flash85}.
\label{subsection:modelJerk}
Cette connaissance de profil de trajectoire correspond
bien à l'hypothèse requise pour l'application de notre méthode 
de reconnaissance, qui est la connaissance d'un modèle de comportement
lors de l'exécution d'une t\^ache.
La trajectoire de la partie du corps d'un humain 
effectuant ce mouvement d'atteinte suit un profil du type minimum jerk
(de la même manière que dans la partie précédente, la trajectoire du robot
correspondait à une décroissance exponentielle).
La trajectoire en position reliant deux points $A$ et $B$ est dite
de jerk minimum si elle correspond à l'optimum du problème:
\begin{eqnarray}
  \min \iiint_0^{\infty} \Vert \dddot{f}(t) \Vert dt \\
  f(0) = A\\
  f(\infty) = B
  \label{eq:minJerkTraj}
\end{eqnarray}

\subsection{Modélisation du minimum jerk}
Plutôt que de raisonner sur la forme implicite du problème~\eqref{eq:minJerkTraj},
il est souvent choisi dans la littérature de représenter les trajectoires 
en minimum jerk comme une classe spécique de polynômes du troisième ordre, 
dont on sait que le jerk est faible.
Nous choisissons de modéliser le jerk non pas comme un polynôme de degré 3,
mais comme l'intégrale triple d'un signal à quatre créneaux correspondant respectivement
aux phases d'accélération linéaires, de décélération, d'accélération et
enfin d'accélération nulle. Ainsi notre modèle comportera 6 paramètres:
les trois valeurs crêtes, et les trois instants relatifs de commutations.
Ce choix est motivé par le fait que la méthode de reconnaissance utilise
une méthode d'optimisation numérique pour la phase d'ajustement
entre les trajectoires reconstruites (par observation) et le modèle de comportement.
Utiliser les coefficients d'un polynôme comme paramètre 
d'optimisations ne permet pas de restreindre l'espace de recherche
aux courbes de type minimum jerk, et par conséquent,
modéliser une t\^ache par un polynôme n'est pas efficace.
Au contraire, la modélisation en signal créneaux
permet d'assurer la minimalité de la fonction de jerk
sans passer par des contraintes artificielles sur les 
coefficients du ploynôme représentant la trajectoire.
La trajectoire en position est
alors obtenue par triple intégration du jerk.

Le jerk est alors défini ainsi:
\begin{equation}
  \mathrm{jerk}(t) =
  \left\{
      \begin{array}{rll}
        K_1& if& 0<t< \Delta t_1 \\
        K_2& if& t_1<t< \Delta t_1 + \Delta t_2\\
        K_3& if& t_2<t< \Delta t_1 + \Delta t_2 + \Delta t_3\\
	0& if& t> \Delta t_1 + \Delta t_2 + \Delta t_3
      \end{array}
    \right.
\label{jerk}
\end{equation}
La figure~\ref{fig:jerk} illustre notre modélisation par un exemple de minimum jerk, et 
sa trajectoire en position correspondante.
\begin{figure}[t]
\begin{center}
%\includegraphics[height=0.9\linewidth, angle = -90]{img/jerkTrajectory.ps}
    \resizebox{0.4\textwidth}{!} {
      \input{figures/chapitre5/jerkTrajectory.pstex_t}
    }
\end{center}
\caption{Un exemple de minimum jerk et sa trajectoire en position correspondante.}
\label{fig:jerk}
\end{figure}

Les mouvements considérés étant des trajectoires
d'atteinte, le mouvement de la main se termine
avec une position constante de celle-ci.
Donc à $t = \Delta t_1 + \Delta t_2 + \Delta t_3$ la vitesse et l'accélération
sont nulles. Ce qui amène à un système de deux équations
avec comme inconnues $\Delta t_3$ et $K_3$, ce qui contraint la durée et le
comportement de la phase finale pour atteindre une vitesse et accélération nulle. 
%The problem is to make this trajectory continuous, so
%that the optimization performance is not poor.
%This is done by constraining the final relative time,
%and the final jerk value~: the velocity and acceleration
%at $t=t_1 + t_ 2+ t_3$ is null. It leads to a system of 2 equations
%which will be solved over $K_3$ and $t_3$.\\
Le problème d'optimisation pour l'ajustement de courbe devient:
\begin{equation}
\begin{array}{ll}
  & \underset{\mbf{x}}\min \int \Vert \mbf{\hat{p}}(t) - \mbf{p}_\mbf{x}(t) \Vert ^2 + K_1^2 + K_2^2 \mathrm{dt}\\
  \text{s.t.} & \mbf{p}_\mbf{x}(t) = \iiint \mathrm{jerk}(t) \mathrm{dt}
\end{array}
\label{minJerkOpti}
\end{equation}
\noindent où $\mbf{x}=[dt_1\ dt_2\ K_1\ K_2]$, $\mbf{\hat{p}}$ est la trajectoire mesurée.
La seconde partie de la fonction de coût $K_1^2 + K_2^2$ est introduite
pour limiter les paramètres $K_1$ et $K_2$,
afin que les trajectoires obtenues par intégration du jerk soient 
d'un ordre de grandeur raisonnable. Comme dans la section~\ref{sec:alg2:proj}, on utilise le solveur 
FSQP avec une forme discrétisée de~\eqref{minJerkOpti}, la distance entre $\mbf{p}_\mbf{x}$ et $\mbf{\hat{p}}$.

\subsection{Validation du modèle choisi}
Dans cette expérimentation, nous validons 
l'algorithme présenté dans le paragraphe~\ref{sec:alg2:proj}
pour la détection de trajectoire d'atteinte chez l'humain.
Un humain est équipé d'un ensemble de 23 marqueurs qui 
vont permettre au système de capture de mouvement
d'enregistrer des démonstrations.
Le sujet doit atteindre le haut d'une bouteille en y posant son doigt, 
tout en gardant un pied au sol et en gardant la position de la main secondaire
constante. La position finale de la main
est donc clairement définie et constante, et son orientation
n'est pas contrainte. La bouteille
est suffisamment éloignée du sujet pour l'obliger à lever un
pied pour atteindre la bouteille. Lors de cette expérimentation, la trajectoire de la main
qui atteint la bouteille est étudiée.
La figure~\ref{fig:humanXP} illustre la position finale du mouvement.

\begin{figure}[t]
\begin{center}
\includegraphics[width=0.6\linewidth]{figures/chapitre5/imgHuman.ps}
\end{center}
\caption{Position finale du mouvement d'atteinte humain.}
\label{fig:humanXP}
\end{figure}
La trajectoire de la main dominante obtenue par capture de mouvement est donnée
à un programme d'optimisation pour réaliser l'ajustement~\eqref{minJerkOpti} de la trajectoire observée
sur notre modèle théorique~\eqref{jerk}.
La figure~\ref{fig:minJerkFitting1}, montre que l'optimisation arrive 
à faire correspondre la trajectoire observée au modèle proposé.
\begin{figure}[t]
\begin{center}
%\includegraphics[height=0.9\linewidth, angle=-90]{img/FittingMinJerk.ps}
\resizebox{.4\textwidth}{!} {
      \input{figures/chapitre5/FittingMinJerk.pstex_t}
    }
\end{center}
\caption{Ajustement du modèle de minimum jerk sur une trajectoire d'atteinte réelle.}
\label{fig:minJerkFitting1}
\end{figure}

%\subsection{Conclusion et futures expérimentations}

\subsection{Analogie avec la pile de t\^aches}
L'hypothèse des \emph{variétés non contrôlées}
définit une séparation entre les espaces contrôlés et les espaces non-contrôlés~\cite{scholz99}.
Ainsi une grande variabilité dans l'espace non-contrôlé (\emph{UCM}) ne 
va pas dégrader les performances d'une t\^ache,
tandis qu'une grande variabilité dans son sous-espace orthogonal va
dégrader ces performances.

Dans le cas particulier des t\^aches d'atteintes considérées
dans les différentes expérimentations menées dans~\cite{jacquierbret09},
la présence de contraintes géométriques matérialisées 
par un obstacle entraîne une grande variabilité au niveau
de l'espace non-contrôlé du coude: pour éviter l'obstacle,
un humain va modifier la trajectoire du coude.
Une approximation de cet espace non-contrôlé est réalisée en utilisant le noyau 
de la jacobienne. Ainsi, cet espace 
représente l'espace nul de la t\^ache d'atteinte.
Les expérimentations menées dans ces travaux
montrent qu'il
est cohérent d'utiliser les mécanismes liés à la redondance,
bien connus dans le cas robotique, pour expliquer
la gestion de la redondance dans les mouvements humains.

 \section*{Conclusion}
Ce chapitre a présenté une méthode pour identifier quelles t\^aches sont
impliquées dans un mouvement observé, sans utiliser d'informations contextuelles:
seule la trajectoire observée est analysée.
L'analyse est conduite par la connaissance du comportement
d'un robot lors de l'exécution d'une t\^ache (par exemple, une décroissance
exponentielle).
Le mouvement analysé est supposé \^etre généré par un ensemble
de contr\^oleurs appartenant à un lot de t\^aches connu.
Le problème de la reconnaissance de t\^aches est traité en procédant par une rétro-ingénierie
du mouvement.
La trajectoire observée est analysée dans chaque espace de t\^aches connu pour décider
quelles t\^aches sont actives en comparant ces trajectoires aux comportements théoriques attendu.
La méthode est généralisable sur les comportements d'une t\^ache, et par conséquent,
n'importe quelle loi de commande utilisée pour générer un mouvement 
peut \^etre utilisée dans cette méthode pour caractériser une t\^ache.

La méthode a été appliquée avec succès dans différent scénarios pour discriminer des mouvements
visuellement proches, en simulation (paragraphe~\ref{sec:simu}) et sur un véritable 
robot HRP-2 (paragraphe~\ref{sec:real}).
Ces mouvements ont été construits spécialement pour présenter des ambiguïtés, dans le but
d'illustrer l'efficacité de la méthode proposée.\\

Dans toutes ces expériences, une hypothèse forte considérée était que le mouvement
analysé ne doit pas comporter de t\^aches qui changent d'état (active, non-active).
Une solution intéressante à explorer serait d'ajouter une dimension temporelle
à la reconnaissance de t\^ache: il s'agirait de déterminer la durée de validité
d'une pile de t\^aches, ainsi un mouvement serait segmenté temporellement
en succession d'instances de piles de t\^aches comme montré dans la Fig.~\ref{fig:trajSegmentation}.
La segmentation temporelle s'effectuera en
introduisant les temps de début et de fin de t\^aches dans les paramètres d'optimisation.

\begin{figure}[t]
\begin{center}
\includegraphics[width=0.6\linewidth]{figures/chapitre5/trajSegmentation2.ps}
\end{center}
\caption{Segmentation temporelle d'une trajectoire en séquence de piles de t\^aches.}
\label{fig:trajSegmentation}
\end{figure}

Dans l'expérience de le paragraphe~\ref{chap:xpHumain},
un modèle de t\^ache adapté aux mouvements humain pour des t\^aches d'atteintes est vérifié.
Par extension, il serait ainsi possible d'appliquer
notre méthode de reconnaissance de t\^aches pour ce type de mouvement.
Le problème étant que les mouvements humains sont beaucoup plus complexes à décrire
que les mouvements des robots. 
Les étapes futures s'attacheront à trouver dans la littérature du domaine
des modèles pour d'autres t\^aches que le mouvement d'atteinte. D'autre part,
pour les mouvements d'atteinte, nous chercherons à valider le modèle du minimum jerk
comme étant un critère discriminant (c'est-à-dire dans le cas du scénario 2.a,
si la main gauche ne suit pas une telle loi).
Une solution pour généraliser à des modèles de t\^aches divers serait 
de s'appuyer sur des méthodes d'apprentissage. On pourrait ainsi extraire
un modèle à partir d'une série de mouvement, par exemple en s'appuyant sur
des processus gaussien. L'utilisation de ces techniques ont montrées
des résultats prometteurs dans le domaine des mouvements humains~\cite{wang08a}.
Les travaux présentés dans~\cite{alvarez09} illustrent une approche hybride
de modélisation de mouvement par apprentissage et modèle physique:
les données de haute dimension représentant des mouvements humains
sont résumé par des variables latentes ayant une interprétation physique.
Le modèle serait ensuite re-utilisé en projetant le mouvement
observé dans l'espace de la variable apprise. La distance 
au modèle appris serait alors utilisée comme une mesure
d'activation de la t\^ache.

Dans notre méthode de reconnaissance,
l'étape des annulations des t\^aches par projections dans les espaces nuls
est primordiale. Outre le fait d'étudier la
validité du modèle de minimum jerk pour caractériser l'exécution
ou la non-exécution d'une t\^ache, une des voies à explorer est d'étudier si 
un modèle cinématique approché d'un humain, pour calculer les projections, peut être suffisant
pour pouvoir déterminer si toute les t\^aches ont été détectées.
L'utilisation d'un modèle plus réaliste et complet,
tel un modèle de génération de mouvement dynamique~\cite{saab11},
pourra être étudié. En effet, en considérant un modèle dynamique,
les inerties des corps seront prises en compte dans le mécanisme de projection.
Enfin, la validité de la pile de t\^aches
sera étudier pour des mouvements humain, c'est-à-dire caractériser si
les mécanismes de fusion d'objectifs multiples (ou comment est gérée la redondance) 
chez l'humain peut \^etre représenter sous une forme de pile de t\^aches.



 %%\section*{Introduction}
%\label{sec:intro}
%-------------------------------------------------------------------%
%-------------------------------------------------------------------%
\chapter{Reconnaissance de t\^ache dynamiques}
\label{chap:6:sec:intro}
%-------------------------------------------------------------------%
%-------------------------------------------------------------------%

Le mecanisme utilis\'e en cin\'ematique est appliqu\'e en
dynamique.

 %%-------------------------------------------------------------------%
%-------------------------------------------------------------------%
\section{Pile de t\^aches dynamique}
\label{chap:basics:sec:sotDyn}
%-------------------------------------------------------------------%
%-------------------------------------------------------------------%

Cette section introduit la pile de t\^aches dynamique.

 %%-------------------------------------------------------------------%
%-------------------------------------------------------------------%
\section{Imitation sans reconnaissance en dynamique}
\label{chap:basics:sec:imitDyn}
%-------------------------------------------------------------------%
%-------------------------------------------------------------------%

Cette section pr\'esente un exemple d'application
de la pile de t\^aches dynamique dans le cadre
d'une imitation de mouvement sans reconnaissance.

 %%-------------------------------------------------------------------%
%-------------------------------------------------------------------%
\section{Reconnaissance de t\^aches dynamique}
\label{chap:basics:sec:recoDyn}
%-------------------------------------------------------------------%
%-------------------------------------------------------------------%

Cette section pr\'esente les r\'esultats obtenus
pour la reconnaissance de t\^aches dynamique.

 %\section*{Conclusion}

Nous avons \'etendu la m\'ethode de reconnaissance \`a la dynamique.


% %%%%%%%%%%%%%%%%%%%%%%%%%%%%%%%%%%%%%%%%%%%%%%%%%%%%%%%%%%%%%%%%%%%%%%%%%%%%
\addcontentsline{toc}{chapter}{\protect\numberline{}Conclusion et perspectives}
 \chapter*{Conclusion et perspectives}
 Les travaux présentés dans cette thèse ont portés
sur la reconnaissance de tâches qu'effectuent un système anthropomorphe.
Nous avons d'abord présenté les fondements sur lesquels reposent nos
travaux: le formalisme de la fonction de t\^aches.
Ce formalisme permet d'exprimer intuitivement des commandes pour un 
robot en décrivant l'objectif dans un espace approprié. Nous avons également donné
des exemples typiques de t\^aches robotiques exprimées
dans ce formalisme. La génération de mouvements est effectuée en construisant une pile
de t\^aches rangées par ordre de priorités.
Les t\^aches 
de haute priorité ne sont pas perturbées par
des t\^aches de priorités inférieures.

Les techniques utilisées pour acquérir des mouvements ont été présentées.
Ces techniques sont applicables aussi bien sur un robot que sur un humain.
Aussi nous avons donné un exemple d'une méthode générique pour reproduire 
un comportement humain sur un robot en utilisant les mécanismes de contrôle
de robot présentés précedemment.

Enfin, nous avons montré qu'en se plaçant directement dans l'espace dans lequel 
une tâche est exprimée, il est possible d'utiliser le modèle
de génération de mouvement comme un modèle de prédiction en résolvant
un problème d'optimisation. Le résidu du problème d'optimisation
produit une mesure d'adéquation entre un mouvement observé et
une loi de génération de mouvements. Ainsi
en supposant que les modèles de générations de mouvements et que les tâches
pouvant intervenir dans un mouvement sont connus et 
qu'aucune tâche ne change d'état (active, inactive) durant le mouvement observé, il est possible 
de reconstruire la pile de t\^aches ayant généré ce mouvement.

%une sémantique aux mouvements consid\'er\'es.
%Cependant, on ne s'est limit\'e qu'\`a faire de la reconnaissance
%dans des segments temporels dans lesquels les
%t\^aches mises en jeu restent dans le m\^eme \'etat.
%Par la suite, la gestion de s\'equence de t\^aches pourrait \^etre
%\'etudi\'e.
%Les travaux pr\'eliminaires sur la reconnaissance de t\^aches humaines
%pr\'esentent des r\'esultats prometteurs.

\section*{Contributions}
\subsection*{Capture de mouvements}
Dans le cadre de l'acquisition de mouvements, il était nécessaire d'observer, de manière externe au robot,
les trajectoires articulaires du robot à partir d'un système de capture de mouvements.
Ainsi, un programme de recallage de modèle par optimisation
a été développé. Le mouvement du squelette défini par la configuration géométrique
des points suivis par la capture de mouvements est transféré au modèle
du robot HRP-2.
Ce programme a aussi été utilisé 
dans les travaux d'éditions de mouvements dynamiques de l'équipe dans lesquels 
les mouvements sont montrés par un humain. Les mouvements obtenus
sont transposés au robot pour donner une première trajectoire articulaire.
Cette trajectoire articulaire est ensuite modifiée en utilisant
des t\^aches afin de corriger les erreurs introduites par le changement de modèle et 
d'adapter la dynamique du mouvement aux contraintes physiques du robot.

\subsection*{Génération de mouvements d'attente}
Pour la génération de mouvements d'attente, un module logiciel a été
créé pour assigner les lots de trajectoires mesurées par capture de mouvements
aux trajectoires de références des t\^aches. Les mouvements
générés ont été utilisés pour \emph{donner une impression de vie} au robot
HRP-2 durant une démonstration publique du robot.

\subsection*{Reconnaissance de t\^aches}
Une méthode de reconnaissance de mouvements capable de détecter 
des t\^aches effectuées en parallèle et possédant éventuellement
des couplages a été développée. Cette méthode s'appuie
sur des opérateurs de projection dans les espaces des t\^aches et leurs 
espaces orthogonaux pour d'une part projeter les mouvements à analyser
dans des espaces caractéristiques d'une t\^ache et d'autre part
pour annuler et découpler les t\^aches détectées.
La méthode de la reconnaissance de t\^aches a été implémentée
en utilisant les différentes librairies développées
en collaboration par l'équipe Gepetto au LAAS-CNRS Toulouse et le JRL Tsukuba.

\section*{Perspectives}
En ce qui concerne la reconnaissance de t\^aches,
les perspectives envisagées sont réliées aux limites
de la méthode. La reconnaissance ne s'applique que sur des segments
temporels dans lesquels les t\^aches mises en jeu ne changent pas d'état.
Ainsi il est envisageable d'ajouter des paramètres de temporels représentant le
début et la fin d'une t\^ache dans le problème 
d'optimisation des paramètres du modèle de génération de mouvements pour détecter
ces changements d'états.
L'étape qui suit logiquement est la reconnaissance de séquence de pile de t\^aches.
Un apprentissage des transitions entre les différentes instances 
de pile de t\^aches pourra être réalisé afin de construire des graphes 
de mouvements offrant une grande réactivité pour la reconnaissance et la 
reproduction.

D'autre part, le formalisme de la fonction de t\^aches
s'étend très bien en dynamique: notre méthode de reconnaissance pourra
être appliquée à des t\^aches dynamiques.
Toutes les propriétés de la fonction de t\^aches sont 
conservées. Seuls les modèles de génération et les espaces d'observation diffèrent.
Il semble donc immédiat d'adapter l'approche à la dynamique.\\

Nous avons donné une définition d'une t\^ache robotique.
Cette définition est accompagnée d'outils et de propriétés
puissantes qui nous ont permis de projeter des trajectoires
dans des espaces spécialisés où la reconnaissance est plus
facile. Mais est-il possible d'utiliser les mêmes concepts
sur un humain ou de définir d'une manière similaire des t\^aches
humaines? 

L'étude du mouvement d'attente nous a amené à réfléchir
sur ce qui caractérise un mouvement ou un comportement.
La méthode utilisée pour reproduire le mouvement
d'attente est très pragmatique: les trajectoires associées 
à certains membres sont reproduites.
L'évaluation du succès d'une imitation d'action 
est immédiate: si l'action à reproduire consiste à déplacer un objet
à un endroit donné, alors la position finale de l'objet
après l'imitation permet de valider le mouvement effectué.
En revanche il n'y a aucun critère objectif permettant de valider
le succès de l'imitation du mouvement d'attente.
Lorsqu'un humain est debout et attend,
des mouvements inconscients apparaissent: le regard
est mobile, la respiration entraîne des mouvements au niveau
du torse. Il n'est pas naturel de rester immobile.
L'exemple des statues vivantes le montre bien, puisque
de gros efforts physiques et de concentration sont nécessaires
pour contrôler sa respiration, retirer toutes les tensions inutiles, maintenir une 
posture constante et se décaler subtilement
lorsque l'attention du spectateur baisse
pour se dégourdir. Dans les deux cas, on peut considérer que l'humain \emph{ne fait rien}.
Cette étude de ce qu'est un modèle générique d'action trouverait une simplification immédiate
en introduisant des méthodes d'apprentissage automatique pour synthétiser plusieurs
démonstrations succéssives en un seul modèle numérique.

% %%%%%%%%%%%%%%%%%%%%%%%%%%%%%%%%%%%%%%%%%%%%%%%%%%%%%%%%%%%%%%%%%%%%%%%%%%%%
 


%---------------------------------------------------------------------------- %
%--- ANNEXES\underline{} ---------------------------------------------------------------- %
%---------------------------------------------------------------------------- %
%---------------------------------------------------------------------------- %
 %\appendix   
 %\chapter{Preuves}
 %\input{chapitres/annexes/annexe_preuveOrdre}
 %\label{annexe:preuve}


% % ---------------------------------------------------------------------------- %
% % --- BIBLIO ----------------------------------------------------------------- %
% % ---------------------------------------------------------------------------- %
% 
%\renewcommand\bibname{Foo}
\bibliographystyle{apalike-fr}
\bibliography{misc/bibfr}


% 
% % ---------------------------------------------------------------------------- %
% % --- TABLES ----------------------------------------------------------------- %
% % ---------------------------------------------------------------------------- %
% 
% % --->
% % ---> TABLE DES MATIERES
% % --->
% %\newpage
% %\cleardoublepage
% %\addcontentsline{toc}{chapter}{Table des mati�es}
% %\markboth{Table des mati�es}{Table des mati�es}
% %\sommaire[2]
% %\shorttableofcontents{Table des mati�es}{2}
% %\tableofcontents{Table des mati�es}
% %
% % --->
% % ---> TABLE DES FIGURES
% % --->
  \newpage
  \cleardoublepage
  \addcontentsline{toc}{chapter}{Table des figures}
  \markboth{Table des figures}{Table des figures}
  \listoffigures
 %%
% % --->
% % ---> TABLE DES ALGOS
% % --->
% %%\newpage
% %%\cleardoublepage
% %%\addcontentsline{toc}{chapter}{Liste des algorithmes}
% %%\markboth{Liste des algorithmes}{Liste des algorithmes}
% %%\listofalgorithms
% %%
% 
% % ---> ???
% %\newpage
% %\cleardoublepage
% %\addcontentsline{toc}{chapter}{Index}
% %\markbo
% 
%  
% 
% 
% 
% 
% 
% %\printindex
% 
% % ---------------------------------------------------------------------------- %
% % --- ABSTRACT --------------------------------------------------------------- %
% % ---------------------------------------------------------------------------- %
% 
% %\commentaire{%-------------------- debut commentaire -------------------
% %% Il faut mettre le resume sur la quatri�e de couverture. 
% %% Suivant que votre liste de figures comporte un nombre pair 
% %% ou impair de pages il faudra mettre un nombre pair ou impair
% %% de commandes \newpage.
\alignquatriemedecouv
 
\markboth{}{}
\pagestyle{empty}

{\small
\noindent\begin{normalsize}\textbf{Reconnaissance de t\^aches par commande inverse}\end{normalsize}\\

Des méthodes efficaces s'appuyant sur des outils statistiques pour réaliser de la reconnaissance de mouvement ont été développé. Ces méthodes reposent sur l'apprentissage de primitives situé dans des espaces approprié, par exemple l'espace latent de l'espace articulaire et/ou d'espace de tâches
adéquat. Les primitives apprises sont souvent séquentielle: un mouvement est segmenté selon l'axe des temps. Dans le cas d'un robot humanoïde, le mouvement peut être décomposé en plusieurs sous-tâches
simultanées. Par exemple dans un scénario de serveur, le robot doit placer une assiette sur la table avec une main tout en maintenant son plateau horizontal avec son autre main. La reconnaissance ne peut donc pas se limiter à une seule et unique tâche par segment de temps consécutif. La méthode présenté dans ces travaux utilise la connaissance des tâches que le robot est capable d'accomplir, ainsi que des contrôleurs qui génèreront les mouvements  pour réaliser une rétro ingénierie sur un mouvement observé. Cette analyse est destiné à reconnaître des tâches qui ont été exécuté de manière simultanées. La méthode repose sur la fonction de tâche et les projection dans l'espace nul des tâches afin de découpler les contrôleurs. L'approche a été appliqué avec succès sur un vrai robot pour distinguer des mouvements visuellement très proches, mais sémantiquement différents. 

\medskip
\noindent 

\textbf{Mots-clefs~:} Analyse de mouvements, pile de t\^aches, robotique

\vspace{0.5cm}

\noindent\begin{normalsize}\textbf{Task recognition by reverse control}\end{normalsize}\\

Efficient methods to perform motion recognition have been developed using statistical tools. Those methods rely on primitives learning in a suitable space, for example the latent space of the joint angle and/or adequate task spaces. The learned primitives are often sequential : a motion is segmented
according to the time axis. When working with a humanoid robot, a motion can be decomposed into simultaneous sub-tasks. For example in a waiter scenario, the robot has to keep some plates horizontal with one of his arms, while placing a plate on the table with its free hand. Recognition can thus not be
limited to one task per consecutive segment of time. The method presented in this work takes advantage of the knowledge of what tasks the robot is able to do and how the motion is generated from this set of known controllers to perform a reverse engineering of an observed motion. This analysis is intended to recognize simultaneous tasks that have been used to generate a motion. The method relies on the task-function formalism and the projection operation into the null space of a task to decouple the controllers. The approach is successfully applied on a real robot to disambiguate motion in different
scenarios where two motions look similar but have different purposes.

\medskip
\noindent 
\textbf{Keywords ~:} Motion analysis, stack of tasks, robotics

}



\end{document}


%<!-- Local IspellDict: francais -->
